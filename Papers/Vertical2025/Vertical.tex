\documentclass[12pt,prb,aps]{revtex4-1}
\usepackage{amsmath}           		          	
\usepackage{graphicx,epstopdf}					
\usepackage{amssymb}
\usepackage{fullpage}
\usepackage{color}
\usepackage{esint}
\pdfoutput = 1 
\newcommand {\bxi}{\mbox{\boldmath$\xi$}}
\allowdisplaybreaks

\begin{document}
\title{Calculation of Vertical Stability in an Inverse Aspect-Ratio Expanded Tokamak Plasma Equilibrium}
\author{Richard Fitzpatrick\,\footnote{rfitzp@utexas.edu}}
\affiliation{Institute for Fusion Studies, Department of Physics, University of Texas at Austin, Austin, TX 78712}
\maketitle

\section{Introduction}
It is well known that an increase in  the net toroidal plasma current flowing around a tokamak  leads to an increase in both the maximum stable $\beta$ value and the energy confinement time.\cite{troyon,goldston}
The conventional method of maximizing the total plasma current, without degrading the  stability of the system to non-axisymmetric magnetohydrodynamical (MHD) modes, is
to modify the plasma's poloidal cross-section such that it is both vertically elongated and triangular.\cite{jet}  Unfortunately, tokamak plasmas possessing strong cross-sectional shaping
are subject to severe axisymmetric instabilities.\cite{v1,v2} Such instabilities involve bulk vertical motion of the plasma on an Alfv\'{e}nic timescale (i.e., $10^{-7}$\,s), which results in the
sudden and violent termination of the  discharge when it comes into contact with the first wall. It is possible to stabilize an axisymmetric mode by placing a perfectly conducting wall
around the plasma. In reality, the mode remains unstable because  the wall inevitably possesses finite electrical conductivity.\cite{rwm}  However, 
growth time of the mode is increased from the Alfv\'{e}n time to the very much longer characteristic L/R time of the wall.\cite{rwm1,rwm1a,rwm2} Usually, the vacuum vessel plays the role of the wall,
and has an L/R time that is  well in excess of  $10^{-3}$\,s. Such a time is much shorter than the length of the plasma 
discharge, but is still long enough to allow active feedback stabilization of the axisymmetric mode with practical power supplies.\cite{rwm3} 

Ref.~\onlinecite{tj} describes the TJ toroidal tearing mode code, which calculates the stability of an inverse aspect-ratio expanded  tokamak plasma equilibrium  to {\em non-axisymmetric}\/ tearing modes via
asymptotic matching techniques. Ref.~\onlinecite{tj1} describes a generalization of the TJ code that permits it to calculate the stability of the plasma to non-axisymmetric
ideal modes in the presence of a perfectly conducting wall surrounding the plasma. The aim of this paper is to describe a further generalization the TJ code that  allows it to calculate
the stability of the plasma to {\em axisymmetric}\/ ideal modes in the presence of a resistive wall that surrounds the plasma. It might be hope that, in order to investigate axisymmetric
modes,  we could simply take the existing
TJ code and set the toroidal mode number, $n$, to zero. Unfortunately, this simple scheme does not work, as evidenced by the large number of terms involving
$n^{-1}$ in the analysis of Ref.~\onlinecite{tj}. Hence, as described in this paper, it is necessary to redo  much of the TJ analysis for the special case $n=0$. 

\section{General Plasma Equilibrium}\label{geq}
All lengths  in this paper  are normalized to  the major radius of the plasma magnetic axis, $R_0$. All magnetic field-strengths
are normalized to the  toroidal field-strength at the magnetic axis, $B_0$. All current densities are normalized to $B_0/(\mu_0\,R_0)$. 
 All plasma pressures are normalized to $B_0^{\,2}/\mu_0$. All energies are normalized to $B_0^{\,2}\,R_0^{\,3}/\mu_0$. 

Let $R$, $\phi$, $Z$ be right-handed cylindrical coordinates whose Jacobian 
is
$(\nabla R\times \nabla\phi\cdot\nabla Z)^{-1} = R$. 
Note that $|\nabla\phi|=1/R$. 
Let $r$, $\theta$, $\phi$ be right-handed flux-coordinates whose
Jacobian is\,\cite{bussac,connor}
\begin{equation}\label{jac}
{\cal J}(r,\theta)\equiv (\nabla r\times \nabla\theta\cdot\nabla\phi)^{-1} \equiv R\left(\frac{\partial R}{\partial\theta}\,\frac{\partial Z}{\partial r} -\frac{\partial R}{\partial r}\,\frac{\partial Z}{\partial \theta}\right)= r\,R^{\,2}.
\end{equation}
Note that $r=r(R,Z)$ and $\theta=\theta(R,Z)$. 
The magnetic axis corresponds to $r=0$. The plasma-vacuum interface corresponds to $r=a$. The inboard mid-plane corresponds to $\theta=0$. 

Consider an axisymmetric tokamak equilibrium whose magnetic field takes the form
\begin{equation}
{\bf B}(r,\theta) = f(r)\,\nabla\phi\times \nabla r + g(r)\,\nabla\phi = f\,\nabla(\phi-q\,\theta)\times \nabla r,
\end{equation}
where
\begin{equation}\label{q}
q(r) = \frac{r\,g}{f}
\end{equation}
is the safety-factor profile. Note that ${\bf B}\cdot\nabla r=0$, which implies that $r$ is a magnetic flux-surface label.
We require $g=1$ on the magnetic axis in order to ensure that the normalized toroidal magnetic field-strength at the  axis is unity.  

It is easily demonstrated that\,\cite{tj}
\begin{align}
B^{\,r}&={\bf B}\cdot\nabla r= 0,\label{bup1}\\[0.5ex]
B^{\,\theta} &={\bf B}\cdot\nabla \theta= \frac{f}{r\,R^{\,2}},\label{bup2}\\[0.5ex]
B^{\,\phi} &={\bf B}\cdot\nabla \phi= \frac{g}{R^{\,2}},\label{bup3}\\[0.5ex]
B_r &={\cal J}\,\nabla\theta\times\nabla\phi\cdot{\bf B}= -r\,f\,\nabla r\cdot\nabla\theta,\label{bdown1}\\[0.5ex]
B_\theta &={\cal J}\,\nabla\phi\times\nabla r\cdot{\bf B}= r\,f\,|\nabla r|^2,\\[0.5ex]
B_\phi &={\cal J}\,\nabla r\times\nabla \theta \cdot{\bf B}= g.\label{bdown3}
\end{align}

The Maxwell equation (neglecting the displacement current, because the plasma velocity perturbations due to axisymmetric modes are far smaller than the velocity of light in vacuum)
${\bf J}= \nabla\times{\bf B}$
yields
\begin{align}
{\cal J}\,J^{\,r} &= \frac{\partial B_\phi}{\partial \theta} =0,\label{jup1}\\[0.5ex]
{\cal J}\,J^{\,\theta} &= -\frac{\partial B_\phi}{\partial r} = - g',\label{jup2}\\[0.5ex]
{\cal J}\,J^{\,\phi}&= \frac{\partial B_\theta}{\partial r} -\frac{\partial B_r}{\partial\theta}=\frac{\partial}{\partial r}\!\left(r\,f\,|\nabla r|^2\right)+ \frac{\partial}{\partial\theta}\!\left(r\,f\,\nabla r\cdot\nabla\theta\right),\label{jup3}
\end{align}
where ${\bf J}$ is the equilibrium current density, $'\equiv d/dr$, and use has been made of  Eqs.~(\ref{bdown1})--(\ref{bdown3}).

Equilibrium force balance requires that
\begin{equation}\label{e15c}
 \nabla P={\bf J}\times {\bf B},
\end{equation}
where $P(r)$ is the equilibrium scalar plasma pressure. Here, for the sake of simplicity, we have neglected the small centrifugal modifications to force balance due to subsonic plasma
rotation.\cite{flow,flow1}
It follows that 
\begin{align}\label{eg1}
P'&= {\cal J}(J^{\,\theta}\,B^{\,\phi}-J^{\,\phi}\,B^{\,\theta})= -g'\,\frac{g}{R^{\,2}} - \frac{f}{r\,R^{\,2}}\left[\frac{\partial}{\partial r}\!\left(r\,f\,|\nabla r|^2\right)+ \frac{\partial}{\partial\theta}\!\left(r\,f\,\nabla r\cdot\nabla\theta\right)\right],
\end{align}
where use has been made of Eqs.~(\ref{bup1})--(\ref{bup3}), and  (\ref{jup1})--(\ref{jup3}). The
other two components of Eq.~(\ref{e15c}) are identically zero. 

Equation~(\ref{eg1}) yields the {\em inverse Grad-Shafranov equation}:\cite{connor}
\begin{equation}\label{gs}
\frac{f}{r}\,\frac{\partial}{\partial r}\!\left(r\,f\,|\nabla r|^2\right) +\frac{f}{r}\,\frac{\partial}{\partial\theta}\!\left(r\,f\,\nabla r\cdot\nabla\theta\right)+g\,g' + R^{\,2}\,P'=0.
\end{equation}
It follows from Eqs.~(\ref{q}), (\ref{jup3}), and (\ref{gs}) that
\begin{equation}\label{jup3a}
{\cal J}\,J^{\,\phi} = -q\,g' - \frac{r\,R^{\,2}\,P'}{f}.
\end{equation}
It is clear from Eqs.~(\ref{jup2}) and (\ref{jup3a}) that $g'=P'=0$ in the  current-free ``vacuum'' region surrounding the plasma, $r>a$. 
We shall also assume that $g'=P'=0$ at the plasma-vacuum interface, so as to ensure that the equilibrium plasma
current density is zero at the interface, $r=a$. 

\section{Axisymmetric Plasma Perturbation}\label{opde}

\subsection{Derivation of Axisymmetric Ideal-MHD P.D.E.s}
Let us assume that all perturbed quantities have no dependence on  the toroidal angle, $\phi$. 
The perturbed plasma equilibrium satisfies the  linearized, marginally-stable, ideal-MHD equations\,\cite{connor,am1,gs1}
\begin{align}
{\bf b} &= \nabla\times (\bxi\times {\bf B}),\label{e21}\\[0.5ex]
\nabla p &={\bf j}\times {\bf B}  +{\bf J}\times {\bf b},\label{e22}\\[0.5ex]
{\bf j} &= \nabla\times {\bf b},\label{e23}\\[0.5ex]
p&= -\bxi\cdot\nabla P,\label{e24}
\end{align}
where $\bxi(r,\theta)$ is the plasma displacement, ${\bf b}(r,\theta)$ the perturbed magnetic field,
${\bf j}(r,\theta)$ the perturbed current density, and $p(r,\theta)$ the perturbed scalar pressure. 

Now,\cite{tj}
\begin{align}
(\bxi\times {\bf B})_\theta&= {\cal J}\,(\xi^{\,\phi}\,B^{\,r} - \xi^{\,r}\,B^{\,\phi}) = -{\cal J}\,B^{\,\phi}\,\xi^{\,r},\\[0.5ex]
(\bxi\times {\bf B})_\phi &= {\cal J}\,(\xi^{\,r}\,B^{\,\theta} - \xi^{\,\theta}\,B^{\,r})= {\cal J}\,B^{\,\theta}\,\xi^{\,r},\label{e23x}
\end{align}
where use has been made of  the fact that $B^{\,r}=J^{\,r}=0$. [See Eqs.~(\ref{bup1}) and (\ref{jup1}).]
 Combining Eqs.~(\ref{e21}) and (\ref{e23x}), we obtain
\begin{align}
{\cal J}\,b^{\,r} &= \frac{\partial}{\partial\theta}\left({\cal J}\,B^{\,\theta}\,\xi^{\,r}\right).
\end{align}
Thus, Eqs.~(\ref{jac}), (\ref{q}),  and (\ref{bup2}) give\,\cite{tj}
\begin{align}\label{e41}
r\,R^{\,2}\,b^{\,r}& = \frac{\partial y}{\partial\theta},
\end{align}
where 
\begin{align}\label{e42}
y(r,\theta) &=f\,\xi^{\,r}.
\end{align}
The constraint $\nabla\cdot{\bf b} =0$, which follows from Eq.~(\ref{e21}),  immediately yields
\begin{equation}\label{e43y}
r\,R^{\,2}\,b^{\,\theta} = - \frac{\partial y}{\partial r}.
\end{equation}
Note that the preceding expression is radically different from the expression, (54), for $b^{\,\theta}$ given in Ref.~\onlinecite{tj}. Thus, it is at this stage that our analysis
starts to diverge from that of Ref.~\onlinecite{tj}.

According to Eq.~(\ref{e24}), 
\begin{equation}
p =-P'\,\nabla r\cdot\bxi=- P'\,\xi^{\,r}.
\end{equation}
So, the perturbed force balance equation, (\ref{e22}), yields
\begin{align}
-\frac{\partial\, (P'\,\xi^{\,r})}{\partial r} &= ({\bf j}\times {\bf B})_r+({\bf J}\times {\bf b})_r,\\[0.5ex]
-\frac{\partial\,(P'\,\xi^{\,r})}{\partial \theta}&= ({\bf j}\times {\bf B})_\theta+({\bf J}\times {\bf b})_\theta,\\[0.5ex]
0&= ({\bf j}\times {\bf B})_\phi+({\bf J}\times {\bf b})_\phi,
\end{align}
giving\,\cite{tj}
\begin{align}
-\frac{\partial\, (P'\,\xi^{\,r})}{\partial r} &=r\,R^{\,2}\,(j^{\,\theta}\,B^{\,\phi}-j^{\,\phi}\,B^{\,\theta}) + r\,R^{\,2}\,(J^{\,\theta}\,b^{\,\phi}-J^{\,\phi}\,b^{\,\theta}),\\[0.5ex]
-\frac{\partial\,(P'\,\xi^{\,r})}{\partial \theta}&=r\,R^{\,2}\,(j^{\,\phi}\,B^{\,r}-j^{\,r}\,B^{\,\phi}) + r\,R^{\,2}\,(J^{\,\phi}\,b^{\,r}-J^{\,r}\,b^{\,\phi}),\\[0.5ex]
0&=r\,R^{\,2}\,(j^{\,r}\,B^{\,\theta}-j^{\,\theta}\,B^{\,r}) + r\,R^{\,2}\,(J^{\,r}\,b^{\,\theta}-J^{\,\theta}\,b^{\,r}),
\end{align}
where use has been made of Eq.~(\ref{jac}). 
Thus, according to Eqs.~(\ref{bup1})--(\ref{bup3}), (\ref{jup1}), (\ref{jup2}), and (\ref{jup3a}), 
\begin{align}
-\frac{\partial\, (P'\,\xi^{\,r})}{\partial r} &= f\,(q\,j^{\,\theta} -j^{\,\phi}) - g'\,b^{\,\phi} + \left(q\,g'+\frac{r\,R^{\,2}\,P'}{f}\right)b^{\,\theta},\label{e51}\\[0.5ex]
-\frac{\partial\,(P'\,\xi^{\,r})}{\partial \theta}&=-r\,g\,j^{\,r} - \left(q\,g'+\frac{r\,R^{\,2}\,P'}{f}\right)b^{\,r},\label{e44}\\[0.5ex]
0&= f\,j^{\,r}+g'\,b^{\,r}.\label{e53}
\end{align}
It follows from Eqs.~(\ref{e41}) and (\ref{e53}) that 
\begin{equation}\label{e54}
r\,R^{\,2}\,j^{\,r} = -\alpha_g\,\frac{\partial y}{\partial\theta},
\end{equation}
where
\begin{align}
\alpha_g (r)&= \frac{g'}{f}.\label{ag}
\end{align}
Note that Eq.~(\ref{e44}) is trivially satisfied. Hence, of the three components of the perturbed force balance equation, only Eq.~(\ref{e51}) remains to be solved. 

Equation~(\ref{e23}) yields\,\cite{tj}
\begin{align}
r\,R^{\,2}\,j^{\,r} &= \frac{\partial b_\phi}{\partial\theta},\label{e57}\\[0.5ex]
r\,R^{\,2}\,j^{\,\theta} &= -\frac{\partial b_\phi}{\partial r},\label{e58}\\[0.5ex]
r\,R^{\,2}\,j^{\,\phi}&= \frac{\partial b_\theta}{\partial r} -\frac{\partial b_r}{\partial \theta},\label{e59}
\end{align}
where use has been made of Eq.~(\ref{jac}). It follows from Eqs.~(\ref{e54}), (\ref{e57}), and (\ref{e58}) that
\begin{align}\label{e43yy}
b_\phi &=-\alpha_g\,y,\\[0.5ex]
r\,R^{\,2}\,j^{\,\theta}&=  \frac{\partial (\alpha_g\,y)}{\partial r}.\label{e44yy}
\end{align}
Note that $\nabla\cdot{\bf j} = 0$, in accordance with Eq.~(\ref{e23}).

Now, 
\begin{equation}
{\bf b} = b_r\,\nabla r + b_\theta\,\nabla\theta+b_\phi\,\nabla\phi,
\end{equation}
so
\begin{align}
b^{\,r} &= {\bf b}\cdot\nabla r = |\nabla r|^2\,b_r + (\nabla r\cdot\nabla\theta)\,b_\theta,\label{e61}\\[0.5ex]
b^{\,\theta} &= {\bf b}\cdot\nabla \theta = (\nabla r\cdot\nabla\theta)\,b_r + |\nabla\theta|^2\,b_\theta,\label{e62}\\[0.5ex]
b^{\,\phi}&={\bf b}\cdot\nabla\phi =\frac{b_\phi}{R^{\,2}}.\label{e63}
\end{align}

Equations~(\ref{jac}), (\ref{e61}), and (\ref{e62}) can be rearranged to give\,\cite{tj}
\begin{align}
b_r &= \left(\frac{1}{|\nabla r|^2}\right)b^{\,r}- \left(\frac{\nabla r\cdot\nabla\theta}{|\nabla r|^2}\right)b_\theta,\label{e69}\\[0.5ex]
\label{e58x}
b^{\,\theta}& = \left(\frac{\nabla r\cdot\nabla\theta}{|\nabla r|^2}\right)b^{\,r} + \left(\frac{1}{r^2\,R^{\,2}\,|\nabla r|^2}\right) b_\theta.
\end{align}
Let 
\begin{equation}\label{zdef}
{\cal Z}(r,\theta)= |\nabla r|^2\,r\,\frac{\partial y}{\partial r} + r\,\nabla r\cdot\nabla\theta\,\frac{\partial y}{\partial\theta}.
\end{equation}
Equations~(\ref{e41}), (\ref{e43y}), (\ref{e43yy}), (\ref{e69}) and (\ref{e58x}) yield 
\begin{align}\label{e53y}
b_r &=\frac{1}{r\,|\nabla r|^2\,R^{\,2}}\,\frac{\partial y}{\partial \theta} + \frac{\nabla r\cdot\nabla\theta}{|\nabla r|^2}\,{\cal Z},\\[0.5ex]
b_\theta &= -{\cal Z},\label{e54y}\\[0.5ex]
b^{\,\phi} &= -\frac{\alpha_g}{R^{\,2}}\,y.\label{e55yy}
\end{align}
Equations~(\ref{e59}), (\ref{e53y}), and (\ref{e54y}) give
\begin{align}\label{e56yy}
r\,R^{\,2}\,j^{\,\phi} &=-\frac{\partial {\cal Z}}{\partial r}-\frac{\partial}{\partial\theta}\!\left[\frac{1}{r\,|\nabla r|^2\,R^{\,2}}\,\frac{\partial y}{\partial \theta} + \frac{\nabla r\cdot\nabla\theta}{|\nabla r|^2}\,{\cal Z}\right].
\end{align}

It follows from Eqs.~(\ref{e42}), (\ref{e43y}), (\ref{e51}), (\ref{e44yy}), (\ref{e55yy}), and (\ref{e56yy}) that
\begin{align}
-\frac{\partial}{\partial r}\!\left(\frac{P'}{f}\,y\right)&= \frac{f\,q}{r\,R^{\,2}}\,\frac{\partial (\alpha_g\,y)}{\partial r} 
+ \frac{f}{r\,R^{\,2}}\,\frac{\partial {\cal Z}}{\partial r}\nonumber\\[0.5ex]
&\phantom{=} +\frac{f}{r\,R^{\,2}}\frac{\partial}{\partial\theta}\!\left[\frac{1}{r\,|\nabla r|^2\,R^{\,2}}\,\frac{\partial y}{\partial \theta} + \frac{\nabla r\cdot\nabla\theta}{|\nabla r|^2}\,{\cal Z}\right]
\nonumber\\[0.5ex]
&\phantom{=} + \frac{g'\,\alpha_g}{R^{\,2}}\,y - \left(q\,g'+\frac{r\,R^{\,2}\,P'}{f}\right)\frac{1}{r\,R^{\,2}}\,\frac{\partial y}{\partial r}.
\end{align}
Hence,
\begin{align}\label{e58u}
-\left[(\alpha_f\,\alpha_p+ r\,\alpha_p')R^{\,2} + q\,r\,\alpha_g'+r^2\,\alpha_g^{\,2}\right]y&= 
 r\,\frac{\partial {\cal Z}}{\partial r}+\frac{\partial}{\partial\theta}\!\left[\frac{1}{|\nabla r|^2\,R^{\,2}}\,\frac{\partial y}{\partial \theta} + \frac{r\,\nabla r\cdot\nabla\theta}{|\nabla r|^2}\,{\cal Z}\right],
\end{align}
where
\begin{align}\label{alpp}
\alpha_p(r)&= \frac{r\,P'}{f^2},\\[0.5ex]
\alpha_f(r) &= \frac{r^2}{f}\,\frac{d}{dr}\!\left(\frac{f}{r}\right).\label{alpf}
\end{align}

Finally, Eqs.~(\ref{zdef}) and (\ref{e58u})  yield the {\em axisymmetric ideal-MHD partial differential equations (p.d.e.s)},
\begin{align}\label{e63y}
r\,\frac{\partial y}{\partial r} &= \frac{{\cal Z}}{|\nabla r|^2} - \frac{r\,\nabla r\cdot\nabla\theta}{|\nabla r|^2}\,\frac{\partial y}{\partial\theta},\\[0.5ex]
r\,\frac{\partial {\cal Z}}{\partial r}&= -\left[(\alpha_f\,\alpha_p+ r\,\alpha_p')R^{\,2} + q\,r\,\alpha_g'+r^2\,\alpha_g^{\,2}\right]y-\frac{\partial}{\partial\theta}\!\left(\frac{1}{|\nabla r|^2\,R^{\,2}}\,\frac{\partial y}{\partial\theta}\right)\nonumber\\[0.5ex]
&\phantom{=} -\frac{\partial}{\partial\theta}\!\left(\frac{r\,\nabla r\cdot\nabla\theta}{|\nabla r|^2}\,{\cal Z}\right),\label{e64y}
\end{align}
which govern the perturbed equilibrium. 

\subsection{Derivation of the Axisymmetric Ideal-MHD O.D.E.s}\label{ode}
Let
\begin{align}\label{e62s}
y(r,\theta)&= \sum_m y_m(r)\,{\rm e}^{\,{\rm i}\,m\,\theta},\\[0.5ex]
{\cal Z}(r,\theta) &= \sum_m Z_m(r)\,{\rm e}^{\,{\rm i}\,m\,\theta}.\label{e63s}
\end{align}
Here, the $m$ are the set of (integer, but not necessarily positive) poloidal mode numbers included in the calculation. 
Equations~(\ref{e63y}) and (\ref{e64y}) yield the {\em axisymmetric ideal-MHD ordinary differential equations (o.d.e.s)}:
\begin{align}\label{e69u}
r\,\frac{dy_m}{dr}&= \sum_{m'}\left(A_{m}^{\,m'}\,Z_{m'} + B_{m}^{\,m'}\,y_{m'}\right),\\[0.5ex]
r\,\frac{dZ_m}{dr}&= \sum_{m'}\left(C_{m}^{\,m'}\,Z_{m'} + D_{m}^{\,m'}\,y_{m'}\right),\label{e70uu}
\end{align}
where
\begin{align}
A_m^{\,m'} &= c_{m}^{\,m'},\\[0.5ex]
B_m^{\,m'} &= - m'\,f_m^{\,m'},\\[0.5ex]
C_{m}^{\,m'} &= -m\,f_m^{\,m'},\\[0.5ex]
D_{m}^{\,m'}&= -(\alpha_f\,\alpha_p+ r\,\alpha_p')\,a_m^{\,m'} - (q\,r\,\alpha_g' +r^{\,2}\,\alpha_g^{\,2})\,\delta_m^{\,m'}+m\,m'\,b_m^{\,m'},\label{Ddef}
\end{align}
and
\begin{align}\label{e73er}
a_m^{\,m'}(r)  &=\oint R^{\,2}\,\exp[-{\rm i}\,(m-m')\,\theta]\,\frac{d\theta}{2\pi},\\[0.5ex]
b_m^{\,m'}(r)  &=\oint|\nabla r|^{-2}\, R^{-2}\,\exp[-{\rm i}\,(m-m')\,\theta]\,\frac{d\theta}{2\pi},\\[0.5ex]
c_m^{\,m'}(r)  &=\oint|\nabla r|^{-2}\,\exp[-{\rm i}\,(m-m')\,\theta]\,\frac{d\theta}{2\pi},\\[0.5ex]
f_{m}^{\,m'}(r)&= \oint \frac{{\rm i}\,r\,\nabla r\cdot\nabla\theta}{|\nabla r|^2}\,\exp[-{\rm i}\,(m-m')\,\theta]\,\frac{d\theta}{2\pi}.\label{e76er}
\end{align}
Here, $\delta_m^{\,m'}$ is a Kronecker delta symbol. Note that $Z_0$ is independent of $r$ in the vacuum region, $r>a$,  in which
$\alpha_g=\alpha_p=0$. 

The axisymmetric ideal-MHD o.d.e.s play the same role for axisymmetric perturbations that the outer-region o.d.e.s, (102) and (103) of Ref.~\onlinecite{tj}, play for
non-axisymmetric perturbations. The main difference is that there are no singularities in the axisymmetric ideal-MHD o.d.e.s because an axisymmetric perturbation 
does not resonate with the plasma. In other words, there are no equilibrium flux-surfaces in the plasma at which ${\bf k}\cdot{\bf B}=0$, where ${\bf k}$ is the
wavevector of the perturbation. This is because $|B^\theta(r)|>0$ throughout a conventional tokamak equilibrium. 

\subsection{Properties of Axisymmetric Ideal-MHD O.D.E.s}
Note   that
$a_{m'}^{\,m} =a_{m}^{\,m'\ast}$, 
$b_{m'}^{\,m} =b_{m}^{\,m'\ast}$, 
$c_{m'}^{\,m} =c_{m}^{\,m'\ast}$, and 
$f_{m'}^{\,m} =-f_{m}^{\,m'\ast}$,
which implies that
\begin{align}\label{e70u}
A_{m'}^{\,m} &= A_{m'}^{\,m\,\ast},\\[0.5ex]
B_{m'}^{\,m} &= -C_{m'}^{\,m\,\ast},\\[0.5ex]
C_{m'}^{\,m} &= -B_{m'}^{\,m\,\ast},\\[0.5ex]
D_{m'}^{\,m} &= D_{m'}^{\,m\,\ast}.\label{e82u}
\end{align}
It follows from Eqs.~(\ref{e69u}), (\ref{e70uu}), and (\ref{e70u})--(\ref{e82u}) that 
\begin{align}
r\,\frac{d}{dr}\!\left[\sum_m (Z_m\,y_m^{\,\ast}- y_m\,Z_m^{\,\ast}) \right]&= 0.
\end{align}
Thus, checking that $\sum_m (Z_m\,y_m^{\,\ast}- y_m\,Z_m^{\,\ast})$ is indeed constant throughout the plasma is a good numerical check on the accuracy of the TJ
code. 

\subsection{Perturbed Electric Field}
Suppose that all perturbed quantities vary in time as ${\rm e}^{-{\rm i}\,\omega\,t}$. 
Let ${\bf e}$ be the perturbed electric field, which satisfies 
\begin{equation}
\nabla\times {\bf e} = {\rm i}\,\omega\,{\bf b}.
\end{equation}
Hence,
\begin{equation}\label{ephi}
e_\phi = {\rm i}\,\omega\,y,
\end{equation}
and 
\begin{equation}\label{e81op}
\frac{\partial e_\theta}{\partial r} - \frac{\partial e_r}{\partial\theta} = - {\rm i}\,\omega\,r\,\alpha_g\,y,
\end{equation}
where use has been made of Eqs.~(\ref{e41}), (\ref{e43y}), and (\ref{e55yy}).
We also expect $\nabla\cdot{\bf e} =0$, which implies that 
$e_r=e_\theta=0$ in the vacuum region, $r\geq a$, in which $\alpha_g=0$. 

\subsection{Toroidal Electromagnetic Torque}\label{storque}
The net toroidal electromagnetic torque exerted on the region of the plasma lying within the magnetic flux-surface whose label is $r$ is\,\cite{tj}
\begin{equation}
T_\phi(r)= \oint\oint r\,R^{\,2}\,b_\phi\,b^{\,r}\,d\theta\,d\phi.
\end{equation}
It follows from Eqs.~(\ref{e41}) and (\ref{e43yy}) that 
\begin{equation}
T_\phi(r) = -\pi\,\alpha_g\oint\left(y^\ast\,\frac{\partial y}{\partial\theta}+y\,\frac{\partial y^\ast}{\partial\theta}\right)d\theta
= -\pi\,\alpha_g\oint\frac{\partial|y|^2}{\partial\theta}\,d\theta = 0.
\end{equation}
We conclude, not surprisingly,  that an axisymmetric perturbation is incapable of exerting a net toroidal electromagnetic torque on the plasma. 

In Ref.~\onlinecite{tj1}, the conservation of toroidal angular momentum is used to prove that the plasma and vacuum perturbed potential energy matrices are Hermitian.
The fact that the toroidal electromagnetic torque is automatically zero in the axisymmetric case negates this proof. Hence, we need to find a new conserved quantity
that does not automatically take the value zero. The electromagnetic energy flux is found to play this role. 

\subsection{Electromagnetic Energy Flux}
The net flux of electromagnetic energy across the plasma-vacuum interface is 
\begin{align}\label{eflux}
{\cal E} &= \left[\oint\oint ({\bf e}\times {\bf b})\cdot\nabla r\,{\cal J}\,d\theta\,d\phi\right]_{r=a}= \left[\oint\oint (e_\theta\,b_\phi-e_\phi\,b_\theta)\,d\theta\,d\phi\right]_{r=a}\nonumber\\[0.5ex]
&={\rm i}\,\pi\,\omega\oint (y\,{\cal Z}^{\,\ast}-y^\ast \,{\cal Z})_{r=a}\,d\theta={\rm i}\,\pi^2\,\omega \sum_{m}\left(Z_m^{\,\ast}\,y_m-y_m^\ast\,Z_m\right)_{r=a}.
\end{align}
Here, use has been made of Eqs.~(\ref{e54y}), (\ref{e62s}), (\ref{e63s}),  and  (\ref{ephi}), as well as  the fact that $e_\theta=0$ for $r\geq a$. 

\subsection{Perturbed Plasma Potential Energy}
The perturbed plasma potential energy in the region of the plasma lying within the magnetic flux-surface whose label is $r$ is\,\cite{tj1,gs1}
\begin{equation}
\delta W_p = \frac{1}{2}\oint\oint r\,R^{\,2}\,\xi^{\,r\ast}(-{\bf B}\cdot{\bf b} + \xi^{\,r}\,P')\,d\theta\,d\phi.
\end{equation}
However,
\begin{equation}
{\bf B}\cdot{\bf b} -\xi^{\,r}\,P' = B^{\,\theta}\,b_\theta+B^{\,\phi}\,b_\phi - \xi^{\,r}\,P'
= - \frac{f}{r\,R^{\,2}}\left({\cal Z} + q\,\alpha_g\,y + \alpha_p\,R^{\,2}\right),
\end{equation}
where use has been made of Eqs.~(\ref{q})--(\ref{bup3}), (\ref{e42}), (\ref{e43yy}), (\ref{e54y}),  and (\ref{alpp}). Hence,
we obtain
\begin{align}\label{e87j}
\delta W_p(r) &=\frac{1}{2}\oint\oint y^{\ast}\left[{\cal Z} + (q\,\alpha_g + \alpha_p\,R^{\,2})\,y\right]d\theta\,d\phi= \pi^2\sum_{m}y_m^{\,\ast}\,\chi_m,
\end{align}
where 
\begin{equation}\label{chidef}
\chi_m(r)=Z_m + q\,\alpha_g\,y_m + \alpha_p\sum_{m'} a_m^{\,m'}\,y_{m'}.
\end{equation}

\section{Inverse Aspect-Ratio Expanded Tokamak Equilibrium}\label{large}

\subsection{Equilibrium Magnetic Flux-Surfaces}\label{flux}
Let us assume that the inverse aspect-ratio of the plasma, $\epsilon=a/R_0=a$ (since $R_0$ is normalized to unity), is such that $0<\epsilon\ll 1$.  
Let $r=\epsilon\,\hat{r}$, $\nabla =\epsilon^{\,-1}\,\hat{\nabla}$, and $'\rightarrow \epsilon^{\,-1}\,'$. 
Suppose that the loci of the equilibrium magnetic flux-surfaces can be written in the parametric form:\,\cite{tj,tj1,connor}
\begin{align}
R(\hat{r},\omega) &= 1 -\epsilon\,\hat{r}\,\cos\omega + \epsilon^{\,2}\,\sum_{j>0}H_j(\hat{r})\,\cos[(j-1)\,\omega] + \epsilon^{\,2}\,\sum_{j>1}V_j(\hat{r})\,\sin[(j-1)\,\omega] \nonumber\\[0.5ex]
&\phantom{=}+\epsilon^{\,3}\,L(\hat{r})\,\cos\omega,\label{e19x}\\[0.5ex]
Z(\hat{r},\omega)&= \epsilon\,\hat{r}\,\sin\omega +\epsilon^{\,2}\,\sum_{j>1}H_j(\hat{r})\,\sin[(j-1)\,\omega]
-\epsilon^{\,2}\,\sum_{j>1}V_j(\hat{r})\,\cos[(j-1)\,\omega]\nonumber\\[0.5ex]&\phantom{=}-\epsilon^{\,3}\,L(\hat{r})\,\sin\omega,\label{e20x}
\end{align}
where $j$ is a positive integer. 
Here, $H_1(\hat{r})$  controls the relative horizontal locations of the flux-surface centroids, $H_2(\hat{r})$ and $V_2(\hat{r})$ control the 
magnitudes and vertical tilts of the flux-surface ellipticities, $H_3(\hat{r})$ and
$V_3(\hat{r})$ control the magnitudes and vertical tilts of the flux-surface triangularities, et cetera, whereas $L(\hat{r})$ is a
flux-surface re-labelling parameter. Moreover, $\omega(R,Z)$ is a  poloidal angle that is distinct from $\theta$. Note that $V_1$ does not appear in Eq.~(\ref{e20x})
because such a factor merely gives rise to a rigid vertical shift of the plasma that can be eliminated by a suitable choice of the
origin of the flux-coordinate system.

Let
\begin{equation}
J(\hat{r},\omega) = \frac{1}{\epsilon^{\,2}}\left(\frac{\partial R}{\partial\omega}\,\frac{\partial Z}{\partial \hat{r}} -\frac{\partial R}{\partial \hat{r}}\,\frac{\partial Z}{\partial \omega}\right)
\end{equation}
be the Jacobian of the $\hat{r}$, $\omega$ coordinate system. We can transform to the $\hat{r}$, $\theta$ coordinate system 
by writing
\begin{align}\label{e11}
\theta(\hat{r},\omega) &= \left.2\pi\int_0^\omega \frac{J(\hat{r},\tilde{\omega})}{R(\hat{r},\tilde{\omega})}\,d\tilde{\omega}\right/\oint\frac{J(\hat{r},\omega)}{R(\hat{r},\omega)}\,d\omega,\\[0.5ex]
\hat{r}&=\frac{1}{2\pi}\oint\frac{J(\hat{r},\omega)}{R(\hat{r},\omega)}\,d\omega.\label{e12}
\end{align}
This transformation ensures that 
\begin{equation}
\frac{\partial\theta}{\partial\omega} = \frac{J}{\hat{r}\,R},
\end{equation}
and, hence, that 
\begin{equation}
{\cal J} \equiv \frac{R}{\epsilon} \left(\frac{\partial R}{\partial\theta}\,\frac{\partial Z}{\partial \hat{r}} -\frac{\partial R}{\partial \hat{r}}\,\frac{\partial Z}{\partial \hat{r}}\right)
=\epsilon\, R\,J\,\frac{\partial\omega}{\partial\theta} =r\,R^{\,2},
\end{equation}
in accordance with Eq.~(\ref{jac}). 

\subsection{Metric Elements}\label{metric}
We can determine the metric elements of the flux-coordinate system by combining Eqs.~(\ref{e19x})--(\ref{e12}).
Evaluating the elements up to ${\cal O}(\epsilon)$, but retaining ${\cal O}(\epsilon^{\,2})$ contributions to terms that are independent of
$\omega$, we obtain,\cite{tj,tj1}
\begin{align}\label{epdef}
L(\hat{r})&= \frac{\hat{r}^{\,3}}{8} -\frac{\hat{r}\,H_1}{2}-\frac{1}{2}\sum_{j>1}(j-1)\,\frac{H_j^{\,2}}{\hat{r}}
-\frac{1}{2}\sum_{j>1}(j-1)\,\frac{V_j^{\,2}}{\hat{r}},\\[0.5ex]
\theta &= \omega+\epsilon\,\hat{r}\,\sin\omega - \epsilon\sum_{j>0}\frac{1}{j}\left[H_j'-(j-1)\,\frac{H_j}{\hat{r}}\right]\sin(j\,\omega)
\nonumber\\[0.5ex]&\phantom{=}+ \epsilon\sum_{j>1}\frac{1}{j}\left[V_j'-(j-1)\,\frac{V_j}{\hat{r}}\right]\cos(j\,\omega),\label{e22y}\\[0.5ex]
|\hat{\nabla} \hat{r}|^2 &= 1 +2\,\epsilon\sum_{j>0}H_j'\,\cos(j\,\theta) +2\,\epsilon\sum_{j>1}V_j'\,\sin(j\,\theta) \nonumber\\[0.5ex]
&\phantom{=}+\epsilon^{\,2}\left(\frac{3\,\hat{r}^{\,2}}{4}-H_1+
\frac{1}{2}\sum_{j>0}\left[H_j'^{\,2}+(j^2-1)\,\frac{H_j^{\,2}}{\hat{r}^{\,2}}\right]\right.\nonumber\\[0.5ex]&\phantom{=}\left.+
\frac{1}{2}\sum_{j>1}\left[V_j'^{\,2}+(j^2-1)\,\frac{V_j^{\,2}}{\hat{r}^{\,2}}\right]
\right),\label{e19}\\[0.5ex]
\hat{\nabla}\hat{r}\cdot\hat{\nabla}\theta&=\epsilon\,\sin\theta
-\epsilon\sum_{j>0}\frac{1}{j}\left[H_j''+\frac{H_j'}{\hat{r}}+(j^2-1)\,\frac{H_j}{\hat{r}^{\,2}}\right]\sin(j\,\theta)\nonumber\\[0.5ex]&
\phantom{=}+\epsilon\sum_{j>1}\frac{1}{j}\left[V_j''+\frac{V_j'}{\hat{r}}+(j^2-1)\,\frac{V_j}{\hat{r}^{\,2}}\right]\cos(j\,\theta),\label{e20uu}
\\[0.5ex]
R^{\,2}&= 1-2\,\epsilon\,\hat{r}\,\cos\theta -\epsilon^{\,2}\left(\frac{\hat{r}^{\,2}}{2}-\hat{r}\,H_1'-2\,H_1\right).\label{e25a}
\end{align}
Here, $'\equiv d/d\hat{r}$. Moreover, we have made use of the fact that $V_j\propto H_j$, for $j>1$, because
$V_j$ and $H_j$ satisfy the identical differential equations, (\ref{e33x}) and (\ref{e28}). 

\subsection{Expansion of Inverse Grad-Shafranov Equation}\label{exp}
Let us write\,\cite{tj,tj1}
\begin{align}\label{e26v}
f(\hat{r})&= \epsilon\,\frac{\hat{r}\,g}{q},\\[0.5ex]
g(\hat{r}) &= 1+ \epsilon^{\,2}\,g_2(\hat{r}) + \epsilon^{\,4}\,g_4(\hat{r}),\label{e27v}\\[0.5ex]
P'(\hat{r}) &= \epsilon^{\,2}\,p_2'(\hat{r}),\label{eq1}
\end{align}
where $q$,  $g_2$, $g_4$, and $p_2$ are all ${\cal O}(1)$. Here, the safety-factor, $q(\hat{r})$, and the second-order plasma
pressure gradient, $p_2'(\hat{r})$, are the two free flux-surface functions that characterize the plasma equilibrium.

Expanding the inverse Grad-Shafranov equation, (\ref{gs}), order by order in the
small parameter $\epsilon$, making use of Eqs.~(\ref{e19})--(\ref{eq1}), we obtain\,\cite{am1,tj,tj1}
\begin{align}
g_2'&=- p_2' - \frac{\hat{r}}{q^2}\,(2-s),\label{e26}\\[0.5ex]
H_1''&= -(3-2\,s)\,\frac{H_1' }{\hat{r}}
-1+\frac{2\,p_2'\,q^2}{\hat{r}},\label{e27}\\[0.5ex]
H_j''&= -(3-2\,s)\,\frac{H_j'}{\hat{r}}+(j^2-1)\,\frac{H_j}{\hat{r}^{\,2}}~~~~~\mbox{for $j>1$},\label{e33x}\\[0.5ex]
V_j''&= -(3-2\,s)\,\frac{V_j'}{\hat{r}}+(j^2-1)\,\frac{V_j}{\hat{r}^{\,2}}~~~~~~\mbox{for $j>1$},\label{e28}\\[0.5ex]
g_4'&= g_2\left[p_2' - \frac{\hat{r}}{q^2}\,(2-s)\right] - \frac{\hat{r}}{q}\,{\mit\Sigma}
+p_2'\left(\frac{\hat{r}^{\,2}}{2}+\frac{\hat{r}^{\,2}}{q^2}-2\,H_1-3\,\hat{r}\,H_1'\right),\label{e31}
\end{align}
where $s=\hat{r}\,q'/q$ is the magnetic shear, and  
\begin{align}{\mit\Sigma} &=\frac{S_2}{q}- \frac{2-s}{q}\,S_3\\[0.5ex]
S_1(\hat{r})&=\frac{1}{2}\sum_{j>0} \left[3\,H_j'^{\,2} -(j^2-1)\,\frac{H_j^{\,2}}{\hat{r}^2}\right]
 +\frac{1}{2}\sum_{j>1} \left[3\,V_j'^{\,2} -(j^2-1)\,\frac{V_j^{\,2}}{\hat{r}^2}\right],\label{e109}
\\[0.5ex]
S_2(\hat{r})&=\frac{3\,\hat{r}^{\,2}}{2} -2\,\hat{r}\,H_1'+\sum_{j>0}\left[H_j'^{\,2}+2\,(j^2-1)\,\frac{H_j'\,H_j}{\hat{r}}-(j^2-1)\,\frac{H_j^{\,2}}{\hat{r}^{\,2}}\right]\nonumber\\[0.5ex]
&\phantom{===}+\sum_{j>1}\left[V_j'^{\,2}+2\,(j^2-1)\,\frac{V_j'\,V_j}{\hat{r}}-(j^2-1)\,\frac{V_j^{\,2}}{\hat{r}^{\,2}}\right],\\[0.5ex]
S_3(\hat{r})&=-\,\frac{3\,\hat{r}^{\,2}}{4}+\frac{\hat{r}^{\,2}}{q^2} +H_1+S_1,\\[0.5ex]
S_4(\hat{r}) &= \frac{7\,\hat{r}^{\,2}}{4}-H_1 -3\,\hat{r}\,H_1' + S_1.
\end{align}
Note that the relative horizontal shift of magnetic flux-surfaces, $-H_1$, otherwise known as the {\em Shafranov shift}, is driven by toroidicity [the second term on
the right-hand side of Eq.~(\ref{e27})], and plasma pressure gradients (the third term). All of the other shaping terms (i.e., the $H_j$, for $j>1$, and
the $V_j$) are driven by axisymmetric currents flowing in external  magnetic field-coils.

Equations~(\ref{ag}), (\ref{alpp}), (\ref{alpf}), and (\ref{e26v})--(\ref{eq1}) yield\,\cite{tj}
\begin{align}
\alpha_p(\hat{r}) &= \frac{p_2'\,q^2}{\hat{r}}\left(1-2\,\epsilon^{\,2}\,g_2\right),\\[0.5ex]
\alpha_g(\hat{r}) &= \frac{q}{\hat{r}}\left(g_2' -\epsilon^{\,2}\,g_2\,g_2'+\epsilon^{\,2}\,g_4'\right),\\[0.5ex]
\alpha_f(\hat{r}) &= -s + \epsilon^{\,2}\,\hat{r}\,g_2'.
\end{align}
Finally, it follows from Eqs.~(\ref{e20uu}) and (\ref{e27})--(\ref{e28}) that
\begin{align}\label{e114}
\hat{\nabla}\hat{r}\cdot\hat{\nabla}\theta &= 2\,\epsilon\left[1-\frac{p_2'\,q^2}{\hat{r}} +(1-s)\,\frac{H_1'}{\hat{r}}\right]\sin\theta\nonumber\\[0.5ex]
&\phantom{=}-2\,\epsilon\sum_{j>1}\frac{1}{j}\left[-(1-s)\,\frac{H_j'}{\hat{r}} + (j^2-1)\,\frac{H_j}{\hat{r}^{\,2}}\right]\sin(j\,\theta)\nonumber\\[0.5ex]
&\phantom{=}+2\,\epsilon\sum_{j>1}\frac{1}{j}\left[-(1-s)\,\frac{V_j'}{\hat{r}} + (j^2-1)\,\frac{V_j}{\hat{r}^{\,2}}\right]\cos(j\,\theta).
\end{align}

\subsection{Calculation of Coupling Coefficients}
Equations~(\ref{e19}) and (\ref{e109}) yield 
\begin{align}\label{e115}
|\hat{\nabla}\hat{r}|^{-2} &= 1 - 2\,\epsilon\sum_{j>0}H_j'\,\cos(j\,\theta) - 2\,\epsilon\sum_{j>0}V_j'\sin(j\,\theta)
+ \epsilon^2\left(-\frac{3\,\hat{r}^{\,2}}{4} + H_1 + S_1\right),
\end{align}
Equation~(\ref{e25a}) gives
\begin{align}
R^{-2}= 1 + 2\,\epsilon\,\hat{r}\,\cos\theta +\epsilon^2\left(\frac{5\,\hat{r}^{\,2}}{2} - \hat{r}\,H_1'-2\,H_1\right).
\end{align}
The previous two equations imply that
\begin{align}\label{e117}
|\hat{\nabla}\hat{r}|^{-2} \,R^{-2}&= 1 +2\,\epsilon\,\hat{r}\,\cos\theta -2\,\epsilon \sum_{j>0}H_j'\,\cos(j\,\theta) -2\,\epsilon\sum_{>1}V_j'\,\sin(j\,\theta)\nonumber\\[0.5ex]
&\phantom{=} + \epsilon^2\left(\frac{7\,\hat{r}^{\,2}}{4} - H_1 -3\,\hat{r}\,H_1' + S_1\right).
\end{align}
Finally, Eqs.~(\ref{e114}) and (\ref{e115}) yield
\begin{align}\label{e118}
\hat{\nabla}\hat{r}\cdot\hat{\nabla}\theta \,|\hat{\nabla}\hat{r}|^{-2} &= 2\,\epsilon\left[1-\frac{p_2'\,q^2}{\hat{r}} +(1-s)\,\frac{H_1'}{\hat{r}}\right]\sin\theta\nonumber\\[0.5ex]
&\phantom{=}-2\,\epsilon\sum_{j>1}\frac{1}{j}\left[-(1-s)\,\frac{H_j'}{\hat{r}} + (j^2-1)\,\frac{H_j}{\hat{r}^{\,2}}\right]\sin(j\,\theta)\nonumber\\[0.5ex]
&\phantom{=}+2\,\epsilon\sum_{j>1}\frac{1}{j}\left[-(1-s)\,\frac{V_j'}{\hat{r}} + (j^2-1)\,\frac{V_j}{\hat{r}^{\,2}}\right]\cos(j\,\theta),
\end{align}
where use has been made of the fact that $V_j(\hat{r})\propto H_j(\hat{r})$ for $j>1$. 

Equations~(\ref{e73er})--(\ref{e76er}), (\ref{e25a}), (\ref{e115}), (\ref{e117}), and (\ref{e118}) imply that
\begin{align}\label{e120}
a_m^{\,m'} &= \delta_{m}^{\,m'} -\epsilon\,\hat{r}\left(\delta_{m'-m-1}+\delta_{m'-m+1}\right)-\epsilon^2\left(\frac{\hat{r}^{\,2}}{2}-\hat{r}\,H_1'-2\,H_1\right)\delta_m^{\,m'},\\[0.5ex]
b_m^{\,m'}&= \delta_m^{\,m'}+\epsilon\,\hat{r}\left(\delta_{m'-m-1}+\delta_{m'-m+1}\right)-\epsilon\sum_{j>0}H_j'\left(\delta_{m'-m-j}+\delta_{m'-m+j}\right)\nonumber\\[0.5ex]
&- \epsilon\sum_{j>1}{\rm i}\,V_j'\left(\delta_{m'-m-j} - \delta_{m'-m+j}\right)+ \epsilon^2\left(\frac{7\,\hat{r}^{\,2}}{4} - H_1 -3\,\hat{r}\,H_1' + S_1\right)\delta_{m}^{\,m'},\\[0.5ex]
c_m^{\,m'} &= \delta_m^{\,m'}-\epsilon\sum_{j>0}H_j'\left(\delta_{m'-m-j}+\delta_{m'-m+j}\right)- \epsilon\sum_{j>1}{\rm i}\,V_j'\left(\delta_{m'-m-j} - \delta_{m'-m+j}\right)\nonumber\\[0.5ex]
&\phantom{=}+ \epsilon^2\left(-\frac{3\,\hat{r}^{\,2}}{4} + H_1 + S_1\right)\delta_{m}^{\,m'},\\[0.5ex]
f_m^{\,m'} &= -\epsilon\left[\hat{r} - p_2'\,q^2+ (1-s)\,H_1'\right]\left(\delta_{m'-m-1}-\delta_{m'-m+1}\right) \nonumber\\[0.5ex]
&\phantom{=}+ \epsilon\sum_{j>1}\frac{1}{j}\left[-(1-s)\,H_j'+(j^2-1)\,\frac{H_j}{\hat{r}}\right]\left(\delta_{m'-m-j}-\delta_{m'-m+j}\right)\nonumber\\[0.5ex]
&\phantom{=}+ \epsilon\sum_{j>1}\frac{{\rm i}}{j}\left[-(1-s)\,V_j'+(j^2-1)\,\frac{V_j}{\hat{r}}\right]
\left(\delta_{m'-m-j}+\delta_{m'-m+j}\right).\label{e123}
\end{align}

If we write
\begin{align}
\alpha_g&= \alpha_g^{\,(0)} + \epsilon^2\,\alpha_g^{\,(2)},\\[0.5ex]
\alpha_p&= \alpha_p^{\,(0)} + \epsilon^2\,\alpha_p^{\,(2)},\\[0.5ex]
\alpha_f&= \alpha_f^{\,(0)} + \epsilon^2\,\alpha_f^{\,(2)},\\[0.5ex]
a_{m}^{m'} &= 1 + \epsilon\,a_m^{\,m'\,(1)} + \epsilon^2\,a_{m}^{\,m'\,(2)},\\[0.5ex]
b_{m}^{\,m'} &= 1 + \epsilon\,b_m^{\,m'\,(1)} + \epsilon^2\,b_{m}^{\,m'\,(2)},\\[0.5ex]
D_{m}^{\,m'} &= D_m^{\,m'\,(0)} + \epsilon\,D_m^{\,m'\,(1)} + \epsilon^2\,D_{m}^{\,m'\,(2)},
\end{align}
where $\alpha_g^{\,(0)}$, $\alpha_g^{\,(2)}$, et cetera, are ${\cal O}(1)$, 
then it follows from Eq.~(\ref{Ddef}) that
\begin{align}\label{e130}
D_m^{\,m\,(0)} &= -\alpha_f^{\,(0)}\,\alpha_p^{\,(0)} - \hat{r}\,\alpha_p^{\prime\,(0)} - q\,\hat{r}\,\alpha_g^{\prime\,(0)} + m^2,\\[0.5ex]
D_{m}^{\,m'\,(1)}&=   - \left[\alpha_f^{\,(0)}\,\alpha_p^{\,(0)} + \hat{r}\,\alpha_p^{\prime\,(0)}\right]a_m^{\,m'\,(1)} + m\,m'\,b_m^{\,m'\,(1)},\\[0.5ex]
D_{m}^{\,m\,(2)}&=  -\left[\alpha_f^{\,(0)}\,\alpha_p^{\,(0)} + \hat{r}\,\alpha_p^{\prime\,(0)}\right]a_m^{\,m'\,(2)} 
-\alpha_f^{\,(0)}\,\alpha_p^{\,(2)}-\alpha_f^{\,(2)}\,\alpha_p^{\,(0)}-\hat{r}\,\alpha_p^{\prime\,(2)} - q\,\hat{r}\,\alpha_g^{\prime\,(2)}\nonumber\\[0.5ex]&\phantom{=}
-\hat{r}^{\,2}\left[\alpha_g^{\,(0)}\right]^2+m^2\,b_m^{\,m\,(2)}.\label{e132}
\end{align}

Finally, Eqs.~(\ref{e73er})--(\ref{e76er}), (\ref{e120})--(\ref{e123}), and (\ref{e130})--(\ref{e132}) give 
\begin{align}
A_m^{\,m}(\hat{r}) &= 1 + \epsilon^2\left(-\frac{3\,\hat{r}^{\,2}}{4} + H_1+S_1\right),\\[0.5ex]
A_m^{\,m\pm 1}(\hat{r}) &= - \epsilon\,H_1',\\[0.5ex]
A_m^{\,m\pm j}(\hat{r}) &= -\epsilon\,(H_j'\pm {\rm i}\, V_j')~~~~~\mbox{for $j>1$},\\[0.5ex]
B_m^{\,m}(\hat{r}) &= 0,\\[0.5ex]
B_m^{\,m\pm 1}(\hat{r}) &= \pm \epsilon\,(m\pm 1)\left[\hat{r}-p_2'\,q^2+(1-s)\,H_1'\right],\\[0.5ex]
B_m^{\,m\pm j}(\hat{r}) &= \pm\epsilon\,\frac{m\pm j}{j}\,\left[(1-s)\,(H_j'\pm {\rm i}\,V_j')- (j^2-1)\,\frac{H_j\pm {\rm i}\,V_j}{\hat{r}}\right]~~~~~\mbox{for $j>1$},\\[0.5ex]
C_m^{\,m}(\hat{r}) &= 0,\\[0.5ex]
C_m^{\,m\pm 1}(\hat{r}) &= \pm \epsilon\,m\left[\hat{r}-p_2'\,q^2+(1-s)\,H_1'\right],\\[0.5ex]
C_m^{\,m\pm j}(\hat{r}) &= \pm\epsilon\,\frac{m}{j}\,\left[(1-s)\,(H_j'\pm {\rm i}\,V_j')- (j^2-1)\,\frac{H_j\pm {\rm i}\,V_j}{\hat{r}}\right]~~~~~\mbox{for $j>1$},\\[0.5ex]
D_m^{\,m}(\hat{r}) &= m^2 + q\,\hat{r}\,\frac{d}{d\hat{r}}\!\left(\frac{2-s}{q}\right) +\epsilon^2\,m^2\,S_4\nonumber\\[0.5ex]
&\phantom{=}+\epsilon^2\left\{-\hat{r}^{\,2}\left(\frac{2-s}{q}\right)^2+q\,\hat{r}\,\frac{d{\mit\Sigma}}{d\hat{r}}- \hat{r}\,\frac{d}{d\hat{r}}(\hat{r}\,p_2') -2\,(1-s)\,\hat{r}\,p_2'
\right.\nonumber\\[0.5ex]
&\phantom{=}\left.+2\,\hat{r}\,p_2'\,q^2\left(-2 + \frac{3\,p_2'\,q^2}{\hat{r}}\right) + 2\,H_1'\,q^2\left[\frac{d}{d\hat{r}}(\hat{r}\,p_2')-4\,(1-s)\,p_2'\right]\right\},\\[0.5ex]
D_m^{\,m\pm 1}(\hat{r}) &=\epsilon\,q^2\left[\frac{d}{d\hat{r}}(\hat{r}\,p_2') - (2-s)\,p_2'\right] + \epsilon\,m\,(m\pm 1)\,(\hat{r}-H_1'),\\[0.5ex]
D_m^{\,m\pm j}(\hat{r}) &= -\epsilon\,m\,(m\pm j)\,(H_j'\pm {\rm i}\,V_j')~~~~~\mbox{for $j>1$}.
\end{align}

\subsection{Toroidal Plasma Current}
The net toroidal plasma current flowing in the plasma is given by
\begin{equation}\label{ipdef}
I_p = \oint_{r=a}{\cal J}\,\nabla\phi\times\nabla r\cdot{\bf B}\,d\theta = \oint_{r=a}\,B_\theta\,d\theta,
\end{equation}
which yields
\begin{equation}
I_p =\gamma_{\rm shape}\, I_{p\,{\rm cly}},
\end{equation}
where
\begin{equation}
I_{p\,{\rm cyl}} = \frac{2\pi\,a^2\,g(a)}{q(a)}
\end{equation}
is the plasma current according to cylindrical theory, whereas
\begin{equation}
\gamma_{\rm shape} = \oint_{r=a} |\hat{\nabla}\hat{r}|^2\,\frac{d\theta}{2\pi}
\end{equation}
is the factor by which the toroidal  current is increased due to the shaping of the plasma's poloidal
boundary. 

\subsection{Behavior Close to Magnetic Axis}\label{axis}
When $\hat{r}\ll 1$, the well-behaved solution of the axisymmetric ideal-MHD o.d.e.s, (\ref{e69u}) and (\ref{e70uu}), that is dominated by the poloidal harmonic whose poloidal mode number is
$m$ is such that
\begin{align}
y_m(\hat{r}) &= \hat{r}^{\,|m|},\\[0.5ex]
Z_m(\hat{r}) &= |m|\,\hat{r}^{\,|m|},
\end{align}
with $y_{m'}(\hat{r})=Z_{m'}(\hat{r})=0$ for $m'\neq 0$. 

\section{Vacuum Solution}\label{vacuum}
\subsection{Toroidal Coordinates}
Let $\mu$, $\eta$, $\phi$ be right-handed {\em orthogonal toroidal coordinates}\/ defined such that\,\cite{tj,tj1,mf}
\begin{align}
R &= \frac{\sinh\mu}{\cosh\mu-\cos\eta},\\[0.5ex]
Z&=\frac{\sin\eta}{\cosh\mu-\cos\eta}.
\end{align}
The scale-factors of the toroidal coordinate system are
\begin{align}
h_\mu&=h_\eta= \frac{1}{\cosh\mu-\cos\eta}\equiv h,\\[0.5ex]
h_\phi &= \frac{\sinh\mu}{\cosh\mu-\cos\eta} = h\,\sinh\mu.
\end{align}
Moreover, 
\begin{equation}
{\cal J}' \equiv (\nabla\mu\times\nabla\eta\cdot\nabla\phi)^{-1}= h^3\,\sinh\mu.
\end{equation}

\subsection{Perturbed Magnetic Field}
The curl-free perturbed magnetic field in the vacuum region is written ${\bf b} = {\rm i}\,\nabla V$,
where
$\nabla^2 V =0$.
The most general axisymmetric  solution to Laplace's equation is\,\cite{tj1,mf1}
\begin{align}
V(z,\eta)&= \sum_m (z-\cos\eta)^{1/2}\,U_m(z)\,{\rm e}^{-{\rm i}\,m\,\eta}, \\[0.5ex]
U_m(z) &= p_m\,\hat{P}_{|m|-1/2}(z)+q_m\,\hat{Q}_{|m|-1/2}(z),
\end{align}
where  $z=\cosh\mu$, the $p_m$ and $q_m$ are arbitrary complex coefficients, and 
\begin{align}\label{e21dd}
\hat{P}_{|m|-1/2}(z) &= \cos(|m|\,\pi)\,\frac{\sqrt{\pi}\,\Gamma(|m|+1/2)\,a^{\,|m|}}{2^{\,|m|-1/2}\,|m|!}\,P_{|m|-1/2}(z),\\[0.5ex]
\hat{Q}_{|m|-1/2}(z)&= \cos(|m|\,\pi)\,\frac{2^{\,|m|-1/2}\,|m|!}{\sqrt{\pi}\,\Gamma(|m|+1/2)\,a^{\,|m|}}\,Q_{|m|-1/2}(z).\label{e22dd}
\end{align}
Here,  the $P_{|m|-1/2}(z)$  and $Q_{|m|-1/2}(z)$ are toroidal functions,\cite{as1} and $\Gamma(z)$ is a
gamma function.\cite{as2}

\subsection{Toroidal Electromagnetic Angular Momentum Flux}
The outward flux of toroidal angular momentum across a constant-$z$ surface is\,\cite{tj,tj1}
\begin{align}
T_\phi(z) &= -\oint\oint {\cal J}' \,b_\phi\,b^{\,\mu}\,d\eta\,d\phi = 0,
\end{align}
because $b_\phi = {\rm i}\,\partial V/\partial\phi = 0$. Of course, the flux has to be zero because the flux of angular momentum across the plasma-vacuum
interface is zero, and there are no sources of angular momentum in the vacuum region surrounding the plasma. (See Sect.~\ref{storque}.)

\subsection{Electromagnetic Energy Flux}
The outward flux of electromagnetic energy flux across a constant-$z$ surface is
\begin{align}
{\cal E}(z) = -\oint\oint {\cal J}'\,{\bf e}\times {\bf b}\cdot\nabla \mu\,d\eta\,d\phi= -{\rm i}\,\pi\oint\left(e_\phi\,\frac{\partial V^\ast}{\partial\eta}
- e_\phi^{\,\ast}\,\frac{\partial V}{\partial\eta}\right)d\eta,
\end{align}
given that $e_\mu=e_\eta=0$ in the vacuum. 
However, $\nabla\times{\bf e}= {\rm i}\,\omega\,{\bf b}$ implies that
\begin{equation}
\frac{\partial e_\phi}{\partial\eta} = - \omega\,h\,\sinh\mu\,\frac{\partial V}{\partial \mu}= - \omega\,h\,\sinh^2\mu\,\frac{\partial V}{\partial z}. 
\end{equation}
Thus,
\begin{align}
{\cal E}(z) &= {\rm i}\,\pi\oint\left(\frac{\partial e_\phi}{\partial\eta}\,V^\ast - \frac{\partial e_\phi^{\,\ast}}{\partial\eta}\,V\right)d\eta= 
-{\rm i}\,\pi\,\omega\oint h\,\sinh^2\mu\left(\frac{\partial V}{\partial z}\,V^\ast - \frac{\partial V^\ast}{\partial z}\,V\right) d\eta\nonumber\\[0.5ex]
&= {\rm i}\,\pi^2\,\omega\sum_m(p_m\,q_m^{\,\ast}-q_m\,p_m^{\,\ast})\,(z^2-1)\,{\cal W}(P_{|m|-1/2},Q_{|m|-1/2}),
 \end{align}
 where ${\cal W}(f,g)= f\,dg/dz-g\,df/dz$. However,\cite{mf2}
 \begin{equation}
 {\cal W}(P_{|m|-1/2},Q_{|m|-1/2}) = \frac{1}{1-z^2},
 \end{equation}
 so
 \begin{equation}\label{e168g}
 {\cal E}(z)=- {\rm i}\,\pi^2\,\omega\sum_m(p_m\,q_m^{\,\ast}-q_m\,p_m^{\,\ast}).
 \end{equation}
 Note that ${\cal E}$ is independent of $z$, as must be the case because there are no energy sources in the vacuum region. 

\subsection{Solution in Vacuum Region}
In the large-aspect ratio limit, $r\ll 1$, it can be demonstrated that\,\cite{mf2}
\begin{align}
z&\simeq \frac{1}{r},\label{e25t}\\[0.5ex]
z^{\,1/2}\,\hat{P}_{-1/2}(z) &\simeq \frac{1}{2}\ln(8\,z),\\[0.5ex]
z^{1/2}\,\hat{P}_{|m|-1/2}(z) &\simeq \frac{\cos(|m|\,\pi)\,(a\,z)^{|m|}}{|m|},\label{ety}\\[0.5ex]
z^{1/2}\,\hat{Q}_{|m|-1/2}(z) &\simeq \frac{\cos(|m|\,\pi)\,(a\,z)^{-|m|}}{2}.\label{e29t}
\end{align}
Note that Eq.~(\ref{ety}) only applies to $|m|>0$.

According to Eq.~(\ref{e41}) and (\ref{e54y}),
\begin{align}\label{e30ay}
\frac{\partial y}{\partial \theta}&= {\cal J}\,{\bf b}\cdot\nabla r = {\rm i}\,{\cal J}\,\nabla V\cdot\nabla r,\\[0.5ex]
{\cal Z}&= -{\cal J}\,\nabla\phi\times\nabla r \cdot{\bf b} = - {\rm i}\,\frac{\partial V}{\partial\theta}.\label{e31ay}
\end{align}
Now, we have already seen that $Z_0\equiv \oint{\cal Z}\,d\theta/(2\,\pi)=0$ is independent of $r$ in the vacuum region.  (See Sect.~\ref{ode}.) 
However, it is clear that Eq.~(\ref{e31ay}) mandates that this constant value is zero. This constraint implies that the axisymmetric ideal modes to
which the plasma equilibrium is subject do not change the net toroidal current flowing in the plasma. [See Eqs.~(\ref{e54y}) and (\ref{ipdef}).]  Note, further, that $y_0\equiv \oint y\,d\theta/(2\,\pi)$ does
not influence the vacuum potential, $V(r,\theta)$. 

The previous two equations yield\,\cite{tj1}
\begin{align}\label{e32ay}
\underline{V}(r)&= \underline{\underline{{\cal P}}}(r)\,\underline{p}+ \underline{\underline{{\cal Q}}}(r)\,\underline{q},\\[0.5ex]
\underline{\psi}(r)&= \underline{\underline{{\cal R}}}(r)\,\underline{p}+ \underline{\underline{{\cal S}}}(r)\,\underline{q},\label{e33ay}
\end{align}
where $V(r,\theta)= \sum_m V_m(r)\,{\rm e}^{\,{\rm i}\,m\,\theta}$, $Z_m(r)=m\,V_m(r)$, $\psi_m(r)=m\,y_m(r)$, 
 $\underline{V}(r)$ is the vector of the $V_m(r)$ values, $\underline{\psi}(r)$ is the vector of the $\psi_m(r)$ values, $\underline{\underline{{\cal P}}}(r)$ is the
matrix of the
\begin{equation}
{\cal P}_{mm'}(r)=\oint_{r}(z-\cos\eta)^{1/2}\,\hat{P}_{|m'|-1/2}(z)\,\exp[-{\rm i}\,(m\,\theta+m'\,\eta)]\,\frac{d\theta}{2\pi}
\end{equation}
values, 
$\underline{\underline{{\cal Q}}}(r)$ is the
matrix of the
\begin{equation}
{\cal Q}_{mm'}(r)=\oint_{r}(z-\cos\eta)^{1/2}\,\hat{Q}_{|m'|-1/2}(z)\,\exp[-{\rm i}\,(m\,\theta+m'\,\eta)]\,\frac{d\theta}{2\pi}
\end{equation}
values, $\underline{\underline{{\cal R}}}(r)$ is the matrix of the 
\begin{align}\label{e354}
{\cal R}_{mm'}(r) &=\oint_{r}
\left\{\left[\frac{1}{2}\,(z-\cos\eta)^{-1/2}\,\hat{P}_{|m'|-1/2}(z)+(z-\cos\eta)^{1/2}\,\frac{d\hat{P}_{|m'|-1/2}}{dz}\right]{\cal J}\,\nabla r\cdot \nabla z
\right.\nonumber\\[0.5ex]&
\left.\phantom{=}+\left[\frac{1}{2}\,(z-\cos\eta)^{-1/2}\,\sin\eta-{\rm i}\,m'\,(z-\cos\eta)^{1/2}\right]\hat{P}_{|m'|-1/2}(z)\,{\cal J}\,\nabla r\cdot \nabla \eta
\right\}\nonumber\\[0.5ex] &
\phantom{=}\times\exp[-{\rm i}\,(m\,\theta+m'\,\eta)]\,\frac{d\theta}{2\pi}
\end{align}
values, 
$\underline{\underline{{\cal S}}}(r)$ is the matrix of the 
\begin{align}\label{e355}
{\cal S}_{mm'}(r) &=\oint_{r}
\left\{\left[\frac{1}{2}\,(z-\cos\eta)^{-1/2}\,\hat{Q}_{|m'|-1/2}(z)+(z-\cos\eta)^{1/2}\,\frac{d\hat{Q}_{|m'|-1/2}}{dz}\right]{\cal J}\,\nabla r\cdot \nabla z
\right.\nonumber\\[0.5ex]&
\left.\phantom{=}+\left[\frac{1}{2}\,(z-\cos\eta)^{-1/2}\,\sin\eta-{\rm i}\,m'\,(z-\cos\eta)^{1/2}\right]\hat{Q}_{|m'|-1/2}(z)\,{\cal J}\,\nabla r\cdot \nabla \eta
\right\}\nonumber\\[0.5ex] &
\phantom{=}\times\exp[-{\rm i}\,(m\,\theta+m'\,\eta)]\,\frac{d\theta}{2\pi}
\end{align}
values, $\underline{p}$ is the vector of the $p_m$ coefficients, and  $\underline{q}$ is the vector of the $q_m$ coefficients. Here, the
subscript $r$ on the integrals indicates that they are taken at constant $r$. 

\subsection{Energy Conservation}
According to Eq.~(\ref{eflux}), the net flux of electromagnetic energy across the plasma-vacuum interface  is
\begin{equation}
{\cal E} = {\rm i}\,\pi^2\,\omega\, (\underline{V}^\dag\,\underline{\psi}- \underline{\psi}^\dag\,\underline{V}).
\end{equation}
However, energy conservation requires this flux to be equal  to the energy flux through the vacuum region, so Eq.~(\ref{e168g}) gives
\begin{equation}\label{e39x}
{\cal E} =-  {\rm i}\,\pi^2\,\omega\,(\underline{q}^\dag\,\underline{p}-\underline{p}^\dag\,\underline{q}).
\end{equation}
Equations~(\ref{e32ay}), (\ref{e33ay}), and the previous two equations, yield\,\cite{tj1}
\begin{align}\label{e41zx}
\underline{\underline{\cal P}}^\dag\,\underline{\underline{\cal R}}&= \underline{\underline{\cal R}}^\dag\,\underline{\underline{\cal P}},\\[0.5ex]
\underline{\underline{\cal Q}}^\dag\,\underline{\underline{\cal S}}&= \underline{\underline{\cal S}}^\dag\,\underline{\underline{\cal Q}},\label{e42zx}\\[0.5ex]
\underline{\underline{\cal P}}^\dag\,\underline{\underline{\cal S}}- \underline{\underline{\cal R}}^\dag\,\underline{\underline{\cal Q}}&=\underline{\underline{1}}.\label{e43zx}
\end{align}
 It can also be demonstrated that 
\begin{align}\label{e46}
\underline{\underline{\cal Q}}\,\underline{\underline{\cal P}}^\dag&= \underline{\underline{\cal P}}\,\underline{\underline{\cal Q}}^\dag,\\[0.5ex]
\underline{\underline{\cal R}}\,\underline{\underline{\cal S}}^\dag&= \underline{\underline{\cal S}}\,\underline{\underline{\cal R}}^\dag.\label{e47x}
\end{align}
The previous five equations hold throughout the vacuum region. 

\subsection{No-Wall Matching Condition}
Suppose that the plasma is surrounded by a vacuum region that extends to infinity. In this case, 
\begin{equation}
\underline{q}=\underline{0},
\end{equation}
 because the 
$\underline{\underline{{\cal Q}}}(r)$ solutions blow up in an unphysical manner as $r\rightarrow\infty$. [See Eqs.~(\ref{e25t}) and (\ref{e29t})].
It immediately follows from Eq.~(\ref{e39x}) that
 \begin{equation}\label{ezero}
 {\cal E}=0.
 \end{equation}
  In other words, there is zero net flux of electromagnetic energy out of  a plasma
 surrounded by a vacuum region that extends to infinity. 
Equations~(\ref{e32ay}) and (\ref{e33ay}) imply that
\begin{equation}\label{hdef}
 \underline{V}(r=a_+)= \underline{\underline{H}}\,\underline{\psi}(r=a),
 \end{equation}
 where
 \begin{equation}
 \underline{\underline{H}}= \underline{\underline{\cal P}}_{\,a}\,\underline{\underline{\cal R}}_{\,a}^{-1}
 \end{equation}
 is termed the {\em no-wall vacuum matrix}. 
   Here, $\underline{\underline{\cal P}}_{\,a}= \underline{\underline{\cal P}}(r=a)$, et cetera. 
Equation~(\ref{e41zx}) ensures that  $\underline{\underline{H}}$ is Hermitian. 

\subsection{Perfect-Wall Matching Condition}
Suppose that the plasma is surrounded by a vacuum region that is bounded by a perfectly conducting wall whose inner surface lies at $r=b_w\,a$, where $b_w\geq 1$. 
Because the wall is perfectly conducting, 
$\underline{\psi}(r=b_w\,a)=0$.\cite{gs1}
In other words, the normal component of the perturbed magnetic field is zero at the inner surface of the wall. 
It follows from Eq.~(\ref{e33ay}) that
\begin{equation}
\underline{p} =- \underline{\underline{I}}_b\,\underline{q},
\end{equation}
where
\begin{equation}
 \underline{\underline{I}}_b= \underline{\underline{\cal R}}_{\,b}^{-1}\,\underline{\underline{\cal S}}_{\,b}
 \end{equation}
 is termed the {\em wall matrix}.
 Here, $\underline{\underline{\cal R}}_{\,b}= \underline{\underline{\cal R}}(r=b_w\,a)$, et cetera. Equation~(\ref{e47x}) ensures that $ \underline{\underline{I}}_b$
 is Hermitian. It immediately follows from Eq.~(\ref{e39x}) that
 \begin{equation}\label{ezero1}
 {\cal E}=0.
 \end{equation}
  In other words, there is zero net electromagnetic energy flux out of  a plasma
 surrounded by a vacuum region that is bounded by perfectly conducting wall. 
 
  Making use of Eqs.~(\ref{e32ay}) and (\ref{e33ay}),  the matching condition at the plasma-vacuum interface  for a perfectly-conducting wall becomes 
 \begin{equation}\label{gdef}
 \underline{V}(r=a_+)= \underline{\underline{G}}\,\underline{\psi}(r=a),
 \end{equation}
 where 
 \begin{equation}
 \underline{\underline{G}}= (\underline{\underline{\cal Q}}_{\,a}-\underline{\underline{\cal P}}_{\,a}\,\underline{\underline{I}}_{\,b})\,(\underline{\underline{\cal S}}_{\,a}-\underline{\underline{\cal R}}_{\,a}\,\underline{\underline{I}}_{\,b})^{-1}
 \end{equation}
 is termed the {\em perfect-wall vacuum matrix}.
It is easily demonstrated from  Eqs.~(\ref{e41zx})--(\ref{e43zx})  that
 \begin{equation}\label{e53xx}
 \underline{\underline{G}}-\underline{\underline{G}}^\dag =- [(\underline{\underline{\cal S}}_{\,a}-\underline{\underline{\cal R}}_{\,a}\,\underline{\underline{I}}_{\,b})^{-1}]^{\dag}\,(\underline{\underline{I}}_{\,b} -\underline{\underline{I}}^\dag_{\,b})\,  (\underline{\underline{\cal S}}_{\,a}+\underline{\underline{\cal R}}_{\,a}\,\underline{\underline{I}}_{\,b})^{-1}.
\end{equation}
Thus,  the vacuum matrix, $\underline{\underline{G}}$, is Hermitian because  the wall matrix, $\underline{\underline{I}}_{\,b}$, is Hermitian. 
 
\subsection{Perturbed Vacuum Potential Energy}
Equation~(\ref{e41}) implies that 
\begin{equation}
{\cal J} \,\nabla r\cdot\nabla V = - {\rm i}\,\frac{\partial y}{\partial \theta}.
\end{equation}
The perturbed potential energy in the vacuum region is\,\cite{tj1,gs1}
\begin{align}
\delta W_v &= \frac{1}{2}\int_{a_+}^\infty \oint\oint {\bf b}^\ast\cdot{\bf b}\,{\cal J}\,dr\,d\theta\,d\phi
=\frac{1}{2}\int_{a_+}^\infty \oint\oint \nabla V^\ast\cdot\nabla V\,{\cal J}\,dr\,d\theta\,d\phi\nonumber\\[0.5ex]
&= -\frac{1}{2}\left(\oint\oint {\cal J}\,\nabla r\cdot\nabla V^\ast\,V\,d\theta\,d\phi\right)_{a_+}
=-\frac{1}{2}\left[\oint\oint \left(-{\rm i}\,\frac{\partial y}{\partial \theta}\right)^\ast\,V\,d\theta\,d\phi\right]_{a_+}\nonumber\\[0.5ex]
&= - \pi^2\left(\sum_m\,m\,y_m^{\,\ast}\,V_m\right)_{a_+},\label{vac}
\end{align}
where use has been made of the facts that $\nabla^2 V =0$ and $\nabla r\cdot \nabla V=0$ at the ideal wall. 

\section{Ideal Stability}
\subsection{Perturbed Plasma Potential Energy}
Suppose that the poloidal harmonics included in the calculation range from $m=-m_{\rm max}$ to $m=m_{\rm max}$, where $m_{\rm max} >0$. 
Let  the $y_{mm'}(r)$ and the $Z_{mm'}(r)$ be linearly independent solutions of the axisymmetric ideal-MHD o.d.e.s, (\ref{e69u}) and
(\ref{e70uu}), that are well-behaved at the magnetic axis. Here, $m$ indexes the poloidal harmonic, whereas $m'$ indexes the dominant poloidal harmonic close to the magnetic
axis. (See Sect.~\ref{axis}.)
Let 
\begin{equation}
\chi_{mm'}(r)=Z_{mm'} + q\,\alpha_g\,y_{mm'} + \alpha_p\sum_{m''} a_m^{\,m''}\,y_{m''m'}.
\end{equation}
We can form $J=2\,m_{\rm max}$ linearly independent solutions that all have $Z_0=0$ at $r=a$, as mandated by the matching condition (\ref{e31ay}). 
The most general solution to the axisymmetric ideal-MHD o.d.e.s that satisfied the constraint $Z_0(a)=0$ is written
\begin{align}
y_m(r) &= \sum_{m'=1,J} y_{mm'}(r)\,\alpha_{m'},\\[0.5ex]
\chi_m(r)&= \sum_{m'=1,J} \chi_{mm'}(r)\,\alpha_{m'},
\end{align}
where the $\alpha_m$ are arbitrary complex coefficients. Here, $m$ indexes the poloidal harmonic and $m'$ indexes the solution. 
 Note that
$ \chi_m(a_-)=Z_m(a_-)=0$ because $\alpha_g(a)=\alpha_p(a)=0$. [See Eq.~(\ref{chidef}).]

According to Eq.~(\ref{e87j}), the net perturbed plasma potential energy is
\begin{equation}
\delta W_p = \pi^2\,\underline{y}^{\dag}\,\underline{\chi},
\end{equation}
where $\underline{y}$ is the vector of the $y_m(a)$ values, excluding the $m=0$ harmonic,  and $\underline{\chi}$  the vector of the $\chi_m(a_-)$ values,
excluding the $m=0$ harmonic.  We can exclude the $m=0$ harmonic because, by construction our solutions are such that $\chi_0(a_-)=0$. It follows that
\begin{equation}
\delta W_p = \pi^2\,\underline{\alpha}^\dag \,\underline{\underline{y}}^\dag \,\underline{\underline{\chi}}\,\underline{\alpha},
\end{equation}
where $\underline{\alpha}$ is the vector of the $\alpha_m$ values, $\underline{\underline{y}}$ the matrix of the $y_{mm'}(a)$ values, excluding the $m=0$ harmonic, 
and $\underline{\underline{\chi}}$ the matrix of the $\chi_{mm'}(a_-)$ values, excluding the $m=0$ harmonic. 
If
\begin{equation}
\underline{\underline{\chi}} = \underline{\underline{W}}_{\,p}\,\underline{\underline{y}}.
\end{equation}
then 
\begin{equation}
\delta W_p = \pi^2\,\underline{\alpha}^\dag \,\underline{\underline{y}}^\dag \,\underline{\underline{W}}_{\,p}\,\underline{\underline{y}}\,\underline{\alpha}.
\end{equation}
Equations~(\ref{eflux}), (\ref{ezero}), and (\ref{ezero1}) imply that $\underline{\underline{W}}_{\,p}$ is Hermitian. 

\subsection{Perturbed Vacuum Potential Energy}
According to Eq.~(\ref{vac}), the perturbed vacuum potential energy is
\begin{equation}
\delta W_v = - \pi^2\sum_m m\,y_m^{\,\ast}(a)\,V_m(a_+),
\end{equation}
excluding the $m=0$ harmonic, which 
obviously does not affect the vacuum energy. 
However, Eqs.~(\ref{hdef}) and (\ref{gdef}) imply that
\begin{equation}
V_m(a_+) = \sum_{m'}\,H_{mm'}\,m'\,y_{m'}(a)
\end{equation}
in the no-wall case, and 
\begin{equation}
V_m(a_+) = \sum_{m'}\,G_{mm'}\,m'\,y_{m'}(a)
\end{equation}
in the perfect-wall case. Here, we have excluded the $m'=0$ harmonic, which also obviously does not affect the vacuum energy. 
Hence, we can write
\begin{equation}
\delta W_v = \pi^2\,\underline{y}^\dag\,\underline{\underline{W}}_{\,v}\,\underline{y},
\end{equation}
where $\underline{\underline{W}}_{\,v}$ is the matrix of the $-m\,H_{mm'}\,m'$ values in the no-wall case, excluding the $m=0$ and $m'=0$ harmonics, and the
$-m\,G_{mm'}\,m'$ values in the perfect-wall case, likewise  excluding the $m=0$ and $m'=0$ harmonics,. Given that $H_{mm'}$ and $G_{mm'}$ are Hermitian, we deduce that $\underline{\underline{W}}_{\,v}$ 
is Hermitian. It follows that
\begin{equation}
\delta W_v = \pi^2\,\underline{\alpha}^\dag\,\underline{\underline{y}}^\dag\,\underline{\underline{W}}_{\,v}\,\underline{\underline{y}}\,\underline{\alpha}.
\end{equation}

\subsection{Total Perturbed Potential Energy}
The total perturbed potential energy is
\begin{equation}
\delta W = \delta W_p + \delta W_v = \pi^2\,\underline{\alpha}^\dag\,\underline{\underline{y}}^\dag\,\underline{\underline{W}}\,\underline{\underline{y}}\,\underline{\alpha},
\end{equation}
where 
\begin{equation}
\underline{\underline{W}} = \underline{\underline{W}}_{\,p} + \underline{\underline{W}}_{\,v}.
\end{equation}
Given that $\underline{\underline{W}}_{\,p}$ and $\underline{\underline{W}}_{\,v}$ are both Hermitian, we deduce that $\underline{\underline{W}}$ is Hermitian.

\subsection{Ideal Stability}
The fact that $\underline{\underline{W}}$ is Hermitian allows us to write
\begin{align}
\underline{\underline{W}}\,\underline{\underline{\beta}} &= \underline{\underline{\beta}}\,\underline{\underline{\mit\Lambda}},\\[0.5ex]
\underline{\underline{\beta}}^\dag\,\underline{\underline{\beta}} &= \underline{\underline{1}},
\end{align}
where $\underline{\underline{\beta}}$ is real, and $\underline{\underline{\mit\Lambda}}$ is the diagonal matrix of the real $\lambda_m$ values. 
If $\underline{\hat{\alpha}}= \underline{\underline{\beta}}^\dag\,\underline{\underline{y}}\,\underline{\alpha}$ then
\begin{equation}
\delta W =\pi^2\,\underline{\hat{\alpha}}^\dag\,\underline{\underline{\mit\Lambda}}\,\underline{\hat{\alpha}}=\pi^2\sum_{m=1,J}|\hat{\alpha}_m|^2\,\lambda_m.
\end{equation}
Thus, if any of the $\lambda_m$ are negative then solutions exist for which $\delta W$ is negative, and the plasma is consequently unstable to
an axisymmetric ideal mode.\cite{gs1}

Suppose that $\sum_{m}|\hat{\alpha}_m|^2=1$. The ideal energy of the $m$th mode,
for which $\hat{\alpha}_{m'} = \delta_{mm'}$,  is
\begin{equation}
\delta W_m = \pi^2\,\lambda_m.
\end{equation}
However,
\begin{equation}
\underline{\underline{\mit\Lambda}} = \underline{\underline{\beta}}^\dag\,\underline{\underline{W}}\,\underline{\underline{\beta}},
\end{equation}
Thus, the diagonal components of $\underline{\underline{\beta}}^\dag\,\underline{\underline{W}}_{\,p}\,\underline{\underline{\beta}}$ 
and $\underline{\underline{\beta}}^\dag\,\underline{\underline{W}}_{\,v}\,\underline{\underline{\beta}}$  are the plasma and vacuum
contributions to the $\lambda_m$, respectively. 
 The eigenfunction of the $m$th mode is conveniently normalized such that $y_{m'}(a)= \beta_{m'm}$. 
 
\section{Resistive Wall Stability}
The unnormalized minor radius of the wall is $b=b_w\,\bar{a}$, where $\bar{a}=\epsilon\,R_0$  is the unnormalized minor radius of the plasma. Suppose that the wall is resistive, and possesses an (unnormalized)  electrical conductivity $\sigma_w$, as well as an (unnormalized)  uniform radial thickness $d$. 
Consider a particular axisymmetric ideal mode for which the total perturbed potential energy in the absence of a wall, 
 $\delta W_{nw}$, is negative,  but the total perturbed potential energy in the presence of a perfectly conducting wall, $\delta W_{pw}$, is positive.  In this case,  the no-wall ideal mode is unstable (because $\delta W_{nw}<0$). In the absence of  the wall, this mode would grow very rapidly on an Alfv\'{e}nic timescale. On the other
 hand, the mode is completely stabilized if the wall is perfectly conducting (because $\delta W_{pw}>0$).  However, because the wall is not perfectly conducting, the unstable ideal mode is
 instead converted into a much more slowly growing resistive wall mode. 
 Let the (unnormalized) 
 growth-rate of the resistive wall mode be
 \begin{equation}
 \gamma= \frac{\hat{\gamma}}{\tau_w},
 \end{equation}
 where
 \begin{equation}
 \tau_w = \mu_0\,\sigma_w\,b\,d
 \end{equation}
 is the (unnormalized) L/R time of the wall. 
 The normalized growth-rate of the resistive wall mode is specified by\,\cite{rwm4,rwm5}
 \begin{equation}
 \sqrt{\frac{\hat{\gamma}}{\delta_w}}\tanh\left(\sqrt{\delta_w\,\hat{\gamma}}\right) = -\frac{\delta W_{nw}}{\alpha_w\,\delta W_{pw}},
 \end{equation}
 where $\delta_w=d/b$, 
and
 \begin{equation}
 \alpha_w = \frac{(1/2)\int|{\bf A}_{nw}\times {\bf n}_w|^2\,dS_w}{\epsilon\,b_w\,(\delta W_{v\,pw}- \delta W_{v\,nw})}.
 \end{equation}
 Here, $\delta W_{v\,nw}$ is the vacuum potential energy in the absence of a wall, whereas $\delta W_{n\,pw}$ is the vacuum energy in the presence of a
 perfectly conducting wall. 
 Moreover, $dS_w$ is an element of the inner  surface of the wall, and the integral is over the whole inner surface. Furthermore,
 the perturbed magnetic field in the vacuum region, in the no-wall case, is written $\nabla\times {\bf A}_{nw}$. 
 It is easily demonstrated from Eqs.~(\ref{e41}) and (\ref{e43y}) that ${\bf A}_{nw}= \sum_{m} y_{m\,nw}(r)\,{\rm e}^{\,{\rm i}\,m\,\theta}\,\nabla\phi$,
 where the $y_{m\,nw}(r)$ are the $y$-components of the no-wall eigenfunction of the mode in question. Finally, $dS_w= {\cal J}\,d\theta\,d\phi$. 
 Hence, we deduce that
 \begin{equation}
 \alpha_w = \frac{\pi^2\left(\sum_{m\neq 0} |y_{m\,nw}|^2\right)_{\hat{r}=b_w}}{\delta W_{v\,pw}- \delta W_{v\,nw}}.
 \end{equation}
 Here, we have neglected $y_{0\,nw}$ because $y_0$ does not affect the vacuum energy. 
 
 Given that $\underline{q}=\underline{0}$ for a no-wall solution, Eq.~(\ref{e33ay}) yields  
 \begin{equation}
\underline{\underline{m}}\, \underline{y}_b = \underline{\underline{\cal R}}_{\,b}\,\underline{\underline{\cal R}}_{\,a}^{-1}\,\underline{\underline{m}}\,\underline{y}_a,
\end{equation}
which enables us to determine the $y_{m\neq 0\,nw}(\hat{r}=b_w)$ from the $y_{m\neq 0\,nw}(\hat{r}=1)$. Here, $\underline{\underline{m}}$ is the diagonal matrix of the poloidal
mode numbers. 

\section*{Acknowledgements}
This research was funded by the  U.S.\ Department of Energy, Office of Science, Office of Fusion Energy Sciences under contract DE-SC0021156.
The author gratefully acknowledges informative discussions with A.H.~Boozer, R.~Granetz, B.~Breizman, and P.C.~de\,Vries. 

\section*{Data Availability Statement}
The digital data used in the figures in this paper can be obtained from the author upon reasonable request. 

\begin{thebibliography}{99}\baselineskip 5ex

\bibitem{troyon}   F.~Troyon, R.~Gruber, H.~Saurenmann, S.~Semenzato and S.~Succi, Plasma Phys.\ Control.\ Fusion {\bf 26}, 209 (1984). 

\bibitem{goldston} R.~Goldston, Plasma Phys.\ Control.\ Fusion {\bf 26}, 87 (1984).

\bibitem{jet} M.~Huguet, K.~Dietz,  J.L.~Hemmerich and J.R.~Last, Fusion Technology  {\bf 11}, 43 (1987).

\bibitem{v1} M.~Okabayashi and G.~Sheffield, Nucl.\ Fusion {\bf 14}, 575 (1974). 

\bibitem{v2} K.~Lackner and A.B.~MacMahon, Nucl.\ Fusion {\bf 14}, 575 (1974). 

\bibitem{rwm} T.~Pfirsch and H.~Tasso, Nucl.\ Fusion {\bf 11}, 259 (1971).

\bibitem{rwm1} S.C.~Jardin, Phys.\ Fluids {\bf 21}, 1851 (1978). 

\bibitem{rwm1a} J.A.~Wesson, Nucl.\ Fusion {\bf 18}, 87 (1978).

\bibitem{rwm2} D.~Dobrott and C.S.~Chang, Nucl.\ Fusion {\bf 21}, 1573 (1981).

\bibitem{rwm3} D.J.~Ward, S.C.~Jardin and C.Z.~Cheng, J.\ Comp.\ Phys.\ {\bf 104}, 221 (1993).

\bibitem{tj} R.~Fitzpatrick, Phys.\ Plasmas {\bf 31}, 102507 (2024).

\bibitem{tj1} R.~Fitzpatrick, Phys.\ Plasmas {\bf 32}, 062509 (2025).

\bibitem{bussac} M.N.~Bussac, R.~Pellat, D.~Edery and J.L.~Soule, Phys.\ Rev.\ Lett.\ {\bf 35}, 1638 (1975).

\bibitem{connor} J.W.~Connor,  S.C.~Cowley, R.J.~Hastie,  T.C.~Hender,  A.~Hood  and T.J.~Martin,  Phys.\ Fluids {\bf 31}, 577 (1988).

\bibitem{flow} R.~Iacono, A.~Bondeson, F.~Troyon and R.~Gruber, Phys.\ Fluids B {\bf 2}, 1794 (1990).

\bibitem{flow1} L.~Guazzotto,  R.~Betti, J.~Manickam and  S.~Kaye, Phys.\ Plasmas {\bf 11}, 604 (2004).

\bibitem{am1} R.~Fitzpatrick, R.J.~Hastie, T.J.~Martin and C.M.~Roach, Nucl.\ Fusion {\bf 33}, 1533 (1993).

\bibitem{gs1} J.P.~Freidberg, {\em Ideal Magnetohydrodynamics}, (Plenum, New York NY, 1987).

\bibitem{mf} P.M.~Morse and H. Feshbach, {\em Methods of Theoretical Physics}, (McGraw-Hill, New York, NY, 1953), p.~1301.

\bibitem{mf1} P.M.~Morse and H. Feshbach, {\em Methods of Theoretical Physics}, (McGraw-Hill, New York, NY, 1953), p.~1302.

\bibitem{as1} M.~Abramowitz and I.A.~Stegun, {\em Handbook of Mathematical Functions}, (Dover, New York NY, 1964), sect.~8.11.

\bibitem{as2} M.~Abramowitz and I.A.~Stegun, {\em Handbook of Mathematical Functions}, (Dover, New York NY, 1964), chap.~6.

\bibitem{mf2} P.M.~Morse and H. Feshbach, {\em Methods of Theoretical Physics}, (McGraw-Hill, New York, NY, 1953), pp.~1302--1309.

\bibitem{rwm4} S.W.~Haney and J.P.~Freidberg, Phys.\ Fluids B {\bf 1}, 1637 (1989).

\bibitem{rwm5} R.~Fitzpatrick, Phys. Plasmas {\bf 31}, 112502 (2024).

\end{thebibliography}
\end{document}