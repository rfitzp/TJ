\documentclass{article}[12pt]
\usepackage{fullpage}
\usepackage{amsmath}
\newcommand {\bxi}{\mbox{\boldmath$\xi$}}

\begin{document}
\begin{center}
{\em Calculation of Vertical Stability in an Inverse Aspect-Ratio
Expanded Tokamak Plasma Equilibrium }\\[1ex]
by R.~Fitzpatrick\\[1ex]
{\bf Reply to Referees' Comments}\\[1ex]
~
\end{center}
Let me thank the  referees for their helpful and insightful comments on my paper. In response to these comments, I
have made the following changes to the paper. 

\begin{enumerate}
\item As suggested, I have relegated some of the more tedious sections of analysis to Appendices. I hope that this
will make the paper more easily readable.

\item I have added some language to p.~3 of the Introduction to point out that my simplified model allows the plasma internal
inductance to be varied, which the simple model of Freidberg et al.\ does not. This feature is significant because, according to Ref.~15, vertical stability
has a strong dependence on $l_i$.

\item I have explained what I mean by the L/R time of the wall (in Appendix A.1), and have referred to this explanation whenever
the L/R time is introduced. 

\item I have modified Sect.~II to give a fuller  explanation of the $r$, $\theta$, $\phi$ coordinate system. This explanation now mentions that
the coordinate system is ``straight field-line'' system. I have also mentioned that $r$ is the mean minor radius of magnetic
flux-surfaces (or the mean inverse aspect-ratio in its normalized form), and that $a$ is the mean minor radius of the plasma boundary. 
Furthermore, I have made it clear that the type of equilibria considered in the paper are ``diffuse'' in the sense that both the equilibrium magnetic
field and current are continuous across the plasma-vacuum interface. 

\item I have introduced the $\exp(\gamma\,t)$ time dependence of perturbed quantities as soon as possible (in Sect.~III.A). 
For the sake of consistency, I have now replaced $\omega$ by ${\rm i}\,\gamma$ throughout the paper (because we are dealing with
purely growing or decaying modes, rather than rotating modes). 

\item I have added a section (Sect.~III.D) in which I discuss plasma incompressibility. As explained in this section, we know, from general
principles, that the most unstable mode has $\nabla\cdot\bxi=0$. However, the question is whether or not we can find a physical
$\bxi$ field that allows $\nabla\cdot\bxi=0$. For up-down symmetric plasma equilibria, it is possible to divide axisymmetric instabilities into two
separate groups. ``Vertical instabilities'' give rise to predominantly vertical plasma displacements, whereas ``horizontal instabilities''
give rise to predominately horizontal motion. My criterion for setting $\nabla\cdot\bxi=0$ is automatically satisfied by vertical
instabilities, but not by horizontal instabilities. It is important to realize that vertical instabilities are not merely rigid shifts of the plasma.
In principle, they can have complicated harmonic content (i.e., they can include poloidal harmonics other than $m=1$.)

\item I have added a section (Sect.~IV.A) in which I explain why it is necessary to use the toroidal coordinate system, $\mu$, $\eta$, $\phi$,  in the vacuum.
I have made it clear that the fact that $\mu$ is not a flux-surface label is not a  problem. 
I have also added a comment at the end of Sect.~IV.G that explains that the effectiveness of the matching between the
$r$, $\theta$, $\phi$ and the $\mu$, $\eta$, $\phi$ coordinates systems can be tested by checking whether the symmetry
requirements (64)--(68) are actually satisfied in the region immediately outside the plasma boundary. (It turns out that they are.)
Section IV.A also explains that, in order to calculate the stability of the resistive wall mode, you need to do two separate {\em ideal}\/
calculations, one in which the wall is treated as a perfect conductor, and one in which the wall is completely absent. The
contribution of the resistive wall to the perturbed energy is taken into account in the Haney-Freidberg formula, (101). (See Refs.~24 and 25.)

\item I have added a section (Sect.~III.I) in which I discuss the contribution of the plasma surface to the perturbed ideal
potential energy. In this section, I explain why this contribution is zero for the type of diffuse equilibria under consideration. 

\item I have rephrased the comment after Eq.~(47) to make it clear that Eq.~(46) applies to $m\neq 0$.

\item In Sect.~VI, I have tried to explain the physical significance of the ``wall parameter'' $\alpha_w$. 

\item In Sect.~VII.B, I have explained that $\omega$ is a geometric poloidal angle that is different from the straight poloidal angle, $\theta$.
In fact, $\omega$ is only introduced as a means of parameterizing the shapes of magnetic flux-surfaces in an intuitive manner.

\item In Sect.~VIII, I have added more language to make it clear exactly what needs to be continuous across the plasma-vacuum interface.
I have also renamed the pressure peaking parameter, $\mu$, since this was causing confusion with the toroidal coordinate, $\mu$. 

\item In Fig.~6, I have added an insert to make it clear that pressure does influence $\delta W_{pw}$. The shifts in the 
no-wall and perfect-wall $\delta W$ values with pressure are about the same size. However, the relative shift in $\delta W_{pw}$ is much
smaller than that in $\delta W_{nw}$ simply because the former is a lot larger than the latter. 

\item In Sect.VIII.F, I have improved the discussion of the $l_i$ scans to more accurately describe the data shown in Fig.~7.

\item I have made it clear at a number of points in the paper that the reason that the TJ toroidal tearing mode code can perform the
calculation is because the the basis solutions used to construct tearing mode eigenfunctions can be rearranged  to
construct ideal eigenfunctions. 

\item I have modified Sect.~A.5.a to make it clear that Eq.~(A75) really does reproduce $\hat{r}$. 

\item I have rephrased the discussion of the Shafranov shift on p.~48 to make it clearer.

\item I have modified the discussion of the effect of triangularity on vertical stability in the Abstract and Summary to more closely accord with the
results shown in the figures.

\item I have corrected the various typographic errors that the referees pointed out. 

\end{enumerate}

\end{document}