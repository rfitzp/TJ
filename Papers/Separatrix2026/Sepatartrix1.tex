%\documentclass[12pt,prb,aps,notitlepage]{revtex4-1}
\documentclass{article}
\usepackage{fullpage}
\usepackage {amsmath}
\pdfoutput = 1 
\usepackage {graphicx}
\usepackage[colorlinks=true, allcolors=blue]{hyperref}
\usepackage{orcidlink}
\usepackage{titlesec}

\titleformat{\subsection}
  {\itshape\normalsize}{\thesubsection}{1em}{}

\begin{document}

\title{\bf Advanced Divertors}

\author{R.~Fitzpatrick\orcidlink{0000-0001-6237-9309}\thanks{rfitzp@utexas.edu}\\[2ex]
{\em Institute for Fusion Studies,  Department of Physics,}\\[0.5ex]{\em   University of Texas at Austin,  Austin TX 78712, USA}}
\date{}
\maketitle

%\begin{abstract}
%\end{abstract}

\section{Plasma Equilibrium}
Let $x$, $y$, $z$ be right-handed Cartesian coordinates. The system is assumed to be periodic in the $z$-direction with period $2\pi\,R_0$, where
$R_0$ is the simulated major radius of the plasma. 
Let $\phi = z/R_0$ be a simulated toroidal angle. The equilibrium magnetic field is written
\begin{equation}
{\bf B} = \nabla\phi\times \nabla\psi_p + B_0\,R_0\,\nabla\phi,
\end{equation}
where $B_0$ is the toroidal magnetic field-strength, and $\psi_p$ the poloidal magnetic flux (divided by $2\pi$). Let $a$ be the mean minor radius of the plasma, and let $I_p$ be the toroidal plasma current. 
Let $X=x/a$, $Y=y/a$, and 
\begin{equation}
\psi_p(x,y) = \frac{\mu_0\,I_p\,R_0}{2\pi}\,\psi(X,Y). 
\end{equation}
The safety-factor on a given magnetic flux-surface is
\begin{equation}
q = \frac{q_\ast}{2\pi}\oint \frac{dL}{|\hat{\nabla}\psi|},
\end{equation}
where 
\begin{equation}
q_\ast = \frac{2\pi\,B_0\,a^2}{\mu_0\,I_p\,R_0},
\end{equation}
and $\hat{\nabla}=a\,\nabla$. Here, $dL$ is an element of normalized length, in the $X$-$Y$ plane, along the flux-surface. 

We require $\hat{\nabla}^2\psi=0$ everywhere that a current does not flow, where $\hat{\nabla}^2\equiv\partial^2/\partial X^2+\partial^2/\partial Y^2$. 
If we let  $z=X+{\rm i}\,Y$ then we can automatically ensure this by writing
\begin{equation}
F(z)= \psi(X,Y)+{\rm i}\,\chi(X,Y),
\end{equation}
where $\psi(X,Y)$ and $\chi(X,Y)$ are real functions. The Cauchy-Reiman relations yield
\begin{align}
\frac{\partial\psi}{\partial X}&= \frac{\partial\chi}{\partial Y},\\[0.5ex]
 \frac{\partial\psi}{\partial Y}&= -\frac{\partial\chi}{\partial X},
\end{align}
from which it follows that $\hat{\nabla}^2\psi=0$. 
The poloidal magnetic field is
\begin{equation}
{\bf B}_p \equiv  \nabla\phi\times \nabla\psi_p=\frac{\mu_0\,I_p}{2\pi\,a}\,\hat{\bf B},
\end{equation}
where
\begin{align}
\hat{B}_{X} &= \frac{\partial\psi}{\partial Y},\\[0.5ex]
\hat{B}_{Y}&= - \frac{\partial\psi}{\partial X}.
\end{align}
Now,
\begin{equation}
\frac{dF}{dz} = \frac{\partial\psi}{\partial X} + {\rm i}\,\frac{\partial\chi}{\partial X}= \frac{\partial\psi}{\partial X} - {\rm i}\,\frac{\partial\psi}{\partial Y},
\end{equation}
which implies that
\begin{align}
\hat{B}_X &= - {\rm Im}\left(\frac{dF}{dz}\right),\\[0.5ex]
\hat{B}_Y &= - {\rm Re}\left(\frac{dF}{dz}\right).
\end{align}

\section{First-Order Magnetic Null}
Let
\begin{equation}
F(z)= \ln z + \zeta\,\ln(z + {\rm i}),
\end{equation}
where $\zeta$ is real and positive. The first term corresponds to the plasma current filament located at the origin, whereas the second corresponds to the divertor coil filament located at $X=0$, $Y=-1$. 
The current in the plasma filament is $I_p$, whereas that in the divertor coil filament is $\zeta\,I_p$. 

Now,
\begin{equation}
\frac{dF}{dz} = \frac{1}{z} + \frac{\zeta}{z+{\rm i}}.
\end{equation}
If $z=z_0$ corresponds to a null  in the poloidal magnetic field, at which $|\hat{\bf B}|=0$,  then we require $(dF/dz)_{z_0}= 0$. It follows that
\begin{equation}
z_0 = - \frac{\rm i}{1+\zeta}.
\end{equation}
In other words, the magnetic null is located at $X=0$, $Y=Y_0$, where $Y_0 = - 1/(1+\zeta)$. 

Now,
\begin{equation}
\frac{d^2F}{dz^2} = -\frac{1}{z^2}- \frac{\zeta}{(z + {\rm i})^2}.
\end{equation}
So, at the magnetic null,
\begin{equation}
\left(\frac{d^2 F}{dz^2}\right)_{z_0} = \frac{(1+\zeta)^3}{\zeta}.
\end{equation}

Thus, in the vicinity of the magnetic null 
\begin{equation}
F(z)\simeq \frac{1}{2}\,\frac{(1+\zeta)^3}{\zeta} \,(z-z_0)^2.
\end{equation}
Let
\begin{equation}
z-z_0 = u + {\rm i}\,v.
\end{equation}
It follows that
\begin{equation}
\psi(u,v) = {\rm Re}(F) = \frac{1}{2}\,\frac{(1+\zeta)^3}{\zeta}\,(u^2-v^2).
\end{equation}
The separatrix curves corresponds to $\psi=0$. Thus, the separatrix curves are the two straight-lines $u=v$ and $u=-v$, which subtend right-angles with respect to one another. 

Consider the  flux-surface
\begin{equation}
v^2-u^2=\epsilon^2,
\end{equation}
whose distance of closest approach to the null point is $\epsilon$. Here, $0<\epsilon\ll 1$. Note that
\begin{equation}
\frac{dv}{du} = \frac{u}{v}.
\end{equation}
This flux-surface corresponds to the contour
\begin{equation}
\psi(u,v) = - \frac{1}{2}\,\frac{(1+\zeta)^3}{\zeta}\,\epsilon^2.
\end{equation}
On the flux-surface,
\begin{align}
|\hat{\nabla}\psi| &= \left[\left(\frac{\partial\psi}{\partial u}\right)^2+\left(\frac{\partial\psi}{\partial v}\right)^2\right]^{1/2}= \frac{1}{2}\,\frac{(1+\zeta)^3}{\zeta}\left[(2\,u)^2+(2\,v)^2\right]^{1/2}
= \frac{(1+\zeta)^3}{\zeta}\,(u^2+v^2)^{1/2},
\end{align}
and
\begin{equation}
dL =\left[1+\left(\frac{dv}{du}\right)^2\right]^{1/2}du=\left(1+\frac{u^2}{v^2}\right)^{1/2} du= \frac{(u^2+v^2)^{1/2}\,du}{v}= \frac{(u^2+v^2)^{1/2}\,du}{(u^2+\epsilon^2)^{1/2}}.
\end{equation}
Thus,
\begin{equation}
\frac{q (\epsilon)}{q_\ast}= \frac{1}{\pi} \, \frac{\zeta}{(1+\zeta)^3}\int_0^1\frac{du}{(u^2+\epsilon^2)^{1/2}}= \frac{1}{\pi} \, \frac{\zeta}{(1+\zeta)^3}\left[\ln\left\{(u^2+\epsilon^2)^{1/2} + u\right\}\right]_0^1
\simeq \frac{1}{\pi} \, \frac{\zeta}{(1+\zeta)^3}\,\ln\left(\frac{2}{\epsilon}\right).
\end{equation}

\section{Second-Order Magnetic Null}
Let
\begin{equation}
F(z) = \ln z + \frac{\zeta}{2}\,\ln(z+{\rm i}+{\mit\Delta}) + \frac{\zeta}{2}\,\ln(z+{\rm i}-{\mit\Delta}),
\end{equation}
where $\zeta$ and ${\mit\Delta}$ are real and positive. The first term corresponds to the plasma current filament located at the origin, whereas the second and third terms corresponds to two
 divertor coil filaments, carrying equal currents, located at $X=\pm{\mit\Delta}$, $Y=-1$. 

Now,
\begin{equation}
\frac{dF}{dz} = \frac{1}{z} + \frac{\zeta}{2\,(z+{\rm i}+{\mit\Delta})} + \frac{\zeta}{2\,(z+{\rm i}-{\mit\Delta})}.
\end{equation}
If $z=z_0$ corresponds to a null  in the poloidal magnetic field, at which $|\hat{\bf B}|=0$,  then we require $(dF/dz)_{z_0}= 0$. It follows that
\begin{equation}
(1+\zeta)\,z_0^{\,2} + {\rm i}\,(2+\zeta)\,z_0 - 1-{\mit\Delta}^2 = 0,
\end{equation}
which gives
\begin{equation}\label{ex}
z_0 = \frac{-{\rm i}\,(2+\zeta) \pm {\rm i}\sqrt{\zeta^2-4\,(1+\zeta)\,{\mit\Delta}^2}}{2\,(1+\zeta)}.
\end{equation}
Thus, in general, there are two magnetic null points. 

Now,
\begin{equation}
\frac{d^2F}{dz^2} = - \frac{1}{z^2} - \frac{\zeta}{2\,(z+{\rm i}+{\mit\Delta})^2} - \frac{\zeta}{2\,(z+{\rm i}-{\mit\Delta})^2}.
\end{equation}
If $z=z_0$ corresponds to a second-order null point then we require $(d^2F/dz^2)_{z_0}= 0$. It follows that 
\begin{equation}
{\mit\Delta} = \frac{\zeta}{2\sqrt{1+\zeta}},
\end{equation}
which is, of course, the condition for the two solutions in Eq.~(\ref{ex}) to merge. The second-order null is located at $X=0$, $Y=Y_0$, where
\begin{equation}
Y_0 = - \frac{(2+\zeta)}{2\,(1+\zeta)}.
\end{equation}

Now, 
\begin{equation}
\frac{d^3F}{dz^3} = \frac{2}{z^3} +  \frac{\zeta}{(z+{\rm i}+{\mit\Delta})^3} + \frac{\zeta}{(z+{\rm i}-{\mit\Delta})^3}.
\end{equation}
So, at the magnetic null, 
\begin{equation}
\left(\frac{d^3 F}{dz^3}\right)_{z_0} = - \frac{16\,{\rm i}\,(1+\zeta)^4}{2+\zeta)^2\,\zeta^2}.
\end{equation}

Thus, in the vicinity of the magnetic null 
\begin{equation}
F(z)\simeq -\frac{1}{6}\,\frac{16\,{\rm i}\,(1+\zeta)^4}{2+\zeta)^2\,\zeta^2}\,(z-z_0)^3.
\end{equation}
Let
\begin{equation}
z-z_0 = u + {\rm i}\,v.
\end{equation}
It follows that
\begin{equation}
\psi(u,v) = {\rm Re}(F) = \frac{8}{3}\,\frac{(1+\zeta)^4}{(2+\zeta)^2\,\zeta^2}\,(3\,u^2- v^2)\,v
\end{equation}
The separatrix curves corresponds to $\psi=0$. Thus, the separatrix curves are the three straight-lines $u=\pm v/\sqrt{3}$ and $v=0$ which subtend $60^\circ$ with respect to one another. 

Consider the 
he  flux-surface
\begin{equation}
(v^2-3\,u^2)\,v=\epsilon^3
\end{equation}
whose distance of closest approach to the null point is $\epsilon$. Here, $0<\epsilon\ll 1$. Note that
\begin{equation}
\frac{du}{dv} = \frac{v^2-u^2}{2\,u\,v}.
\end{equation}
This flux-surface corresponds to the contour
\begin{equation}
\psi(u,v) = -\frac{8}{3}\,\frac{(1+\zeta)^4}{(2+\zeta)^2\,\zeta^2}\,\epsilon^3.
\end{equation}
On the flux-surface,
\begin{align}
|\hat{\nabla}\psi| &= \left[\left(\frac{\partial\psi}{\partial u}\right)^2+\left(\frac{\partial\psi}{\partial v}\right)^2\right]^{1/2}= \frac{8}{3}\,\frac{(1+\zeta)^4}{(2+\zeta)^2\,\zeta^2} \left[(6\,u\,v)^2+(3\,u^2-3\,v^2)^2\right]^{1/2}
\nonumber\\[0.5ex]
&=\frac{8\,(1+\zeta)^4}{(2+\zeta)^2\,\zeta^2} \,(u^2+v^2), 
\end{align}
and
\begin{align}
dL &=\left[\left(\frac{du}{dv}\right)^2+1\right]^{1/2}dv=\left[\left(\frac{v^2-u^2}{2\,u\,v}\right)^2+1\right]^{1/2} dv= \frac{(u^2+v^2)\,dv}{2\,u\,v}.
\end{align}
Thus,
\begin{align}
\frac{q (\epsilon)}{q_\ast}&= \frac{1}{8\,\pi}\,\frac{(2+\zeta)\,\zeta^2}{(1+\zeta)^4} \,\int_\epsilon^\infty\frac{dv}{2\,u\,v}= \frac{\sqrt{3}}{16\,\pi} \, \frac{(2+\zeta)\,\zeta^2}{(1+\zeta)^4} \int_\epsilon^\infty
\frac{dv}{v^{1/2}\,(v^3-\epsilon^3)^{1/2}}\nonumber\\[0.5ex]
&= \frac{\sqrt{3}}{16\,\pi} \, \frac{(2+\zeta)\,\zeta^2}{(1+\zeta)^4} \,\frac{1}{\epsilon}\int_1^\infty
\frac{dx}{x^{1/2}\,(x^3-1)^{1/2}}=\frac{\sqrt{3}\,\Gamma(4/3)}{16\,\pi^{1/2}\,\Gamma(5/6)}\,\frac{(2+\zeta)\,\zeta^2}{(1+\zeta)^4} \,\frac{1}{\epsilon}.
\end{align}


\end{document}
