\documentclass{beamer}

\usepackage{color}
\usepackage{graphicx}
\usepackage{amsmath}

\newcommand{\blue}[1]{\textcolor{blue}{#1}}
\newcommand{\red}[1]{\textcolor{red}{#1}}
\newcommand {\bmu} {\mbox{\boldmath$\mu$}}

\setbeamertemplate{frametitle}[default][center]

\begin{document}

\begin{frame}
\begin{center}
\blue{\Large\bf Investigation of Neoclassical Tearing Mode}\\[1ex]
 \blue{\Large\bf Detection by ECE Radiometry in Tokamaks}\\[1ex]
 \blue{\Large\bf via Asymptotic Matching}\\[1ex]
 \blue{\Large\bf Techniques}\,\footnote{arXiv:2506.05553}\\[4ex]
Richard Fitzpatrick\\[2ex]
{\em Institute of Fusion Studies, Department of Physics},\\[0.5ex] {\em University of Texas at Austin}
\vspace{4cm}
\end{center}
\end{frame}

\begin{frame}
\frametitle{Disruption Avoidance}
 
\begin{itemize}
\item Next generation tokamaks, such as ITER, need to avoid \textcolor{red}{disruptions} (i.e., sudden catastrophic losses of plasma confinement) which could severely damage devices. 

\item Disruptions can be avoided by keeping net toroidal plasma current, mean plasma pressure, and mean plasma density, below well-known
limiting values.\footnote{P.C.~de Vries, et al.\ Nucl.\ Fusion {\bf 51}, 053018 (2011).}

\item However, there is one type of disruption (i.e., that associated with neoclassical tearing modes) that cannot be passively avoided during normal operation. 

\item  Instead, active measures are needed to prevent NTM-triggered disruptions. 

\end{itemize}
\end{frame}

\begin{frame}
\frametitle{Neoclassical Tearing Modes}
 
\begin{itemize}
\item Nor surprisingly, \textcolor{red}{neoclassical tearing modes} (NTMs) are observed to be main cause of disruptions in ITER-relevant discharges in present-day 
tokamaks.\footnote{R.J.~La\,Haye, Phys.\ Plasmas {\bf 13}, 055501 (2006).} 
\item NTMs are linearly stable, but are destabilized nonlinearly by loss of non-inductive \textcolor{red}{bootstrap current}, due to flattening of pressure profile within magnetic separatrix of
associated island chain.

\item NTMs are triggered in fairly random fashion by otherwise benign plasma instabilities such as sawtooth crashes. 
\item Once triggered, NTMs in ITER predicted to  grow as rotating modes on timescale of about \textcolor{blue}{3 s}, eventually lock to vacuum vessel when \textcolor{blue}{$W/a\sim 0.08$}, and then trigger disruption.\footnote{R. Fitzpatrick. Phys.\ Plasmas {\bf 30}, 042514 (2023).} 

\item Remedial action must be taken prior to locking!
\end{itemize}
\end{frame}

\begin{frame}
\frametitle{Electron Cyclotron Current Drive}
 
\begin{itemize}
\item NTMs can be stabilized by means of localized \textcolor{red}{electron cyclotron current drive} (ECCD) that is targeted at one of  O-points of  NTM
island chain.\footnote{G.~Gantenbien, et al.\ Phys.\ Rev.\ Lett.\ {\bf 85}, 1242 (2000).} Idea is to replace missing bootstrap current.

\item ECCD driven by suitably phased and directed electron cyclotron waves injected into plasma.\footnote{R.~Prater, Phys.\ Plasmas {\bf 13}, 055501 (2006).}

\item \textcolor{red}{Early detection} of NTM island chain, together with \textcolor{red}{accurate location} of island O-points,  is key to successful disruption avoidance by means of ECCD. 
\end{itemize}
\end{frame}

\begin{frame}
\frametitle{Electron Cyclotron Emission}
 
\begin{itemize}
\item Tokamak plasma spontaneously emit electron cyclotron waves.\footnote{ M.~Bornatici, et al.\ Nucl.\ Fusion {\bf 23}, 1153 (1983).}

\item Intensity of  \textcolor{red}{electron cyclotron emission} (ECE) (almost exactly) proportion to electron temperature of emitting region.

\item Frequency of \textcolor{red}{electron cyclotron emission} (ECE) depends mostly on magnetic field-strength, which is (almost exactly) inversely
proportion to major radius of emitting region. 

\item Hence, ECE emission can be used to simultaneously  detect local flattening of electron temperature due NTM island chain, and 
accurately determine major radius of  chain. 
\end{itemize}
\end{frame}

\begin{frame}
\frametitle{Synthetic ECE Diagnostics}
\begin{itemize}
\item  Accurate modeling of expected ECE signal due to NTM is important component of design of ECCD suppression  system. 
\item In existing synthetic ECE diagnostics,  a single-harmonic, radially-symmetric, NTM island chain, centered on 
rational surface, is
crudely inserted into plasma equilibrium.\footnote{J.P.~Ziegel, et al., Nucl.\ Fusion {\bf 64}, 126032 (2024).} 
\item In reality, an NTM consists of \textcolor{red}{multiple} helical harmonics. 
\item Harmonics with  same toroidal mode number as  NTM, but different poloidal mode numbers, are
coupled \textcolor{red}{linearly} via toroidicity and flux-surface shaping throughout  plasma. 
\item Harmonics whose poloidal and toroidal mode numbers are in  same ratio as those of  NTM are coupled \textcolor{red}{nonlinearly} in  immediate vicinity of island chain. 
\item NTM island chains are \textcolor{red}{radially asymmetric} due to mean gradient in tearing eigenfunction at rational surface. 
\end{itemize}
\end{frame}

\begin{frame}
\frametitle{Determining Magnetic Structure of NTM}
 
\begin{itemize}
\item By far,  most efficient method of determining stability and structure of NTM is via \textcolor{red}{asymptotic matching}.\footnote{H.P.~Furth,  J.~Killeen and M.N.~Rosenbluth,  Phys.\ Fluids {\bf 6}, 459 (1963).}
\item  In this approach, plasma divided into \textcolor{red}{outer region}, that comprises most of plasma, and \textcolor{red}{inner region} that is localized in vicinity of NTM rational surface. 
\item Tearing perturbation in outer region determined by solving linearized, marginally-stable,  ideal-MHD  equations in full toroidal geometry (e.g. using \textcolor{blue}{TJ} code.\footnote{R. Fitzpatrick, Phys. Plasmas {\bf 31}, 102507 (2024).})
\item Tearing perturbation in inner region consists of nonlinear, radially-asymmetric, island equilibrium.
\item Solutions in inner and outer regions asymptotically matched to one another at boundary between two regions to determine properties of island chain (e.g., radial asymmetry) in terms of
tearing eigenfunction in outer region. 
\end{itemize}

\end{frame}

\begin{frame}
\frametitle{Determining Electron Temperature Perturbation}
 
\begin{itemize}
\item No change in topology of magnetic flux-surfaces in outer region. Electron temperature assumed to be passively convected by plasma. Hence,
perturbed temperature is minus product of radial plasma displacement and equilibrium temperature gradient.
\item In inner region, electron temperature assumed to be constant on island magnetic flux-surfaces. Temperature profile determined by solution of \textcolor{blue}{$\langle \nabla^2 T_e\rangle=0$}, where
\textcolor{blue}{$\langle\cdots\rangle$} denotes island  flux-surface average.
\item Global temperature profile determined by asymptotically matching temperatures in inner and outer region at boundary between two regions. Determines reduction in core temperature due to island chain.
\end{itemize}

\end{frame}

\begin{frame}
\frametitle{Asymmetric Magnetic Island - I}
 
\begin{itemize}
\item Let \textcolor{blue}{$x=r-r_s$}, \textcolor{blue}{$X=x/W$}, and \textcolor{blue}{$\zeta=m\,\theta-n\,\phi$}, where \textcolor{blue}{$W$} is full width  of NTM island chain's magnetic separatrix. 

\item If island magnetic flux-surfaces are contours of \textcolor{blue}{${\mit\Omega}(X,\zeta)$} then equilibrium force-balance yields
$$
\textcolor{blue}{\left[\left.\frac{\partial^2{\mit\Omega}}{\partial X^2}\right|_\zeta,{\mit\Omega}\right]=0},
$$
where
$$
\textcolor{blue}{[A,B] \equiv \left.\frac{\partial A}{\partial X}\right|_\zeta \left.\frac{\partial B}{\partial\zeta}\right|_X- \left.\frac{\partial B}{\partial X}\right|_\zeta \left.\frac{\partial A}{\partial\zeta}\right|_X}.
$$
\item This requirement stipulates that the current density in the island region must be constant on magnetic flux-surfaces. 

\end{itemize}
\end{frame}

\begin{frame}
\frametitle{Asymmetric Magnetic Island - II}
 
\begin{center}
\includegraphics[width=0.7\textwidth]{../Fig3.pdf}
\end{center} 

\begin{itemize}
\item Suitable solution is 
$$
\textcolor{blue}{{\mit\Omega}(X,\zeta) = 8\,X^2 + \cos(\zeta-\delta^2\,\sin\zeta) - 2\sqrt{8}\,\delta\,X\,\cos\zeta+\delta^2\,\cos^2\zeta}. 
$$
\item Free parameter \textcolor{blue}{$\delta$} determines degree of radial asymmetry. 

\end{itemize}
\end{frame}

\begin{frame}
\frametitle{Asymmetric Magnetic Island - III}
 
\begin{center}
\includegraphics[width=0.7\textwidth]{../Fig6.pdf}
\end{center} 

\begin{itemize}
\item Perturbed electron temperature in island region written
$$
\textcolor{blue}{\delta T_e(X,\zeta) = \delta T_0(x)+\sum_{\nu>0} \delta T_\nu(X)\,\cos(\nu\,\zeta)}.
$$
\end{itemize}
\end{frame}


\begin{frame}
\frametitle{Asymmetric Magnetic Island - IV}
 
\begin{center}
\includegraphics[width=0.7\textwidth]{../Fig7.png}
\end{center}
\begin{itemize}
\item Thick curve: island separatrix. Dot: island O-point. Dashed line: 
rational surface. Contours show \textcolor{blue}{$T_e(X,\zeta)$}.
\item Note that island O-point, which is true ECCD target, is shifted radially inward from rational surface. 
\end{itemize}
\end{frame}

\begin{frame}
\frametitle{Perturbed Electron Temperature}
 
\begin{center}
\includegraphics[width=0.85\textwidth]{../Fig13.png}
\end{center}
\begin{itemize}
\item Left/right panels: \textcolor{blue}{$3,2$}/\textcolor{blue}{$2,1$} NTMs. Black curves: toroidally coupled rational surfaces. Black dot: magnetic axis. 
\item Realistic temperature perturbation associated with NTM is surprisingly complicated.   
\end{itemize}
\end{frame}

\begin{frame}
\frametitle{Total Electron Temperature}
 
\begin{center}
\includegraphics[width=0.85\textwidth]{../Fig14.png}
\end{center}
\begin{itemize}
\item Nevertheless, when perturbed electron temperature added to equilibrium temperature, result is appropriate helical flat-spot at NTM rational surface. 
\end{itemize}
\end{frame}

\begin{frame}
\frametitle{ECE Signal - I}
 
 \begin{itemize}
 \item In optically thick plasma, intensity of ECE emission directly proportional to electron temperature.
 \item Angular frequency of \textcolor{blue}{$j$}th harmonic ECE signal is
 $$
 \textcolor{blue}{\omega = \frac{j\,{\mit\Omega}_0\,R_0}{R}\left[1-\left(\frac{v}{c}\right)^2\right]^{1/2}},
 $$
 where $\textcolor{blue}{{\mit\Omega}_0=e\,B_0/m_e}$, $\textcolor{blue}{B_0}$ is on-axis toroidal magnetic field-strength,  \textcolor{blue}{$R_0$} is major radius of magnetic axis, and \textcolor{blue}{$v$}
 is electron speed. 
 \item Let 
 $$
 \textcolor{blue}{R_\omega(\omega) = \frac{j\,{\mit\Omega}_0\,R_0}{\omega}}
 $$
be the major radius from which ECE of frequency \textcolor{blue}{$\omega$} is emitted  in  absence of relativistic mass increase. \textcolor{blue}{$R$}
is actual major radius from which signal emitted. 
\end{itemize}
 \end{frame}
 
\begin{frame}
\frametitle{ECE Signal - II}
 
\begin{itemize}

\item It follows that
\begin{align*}
\textcolor{blue}{\frac{v}{c} =\left\{\begin{array}{ccc}\left[1-\left(\frac{R}{R_\omega}\right)^2\right]^{1/2}&~~~~~&R\leq R_\omega\\[1ex]
0&&R>R_\omega
\end{array}\right.}.
\end{align*}

 \item Distribution of electron speeds is
$$
\textcolor{blue}{f(v)= A\,v^2\,{\rm exp}\left(-\frac{1}{\theta_\omega}\left[1-\left(\frac{v}{c}\right)^2\right]^{-1/2}\right)},
$$
where
\textcolor{blue}{$\theta_\omega(\omega) = T_e(R_\omega)/(m_e\,c^{\,2})\ll 1$}. 
\end{itemize}

\end{frame}

\begin{frame}
\frametitle{ECE Signal - III}
 
\begin{itemize}
\item Previous two equations allow us to define 
$$
\textcolor{blue}{F(R,R_\omega) = \left[1-\left(\frac{R}{R_\omega}\right)^2\right]\exp\!\left(-\frac{1}{\theta_\omega}\,\frac{R_\omega}{R}\right)}.
$$

\item Electron temperature measured by  ECE diagnostic is  convolution of actual signal, \textcolor{blue}{$T_e(R)$},  and function \textcolor{blue}{$F(R,R_\omega)$}:
$$
\textcolor{blue}{T_e(R_\omega) = \frac{\int_{R_{\rm min}}^{R_\omega} T_e(R)\,F(R,R_\omega)\,dR}  {\int_{R_{\rm min}}^{R_\omega} F(R,R_\omega)\,dR}}.
$$
 \end{itemize}
 \end{frame}
 
 \begin{frame}
\frametitle{ECE Signal - IV}
 
\begin{itemize}
\item Convolution specifies distortion of ECE signal due to relativistic mass increase. 
\item Taylor expanding: 
\begin{align*}
\textcolor{blue}{T_e(R_\omega)}&\textcolor{blue}{~= T_e(R_\omega) - 2\,\theta_\omega\left(1-\frac{13}{2}\,\theta_\omega\right) R_\omega\,\frac{d T_e(R_\omega)}{dR}}\\[0.5ex]
&\phantom{=}\textcolor{blue}{+ 3\,\theta_\omega^{\,2}\,R_\omega^{\,2}\,
\frac{d^2 T_e(R_\omega)}{dR^2} + {\cal O}(\theta_\omega^{\,3})}.
\end{align*}
\item To first order  in \textcolor{blue}{$\theta_\omega$}, 
measured temperature profile is \textcolor{blue}{$T_e[R_\omega\,(1-2\,\theta_\omega)]$}: i.e.,  measured profile shifted outward in major radius 
distance \textcolor{blue}{$2\,\theta_\omega\,R_\omega$}. 
\item To second order,  measured profile  smeared out in major radius. 
\end{itemize}
\end{frame}

\begin{frame}
\frametitle{Synthetic ECE Diagnostic - I}
 
 \begin{center}
\includegraphics[width=0.65\textwidth]{../Fig15.pdf}
\end{center}

\begin{itemize}
\item Left/right-panels: \textcolor{blue}{3,2}/\textcolor{blue}{2,1} NTMs. Top/bottom panels: two different toroidal angles. Red/blue curves: distorted/ undistorted \textcolor{blue}{$\delta T_e$} ECE signals. Dashed lines: rational surfaces. 
\item Relativistic mass increase shifts inferred location of ECE signal outward in major radius, and also smears out signal. 
\end{itemize}
 \end{frame}


\begin{frame}
\frametitle{Synthetic ECE Diagnostic - II}
 
 \begin{center}
\includegraphics[width=0.65\textwidth]{../Fig16.pdf}
\end{center}

\begin{itemize}
\item Left/right-panels: \textcolor{blue}{3,2}/\textcolor{blue}{2,1} NTMs. Top/bottom panels: two different toroidal angles. Red/blue curves: distorted /undistorted \textcolor{blue}{$T_e$} ECE signals. Dashed lines: rational surfaces. 
\item Relativistic mass increase shifts inferred location of ECE signal outward in major radius, and also smears out signal. 
\end{itemize}
 \end{frame}
 
\begin{frame}
\frametitle{Synthetic Berrino Algorithm - I}
 
\begin{center}
\includegraphics[width=0.65\textwidth]{../Fig17.pdf}
\end{center}

\begin{itemize}
\item Synthetic Berrino algorithm\footnote{J.~Berrino, et al, Nucl.\ Fusion {\bf 45}, 1350 (2005).} (radial gradient of ECE signal averaged over toroidal angle) gives clear signal for NTM islands whose widths are as small as \textcolor{blue}{1\%} of minor radius. 
\item Minimum of Berrino signal (which is usually taken as target radius for ECCD) shifted outward in major radius due to relativistic mass increase. 
\end{itemize}

\end{frame}

\begin{frame}
\frametitle{Synthetic Berrino Algorithm - II}
 
\begin{center}
\includegraphics[width=0.65\textwidth]{../Fig18.pdf}
\end{center}

\begin{itemize}
\item If inferred major radius of signal taken to be $\textcolor{blue}{R_\omega\,(1-\theta_\omega)}$, instead of $\textcolor{blue}{R_\omega}$,
then minimum in Berrino signal gives much better indication of location of rational surface. 
\item However, signal still need to be corrected for inward shift of island O-points. 
\end{itemize}

\end{frame}


\begin{frame}
\frametitle{Conclusions}
 
\begin{itemize}
\item Asymptotic matching techniques permit rapid and realistic calculation of ECE signal generated by an NTM. 

\item Asymptotic matching techniques can calculate magnetic, temperature, and density perturbations associated with tearing mode in
both linear and nonlinear regimes. 

\item As such, asymptotic matching techniques could be used to simulate any diagnostic used to study tearing modes. 
\end{itemize}
\end{frame}



\end{document}