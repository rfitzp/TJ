\documentclass{article}[12pt]
\usepackage{fullpage}
\usepackage{amsmath}
\newcommand {\bxi}{\mbox{\boldmath$\xi$}}

\begin{document}
\begin{center}
{\em Investigation of Neoclassical Tearing Mode Detection by ECE Radiometry in  an ITER-like Tokamak via Asymptotic Matching Techniques}\\[1ex]
by R.~Fitzpatrick\\[1ex]
{\bf Reply to Referees' Comments}\\[1ex]
~
\end{center}
Let me thank the referees for their helpful and insightful comments on my paper. Here are my responses to these comments. 

\begin{enumerate}

\item The main change that I have made to the paper is to move most of the technical parts  into Appendices. I hope
that this will make the paper more easily readable. 

\item I have changed Eq.~(19) in the paper to remove the offset in the temperature, $T_{e\,{\rm ped}}$, because this was actually zero in the calculation. This means that
the pressure now has the regular definition as the product of the temperature and density. 

\item The fact that the plasma pressure is limited to a rather low value by classical tearing mode stability is symptomatic of the fact
that the inverse aspect-ratio of the equilibrium is limited to $\epsilon=0.2$. If TJ were able to deal with the realistic inverse aspect-ratio of $\epsilon= 0.32$ then the pressure
could be increased (because the true limit is set by $\beta_N\propto \beta_t/\epsilon$). It is also possible that the pressure could be increased by
optimizing the safety-factor and pressure profiles. It have added comments to this effect in Sect.~3.2.

\item In Eq.~(42), it is true that $8\,X^2$ is the first term in a Taylor expansion. The next term in the expansion is of order $(q''\,W/q')\,X^3$. Obviously,
this next term can be neglected for a sufficiently thin island. However, the real reason for neglecting the term is that I do not know how to
solve the island force balance equation (40) if the term is retained. I have added a comment to this effect in Sect.~C.3. 

\item In my statement (in Sect.~D.4) that ECCD can successfully stabilize an NTM provided that the offset of the peak of the current drive profile
from the rational surface, $d$, is less than twice the standard deviation of the deposition profile, $D$, there is an implicit assumption that the
island width is less than the deposition width. This is a reasonable assumption because, to be successful, an ECCD mitigation system need to 
detect and suppress an NTM before the island width grows to a large value. (Incidentally, $D$ and $d$ are normalized to the plasma
minor radius rather than the island width.) I have added  comments to this effect  whenever this result is mentioned. 

\item In Sect.~D.4, I have added a discussion of why I think that NTMs are probably going to be rotating in ITER. 

\item In Sect.~8.5, I have added a comment to the effect that the 1\% of the plasma minor radius island width detection threshold excludes the effect of experimental noise, and
should be treated as an absolute, rather than a practical, detection threshold. 

\item I have added the critical island width for temperature flattening to Table~1. It can be seen that the critical island width is less
than 1\% of the plasma minor radius, which means that incomplete flattening is not going to modify my calculation. 

\item I have added more explanation of the model equilibrium to Sect.~A.3. 

\item I have added a brief discussion of Thomson scattering to Sect.~9. As far as I can ascertain, Thomson scattering is only used to detect
NTMs when ECE is not practical: e.g., in high-density ITER baseline scenario discharges in DIII-D, or in spherical tokamaks. I could not find an
example of the use of Thomson scattering as part of an NTM mitigation scheme in an experiment. The main drawback of Thomson
scattering is that its temporal resolution is not as good as ECE. 

\end{enumerate}

\end{document}