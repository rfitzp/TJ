\documentclass{iopjournal}
\usepackage{amsmath}           		          	
\usepackage{graphicx,epstopdf}					
\usepackage{amssymb}
\newcommand {\bxi}{\mbox{\boldmath$\xi$}}
\pdfoutput = 1 
\allowdisplaybreaks

\begin{document}
\articletype{paper}
\title{Calculation of Tearing Mode Stability in a Magnetically Diverted Tokamak Plasma}
\author{Richard Fitzpatrick\orcid{0000-0001-6237-9309}}
\affil{Institute for Fusion Studies, Department of Physics, University of Texas at Austin, Austin TX 78712, USA}
\email{rfitzp@utexas.edu}

\section{Two-Filament Model of a Magnetically Diverted Plasma}
\subsection{Introduction}
The aim of this section is to construct a very simple model of a magnetically diverted tokamak plasma.

\subsection{Equilibrium Magnetic Field}
Suppose that two current filaments run parallel to the $z$-axis \cite{pom}. Let the first filament carry the
current $I_p$, and pierce the $x$-$y$ plane at $x=y=0$. Let the second filament carry the current $I_c$, and
pierce the $x$-$y$ plane at $x=0$, $y=-a$.  Here, $a$ is the effective minor radius of the plasma. The first filament represents the ``toroidal'' (i.e., $z$-directed) plasma
current, whereas the second represents the current flowing in the magnetic divertor coil. 
Suppose that there is a uniform externally generated ``toroidal'' (i.e., $z$-directed) magnetic
field of strength $B_0$. Let the system be periodic in the $z$ direction with period $2\pi\,R_0$, where $R_0$ is the simulated major
radius of the plasma.  
It is helpful to define the simulated toroidal angle,  $\phi= z/R_0$. 

The equilibrium  magnetic field can be written in the divergence-free form
\begin{equation}\label{e1}
{\bf B} = \nabla\phi\times \nabla\psi_p + B_0\,R_0\,\nabla\phi,
\end{equation}
where 
\begin{equation}
\psi_p(x,y)=  \frac{\mu_0\,I_0\,R_0}{4\pi}\,\ln(x^2+y^2) + \frac{\mu_0\,I_1\,R_0}{4\pi} \,\ln[x^2+(y+a)^2]
\end{equation}
is the ``poloidal'' magnetic flux (divided by $2\pi$) generated by the two current filaments.

\subsection{Flux Coordinates}
It is convenient to re-express the magnetic field as 
\begin{equation}\label{e2}
{\bf B} = \nabla(\phi-q\,\theta)\times \nabla \psi_p,
\end{equation}
where $\theta$ is a poloidal (i.e., in the $x$-$y$ plane) angle, and  $q=q(\psi_p)$ is the safety-factor  \cite{boz}. Equations~(\ref{e1}) and (\ref{e2}) can be reconciled provided 
\begin{equation}\label{e3}
\nabla\psi_p\times\nabla\theta\cdot\nabla\phi = \frac{B_0}{R_0\,q}.
\end{equation}
Note, from Eq.~(\ref{e2}),  that ${\bf B}\cdot\nabla\psi_p=0$, which implies that  $\psi_p$ is a magnetic flux-surface label. Furthermore,
${\bf B}\cdot\nabla(\phi-q\,\theta)=0$, which implies that magnetic field-lines within a given flux-surface appear as straight lines, 
with gradient $d\phi/d\theta = q$, when 
plotted in the $\theta$-$\phi$ plane. In fact, $\psi_p$, $\theta$, $\phi$ are known as {\em flux-coordinates}, and $\theta$
is termed a ``straight''  poloidal angle \cite{boz}.

\subsection{Non-Diverted Edge Safety-Factor}
In the absence of the divertor current, the plasma would have a circular cross-section of minor radius $a$, and
an edge safety-factor of
\begin{equation}\label{e4}
q_\ast = \frac{2\pi\,B_0\,a^2}{\mu_0\,I_p\,R_0}.
\end{equation}

\subsection{Normalization Scheme}
Let  $x=a\,X$, $y=a\,Y$,  $\nabla = a^{-1}\,\hat{\nabla}$, and $\psi_p = \mu_0\,I_p\,R_0\,\psi/(2\pi)$.
It follows that
\begin{align}
\hat{\nabla}\psi\times \hat{\nabla}\theta\cdot\hat{\nabla}\phi &= \frac{a}{R_0}\,\frac{q_\ast}{q},\label{ex}\\[0.5ex]
\psi &= \frac{1}{2}\,\ln\left(X^2+Y^2\right) + \frac{\beta}{2}\,\ln\left[X^2+(Y+1)^2\right],\\[0.5ex]
\psi_X &= \frac{X}{X^2+Y^2} + \frac{\beta\,X}{X^2+(Y+1)^2},\\[0.5ex]
\psi_Y &= \frac{Y}{X^2+Y^2} +\frac{\beta\,(Y+1)}{X^2+(Y+1)^2},
\end{align}
where $\beta=I_c/I_p$. Here, $\psi_X\equiv\partial\psi/\partial X$, et cetera. 

\subsection{Magnetic X-point}
The magnetic $X$-point forms at the point in the $X$-$Y$ plane at which $\psi_X=\psi_Y=0$.
As is easily demonstrated, the coordinates of this point are ($X_c$, $Y_c$), where  $X_c=0$ and $Y_c=-1/(1+\beta)$. 
The magnetic separatrix corresponds to the curve $\psi(X,Y)= \psi_c$, where 
\begin{equation}
\psi_c \equiv \psi(X_c,Y_c) = \ln\left[\frac{\beta^{\,\beta}}{(1+\beta)^{1+\beta}}\right].
\end{equation}
 It is helpful to define the normalized poloidal flux ${\mit\Psi}= \psi_c/\psi$. 

\subsection{Construction of Flux Coordinate System}
Equation~(\ref{ex})  yields
\begin{equation}
\frac{d\theta}{dL} = \frac{q_\ast}{q\,|\hat{\nabla}\psi|}.
\end{equation}
where $\hat{\nabla}= a\,\nabla$, and $dL$ is an element of normalized length around a magnetic flux-surface. 
It follows that
\begin{equation}
q(\psi) = \frac{q_\ast}{2\pi}\oint\frac{dL}{|\hat{\nabla}\psi|},
\end{equation}
where $\oint$ implies a complete circuit in $\theta$ at constant $\psi$. 
It is easily demonstrated that, on such a circuit, 
\begin{align}\label{e11}
\frac{dX}{dL} &= -\frac{\psi_Y}{\sqrt{\psi_X^{\,2} + \psi_Y^{\,2}}},\\[0.5ex]
\frac{dY}{dL}&= \frac{\psi_X}{\sqrt{\psi_X^{\,2} + \psi_Y^{\,2}}},\\[0.5ex]
\frac{d\phi}{dL} &=  \frac{q_\ast}{\sqrt{\psi_X^{\,2} + \psi_Y^{\,2}}},\\[0.5ex]
\frac{d\omega}{dL}  &= \frac{X\,\psi_X + Y\,\psi_Y}{(X^2+Y^2)\sqrt{\psi_X^{\,2} + \psi_Y^{\,2}}},\label{e22}
\end{align}
where
\begin{equation}
\omega =\tan^{-1} \left(\frac{Y}{X}\right)
\end{equation}
is a geometric poloidal angle. 
Here, $\phi$ is calculated on the assumption that we are following a magnetic field-line within the flux-surface (i.e., $d\phi/d\theta = q$). We need to
integrate Eqs.~(\ref{e11})--(\ref{e22}) from $\omega=0$ to $\omega = 2\pi$, subject to the initial
condition $\phi(\omega=0)=0$, and then set $q(\psi)= \phi(\omega=2\pi)/(2\pi)$.
We can then compute $\theta$ using
\begin{align}
\frac{d\theta}{dL}& = \frac{1}{q\sqrt{\psi_X^{\,2} + \psi_Y^{\,2}}}.
\end{align}

\subsection{Results}
Let $q_\ast = 12$ and $\beta=0.2$. 
Figure~\ref{fig1} shows the magnetic flux-surfaces ${\mit\Psi}=0.9$, ${\mit\Psi}=1.0$, and ${\mit\Psi}=1.1$, plotted in the 
$X$-$Y$ plane. Flux-surfaces characterized by ${\mit\Psi}<1$ do not enclose the divertor current filament, whereas those
characterized by ${\mit\Psi}<1$ do enclose the filament. The magnetic separatrix, ${\mit\Psi}=1$, separates flux-surfaces
 that do and do not enclose the divertor current filament, and crosses itself at the magnetic X-point. 
 
Figure~\ref{fig2} shows the safety-factor profile $q({\mit\Psi})$. It is clear that the safety-factor generally increases with increasing ${\mit\Psi}$. 
However,   $q\rightarrow\infty$ as ${\mit\Psi}\rightarrow 1$.
 In other words, the safety-factor tends to infinity as the magnetic separatrix is approached. It is apparent from the bottom
 panel that $q$ approaches infinity logarithmically as ${\mit\Psi}\rightarrow 1$ (because the plot of $q$ versus
 $\log_{10}(|{\mit\Psi}-1|)$ asymptotes to a straight line as $|{\mit\Psi}-1|\rightarrow 0$). In other words, close to
 the separatrix we can write
 \begin{equation}
 q({\mit\Psi})\simeq -\alpha_-\,\log(1-{\mit\Psi})
 \end{equation}
 for ${\mit\Psi}<1$, and 
 \begin{equation}
 q({\mit\Psi})\simeq -\alpha_+\,\log({\mit\Psi}-1)
 \end{equation}
 for ${\mit\Psi}>1$. 
 Moreover, it is clear from the figure that $\alpha_+>\alpha_-$. 
 
 Figures~\ref{fig3} and \ref{fig4} show the flux coordinate system inside and outside the magnetic separatrix, respectively. It can be seen that, as the magnetic
 separatrix is crossed, all the contours of $\theta$ converge onto the X-point, and then diverge away from it. This singular behavior occurs
 because the Jacobian of the coordinate system, ${\cal J} \equiv (\hat{\nabla}\psi\times\hat{\nabla}\theta\cdot\hat{\nabla}\phi)^{-1} = (R_0/a)\,(q/q_\ast)$,
 is infinite at the X-point. 
 
\subsection{Significance of Flux Coordinates}
To understand the significance of the flux coordinate system, suppose that the plasma is subject to a magnetic perturbation that
varies with $\theta$, $\phi$ and time, $t$, as  $\exp[\,{\rm i}\,(m\,\theta-n\,\phi)+\gamma\,t]$. Here, $m$ and $n$ are integers. 
In other words, the perturbation, which is, of course, single-valued in the angular coordinates $\theta$ and $\phi$, possesses
$m$ periods in the poloidal angle and $n$ periods in the toroidal angle. The scalar product of the curl of the linearized MHD
Ohm's law yields
\begin{equation}
-\gamma\,b^{\psi_p} + {\rm i}\,\gamma\,\frac{B_0}{R_0\,q}\,(m-n\,q)\,\xi^{\psi_p} \simeq \frac{B_0}{R_0\,q}\,{\rm i}\,m \,\eta_\parallel \,j_\phi,
\end{equation}
where subscripts and superscripts denote covariant and contravariant components in the $\psi_p$, $\theta$, $\phi$ coordinate
system. Here, ${\bf b}$ is the perturbed magnetic field, $\bxi$ the plasma displacement,  ${\bf j}$ the perturbed current
density, and $\eta_\parallel$ the parallel plasma electrical resistivity. 

The previous equation described how the inductive electric
field generated by a time-varying magnetic field drives a current parallel to magnetic field-lines. On a general magnetic flux-surface, the
first two terms on the left-hand side of the previous equation cancel one another out, and there is no  driven current. In other words, 
the plasma displaces rather than allowing a parallel inductive current to flow. However, it is clear from the equation that there exist
special magnetic flux-surfaces, termed {\em rational}\/ flux-surfaces, at which the safety-factor takes the rational
value $q=m/n$. On a rational flux-surface, the two terms on the left-hand side of the previous equation cannot cancel one another
out, because the second term is zero. Hence, in general, a parallel inductive current flows on a rational flux-surface. The current is a
shielding current that acts to suppress magnetic reconnection on the flux-surface, or, at least, to slow down reconnection such that it
takes place on the comparatively long resistive timescale. 

We can now appreciate that by employing a flux coordinate system we can distinguish rational magnetic flux-surfaces from
irrational flux-surfaces. We can also calculate the angular variation of the particular magnetic perturbation that drives a
shielding current at a particular rational surface. Note, finally, that it is clear from Figs.~\ref{fig2}--\ref{fig4} that rational
flux-surfaces exist both inside and outside the magnetic separatrix. 

\iffalse
\section{Safety-Factor}
Let $\hat{r}=r/a$.  Suppose that the last closed magnetic flux surface is at $\hat{r}=1$. Suppose that the safety-factor profile takes the
form
\begin{equation}
q(\hat{r}) = q_0 + (q_{95}-q_0)\,\frac{\ln(1-\hat{r}^{\,2})}{\ln(1-\hat{r}_{95}^{\,2})}.
\end{equation}
[Logarithmic increase in $q$ at separatrix:  N.~Pomphrey and
 A.~Reiman, Phys.\ Fluids B {\bf 4}, 938 (1992).]
\section{Poloidal Magnetic Flux}
Now,
\begin{equation}
\frac{d\psi_p}{dr} = B_0\,R_0\,f = B_0\,a\,\frac{\hat{r}}{q},
\end{equation}
assuming that $g(\hat{r})=1$. Let $\hat{\psi}_p = \psi_p/(B_0\,a^2)$. It follows that
\begin{equation}
\frac{d\hat{\psi}_p}{d\hat{r}} = \frac{\hat{r}}{q}=\frac{\hat{r}}{q_0-\alpha\,\ln(1-\hat{r}^{\,2})},
\end{equation}
where
\begin{equation}
\alpha = -\frac{q_{95}-q_0}{\ln(1-\hat{r}_{95}^{\,2})}.
\end{equation}
So,
\begin{align}
\hat{\psi}_p(\hat{r})&=\int_0^{\hat{r}}\frac{d\hat{\psi}_p}{d\hat{r}}\,d\hat{r} = \frac{1}{2}\int_0^{\hat{r}^{\,2}}\frac{dx}{q_0 -\alpha\,\ln(1-x)} = \frac{1}{2}\int_{1-\hat{r}^{\,2}}^1\frac{dy}{q_0-\alpha\,\ln y}\nonumber\\[0.5ex]
&=
\frac{{\rm e}^{\,q_0/\alpha}}{2\,\alpha}\int^{q_0/\alpha-\ln(1-\hat{r}^{\,2})}_{q_0/\alpha}\frac{{\rm e}^{-z}\,dz}{z}
= \frac{{\rm e}^{\,q_0/\alpha}}{2\,\alpha}\left[\int_{q_0/\alpha}^\infty \frac{{\rm e}^{-z}\,dz}{z}-\int_{q_0/\alpha-\ln(1-\hat{r}^{\,2})}^\infty \frac{{\rm e}^{-z}\,dz}{z}\right]\nonumber\\[0.5ex]
&= \frac{{\rm e}^{\,q_0/\alpha}}{2\,\alpha}\left[\int_{1}^\infty \frac{{\rm e}^{-(q_0/\alpha)\,t}\,dt}{t}-
\int_1^\infty \frac{{\rm e}^{-[q_0/\alpha-\ln(1-\hat{r}^{\,2})]\,t}\,dt}{t}\right],
\end{align}
giving
\begin{equation}
\hat{\psi}_p(\hat{r}) = \frac{{\rm e}^{\,q_0/\alpha}}{2\,\alpha}\left\{E_1(q_0/\alpha)-E_1[q_0/\alpha-\ln(1-\hat{r}^{\,2})]\right\}.
\end{equation}
Thus,
\begin{equation}
\hat{\psi}_p(1) =  \frac{{\rm e}^{\,q_0/\alpha}}{2\,\alpha}\,E_1(q_0/\alpha).
\end{equation}

Let ${\mit\Psi}(\hat{r})=\psi_p(\hat{r})/\psi_p(1)$. It follows that
\begin{equation}
{\mit\Psi}(\hat{r})= 1- \frac{E_1[q_0/\alpha-\ln(1-\hat{r}^{\,2})]}{E_1(q_0/\alpha)}.
\end{equation}
By definition, ${\mit\Psi}(\hat{r}_{95}) = 0.95$, so 
\begin{equation}
0.95 = 1- \frac{E_1[q_0/\alpha-\ln(1-\hat{r}_{95}^{\,2})]}{E_1(q_0/\alpha)}= 1- \frac{E_1(q_{95}/\alpha)}{E_1(q_0/\alpha)},
\end{equation}
giving
\begin{equation}
0.05 = \frac{E_1(q_{95}/\alpha)}{E_1(q_0/\alpha)}.
\end{equation}
Assuming that $q_0$ and $q_{95}$ are specified, the  previous equation can be solved to give $\alpha$, which then determines $\hat{r}_{95}$. 

\section{Rational Surfaces}
Suppose that $n$ is the toroidal mode number. The $m$, $n$ rational surface lies at $\hat{r}=\hat{r}_m$, where $q(\hat{r}_m)=q_m$ and $q_m=m/n$. 
Thus,
\begin{equation}
\hat{r}_m= \left[1-\exp\left(\frac{q_0 - q_m}{\alpha}\right)\right]^{1/2}.
\end{equation}
Hence, if ${\mit\Psi}_m \equiv {\mit\Psi}(\hat{r}_m)$ then
\begin{equation}
{\mit\Psi}_m =1 - \frac{E_1(q_m/\alpha)}{E_1(q_0/\alpha)}.
\end{equation}
The spacing (in $\hat{r}$) between successive rational surfaces is
\begin{equation}
\hat{\delta}_{\rm rational} = \frac{d\hat{r}_m}{dm} = \frac{1-\hat{r}_m^{\,2}}{2\,\hat{r}_m\,n\,\alpha}.
\end{equation}
In the continuum limit, assuming that $\hat{r}_m\simeq 1$, we get
\begin{equation}
\hat{\delta}_{\rm rational}(\hat{r})  \simeq \frac{1-\hat{r}}{n\,\alpha}=\frac{x}{n\,\alpha},
\end{equation}
where $x=1-\hat{r}$. 

\section{Magnetic Shear}
The magnetic shear is
\begin{equation}
s(\hat{r}) = \frac{\hat{r}}{q}\,\frac{dq}{d\hat{r}} = \frac{2\,\alpha\,\hat{r}^{\,2}}{(1-\hat{r}^{\,2})\,q} \simeq \frac{1}{f(x)},
\end{equation}
where
\begin{equation}
f(x) = - x\,\ln(2\,x).
\end{equation}

\section{Profiles}
Let
\begin{align}
n_e({\mit\Psi}) &= 20\,n_{e\,95}\,(1-{\mit\Psi}) + n_{e\,100},\\[0.5ex]
T_e({\mit\Psi}) &=20\,T_{e\,95}\,(1-{\mit\Psi}) + T_{e\,100},\\[0.5ex]
T_i({\mit\Psi}) &= 20\,T_{i\,95}\,(1-{\mit\Psi}) + T_{i\,100}.
\end{align}
Thus,
\begin{align}
\omega_{\ast\,e}({\mit\Psi})& = \frac{T_e}{e}\left(\frac{d\ln n_e}{d\psi_p}+ \frac{d\ln T_e}{d\psi_p}\right)=
\frac{T_e}{e\,B_0\,a^2\,\psi_p(1)}\left(\frac{d\ln n_e}{d{\mit\Psi}}+ \frac{d\ln T_e}{d{\mit\Psi}}\right)\nonumber\\[0.5ex]
&\simeq - \frac{20\,T_{e\,100}}{e\,B_0\,a^2\,\psi_p(1)}\left(\frac{n_{e\,95}}{n_{e\,100} }+ \frac{T_{e\,95}}{T_{e\,100}}\right),\\[0.5ex]
\omega_{\ast\,i}({\mit\Psi}) &= -\frac{T_i}{e}\left(\frac{d\ln n_e}{d\psi_p}+ \frac{d\ln T_i}{d\psi_p}\right)=
-\frac{T_i}{e\,B_0\,a^2\,\psi_p(1)}\left(\frac{d\ln n_e}{d{\mit\Psi}}+ \frac{d\ln T_i}{d{\mit\Psi}}\right)\nonumber\\[0.5ex]
&\simeq - \frac{20\,T_{i\,100}}{e\,B_0\,a^2\,\psi_p(1)}\left(\frac{n_{e\,95}}{n_{e\,100} }+ \frac{T_{i\,95}}{T_{i\,100}}\right).
\end{align}

\section{Layer Quantities}
Now, 
\begin{align}
\tau_{ee} &= \frac{6\sqrt{2}\,\pi^{3/2}\,\epsilon_0^{\,2}\,m_e^{1/2}\,T_e^{\,3/2}}{\ln{\mit\Lambda}\,e^4\,n_e}\simeq  \frac{6\sqrt{2}\,\pi^{3/2}\,\epsilon_0^{\,2}\,m_e^{1/2}\,T_{e\,100}^{\,3/2}}{\ln{\mit\Lambda}\,e^4\,n_{e\,100}} ,\\[0.5ex]
\sigma_\parallel &= 1.96\,\frac{n_e\,e^2\,\tau_{ee}}{m_e},\\[0.5ex]
\tau_R &= \mu_0\,a^2\,\sigma_\parallel\,\hat{r}^{\,2}\simeq \mu_0\,a^2\,\sigma_\parallel,\\[0.5ex]
\tau_H &= \frac{R_0}{B_0}\,\frac{\sqrt{\mu_0\,m_i\,n_e}}{n\,s}\simeq \tau_A\,f,\\[0.5ex]
\tau_A &= \frac{R_0}{B_0}\,\frac{\sqrt{\mu_0\,m_i\,n_{e\,100}}}{n},\\[0.5ex]
\tau_\perp &= \frac{a^2\,\hat{r}^{\,2}}{D_\perp} \simeq \frac{a^2}{D_\perp},\\[0.5ex]
\tau_\varphi &= \frac{a^2\,\hat{r}^{\,2}}{\chi_\varphi}\simeq \frac{a^2}{\chi_\varphi},\\[0.5ex]
\tau &= -\frac{\omega_{\ast\,e}}{\omega_{\ast\,i}},\\[0.5ex]
d_\beta &= \frac{\sqrt{(5/3)\,m_i\,(T_e+T_i)}}{e\,B_0}\simeq  \frac{\sqrt{(5/3)\,m_i\,(T_{e\,100}+T_{i\,100})}}{e\,B_0},\\[0.5ex]
S &= \frac{\tau_R}{\tau_H} = {\cal S}\,f^{-1}\\[0.5ex]
{\cal S} &= \frac{\tau_R}{\tau_A},\\[0.5ex]
P_\varphi &=\frac{\tau_R}{\tau_\varphi},\\[0.5ex]
P_\perp &= \frac{\tau_R}{\tau_\perp},\\[0.5ex]
D &= S^{\,1/3}\left(\frac{\tau}{1+\tau}\right)^{1/2}\,\frac{d_\beta}{a\,\hat{r}}\simeq {\cal D}\,f^{-1/3},\\[0.5ex]
{\cal D} &= {\cal S}^{\,1/3}\left(\frac{\tau}{1+\tau}\right)^{1/2}\,\frac{d_\beta}{a},\\[0.5ex]
Q_E &= -S^{\,1/3}\,n\,\omega_E\,\tau_H\simeq {\cal Q}_E\,f^{\,2/3},\\[0.5ex]
{\cal Q}_E&= -{\cal S}^{\,1/3}\,n\,\omega_E\,\tau_A,\\[0.5ex]
{\cal Q}_{e,i} &= -S^{\,1/3}\,n\,\omega_{\ast\,e,i}\,\tau_H\simeq {\cal Q}_{\ast\,e,i}\,f^{\,2/3},\\[0.5ex]
{\cal Q}_{e\,i}&= -{\cal S}^{\,1/3}\,n\,\omega_{\ast\,e\,i}\,\tau_A,\\[0.5ex]
Q &= S^{1/3}\,\omega\,\tau_H= {\cal Q}\,f^{\,2/3},\\[0.5ex]
{\cal Q}&= {\cal S}^{1/3}\,\omega\,\tau_A.
\end{align}

\section{Layer Equation}
The layer equation is written
\begin{equation}
\frac{d}{dp}\!\left[A(p)\,\frac{dY_e}{dp}\right] - \frac{B(p)}{C(p)}\,p^2\,Y_e=0,
\end{equation}
where
\begin{align}
A &= \frac{p^2}{-{\rm i}\,(Q-Q_E-Q_e)+p^2},\\[0.5ex]
B &= - {\rm i}\,(Q-Q_E)\,(Q-Q_E-Q_i)-{\rm i}\,(Q-Q_E-Q_i)\,(P_\varphi+P_\perp)\,p^2 + P_\varphi\,P_\perp\,p^4,\\[0.5ex]
C &= - {\rm i}\,(Q-Q_E-Q_e)+ [P_\perp-{\rm i}\,(Q-Q_E-Q_i)\,D^{\,2}]\,p^2
+ (1+1/\tau)\,P_\varphi\,D^{\,2}\,p^4.
\end{align}
Let
\begin{equation}
p = f^{-1/6}\,\hat{p}.
\end{equation}
It follows that 
\begin{equation}\label{elayer}
\frac{d}{d\hat{p}}\!\left[{\cal A}\,\frac{dY_e}{d\hat{p}}\right] - \frac{{\cal B}}{\cal{C}}\,\hat{p}^{\,2}\,Y_e=0,
\end{equation}
where
\begin{align}
{\cal A} &= \frac{\hat{p}^{\,2}}{-{\rm i}\,({\cal Q}-{\cal Q}_E-{\cal Q}_e)\,f+\hat{p}^{\,2}},\\[0.5ex]
{\cal B} &= - {\rm i}\,({\cal Q}-{\cal Q}_E)\,({\cal Q}-{\cal Q}_E-{\cal Q}_i)\,f^{\,2}-{\rm i}\,({\cal Q}-{\cal Q}_E-{\cal Q}_i)\,(P_\varphi+P_\perp)\,\hat{p}^{\,2}\,f + P_\varphi\,P_\perp\,\hat{p}^{\,4},\\[0.5ex]
{\cal C} &= - {\rm i}\,({\cal Q}-{\cal Q}_E-{\cal Q}_e)\,f^{\,2}+ [P_\perp-{\rm i}\,({\cal Q}-{\cal Q}_E-{\cal Q}_i)\,{\cal D}^{\,2}]\,\hat{p}^{\,2}\,f
+ (1+1/\tau)\,P_\varphi\,{\cal D}^{\,2}\,\hat{p}^{\,4}.
\end{align}
Thus, Eq.~(\ref{elayer}) reduces to 
\begin{equation}
\frac{d^2\,Y_e}{d\hat{p}^{\,2}}- \frac{P_\perp}{(1+1/\tau)\,{\cal D}^{\,2}}\,\hat{p}^{\,2}\,Y_e\simeq =0.
\end{equation}
The characteristic layer width in $\hat{p}$ space is
\begin{equation}
\hat{p}_\ast = \left[\frac{{\cal D}^{\,2}\,(1+1/\tau)}{P_\perp}\right]^{1/4}.
\end{equation}
Thus, the characteristic layer width in $\hat{r}$ space is
\begin{equation}
\hat{\delta}_{\rm layer} = \frac{S^{-1/3}}{f^{-1/6}\,\hat{p}_\ast}= {\cal S}^{-1/3}\left[\frac{P_\perp}{{\cal D}^{\,2}\,(1+1/\tau)}\right]^{1/4}\,f^{\,1/2}
= n^{-1/2}\,{\mit\Delta}_{\rm layer}\,f^{\,1/2},
\end{equation}
where
\begin{align}
{\mit\Delta}_{\rm layer}&=\frac{\tau_A^{\,1/2}}{\tau_R^{\,1/4}\,\tau_\perp^{\,1/4}\,\hat{d}_\beta ^{\,1/2}}\\[0.5ex]
\hat{\tau}_{A} &= \frac{R_0}{B_0}\,\sqrt{\mu_0\,m_i\,n_{e\,100}},\\[0.5ex]
\hat{d}_\beta &=  \frac{\sqrt{(5/3)\,m_i\,(T_{e\,100}+T_{i\,100})}}{e\,B_0\,a}.
\end{align}

\section{Overlap Criterion}
The resistive layer width exceeds the spacing between rational surfaces when
\begin{equation}
\hat{\delta}_{\rm layer} > \hat{\delta}_{\rm rational},
\end{equation}
or
\begin{equation}
x<x_c
\end{equation}
where
\begin{equation}
\frac{x_c}{-\ln(2\,x_c)}= n\,(\alpha\,{\mit\Delta}_{\rm layer})^2.
\end{equation}
\fi

\funding{This research was funded by the  U.S.\ Department of Energy, Office of Science, Office of Fusion Energy Sciences under contract DE-FG02-04ER54742.}
\ack{} 

\data{The digital data used in the figures in this paper can be obtained from the author upon reasonable request.}

\begin{thebibliography}{99}\baselineskip 5ex

\bibitem{pom} N.~Pomphrey and A.~Reiman, Phys.\ Fluids B {\bf 4}, 938 (1992). 

\bibitem{boz} A.H.~Boozer, Rev. Mod.\ Phys.\ {\bf 76}, 1071 (2004).

\end{thebibliography}

\begin{figure}
\centerline{\includegraphics[width=\textwidth]{Figure1.pdf}}
\caption{The magnetic flux-surfaces ${\mit\Psi}=0.9$ (red), ${\mit\Psi}=1.0$ (green), and ${\mit\Psi}=1.1$ (blue), plotted in the $X$-$Y$
plane. The black dots shows the locations of the two current filaments. Here, $\beta=0.2$. }\label{fig1}
\end{figure}

\begin{figure}
\centerline{\includegraphics[width=\textwidth]{Figure2.pdf}}
\caption{The safety-factor $q({\mit\Psi})$ calculated for $q_\ast=12$ and $\beta=0.2$. }\label{fig2}
\end{figure}

\begin{figure}
\centerline{\includegraphics[width=\textwidth]{Figure3.pdf}}
\caption{The flux coordinate system inside the magnetic separatrix calculated for $q_\ast=12$ and $\beta=0.2$. The red curves
are surfaces of constant $\hat{\psi}$, whereas the blue curves are surfaces of constant $\theta$. The black dots shows the locations of the plasma current filaments. }\label{fig3}
\end{figure}

\begin{figure}
\centerline{\includegraphics[width=\textwidth]{Figure4.pdf}}
\caption{The flux coordinate system outside the magnetic separatrix calculated for $q_\ast=12$ and $\beta=0.2$. The red curves
are surfaces of constant $\hat{\psi}$, whereas the blue curves are surfaces of constant $\theta$. The black dots shows the locations of the two current filaments. }\label{fig4}
\end{figure}

\begin{figure}
\centerline{\includegraphics[width=\textwidth]{Figure5.pdf}}
\caption{The variation of the toroidal angle, $\phi$, with the geometric poloidal angle, $\omega$, on a magnetic field-line lying within a magnetic
flux-surface characterized by ${\mit\Psi}=0.999$, calculated with $q_\ast=12$ and $\beta=0.2$. The dashed line shows the mean gradient of the
field-line. }\label{fig5}
\end{figure}
\end{document}