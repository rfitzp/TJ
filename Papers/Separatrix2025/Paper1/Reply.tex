\documentclass{article}[12pt]
\usepackage{fullpage}
\usepackage{amsmath}

\begin{document}
\begin{center}
{\em Response of a magnetically diverted tokamak plasma to a resonant magnetic perturbation}\\[1ex]
by R.~Fitzpatrick\\[1ex]
{\bf Reply to Referees' Comments}\\[1ex]
~
\end{center}
Let me thank the  referees for their helpful and insightful comments on my paper. Here are my responses to
their comments.

\section*{Referee 1}
\begin{enumerate}
\item In Sect.~4.5, I have added a discussion of mode coupling. I have also given an argument that shows that
the criterion $|{\mit\Delta}_k|/(2\,m_k)\gg 1$ for an ideal response, and vice versa for a vacuum response, remains
approximately true even in the presence of mode coupling. 
\item In Sect.~5.9, I have added further discussion of the significance of the overlap of resistive layers close to the separatrix.
I have also tried to show how the overlap combined with the fact that $|{\mit\Delta}_k|/(2\,m_k)\sim {\cal O}(1)$ in the overlap region, lead
to the conclusion that the response of the region is vacuum-like. 

\item As to whether there is a casual relationship between the
overlap and the weak layer response, I think that there is, because both effects are directly related to the enormously large
magnetic shear close to the separatrix. If all of the plasma parameters stayed the same in the separatrix region, but
the shear were ${\cal O}(1)$, then there would be no overlap of the layers, and the shielding response of the layers to the RMP
would be very strong. 

\item I have adjusted the JET parameters to be consistent with discharge 84800 ($B_0=2.0$\,T and $a=0.9$\,m).

\item I have added a table (Table~1) that shows how the width $\epsilon_c$ varies with the effective charge number, $Z$. 
\end{enumerate}

\section*{Referee 2}
\begin{enumerate}
\item I have removed the specific values for $\epsilon_c$ from the Abstract.
\item I have removed the unnecessary  ``Introduction" sub-sections from Sects.~2, 3, 4, and 5.
\item In Sect.~2,1, I have changed ``the coordinate system is uniquely defined" to ``the straight-field-line coordinate system is uniquely defined".  
\item In Sect.~2.6, I have added further explanation for the derivation of Eq.~(13).
\item In Sect.~2.6, I have also added some discussion as to why I think that the result $\alpha_+>\alpha_-$ is generally true. 
\item In Sect.~2.8, I have added a reference to derivation of the crucial induction equation, (22). The derivation itself is  in Appendix~A. (It took me a whole day to remember the
derivation, so it clearly needs to appear explicitly in the paper.)
\item I have now stated that the term LCFS is  ``misleading", rather than ``inaccurate". 
\item I have replaced the term ``marginally-stable" by ``inertia-free". The response of the equilibrium to a static RMP is governed by inertia-free
ideal-MHD (except in the resistive layers) because the original equilibrium remains an equilibrium in the presence of the RMP. In other words,
a static RMP does not cause the plasma to move Alfv\'{e}nically. 
\item In Sect.~4.1, I have added more discussion as to the significance of the terms appearing in Eqs.~(48) and (49). I have also added
more discussion of the terms appearing in the normalized layer equations to Sect.~5.4.
\item I have added an additional figure (Fig.~9) that shows explicitly that the ordering $(n\,s)^2\gg |Q_e|$, $|Q_i|$, $|Q_E|$, $D$, $P_E$, $P_\varphi$ holds
very well close to the separatrix. 
\item I have changed the final comma to a full-stop in the final sentence. 

\end{enumerate}

\end{document}