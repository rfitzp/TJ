%\documentclass[12pt,prb,aps,notitlepage]{revtex4-1}
\documentclass{article}
\usepackage{fullpage}
\usepackage {amsmath}
\pdfoutput = 1 
\usepackage {graphicx}

\begin{document}

\title{\bf A Four-Field Resonant Response Model for Tokamak Plasmas}

\author{R.~Fitzpatrick\,\footnote{rfitzp@utexas.edu} \& Y.~Lin\\[2ex]{\em Institute for Fusion Studies,  Department of Physics,}\\[0.5ex]{\em   University of Texas at Austin,  Austin TX, 78712, USA}}
\date{}
\maketitle
%\begin{abstract}

%\end{abstract}

\section{Introduction}
Externally generated, static, {\em resonant magnetic perturbations}\/ (RMPs) can drive magnetic reconnection in tokamak plasmas that
are intrinsically stable to tearing perturbations \cite{sco,compass,fish,rf1993,rf1998}.
Driven reconnection leads to the formation of magnetic island chains at so-called {\em rational}\/ magnetic flux-surfaces \cite{boz}. Such chains locally flatten the pressure profile, and, thereby, degrade the plasma confinement \cite{chang}. 

The analysis  of the response of a tokamak plasma to an RMP  is most efficiently formulated as an asymptotic matching problem in which the  plasma is  divided into two distinct regions \cite{fkr}.    In the so-called {\em outer region}, which comprises most
of the plasma, the plasma perturbation is governed by the equations of linearized, marginally-stable, ideal-magnetohydrodynamics (MHD).
However, these equations become singular on    rational  magnetic flux-surfaces at which the perturbed magnetic field resonates with the equilibrium field. In the {\em inner region}, which
consists of a set of narrow layers centered on the various rational surfaces, non-ideal-MHD effects   become important.

It is well known that single-fluid resistive-MHD offers a very poor description of
the response of the inner region to the magnetic perturbation in the outer region. 
For instance, the strong  diamagnetic flows present in tokamak plasmas imply that the  electron and ion fluid velocities are significantly different from one another,
necessitating a two-fluid treatment \cite{ara}. Moreover, resistive-MHD does not take  into account the important  ion sound radius lengthscale below which electron and ion dynamics become decoupled from one another \cite{drake,wal}. Previously, Cole \& Fitzpatrick\,\cite{cole} used the four-field model
of Fitzpatrick \& Waelbroeck \cite{fw} (which is based on the original four-field model of Hazeltine, Kotschenreuther \& Morrison \cite{haz}) 
to determine the linear two-fluid response of a resonant layer to the perturbation in the outer region. This treatment was extended in Ref.~\cite{rf2022} to take into account the
 anomalously large perpendicular  energy diffusivity  present in tokamak plasmas. 
 
 In configuration space, the four-field models of Refs.~\cite{cole} and \cite{rf2022} yield a set of resonant layer equations that can be expressed as 
 {\em ten}\/ coupled
 first-order linear differential equations \cite{lee}. However, in Fourier space, the resonant layer equations can be written as 
 {\em four}\/ first-order linear differential equatons \cite{cole}. It is clearly advantageous to solve the equations in Fourier space. In Refs.~\cite{cole}
 and \cite{rf2022}, an approximation is made by which one of the terms in the Fourier-transformed layer equations is neglected. This
 approximation, which is valid in low-$\beta$ plasmas,  is such that the layer equation that governs the parallel ion dynamics decouples from the other three equations, effectively
 converting a four-field resonant response model into a three-field model. Furthermore, the three remaining layer equations can be combined to give a
 single second-order linear ordinary differential equation. This second-order equation is most conveniently solved numerically by means of a Riccati transformation
 that converts it into a first-order nonlinear differential equation \cite{ric1,ric2}. The advantage of the Riccati approach is that it can deal with numerically problematic  solutions
 that blow up as $\exp(p^2)$, or faster, at large $p$, where $p$ is the Fourier-space variable. 
 
 Lee, Park \& Na \cite{lee} recently demonstrated  how to solve the full tenth-order four-field resonant layer equations in configuration  space using a Riccati transformation. In the process, they discovered that, in a high-$\beta$ plasma,  the RMP frequency (as seen in the local ${\bf E}\times {\bf B}$ frame
 at the rational surface)  at which driven magnetic reconnection is maximized is shifted in the ion diamagnetic direction from the electron diamagnetic frequency.
 This result is significant because there is some experimental evidence for such a shift \cite{paz}. In the present paper, we demonstrate how the calculation of
 Lee et alia can be reimplemented in Fourier space. The Fourier version of the calculation is more convenient, from a numerical point of view, because
 it involves the solution of a fourth-order, rather than a tenth-order, system of equations. 

\section{Asymptotic Matching}\label{sect1}
\subsection{Plasma Equilibrium}
Consider a large aspect-ratio tokamak plasma equilibrium whose magnetic flux-surfaces map out
(almost) concentric circles in the poloidal plane. Such an equilibrium can be approximated as a
periodic cylinder \cite{rf1993}. Let $r$, $\theta$, $z$ be right-handed cylindrical coordinates. 
The magnetic axis corresponds to $r=0$, and the plasma boundary to $r=a$, where $a$ is the simulated minor radius of the plasma. The system is assumed to be periodic in the $z$-direction with periodicity length $2\pi\,R_0$, where $R_0$ is the simulated major radius of the plasma. The
safety-factor profile takes the form $q(r)=r\,B_z/[R_0\,B_\theta(r)]$, where $B_z$ is the constant
``toroidal'' magnetic field-strength, and $B_\theta(r)$ is the poloidal magnetic field-strength. The equilibrium poloidal and toroidal magnetic
fluxes (divided by $2\pi$) are written $\psi_p(r)=B_z\int_0^r r'/q(r')\,dr'$ and $\psi_t(r)=B_z\,r^2/2$, respectively.
The standard
large aspect-ratio orderings, $r/R_0\ll 1$ and $B_\theta/B_z\ll 1$, are adopted. 

\subsection{Outer Region}\label{cyl}
Consider a static (in the laboratory frame) magnetic perturbation that has $m$ periods in the poloidal direction, and
$n$ periods in the toroidal direction. 
The response of the plasma to the perturbation is
governed by the linearized equations of marginally-stable ideal-MHD everywhere in the plasma, apart from a (radially) narrow
layer centered on the rational surface whose minor radius, $r_s$,  is such that $q(r_s)=m/n$ \cite{fkr}.

The perturbed magnetic field associated with the tearing mode is written $\delta{\bf B} \simeq \nabla\delta\psi\times{\bf e}_z$,  
where
$\delta\psi(r,\theta,\varphi)= \delta\psi(r)\,\exp[\,{\rm i}\,(m\,\theta-n\,\varphi)]$, 
and $\varphi=z/R_0$ is a simulated toroidal angle. 
In the outer region  (i.e., everywhere in the plasma
apart from the resonant layer), the perturbed helical magnetic flux, $\delta\psi(r)$, satisfies the
{\em cylindrical tearing mode equation}\/  \cite{cyl,wes},
\begin{equation}\label{e3}
\frac{d^2 \delta\psi}{d r^2} + \frac{1}{r}\,\frac{d \delta\psi}{dr}-\frac{m^2}{r^2}\,\delta\psi - \frac{J_z'\,\delta\psi}{r\,(1/q-n/m)}=  0,
\end{equation}
where 
$J_z(r)= R_0\,\mu_0\,j_z(r)/B_z$,
and $j_z(r)$ is the equilibrium ``toroidal'' current density. Here, $'\equiv d/dr$. 

In general, the solution of Eq.~(\ref{e3}) that satisfies physical
boundary conditions at the magnetic axis and the plasma boundary is such that $\delta\psi(r)$ is continuous
across the rational surface, whereas $d\delta\psi/dr$ is discontinuous. The discontinuity of
 $d\delta\psi/dr$ across the rational surface is indicative of the presence of a helical current
 sheet at the surface.  The complex quantity ${\mit\Psi}_s= \delta\psi(r_s)$ determines the amplitude
 and phase of the reconnected helical magnetic flux at the rational surface, whereas the complex quantity 
 \begin{equation}
{\mit\Delta\Psi}_{\!s} = \left[r\,\frac{d\delta\psi}{dr}\right]_{r_{s-}}^{r_{s+}}
\end{equation}
parameterizes the amplitude and phase of the helical current sheet \cite{rf1993}. 

The solution of
the cylindrical tearing mode equation in the outer region, in the presence of an externally generated RMP (with $m$ periods in the poloidal
direction and $n$ periods in the toroidal direction), leads to the {\rm tearing mode dispersion relation}\/ \cite{rf1993,fkr}
\begin{equation}\label{e110}
{\mit\Delta\Psi}_{\!s}- E_{ss}\,{\mit\Psi}_s = -E_{ss}\,{\mit\Psi}_v,
\end{equation}
where $E_{ss}$ is a real dimensionless quantity known as the {\em tearing stability index}\/ \cite{fkr}. Moreover,
${\mit\Psi}_v$ is the so-called {\em vacuum flux}, and is defined as the reconnected magnetic flux
that would be driven at the rational surface by the RMP were the
plasma intrinsically tearing-stable (i.e., $E_{ss} <0$), and were there no current sheet at the rational surface
(i.e., ${\mit\Delta\Psi}_{\!s}=0$) \cite{rf1993}.

\subsection{Inner Region}
The current sheet at the rational surface is resolved by employing a extended-MHD resonant response model
in the  inner region (i.e., the region of the plasma in the immediate vicinity of the rational surface) to determine the
complex {\em layer response index}, ${\mit\Delta}_s$. Asymptotic matching between the solutions in the inner and the outer regions
yields
\begin{equation}
{\mit\Delta}_s= \frac{{\mit\Delta\Psi}_{\!s}}{{\mit\Psi}_s}.
\end{equation}
The previous two equations lead to the {\em plasma response equation}, 
\begin{equation}
{\mit\Psi}_s = \frac{{\mit\Psi}_v}{1+{\mit\Delta}_s/(-E_{ss})}.
\end{equation}

Note that if $|{\mit\Delta}_s|/(-E_{ss})\ll 1$ then ${\mit\Psi}_s\simeq {\mit\Psi}_v$ and ${\mit\Delta\Psi}_{\!s}\simeq 0$, 
which is termed a {\em vacuum response}. On the other hand, if $|{\mit\Delta}_s|/(-E_{ss})\gg 1$ then $|{\mit\Psi}_s|\ll |{\mit\Psi}_v|$,
which is termed an {\em ideal response}. In the vacuum response regime, the current sheet that is excited in the resonant layer is too feeble
to prevent driven magnetic reconnection, and the reconnected flux at the rational surface is the same as that which would
be driven if there were no plasma present at the surface. On the other hand, in the ideal response regime, the  current sheet excited in the layer is
large enough to almost completely suppress driven magnetic reconnection, which implies that the response of the plasma is equivalent
to that which would occur if the ideal-MHD flux freezing constraint, ${\mit\Psi}_s=0$, were imposed at the rational surface \cite{rf2025}. 

\section{Four-Field Resonant Response Model}
\subsection{Description}
Our four-field resonant response model is described in detail in Appendix~\ref{sfour}. The model is defined by the following eight dimensionless parameters:
$S=\tau_R/\tau_H$, $Q_E=-S^{\,1/3}\,n\,\omega_E\,\tau_H$, $Q_e=-S^{\,1/3}\,n\,\omega_{\ast\,e}\,\tau_H$, $Q_i=
-S^{\,1/3}\,n\,\omega_{\ast\,i}\,\tau_H$, $c_\beta=\sqrt{\beta/(1+\beta})$, $D=S^{\,1/3}\,\iota_e^{\,1/2}\,\hat{d}_\beta$, $P_E=\tau_R/\tau_E$,  and $P_\varphi=\tau_R/\tau_\varphi$. 
Here, $\tau_R$ is the resistive diffusion timescale, $\tau_H$ the hydrodynamic timescale, $\tau_E$ the energy confinement timescale,
$\tau_\varphi$ the momentum confinement timescale, $\omega_E$ the ${\bf E}\times {\bf B}$ frequency, $\omega_{\ast\,e}$ the
electron diamagnetic frequency, $\omega_{\ast\,i}$ the ion diamagnetic frequency, $\beta=(5/3)\,\mu_0\,p/B_z^{\,2}$ the square of the ratio of the sound speed
to the Alfv\'{e}n speed, $p$ the plasma pressure, $d_\beta=c_\beta\,d_i$ the ion sound radius, and $d_i$ the collisionless ion skin-depth.  Moreover,
$\iota_e = - \omega_{\ast\,e}/(\omega_{\ast\,i}-\omega_{\ast\,e})$, and $\hat{d}_\beta= d_\beta/r_s$. All quantities are evaluated at the
rational surface. 

After linearization and Fourier transformation (with respect to $x=r-r_s$), our four-field model reduces to the $2\times 2$ Ricatti matrix
differential equation specified in Eq.~(\ref{ricc}). Here, the elements of the $2\times 2$ matrices $\underline{\underline{E}}$ and
$\underline{\underline{F}}$ are given in Eqs.~(\ref{ee1})--(\ref{ee4}) and (\ref{f1})--(\ref{f4}). The method of solution is as follows. A solution of the Ricatti equation
is launched from a large value of the Fourier transform variable $p$, with the initial conditions specified in Eq.~(\ref{e147}), and
integrated to a small value of $p$. The layer response index, ${\mit\Delta}_s=S^{\,1/3}\,\hat{\mit\Delta}$, is then deduced from Eqs.~(\ref{e48}) and (\ref{e111}). 

\section*{Acknowledgements}
The authors would like to thank   Y.~Lee, J.-K.~Park, J.~Waybright, and D.~Burgess for help benchmarking the results of this paper against those of
Ref.~\cite{lee}. 

\section*{Funding}
This research was supported by the U.S.\ Department of Energy, Office of Science, Office of Fusion Energy Sciences,  under  contract DE-FG02-04ER54742. 

\section*{Data Availability Statement}
The data that support the findings of this study are available from the corresponding author upon reasonable request.

%\section*{References}
\begin{thebibliography}{99}\baselineskip 5ex

\bibitem{sco} J.T.~Scoville, R.J.~La\,Haye, A.G.~Kellman, T.H.~Osborne, R.D.~Stambaugh, E.J.~Strait and T.S.~Taylor, Nucl.\ Fusion {\bf 31}, 875 (1991).

\bibitem{compass} T.C.~Hender, R.~Fitzpatrick, A.W.~Morris,  P.G.~Carolan, R.D.~Durst,
T.~Edlington, J.~Ferreira, S.J.~Fielding, P.S.~Haynes, J.~Hugill, et al., Nucl.\ Fusion {\bf 32}, 2091 (1992).

\bibitem{fish} G.M.~Fishpool and P.S.~Haynes, Nucl.\ Fusion {\bf 34}, 109 (1994).

\bibitem{rf1993} R.~Fitzpatrick, Nucl.\ Fusion {\bf 33}, 1049 (1993).

\bibitem{rf1998} R.~Fitzpatrick, Phys.\ Plasmas {\bf 5}, 3325 (1998).

\bibitem{boz} A.H.~Boozer, Rev.\ Mod.\ Phys.\ {\bf 76}, 1071 (2004).

\bibitem{chang}  Z.~Chang and J.D.~Callen, Nucl.\ Fusion {\bf 30}, 219 (1990).

\bibitem{fkr} H.P.~Furth, J.~Killeen and M.N.~Rosenbluth, Phys.\ Fluids {\bf 6}, 459 (1963).

\bibitem{ara} G.~Ara, B.~Basu, B.~Coppi, G.~Laval, M.N.~Rosenbluth and B.V.~Waddell, Ann.\ Phys.\ (NY) {\bf 112}, 443 (1978).

\bibitem{drake} J.F.~Drake and Y.C.~Lee, Phys.\ Fluids {\bf 20}, 1341 (1977).

\bibitem{wal} F.L.~Waelbroeck, Phys.\ Plasmas {\bf 10}, 4040 (2003).

\bibitem{cole} A.~Cole and R.~Fitzpatrick, Phys.\ Plasmas {\bf 13}, 032503 (2006).

\bibitem{fw} R.~Fitzpatrick and F.L.~Waelbroeck, Phys. Plasmas {\bf 12}, 022307 (2005).

\bibitem{haz} R.D.~Hazeltine, M.~Kotschenreuther and P.G.~Morrison, Phys.\ Fluids {\bf 28}, 2466 (1985).

\bibitem{rf2022} R. Fitzpatrick,  Phys.\ Plasmas {\bf 29}, 032507 (2022).

\bibitem{lee} Y.~Lee, J.-K.~Park and Y.-S.~Na, Nucl.\ Fusion {\bf 64}, 106058 (2024).

\bibitem{ric1} D.P.~Brennan, A.J.~Cole, C.~Akcay and J.M.~Finn, Bul.\ Am.\ Phys.\ Soc.\ {\bf 64}, 249 (2019).

\bibitem{ric2} J.-K.~Park, Phys.\ Plasmas {\bf 29}, 072506 (2022).

\bibitem{paz} C.~Paz-Soldan, R.~Nazikian, L.~Cui, B.C.~Lyons, D.M.~Orlov, A.~Kirk, 
N.C.~Logan, T.H.~Osborne, W.~Suttrop and D.B.~Weisberg, Nucl.\ Fusion {\bf 59},
056012 (2019).

\bibitem{cyl} H.P.~Furth, P.H.~Rutherford and H.~Selberg, Phys.\  Fluids {\bf 16}, 1054 (1973). 

\bibitem{wes} J.A.~Wesson, Nucl.\ Fusion {\bf 18}, 87 (1978).
\bibitem{rf2025} R.~Fitzpatrick, {\em Response of a magnetically diverted tokamak plasma to a resonant magnetic perturbation}, 
arXiv 2511.07666 (2025).

\bibitem{haz1} R.D.~Hazeltine and J.D.~Meiss, {\em Plasma Confinement}, (Dover, New York NY, 2003).

\end{thebibliography}

\appendix
\section{Four-Field Resonant Plasma Response Model}\label{sfour}
\subsection{Fundamental Definitions}
The plasma is assumed to consist of two species. First, electrons of mass $m_e$, electrical charge $-e$, 
number density $n_e$, and temperature $T_e$.  Second, ions of mass $m_i$, electrical charge $+e$,  
number density $n_e$, and temperature $T_i$. Let $p=n_e\,(T_e+T_i)$ be the total plasma pressure. 

Let $r_s$ be the minor radius of the rational surface. It is helpful to define $n_0 = n_e(r_s)$, $p_0= p(r_s)$,
\begin{align}
\eta_e &=\left.\frac{d\ln T_e}{d\ln n_e}\right|_{r=r_s},\label{e211}\\[0.5ex]
\eta_i &= \left.\frac{d\ln T_i}{d\ln n_e}\right|_{r=r_s},\\[0.5ex]
\iota&= \left(\frac{T_e}{T_i}\right)_{r=r_s}\left(\frac{1+\eta_e}{1+\eta_i}\right)=\left(\frac{dp_e}{dp_i}\right)_{r_s},\label{e213}
\end{align}
where $n_e(r)$, $p(r)$, $p_e(r)$, $p_i(r)$, $T_e(r)$, and $T_i(r)$ refer to
electron number density, total pressure, electron pressure, ion pressure, electron temperature, and ion temperature profiles, respectively, in the absence of the magnetic perturbation. 

For the sake of simplicity, the perturbed electron and ion temperature profiles are assumed to be functions of
the perturbed electron number density profile in the immediate vicinity of the rational surface. In other words, $T_e=T_e(n_e)$ and $T_i=T_i(n_e)$. This
implies that $p=p(n_e)$. 
The ``MHD velocity'', which is the velocity of a
fictional MHD fluid \cite{haz1}, is defined ${\bf V}={\bf V}_E + V_{\parallel\,i}\,{\bf b}$, where ${\bf V}_E$ is the
${\bf E}\times{\bf B}$ drift velocity, $V_{\parallel\,i}$ is the parallel component of the ion fluid
velocity, ${\bf b}= {\bf B}/|{\bf B}|$, and ${\bf B}$ is the magnetic field-strength.

\subsection{Fundamental Fields}
The four fundamental fields in our four-field model---namely, $\psi$, $N$, $\phi$, and $V$---have the following
definitions:
\begin{align}\label{e10}
\nabla\psi &= \frac{{\bf e}_\parallel\times{\bf B}} {r_s\,B_z},\\[0.5ex]
N &=-\hat{d}_i\left(\frac{p-p_0}{B_z^{\,2}/\mu_0}\right),\\[0.5ex]
\nabla\phi &= \frac{{\bf e}_{\parallel}\times {\bf V}}{r_s\,V_A},\\[0.5ex]
V &= \hat{d}_i\left(\frac{{\bf e}_{\parallel}\cdot {\bf V}}{V_A}\right).\label{e13}
\end{align}
Here,   ${\bf e}_{\parallel} = (0,\,\epsilon/q_s,\,1)$, $\epsilon = r/R_0$, $q_s=m/n$, 
$V_A =(B_z/\sqrt{\mu_0\,n_0\,m_i})_{r_s}$, 
$d_i =[m_i/(n_0\,e^{\,2}\,\mu_0)]^{1/2}_{r_s}$,
and $\hat{d}_i=d_i/r_s$. 
 Our
model also employs the auxiliary field
\begin{align}\label{e16}
J=-\frac{2\,\epsilon_s}{q_s}+\hat{\nabla}^2\psi,
\end{align}
where 
$\epsilon_s=r_s/R_0$, and $\hat{\nabla} = r_s\,\nabla$. Note that $V_A$ is the {\em Alfv\'{e}n speed}\/ at the rational surface, whereas $d_i$ is the {\em collisionless ion skin-depth}. 

\subsection{Fundamental Equations}
For the case of a static (in the laboratory frame) magnetic perturbation, the four-field model takes the form \cite{fw,cole,rf2022}:
\begin{align}\label{e12v}
0&= [\phi,\psi] -\iota_e\,[N,\psi]
+\hat{\eta}_\parallel\,J + \hat{E}_\parallel,\\[0.5ex]
0&= [\phi,N] +\hat{d}_\beta^{\,2}\,[J,\psi]+c_\beta^{\,2}\,[V,\psi] 
+ \hat{\chi}_E\,\hat{\nabla}^{2}N,\\[0.5ex]
0&= [\phi,\hat{\nabla}^2\phi] - \frac{\iota_i}{2}\left(\hat{\nabla}^2[\phi,N] + [\hat{\nabla}^2\phi,N] + [\hat{\nabla}^2 N,\phi]\right) + [J,\psi] +\hat{\chi}_\varphi  \,\hat{\nabla}^4\!\left(\phi + \iota_i\,N\right), \\[0.5ex]
0&= [\phi,V] +[N,\psi] + \hat{\chi}_\varphi\,\hat{\nabla}^2 V.\label{e21}
\end{align}
Here, $[A,B]\equiv \hat{\nabla} A\times \hat{\nabla} B\cdot {\bf e}_{\parallel}$, $\iota_e=\iota/(1+\iota)$, $\iota_i=1/(1+\iota)$, $\hat{t} = t/(r_s/V_A)$, $\hat{\eta}_{\parallel} = \eta_{\parallel}/(\mu_0\,r_s\,V_A)$, $\hat{E}_\parallel = E_\parallel/(B_z\,V_A)$, 
$\hat{\chi}_E= \chi_E/(r_s\,V_A)$, $\hat{\chi}_\varphi= \chi_\varphi/(r_s\,V_A)$, where $\eta_{\parallel}$ is the parallel  plasma electrical
resistivity at the rational surface, $E_\parallel$ the parallel inductive electric field that maintains the equilibrium toroidal
plasma current in the vicinity of the rational surface, $\chi_E$ the anomalous perpendicular heat diffusivity at the
rational surface, and $\chi_\varphi$  the anomalous perpendicular ion momentum
diffusivity at the rational surface. 
Moreover, $d_\beta=c_\beta\,d_i$, and $\hat{d}_\beta=d_\beta/r_s$, where $c_\beta =[\beta/(1+\beta)]^{1/2}_{r_s}$, and
$\beta=(5/3)\,\mu_0\,p_0/B_z^{\,2}$. Here, $d_\beta$ is usually referred to as the {\em ion sound radius}.  

\subsection{Matching to Plasma Equilibrium}
The unperturbed plasma equilibrium is such that
${\bf B} = (0,\,B_\theta(r),\,B_z)$,  $p = p(r)$,
${\bf V} = (0,\,V_E(r),\,V_z(r))$,
where 
$V_E(r)\simeq -E_r/B_z$
 is the (dominant $\theta$-component of the) ${\bf E}\times {\bf B}$ velocity. Now, the resonant layer is assumed to have a radial thickness that is
much smaller than $r_s$.   Hence, we only need to evaluate plasma equilibrium quantities in the immediate vicinity of the rational
surface. Equations~(\ref{e10})--(\ref{e13}) suggest that 
\begin{align}
\psi(\hat{x})&= \frac{\hat{x}^{\,2}}{2\,\hat{L}_s},\label{e23}\\[0.5ex]
N(\hat{x}) &= -\hat{V}_\ast\,\hat{x},\\[0.5ex]
\phi(\hat{x}) &= - \hat{V}_E\,\hat{x},\\[0.5ex]
V (\hat{x})&= \hat{V}_\parallel,\label{e26}
\end{align}
where 
$\hat{x}=(r-r_s)/r_s$,
 $\hat{L}_s=L_s/r_s$,  $L_s=R_0\,q_s/s_s$, 
  $\hat{V}_E= V_E(r_s)/V_A$,
$\hat{V}_\ast= V_\ast(r_s)/V_A$,
$V_\ast(r) = (dp/dr)/(e\,n_0\,B_z)$ 
is the (dominant $\theta$-component of the) diamagnetic velocity,
  and 
 $\hat{V}_\parallel=\hat{d}_i\, V_z(r_s)/V_A$. Here, $s_s=s(r_s)$ and $s(r)=d\ln q/d\ln r$. We also have
 \begin{align}\label{e28}
 J (\hat{x})&= -\left(\frac{2}{s_s}-1\right)\frac{1}{\hat{L}_s},
\end{align} 
and $ \hat{E}_\parallel(\hat{x}) =(2/s_s-1)\, (\hat{\eta}_\parallel/\hat{L}_s)$.

\subsection{Derivation of Linear Layer Equations}
In accordance with Eqs.~(\ref{e23})--(\ref{e28}), let us write
\begin{align}\label{e31}
\psi(\hat{x},\zeta) &= \frac{\hat{x}^{\,2} }{2\,\hat{L}_s}+ \tilde{\psi}(\hat{x})\,{\rm e}^{\,{\rm i}\,\zeta},\\[0.5ex]
\phi(\hat{x},\zeta) &=-\hat{V}_E\,\hat{x}+ \tilde{\phi}(\hat{x})\,{\rm e}^{\,{\rm i}\,\zeta},\\[0.5ex]
N(\hat{x},\zeta) &= -\hat{V}_\ast\,\hat{x}+\iota_e\,\tilde{N}(\hat{x})\,{\rm e}^{\,{\rm i}\,\zeta},\\[0.5ex]
V(\hat{x},\zeta) &= \hat{V}_\parallel +\iota_e\,\tilde{V}(\hat{x})\,{\rm e}^{\,{\rm i}\,\zeta},\\[0.5ex]
J(\hat{x},\zeta) &=-\left(\frac{2}{s_s}-1\right)\!\frac{1}{\hat{L}_s}+ \hat{\nabla}^2\tilde{\psi}(\hat{x})\,{\rm e}^{\,{\rm i}\,\zeta},\label{e36}
\end{align}
where $\zeta=m\,\theta-n\,\varphi$. 
Substituting Eqs.~(\ref{e31})--(\ref{e36}) into Eqs.~(\ref{e16})--(\ref{e21}), 
 and only retaining terms that
are first order in perturbed quantities, we obtain the following set of linear equations:
\begin{align}\label{e37}
-{\rm i}\,n\,(\omega_E +\omega_{\ast\,e})\,\tau_H\,\tilde{\psi} &= -{\rm i} \,\hat{x}\,(\tilde{\phi}-\tilde{N})+ S^{-1}\,\hat{\nabla}^2\tilde{\psi},\\[0.5ex]
-{\rm i}\, n\,\omega_E\,\tau_H\,\tilde{N} &= {\rm i}\,n\,\omega_{\ast\,e}\,\tau_H\,\tilde{\phi}- {\rm i}\,\iota_e\,\hat{d}_\beta^{\,2}\,\hat{x}\,\hat{\nabla}^2\tilde{\psi}-{\rm i}\,c_\beta^{\,2}\,\hat{x}\,\tilde{V}  + S^{-1}\,P_E\,\hat{\nabla}^2 \tilde{N},\\[0.5ex]
-{\rm i}\,n\,(\omega_E+\omega_{\ast\,i})\,\tau_H\,\hat{\nabla}^2\tilde{\phi} &= -{\rm i}\,\hat{x}\,\hat{\nabla}^2\tilde{\psi}+S^{-1}\,P_\varphi\,\hat{\nabla}^4\!\left(\tilde{\phi} + \frac{\tilde{N}}{\iota}\right),\\[0.5ex]
-{\rm i}\,n\,\omega_E\,\tau_H\,\tilde{V} &= -{\rm i}\,n\,\omega_{\ast\,e}\,\tau_H\,\tilde{\psi} - {\rm i}\,\hat{x}\,\tilde{N}
+ S^{-1}\,P_\varphi\,\hat{\nabla}^2\tilde{V}.\label{e40}
\end{align}
Here, 
$\tau_H = L_s/(m\,V_A)$ 
is the  hydromagnetic time, 
$\omega_E =-(q_s/r_s)\,V_E(r_s) = -(d{\mit\Phi}/d\psi_p)_{r_s}$
  the 
 ${\bf E}\times {\bf B}$ frequency, 
$\omega_{\ast\,e} = \iota_e\,(q_s/r_s)V_\ast(r_s)= [(dp_e/d\psi_p)/(e\,n_e)]_{r_s}$
the electron diamagnetic frequency,
$\omega_{\ast\,i} =-\iota_i\,(m/r_s)V_\ast(r_s)=-[(dp_i/d\psi_p)/(e\,n_e)]_{r_s}$
 the  ion diamagnetic frequency,  ${\mit\Phi}(r)$ the equilibrium electrostatic potential, $S=\tau_R/\tau_H$ the  Lundquist number, 
$\tau_R = \mu_0\,r_s^{\,2}/\eta_\parallel$
 the
 resistive diffusion time,  $\tau_E = r_s^{\,2}/\chi_E$ the energy confinement time,
and $\tau_\varphi
= r_s^{\,2}/\chi_\varphi$
the  toroidal momentum confinement time.
Furthermore, $P_E = \tau_R/\tau_E$ and $P_\varphi = \tau_R/\tau_\varphi$ are magnetic Prandtl numbers.

 Let us define the stretched radial variable $X = S^{\,1/3}\,\hat{x}$.
Assuming that $X\sim{\cal O}(1)$ in the layer (i.e., assuming that the layer thickness is roughly of order $S^{-1/3}\,r_s$),
and making use of the fact that $S\gg 1$ in conventional tokamak plasmas,  Eqs.~(\ref{e37})--(\ref{e40}) reduce to the following
set of linear layer equations \cite{cole,rf2022}:
\begin{align}\label{e43xx}
{\rm i}\,(Q_E+ Q_e)\,\tilde{\psi}&= - {\rm i}\,X\left(\tilde{\phi}-\tilde{N}\right)+ \frac{d^{\,2}\tilde{\psi}}{d X^2},\\[0.5ex]
{\rm i}\,Q_E\,\tilde{N} &=  -{\rm i}\,Q_{e}\,\tilde{\phi}   - {\rm i} \,D^{2}\,X\,\frac{d^{\,2}\tilde{\psi}}{dX^{2}}- {\rm i}\,c_\beta^{\,2}\,X\,\tilde{V}
+ P_E\,\frac{d^{\,2} \tilde{N}}{dX^{2}},\label{e44xx}\\[0.5ex]
{\rm i}\,(Q_E+Q_i)\,\frac{d^{\,2}\tilde{\phi}}{dX^2}&= - {\rm i}\,X\,\frac{d^{\,2}\tilde{\psi}}{dX^2}+ P_\varphi\,\frac{d^{\,4}}{dX^4}\!\left(\tilde{\phi} + \frac{\tilde{N}}{\iota}\right),\\[0.5ex]
{\rm i}\,Q_E\,\tilde{V} &= {\rm i}\,Q_{e}\,\tilde{\psi} - {\rm i}\,X\,\tilde{N} + P_\varphi\,\frac{d^{\,2}\tilde{V}}{dX^{2}}.\label{e46}
\end{align}
Here, $Q_E=-S^{1/3}\,n\,\omega_E\,\tau_H$, $Q_{e,i} = -S^{\,1/3}\,n\,\omega_{\ast\,e,i}\,\tau_H$,
and $D = S^{\,1/3}\,\iota_e^{1/2}\,\hat{d}_\beta$. If we write $P_E = c_\beta^{\,2}$
then Eqs.~(\ref{e43xx})--(\ref{e46}) become equivalent to the set of  layer equations solved by Lee et alia. (To be slightly more exact, we
get from our equations to those of Lee et alia by making the following transformation: $Q_E\rightarrow Q$, $Q_e\rightarrow -Q_{\ast\,e}$, $Q_i\rightarrow -Q_{\ast\,i}$, $\iota\rightarrow 1/\tau$, $P_E\rightarrow c_\beta^{\,2}$,
 $P_\varphi\rightarrow P$, $\tilde{\psi}\rightarrow-\tilde{\psi}$, $\tilde{N}\rightarrow \tilde{Z}$, $\tilde{\phi}\rightarrow\tilde{\phi}$, and
$\tilde{V}\rightarrow-\tilde{V}_z$.)

The previously mentioned low-$\beta$ approximation used in Refs.~\cite{cole} and \cite{rf2022} involves neglecting the term
containing $c_\beta^{\,2}$ in Eq.~(\ref{e44xx}). This approximation decouples Eq.~(\ref{e46}) from the three preceding equations, and effectively
converts a four-field resonant response model into a three-field model. In the following, we shall not use this approximation.

\subsection{Asymptotic Matching}
The  linear layer equations, (\ref{e43xx})--(\ref{e46}), possess tearing parity solutions
characterized by the symmetry $\tilde{\psi}(-X)=\tilde\psi(X)$, 
$\tilde{N}(-X)= - \tilde{N}(X)$, $\tilde{\phi}(-X)=-\tilde{\phi}(X)$,  $\tilde{V}(-X)=\tilde{V}(X)$.
As is easily demonstrated, the asymptotic behavior of the tearing parity solutions to Eqs.~(\ref{e43xx})--(\ref{e46}) are such that 
\begin{align}\label{e311}
\tilde{\psi}(X)&\rightarrow  \psi_0\left[\frac{\skew{6}\hat{\mit\Delta}}{2}\,|X| +1+ {\cal O}\left(\frac{1}{X^2}\right)\right],\\[0.5ex]
\skew{3}\tilde{\phi}(X)&\rightarrow -\psi_0\,Q_E\,\left[\frac{\skew{6}\hat{\mit\Delta}}{2}\,{\rm sgn}(X) +\frac{1}{X}+ {\cal O}\left(\frac{1}{X^2}\right)\right],\label{e312a}\\[0.5ex]
\tilde{N}(X) & \rightarrow \psi_0\,Q_{e}\left[\frac{\skew{6}\hat{\mit\Delta}}{2}\,{\rm sgn}(X) +\frac{1}{X}+ {\cal O}\left(\frac{1}{X^2}\right)\right],\label{e3.13a}\\[0.5ex]
\tilde{V}(X)&\rightarrow {\cal O}\left(\frac{1}{X^3}\right)\label{e3.14a}
\end{align}
as $|X|\rightarrow\infty$, where $\psi_0$ is an arbitrary constant
The layer response index is
\begin{equation}\label{e48}
{\mit\Delta}_s= S^{1/3}\,\hat{\mit\Delta}.
\end{equation}

\subsection{Fourier Transformation}
Equations~(\ref{e43xx})--(\ref{e46}) are most conveniently solved in Fourier transform space \cite{cole}.
Let
\begin{equation}
\bar{\phi}(p) = \int_{-\infty}^\infty \tilde{\phi}(X)\,{\rm e}^{-{\rm i}\,p\,X}\,dX,
\end{equation}
et cetera. The Fourier transformed linear layer equations become
\begin{align}\label{e314}
{\rm i}\,(Q_E+Q_e)\,\bar{\psi}&=\frac{d}{dp}\!\left(\bar{\phi}-\bar{N}\right)-p^2\,\bar{\psi},\\[0.5ex]
{\rm i}\,Q_E\,\bar{N} &= - {\rm i}\,Q_{e}\,\bar{\phi} -D^{\,2}\,\frac{d(p^2\,\bar{\psi})}{dp}+ c_\beta^{\,2}\,\frac{d\bar{V}}{dp}
  - P_E\,p^2\bar{N},\label{e316}\\[0.5ex]
{\rm i}\,(Q_E+Q_i)\,p^2\,\bar{\phi}&=  \frac{d(p^2\,\bar{\psi})}{dp}- P_\varphi\,p^4\!\left(\bar{\phi} + \frac{\bar{N}}{\iota}\right),\\[0.5ex]
{\rm i}\,Q_E\,\bar{V} &= {\rm i}\,Q_{e}\,\bar{\psi} +\frac{d\bar{N}}{dp}- P_\varphi\,p^2\,\bar{V},\label{e317}
\end{align}
where, for a tearing parity solution, 
\begin{equation}\label{e41}
\bar{\phi}(p)-\bar{N}(p)\equiv\bar{Y} (p)\rightarrow \bar{Y}_0\!\left[\frac{\hat{\mit\Delta}}{\pi\,p} + 1+ {\cal O}(p)\right]
\end{equation}
as $p\rightarrow 0$, where $Y_0$ is an arbitrary constant. 

Finally, if we define 
\begin{align}
\bar{J}(p)&= p^2\,\bar{\psi},
\end{align}
then Eqs.~(\ref{e314})--(\ref{e317}) can be converted into the following equivalent set of four coupled first-order differential equations:
\begin{align}\label{e1}
\frac{d\bar{Y}}{dp} &= \left[\frac{\,{\rm i}\,(Q_E+Q_e)+p^2}{p^2}\right]\bar{J},\\[0.5ex]
\frac{d\bar{N}}{dp} &= -\left(\frac{{\rm i}\,Q_e}{p^2}\right)\bar{J} + (\,{\rm i}\,Q_E + P_\varphi\,p^2)\,\bar{V},\\[0.5ex]
\frac{d\bar{J}}{dp} &= [\,{\rm i}\,(Q_E+Q_i)\,p^2 + P_\varphi\,p^4]\,\bar{Y}
+ [\,{\rm i}\,(Q_E+Q_i)\,p^2 + \iota_e^{\,-1}\,P_\varphi\,p^4]\,\bar{N},\\[0.5ex]
c_\beta^{\,2}\,\frac{d\bar{V}}{dp} &= [\,{\rm i}\,Q_e+ {\rm i}\,(Q_E+Q_i)\,D^{\,2}\,p^2 + D^{\,2}\,P_\varphi\,p^4]\,\bar{Y}
\nonumber\\[0.5ex]&\phantom{=} +\{\,{\rm i}\,(Q_E+Q_e) + [P_E+{\rm i}\,(Q_E + Q_i) \,D^{\,2}]\,p^2 + \iota_e^{\,-1}\,D^{\,2}\,P_\varphi\,p^4\}\,\bar{N},\label{e4}
\end{align}
Note that $\iota_e= -Q_e/(Q_i-Q_e)$. 

\subsection{Small Argument Expansion}
Let us search for power-law solutions of Eqs.~(\ref{e1})--(\ref{e4}) at small values of $p$. Given that we have four coupled first-order differential equations,
we expect to find four independent power-law solutions.
The first solution is
such that
\begin{align}
\bar{Y}(p) &= {\rm i}\,(Q_E+Q_e)\,a_{-1}\,p^{-1} -\left[\frac{{\rm i}}{2}\,Q_E\,(Q_E+Q_e)\,(Q_E+Q_i) +1\right]a_{-1}\,p +{\cal O}(p^3),\\[0.5ex]
\bar{N}(p)&= -{\rm i}\,Q_e\,a_{-1}\, p^{-1} + \frac{{\rm i}}{2}\,Q_E\,Q_e\,(Q_E+Q_i)\,a_{-1}\,p + {\cal O}(p^3),\\[0.5ex]
\bar{J}(p)&= -a_{-1}- \frac{1}{2}\,Q_E\,(Q_E+Q_i)\,a_{-1}\,p^2 +{\cal O}(p^4),\\[0.5ex]
\bar{V}(p)&=- \frac{[\,{\rm i}\,Q_e\,(1+P_E) +Q_E\,(Q_E+Q_i)\,D^{\,2}]}{2\,c_\beta^{\,2}}\,a_{-1}\,p^2+{\cal O}(p^4),
\end{align}
where $a_{-1}$ is an arbitrary constant. 
The second solution is such that
\begin{align}
\bar{Y}(p) &= {\rm i}\,(Q_E+Q_e)\,a_0-\frac{{\rm i}}{6}\,Q_E\,(Q_E+Q_e)\,(Q_E+Q_i)\,a_0\,p^2 +{ \cal O}(p^4),\\[0.5ex]
\bar{N}(p) &= -{\rm i}\,Q_e\,a_0+\frac{{\rm i}}{6}\,Q_E\,Q_e\,(Q_E+Q_i)\,a_0\,p^2+{\cal O}(p^4),\\[0.5ex]
\bar{J}(p)&= -\frac{1}{3}\,Q_E\,(Q_E+Q_i)\,a_0\,p^3+{\cal O}(p^5),\\[0.5ex]
\bar{V}(p)&= -\frac{1}{3}\,\frac{[\,{\rm i}\,Q_e\,P_E + Q_E\,(Q_E+Q_i)\,D^{\,2}]}{c_\beta^{\,2}}\,a_0\,p^3+{\cal O}(p^5),
\end{align}
where $a_0$ is an arbitrary constant. 
The third solution is such that
\begin{align}
\bar{Y}(p) &= -\frac{1}{6}\,(Q_E+Q_e)\,(Q_E+Q_i)\,a_2\,p^2+{\cal O}(p^4),\\[0.5ex]
\bar{N}(p) &= a_2+\frac{{\rm i}}{2}\,(Q_E+Q_i)\left(-\frac{{\rm i}}{3}\,Q_e\,+ \frac{g}{c_\beta^{\,2}}\right)a_2\,p^2+{\cal O}(p^4),\\[0.5ex]
\bar{J}(p) &=\frac{{\rm i}}{3}\,(Q_E+Q_i)\,a_2\,p^3+{\cal O}(p^5),\\[0.5ex]
\bar{V}(p) &={\rm i}\,\frac{(Q_E+Q_e)}{c_\beta^{\,2}}\,a_2\,p+{\cal O}(p^3),
\end{align}
where $a_2$ is an arbitrary constant. 
The final solution is such that
\begin{align}
\bar{Y}(p)& =- \frac{{\rm i}}{12}\,Q_E\,(Q_E+Q_e)\,(Q_E+Q_i)\,a_3\,p^3+{\cal O}(p^5),\\[0.5ex]
\bar{N}(p) &=-{\rm i}\, Q_E\,a_3\,p+{\cal O}(p^3),\\[0.5ex]
\bar{J}(p)&= -\frac{1}{4}\,Q_E\,(Q_E+Q_i)\,a_3\,p^4+{\cal O}(p^6),\\[0.5ex]
\bar{V}(p) &=- \frac{Q_E\,(Q_E+Q_e)}{2\,c_\beta^{\,2}}\,a_3\,p^2+{\cal O}(p^4),
\end{align}
where $a_3$ is an arbitrary constant. Note, however, that the third solution is not consistent with Equations~(\ref{e311})--(\ref{e3.14a}), which mandate that 
$\bar{N}(p) \rightarrow- [Q_e/(Q_E+Q_e)]\,\bar{Y}(p) + {\cal O}(p)$ and  $\bar{V}(p)\rightarrow {\cal O}(p^2)$
as $p\rightarrow 0$. Hence, we deduce that $a_2=0$. 

We conclude that, at small values of $p$, the most general solution for $\bar{Y}(p)$ and $\bar{N}(p)$ takes the form 
\begin{align}\label{yy}
\bar{Y}(p) &= {\rm i}\,(Q_E+Q_e)\,(a_{-1} \,p^{-1}+a_0)+ {\cal O}(p),\\[0.5ex]
\bar{N}(p) &= -{\rm i}\,Q_e\,(a_{-1}\,p^{-1} +a_0)+ {\cal O}(p),\label{nn}
\end{align}
which is consistent with Eqs.~(\ref{e311})--(\ref{e3.14a}).

\subsection{Ricatti Matrix Differential Equation}
Let
\begin{align}
\underline{u}&= \left(\begin{array}{c}\bar{Y}\\\bar{N}\end{array}\right),\\[0.5ex]
\underline{v}&= \left(\begin{array}{c}\bar{J}\\c_\beta^{\,2}\,\bar{V}\end{array}\right).
\end{align}
Equations~(\ref{e1})--(\ref{e4}) can be written in the form 
\begin{align}
\frac{d\underline{u}}{dp}= \underline{\underline{A}}\,\underline{v},\\[0.5ex]
\frac{d\underline{v}}{dp}= \underline{\underline{B}}\,\underline{u},
\end{align}
where
\begin{align}
A_{11} &=  \frac{{\rm i}\,(Q_E+Q_e)+p^2}{p^2},\\[0.5ex]
A_{12}&=0,\\[0.5ex]
A_{21} &=- \frac{{\rm i}\,Q_e}{p^2},\\[0.5ex]
A_{22} &= \frac{{\rm i}\,Q_E + P_\varphi\,p^2}{c_\beta^{\,2}},\\[0.5ex]
B_{11} &= {\rm i}\,(Q_E+Q_i)\,p^2 + P_\varphi\,p^4,\\[0.5ex]
B_{12} &= {\rm i}\,(Q_E+Q_i)\,p^2 + \iota_e^{\,-1}\,P_\varphi\,p^4,\\[0.5ex]
B_{21}&= {\rm i}\,Q_e+ {\rm i}\,(Q_E+Q_i)\,D^{\,2}\,p^2 + D^{\,2}\,P_\varphi\,p^4,\\[0.5ex]
B_{22} & ={\rm i}\,(Q_E+Q_e) + [P_E + {\rm i}\,(Q_E +Q_i)\,D^{\,2}]\,p^2 + \iota_e^{\,-1}\,D^{\,2}\,P_\varphi\,p^4.
\end{align}
Thus, we obtain the following matrix differential equation: 
\begin{equation}\label{mat}
\frac{d}{dp}\!\left(\underline{\underline{A}}^{-1}\,\frac{d\underline{u}}{dp}\right) = \underline{\underline{B}}\,\underline{u}.
\end{equation}

Let
\begin{equation}\label{wdef}
 p\,\frac{d\underline{u}}{dp}=\underline{\underline{W}}\,\underline{u}.
\end{equation}
The previous two equations can be combined to give 
\begin{equation}
\left(p\,\frac{d\underline{\underline{W}}}{dp} - \underline{\underline{W}} 
+ \underline{\underline{W}}\,\underline{\underline{W}} + \underline{\underline{A}}\,p\,\frac{d\underline{\underline{A}}^{-1}}{dp}\,\underline{\underline{W}}- p^2\,\underline{\underline{A}}\,\underline{\underline{B}}\right)\underline{u} = \underline{0},
\end{equation}
which  yields the Riccati matrix differential equation, 
\begin{equation}\label{ricc}
p\,\frac{d\underline{\underline{W}}}{dp} = \underline{\underline{W}} - \underline{\underline{W}}\,\underline{\underline{W}} - \underline{\underline{E}}\,\underline{\underline{W}}
+\underline{\underline{F}},
\end{equation}
where 
\begin{align}
\underline{\underline{E}}(p) &= 
\underline{\underline{A}}\,p\,\frac{d\underline{\underline{A}}^{-1}}{dp},\\[0.5ex]
\underline{\underline{F}}(p)&= p^2\,\underline{\underline{A}}\,\underline{\underline{B}}.
\end{align}
In fact, it is easily demonstrated that
\begin{align}\label{ee1}
E_{11} &= \frac{2\,{\rm i}\,(Q_E+Q_e)}{{\rm i}\,(Q_E+ Q_e)+p^2},\\[0.5ex]
E_{12}&= 0,\\[0.5ex]
E_{21} &=-\frac{2\,{\rm i}\,Q_e\,(\,{\rm i}\,Q_E+2\,P_\varphi\,p^2)}{[\,{\rm i}\,(Q_E+Q_e)+p^2]\,(\,{\rm i}\,Q_E+P_\varphi\,p^2)},\\[0.5ex]
E_{22} &= -\frac{2\,P_\varphi\,p^2}{{\rm i}\,Q_E+ P_\varphi\,p^2},\label{ee4}
\end{align}
and
\begin{align}\label{f1}
F_{11} &= p^2\,[\,{\rm i}\,(Q_E+Q_e)+p^2]\,[\,{\rm i}\,(Q_E+Q_i) + P_\varphi\,p^2],\\[0.5ex]
F_{12} &= p^2\,[\,{\rm i}\,(Q_E+Q_e)+p^2]\,[\,{\rm i}\,(Q_E+Q_i) + \iota_e^{\,-1}\,P_\varphi\,p^2],\\[0.5ex]
F_{21} &=-{\rm i}\,Q_e\,p^2\,[\,{\rm i}\,(Q_E+Q_i)+P_\varphi\,p^2]\nonumber\\[0.5ex]
&\phantom{=} + c_\beta^{\,-2}\,p^2\,(\,{\rm i}\,Q_E+P_\varphi\,p^2)\,[\,{\rm i}\,Q_e+ {\rm i}\,(Q_E+Q_i)\,D^{\,2}\,p^2 + D^{\,2}\,P_\varphi\,p^4],\\[0.5ex]
F_{22} &=-{\rm i}\,Q_e\,p^2\,[\,{\rm i}\,(Q_E+Q_i)+\iota_e^{\,-1}\,P_\varphi\,p^2]\nonumber\\[0.5ex] 
&\phantom{=}+c_\beta^{\,-2}\,p^2\,(\,{\rm i}\,Q_E+P_\varphi\,p^2)\,\{\,{\rm i}\,(Q_E+Q_e) + [P_E +{\rm i}\,(Q_E + Q_i)\,D^{\,2}]\,p^2 + \iota_e^{\,-1}\,D^{\,2}\,P_\varphi\,p^4\}.\label{f4}
\end{align}

Finally, if 
\begin{equation}
\underline{\underline{W}}(p)= \left(\begin{array}{cc} W_{11},&W_{12}\\W_{21},&W_{22}\end{array}\right)
\end{equation}
then Eq.~(\ref{ricc}) yields 
\begin{align}\label{err1}
p\,\frac{dW_{11}}{dp}  &= W_{11}  - W_{11}\,W_{11}-W_{12}\,W_{21}- E_{11}\,W_{11} + F_{11},\\[0.5ex]
p\,\frac{dW_{12}}{dp} &= W_{12} - W_{11}\,W_{12} - W_{12}\,W_{22} - E_{11}\,W_{12} + F_{12},\\[0.5ex]
p\,\frac{dW_{21}}{dp} &= W_{21} -W_{21}\,W_{11}- W_{22}\,W_{21} - E_{21}\,W_{11} - E_{22}\,W_{21} + F_{21},\\[0.5ex]
p\,\frac{dW_{22}}{dp} &= W_{22} -W_{21}\,W_{12}- W_{22}\,W_{22}- E_{21}\,W_{12} - E_{22}\,W_{22} + F_{22}.\label{err4}
\end{align}
Thus, our final system of equations consists of a set of four coupled nonlinear differential equations. 

\subsection{Small Argument Behavior of Riccati Matrix Differential Equation}
It follows from Eqs.~(\ref{ee1})--(\ref{ee4})  that  $\underline{\underline{E}}(p) = \underline{\underline{E}}^{\,(0)} + {\cal O}(p^2)$ at small values of $p$,
where 
\begin{align}
E_{11}^{\,(0)} &= 2,\\[0.5ex]
E_{12}^{\,(0)}&=0,\\[0.5ex]
E_{21}^{\,(0)} &= - \frac{2\,Q_e}{Q_E+Q_e},\label{e210}\\[0.5ex]
E_{22}^{\,(0)}&= 0.
\end{align}
Likewise,  Eqs.~(\ref{f1})--(\ref{f4}) imply that $\underline{\underline{F}}(p) ={\cal O}(p^2)$.

Suppose that  $\underline{\underline{W}}(p)= \underline{\underline{W}}^{\,(0)}+ \underline{\underline{W}}^{\,(1)}\,p$ at small values of $p$,
where the elements of  $\underline{\underline{W}}^{\,(0)}$ and $\underline{\underline{W}}^{\,(1)}$ are independent of $p$. 
Equation~(\ref{ricc}) gives
\begin{align}
\underline{\underline{0}}&= \underline{\underline{W}}^{\,(0)}-\underline{\underline{W}}^{\,(0)}\,\underline{\underline{W}}^{\,(0)}- 
\underline{\underline{E}}^{\,(0)}\,\underline{\underline{W}}^{\,(0)},\\[0.5ex]
\underline{\underline{0}}&= -\underline{\underline{W}}^{\,(1)}\,\underline{\underline{W}}^{\,(0)}-\underline{\underline{W}}^{\,(0)}\,\underline{\underline{W}}^{\,(1)}- 
\underline{\underline{E}}^{\,(0)}\,\underline{\underline{W}}^{\,(1)}.
\end{align}
Suitable solutions are
\begin{align}\label{w0def}
\underline{\underline{W}}^{\,(0)} &=  \left(\begin{array}{cc} -1,&0\\-E_{21}^{\,(0)}/2,&0\end{array}\right),\\[0.5ex]
W_{12}^{\,(1)} &= 0,\\[0.5ex]
W_{21}^{\,(1)}& = \frac{E_{21}^{\,(0)}}{2}\left[W_{11}^{\,(1)}- W_{22}^{\,(1)}\right].\label{w21def}
\end{align}

At small values of $p$, let
\begin{align}
\underline{u}(p) &= \underline{u}_{-1}\,p^{-1} + \underline{u}_0,
\end{align}
where the elements of  $\underline{u}_{-1}$ (which are $y_{-1}$ and $n_{-1}$, respectively) and the elements of $\underline{u}_{0}$ (which are $y_0$ 
and $n_0$, respectively) are all constants.
Equation~(\ref{wdef}) gives 
\begin{align}
\underline{\underline{W}}^{\,(0)}\,\underline{u}_{-1} &= - \underline{u}_{-1},\\[0.5ex]
\underline{\underline{W}}^{\,(0)}\,\underline{u}_{0} + \underline{\underline{W}}^{\,(1)}\,\underline{u}_{-1} &=\underline{0}.\label{w1def}
\end{align}
Thus, making use of Eq.~(\ref{w0def}), we get
\begin{equation}
\left(\begin{array}{cc} -1,&0\\ -E_{21}^{\,(0)}/2,&0\end{array}\right) \left(\begin{array}{c}y_{-1}\\ n_{-1}\end{array}\right)= -\left(\begin{array}{c}y_{-1}\\ n_{-1}\end{array}\right),
\end{equation}
which implies that
\begin{equation}
\frac{E_{21}^{\,(0)}}{2}\,y_{-1}= - \frac{Q_e}{Q_E+Q_e}\,y_{-1} = n_{-1},
\end{equation}
in accordance with Eqs.~(\ref{yy}) and (\ref{nn}), where use has been made of Eq.~(\ref{e210}). 
Equations~(\ref{w0def})--(\ref{w21def})  and (\ref{w1def})  yield
\begin{equation}
\frac{y_0}{y_{-1}}= W_{11}^{\,(1)},
\end{equation}
with $n_0$ undetermined. 
It follows from Equation~(\ref{e41}) that 
\begin{equation}\label{e111}
\frac{\pi}{\hat{\mit\Delta}}\equiv \frac{y_0}{y_{-1}} = W_{11}^{\,(1)} = \frac{dW_{11}(0)}{dp}.
\end{equation}
Note, incidentally, that the result $\pi/\hat{\mit\Delta}=y_0/y_{-1}$ follows directly from Equation~(\ref{e43xx}), and would hold even if $a_2$ were non-zero. 

\subsection{Large Argument Behavior of Riccati Matrix Differential Equation}
At large values of $p$, it is clear from Eqs.~(\ref{f1})--(\ref{f4}) that  $\underline{\underline{F}}(p)=\underline{\underline{F}}^{\,(6)}\,p^6+ 
\underline{\underline{F}}^{\,(8)}\,p^8$, where the elements of $\underline{\underline{F}}^{\,(6)}$ and $\underline{\underline{F}}^{\,(8)}$ are constants.
On the other hand, Eqs.~(\ref{ee1})--(\ref{ee4}) imply that $\underline{\underline{E}}(p)=\underline{\underline{E}}^{\,(0)}$, 
where the elements of $\underline{\underline{E}}^{\,(0)}$ are constants. 
Thus, if we write $\underline{\underline{W}}(p) =\underline{\underline{W}}^{\,(2)}\,p^2+\underline{\underline{W}}^{\,(4)}\,p^4$,
where the elements of $\underline{\underline{W}}^{\,(2)}$ and  $\underline{\underline{W}}^{\,(4)}$ are constants, then Eq.~(\ref{ricc}) gives 
\begin{align}\label{e138}
\underline{\underline{W}}^{\,(4)}\,\underline{\underline{W}}^{\,(4)}&= \underline{\underline{F}}^{\,(8)},\\[0.5ex]
\underline{\underline{W}}^{\,(2)}\,\underline{\underline{W}}^{\,(4)}+ \underline{\underline{W}}^{\,(4)}\,\underline{\underline{W}}^{\,(2)}&= 
\underline{\underline{F}}^{\,(6)}.\label{e139}
\end{align}
Now, according to Eqs.~(\ref{f1})--(\ref{f4}), 
\begin{align}
F^{\,(8)}_{11} &=0,\\[0.5ex]
F^{\,(8)}_{12} &= 0,\\[0.5ex]
F^{\,(8)}_{21} &= c_\beta^{\,-2}\,D^{\,2}\,P_\varphi^{\,2},\\[0.5ex]
F^{\,(8)}_{22} &= c_\beta^{\,-2}\,\iota_e^{\,-1}\,D^{\,2}\,P_\varphi^{\,2},
\end{align}
so Eq.~(\ref{e138}) yields 
\begin{align}
W^{\,(4)}_{11} &=0,\\[0.5ex]
W^{\,(4)}_{12} &= 0,\\[0.5ex]
W^{\,(4)}_{21} &= -c_\beta^{\,-1}\,\iota_e^{\,1/2}\,D\,P_\varphi,\\[0.5ex]
W^{\,(4)}_{22} &=-c_\beta^{\,-1}\,\iota_e^{\,-1/2}\, D\,P_\varphi,
\end{align}
where we have chosen the sign of the square root that is associated with well-behaved solutions at large values of $p$. Here, we are  assuming that $\iota_e>0$. 
Equations~(\ref{f1})--(\ref{f4}) also give 
\begin{align}
F^{\,(6)}_{11} &=P_\varphi,\\[0.5ex]
F^{\,(6)}_{12} &= \iota_e^{\,-1}\,P_\varphi,\\[0.5ex]
F^{\,(6)}_{21} &={\rm i}\,c_\beta^{\,-2}\,Q_E\,D^{\,2}\,P_\varphi + {\rm i}\,c_\beta^{\,-2}\,(Q_E+Q_i)\,D^{\,2}\,P_\varphi,\\[0.5ex]
F^{\,(6)}_{22} &={\rm i}\,c_\beta^{\,-2}\,\iota_e^{\,-1}\,Q_E\,D^{\,2}\,P_\varphi + c_\beta^{\,-2}\,[P_E+{\rm i}\,(Q_E+Q_i)\,D^{\,2}]\,P_\varphi.
\end{align}
Thus, Eq.~(\ref{e139}) yields
\begin{align}
W_{12}^{\,(2)}\,W_{21}^{\,(4)} &= F_{11}^{\,(6)},\\[0.5ex]
W_{12}^{\,(2)}\,W_{22}^{\,(4)} &= F_{12}^{\,(6)},
\end{align}
which gives 
\begin{align}
W_{12}^{\,(2)} &= - c_\beta\,\iota_e^{\,-1/2}\,D^{-1}.
\end{align}

Now, if
\begin{equation}\label{e155}
\underline{\underline{W}}\,\underline{u}= \lambda(p)\,\underline{u}
\end{equation}
then Eq.~(\ref{wdef}) yields
\begin{equation}
p\,\frac{d\underline{u}}{dp} = \lambda\,\underline{u},
\end{equation}
which implies that
\begin{equation}
\underline{u}(p) = \underline{u}(p_0)\,\exp\left[\int_{p_0}^p\frac{\lambda_r(p')}{p'}\,dp'\right]\exp\left[\,{\rm i}\!\int_{p_0}^p\frac{\lambda_i(p')}{p'}\,dp'\right], 
\end{equation}
where $\lambda_r$  and $\lambda_i$ are the real and imaginary parts of $\lambda$, respectively. 
Of course, a solution that is well behaved at large values of $p$ is such that $\lambda_r$ is negative. As we have seen, the large-$p$ limit of
Eq.~(\ref{ricc}) is
\begin{equation}
\underline{\underline{W}}\,\underline{\underline{W}} = \underline{\underline{F}}.
\end{equation}
Hence, if
\begin{equation}\label{e158}
 \underline{\underline{F}}\,\underline{u} = {\mit\Lambda}\,\underline{u}
 \end{equation}
 then Eqs.~(\ref{e155}) and (\ref{e158}) imply that 
 \begin{equation}
 \lambda^2 = {\mit\Lambda}.
 \end{equation}
 The eigenvalue problem for the $F$-matrix reduces to
 \begin{equation}
 {\mit\Lambda}^{\,2}- (F_{11}+ F_{22})\,{\mit\Lambda} + F_{11}\,F_{22} - F_{12}\,F_{21}=0.
 \end{equation}
 Now,
 \begin{align}
 F_{11}+F_{22}&\simeq  F_{22}^{\,(8)}\,p^8 =c_\beta^{\,-2}\, \iota_e^{\,-1}\,D^{\,2}\,P_\varphi^{\,2}\,p^8,\\[0.5ex]
 F_{11}\,F_{22} - F_{12}\,F_{21}&  \simeq \left[F_{11}^{\,(6)}\,F_{22}^{\,(8)} - F_{12}^{\,(6)}\,F_{21}^{\,(8)}\right]p^{14}\nonumber\\[0.5ex]&\phantom{=}+\left[F_{11}^{\,(6)}\,F_{22}^{\,(6)} - F_{12}^{\,(6)}\,F_{21}^{\,(6)}\right]p^{12}
=c_\beta^{\,-2}\,R\,P_\varphi^{\,2}\,p^{12},
 \end{align}
  where 
 \begin{equation}
 R= P_E + {\rm i}\,(1-\iota_e^{\,-1})\,(Q_E+Q_i)\,D^{\,2},
 \end{equation}
Hence, the two eigenvalues of the $F$-matrix are
 \begin{align}
 {\mit\Lambda}_1&\simeq F_{22}^{\,(8)}\,p^8=  c_\beta^{\,-2}\,\iota_e^{\,-1}\,D^2\,P_\varphi^{\,2}\,p^8,\\[0.5ex]
 {\mit\Lambda}_2 &\simeq \frac{[F_{11}^{\,(6)}\,F_{22}^{\,(6)} - F_{12}^{\,(6)}\,F_{21}^{\,(6)}]}{F_{22}^{\,(8)}}\,p^4= \iota_e\,D^{-2}\,R\,p^4.
 \end{align}
 Thus, we deduce that the two eigenvalues of the $W$-matrix are 
 \begin{align}\label{l1}
 \lambda_1&=-{\mit\Lambda}_1^{\,1/2}= -c_\beta^{\,-1}\,\iota_e^{\,-1/2}\,D\,P_\varphi\,p^4,\\[0.5ex]
 \lambda_2&=-{\mit\Lambda}_2^{\,1/2}=-\iota_e^{\,1/2}\,D^{-1}\,R^{\,1/2}\,p^2,
 \end{align}
 Here, the square root of $R$ is taken such that the real part of $\lambda_2$ is negative. 
  Now, the eigenvalue problem for the $W$-matrix reduces to 
 \begin{equation}
 \lambda^{2} - W_{22}^{\,(4)}\,p^4\,\lambda + \left[W_{11}^{\,(2)}\,W_{22}^{\,(4)} - W_{12}^{\,(2)}\,W_{21}^{\,(4)}\right]p^6 = 0.
 \end{equation}
which yields
\begin{equation}
\lambda_1\simeq W_{22}^{\,(4)}\,p^4,
\end{equation}
which is in agreement with Eq.~(\ref{l1}), and
\begin{equation}
\lambda_2 \simeq \left[W_{11}^{\,(2)} - \frac{W_{12}^{\,(2)}\,W_{21}^{\,(4)}}{W_{22}^{\,(4)}}\right]p^2,
\end{equation}
 which implies that
 \begin{equation}
 W_{11}^{\,(2)} = -\iota_e^{\,1/2}\,D^{-1}\,R^{\,1/2}-c_\beta\,\iota_e^{\,1/2}\,D^{-1}.
 \end{equation}
 Hence, the large-$p$ boundary condition for the $W$-matrix
 is
 \begin{equation}\label{e147}
 \underline{\underline{W}}(p) =  \left(\begin{array}{cc} -\iota_e^{\,1/2}\,D^{-1}\,R^{\,1/2}\,p^2-c_\beta\,\iota_e^{\,1/2}\,D^{-1}\,p^2,& - c_\beta\,\iota_e^{\,-1/2}\,D^{-1}\,p^2\\-c_\beta^{\,-1}\,\iota_e^{\,1/2}\,D\,P_\varphi\,p^4,&-c_\beta^{\,-1}\,\iota_e^{\,-1/2}\,D\,P_\varphi\,p^4\end{array}\right).
 \end{equation}

\subsection{Method of Solution}
The method of solution is to launch the well-behaved asymptotic solution (\ref{e147}) of Eqs.~(\ref{err1})--(\ref{err4}) from large $p$, and then
integrate the equations backward to small $p$. The complex layer response index, ${\mit\Delta}_s$, is then determined  from Eqs.~(\ref{e48}) and (\ref{e111}). 
Note that, because there are no free parameters in expression (\ref{e147}), the layer response index is uniquely determined by this procedure. 



\end{document}