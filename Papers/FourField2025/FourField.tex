\documentclass[12pt,prb,aps]{revtex4-1}
\usepackage {amsmath}
\pdfoutput = 1 
\usepackage {graphicx}

\begin{document}

\title{A Four-Field Resonant Response Model for Tokamak Plasmas}

\author{R.~Fitzpatrick\,\footnote{rfitzp@utexas.edu}}
\affiliation{Institute for Fusion Studies,  Department of Physics,  University of Texas at Austin,  Austin TX, 78712, USA}

\begin{abstract}

\end{abstract}

\maketitle

\section{Introduction}
Tearing modes are slowly growing instabilities of ideally-stable tokamak plasmas that reconnect magnetic field-lines
at various rational surfaces within the plasma, in the process forming magnetic island chains that locally flatten the pressure profile and, thereby, degrade the plasma confinement.\cite{wes}
If tearing modes grow to sufficiently large amplitude then they can trigger major disruptions.\cite{wes1}  In fact,  tokamak
plasmas are observed to be particularly disruption-prone when tearing modes lock (i.e., become stationary in the
laboratory frame) to externally generated, resonant magnetic perturbations.\cite{vries}  

The analysis  of tearing mode dynamics in tokamak plasmas is most efficiently formulated as an asymptotic matching problem in which the  plasma is  divided into two distinct regions.\cite{fkr}    In the so-called {\em outer region}, which comprises most
of the plasma, the perturbation is governed by the equations of linearized, marginally-stable, ideal-magnetohydrodynamics (MHD).
However, these equations become singular on   {\em rational}\/ magnetic flux-surfaces at which the perturbed magnetic field resonates with the equilibrium field. In the {\em inner region}, which
consists of a set of narrow layers centered on the various rational surfaces, non-ideal-MHD effects   become important.

It is well known that single-fluid, resistive magnetohydrodynamics (MHD) offers a very poor description of
the response of the inner region to the tearing perturbation in the outer region. 
For instance, the strong  diamagnetic flows present in tokamak plasmas imply that the  electron and ion fluid velocities are significantly different from one another,
necessitating a two-fluid treatment.\cite{ara} Moreover, resistive-MHD does not take  into account the important  ion sound radius lengthscale below which electron and ion dynamics become decoupled from one another.\cite{drake,wal} Previously, Cole \& Fitzpatrick\,\cite{cole} used the four-field model
of Fitzpatrick \& Waelbroeck\,\cite{fw} (which is based on the original four-field model of Hazeltine, Kotschenreuther \& Morrison\,\cite{haz}) 
to determine the two-fluid response of a linear tearing layer to the perturbation in the outer region. This treatment was extended in Ref.~\onlinecite{rf2022} to take into account the
 anomalously large perpendicular  particle diffusivity  present in tokamak plasmas. 
 
 In configuration space, the four-field models of Refs.~\onlinecite{cole} and \onlinecite{rf2022} yield a set of resonant layer equations that can be expressed as 
 {\em ten}\/ coupled
 first-order linear differential equations.\cite{lee} However, in Fourier space, the resonant layer equations can be written as 
 {\em four}\/ first-order linear differential equatons.\cite{cole} It is clearly advantageous to solve the equations in Fourier space. In Refs.~\onlinecite{cole}
 and \onlinecite{rf2022}, an approximation is made by which one of the terms in the Fourier-transformed layer equations is neglected. This
 approximation, which is valid in low-$\beta$ plasmas,  is such that the layer equation that governs the parallel ion dynamics decouples from the other three equations, effectively
 converting a four-field resonant response model into a three-field model. Furthermore, the three remaining equations can be combined to give a
 single second-order linear ordinary differential equation. This second-order equation is most conveniently solved numerically by means of a Riccati transformation
 that converts it into a first-order nonlinear differential equation.\cite{ric1,ric2} The advantage of the Riccati approach is that it can deal with numerically problematic  solutions
 that blow up as $\exp(p^2)$, or faster, at large $p$, where $p$ is the Fourier-space variable. 
 
 Lee, Park \& Na\,\cite{lee} recently demonstrated  how to solve the full tenth-order four-field resonant layer equations in configuration  space using a Riccati transformation. In the process, they discovered that, in a high-$\beta$ plasma,  the frequency (in the local ${\bf E}\times {\bf B}$ frame)  at which resonant response of
 a tearing layer attains its maximum value  is shifted in the ion diamagnetic direction from the electron diamagnetic frequency.
 This result is significant because there is some experimental evidence for such a shift.\cite{paz} In the present paper, we demonstrate how the calculation of
 Lee et alia can be reimplemented in Fourier space. The Fourier version of the calculation is more convenient, from a numerical point of view, because
 it involves the solution of a fourth-order, rather than a tenth-order, system of equations. 

\section{Preliminary Analysis}\label{sect1}
\subsection{Plasma Equilibrium}
Consider a large aspect-ratio tokamak plasma equilibrium whose magnetic flux-surfaces map out
(almost) concentric circles in the poloidal plane. Such an equilibrium can be approximated as a
periodic cylinder.\cite{rf1993} Let $r$, $\theta$, $z$ be right-handed cylindrical coordinates. 
The magnetic axis corresponds to $r=0$, and the plasma boundary to $r=a$. The system is assumed to be periodic in the $z$
direction with periodicity length $2\pi\,R_0$, where $R_0$ is the simulated major radius of the plasma. The
safety-factor profile takes the form $q(r)=r\,B_z/[R_0\,B_\theta(r)]$, where $B_z$ is the constant
``toroidal'' magnetic field-strength, and $B_\theta(r)$ is the poloidal magnetic field-strength. The standard
large aspect-ratio orderings, $r/R_0\ll 1$ and $B_\theta/B_z\ll 1$, are adopted. 

\subsection{Asymptotic Matching}\label{cyl}
Consider a tearing mode perturbation that has $m$ periods in the poloidal direction, and
$n$ periods in the toroidal direction. 
The response of the plasma to the tearing mode is
governed by marginally stable, ideal-MHD everywhere in the plasma, apart from a (radially) narrow
layer centered on the rational surface, minor radius $r_s$, at which $q(r_s)=m/n$.\cite{fkr}

The perturbed magnetic field associated with the tearing mode is written $\delta{\bf B} \simeq \nabla\delta\psi\times{\bf e}_z$,  
where
$\delta\psi(r,\theta,\varphi,t)= \delta\psi(r,t)\,\exp[\,{\rm i}\,(m\,\theta-n\,\varphi)]$, 
and $\varphi=z/R_0$ is a simulated toroidal angle. 

In the outer region  (i.e., everywhere in the plasma
apart from the resonant layer), the perturbed helical magnetic flux, $\delta\psi(r,t)$, satisfies the
{\em cylindrical tearing mode equation},\,\cite{wes}
\begin{equation}\label{e3}
\frac{\partial^2\delta\psi}{\partial r^2} + \frac{1}{r}\,\frac{\partial\delta\psi}{\partial r}-\frac{m^2}{r^2}\,\delta\psi - \frac{J_z'\,\delta\psi}{r\,(1/q-n/m)}=  0,
\end{equation}
where 
$J_z(r)= R_0\,\mu_0\,j_z(r)/B_z$,
and $j_z(r)$ is the equilibrium ``toroidal'' current density. Here, $'\equiv d/dr$. In general, the solution of Eq.~(\ref{e3}) that satisfies physical
boundary conditions at the magnetic axis and the plasma boundary is such that $\delta\psi$ is continuous
across the rational surface, whereas $\partial\delta\psi/\partial r$ is discontinuous. The discontinuity of
 $\partial\delta\psi/\partial r$ across the rational surface is indicative of the presence of a helical current
 sheet at the surface.  The complex quantity ${\mit\Psi}_s(t)= \delta\psi(r_s,t)$ determines the amplitude
 and phase of the reconnected helical magnetic flux at the rational surface, whereas the complex quantity\,\cite{rf1993}
 \begin{equation}
{\mit\Delta\Psi}_s = \left[r\,\frac{\partial \delta\psi}{\partial r}\right]_{r_{s-}}^{r_{s+}}
\end{equation}
parameterizes the amplitude and phase of the helical current sheet flowing at the surface. The solution of
the cylindrical tearing mode equation in the outer region, in the presence of an externally generated, resonant magnetic perturbation 
of the same helicity as the tearing mode, leads to the relation\,\cite{rf1993,fkr}
\begin{equation}\label{e110}
{\mit\Delta\Psi}_s= ({\mit\Delta}_s')\,{\mit\Psi}_s + (-{\mit\Delta}_s')\,{\mit\Psi}_v,
\end{equation}
where ${\mit\Delta}_s'$ is a real dimensionless quantity known as the {\em tearing stability index}. Moreover,
${\mit\Psi}_v$ is the so-called {\em vacuum flux}, and is defined as the reconnected magnetic flux
that would be driven at the rational surface by the resonant magnetic perturbation were the
plasma intrinsically tearing stable (i.e., ${\mit\Delta}_s' <0$), and were there no current sheet at the rational surface
(i.e., ${\mit\Delta\Psi}_s=0$).

The current sheet at the rational surface can only be resolved by solving  a resistive-MHD plasma response model
in the  inner region (i.e., the region of the plasma in the immediate vicinity of the rational surface), 
and asymptotically matching the solution so obtained to the ideal-MHD solution in the outer region. The four-field 
plasma response model used in this paper is
described in the following section. 

\section{Four-Field Resonant Plasma Response Model}\label{sfour}
\subsection{Useful Definitions}
The plasma is assumed to consist of two species. First, electrons of mass $m_e$, electrical charge $-e$, 
number density $n$, and temperature $T_e$.  Second, ions of mass $m_i$, electrical charge $+e$,  
number density $n$, and temperature $T_i$. Let $p=n\,(T_e+T_i)$ be the total plasma pressure. 

It is helpful to define $n_0 = n(r_s)$, $p_0= p(r_s)$,
\begin{align}
\eta_e &=\left.\frac{d\ln T_e}{d\ln n}\right|_{r=r_s},\label{e211}\\[0.5ex]
\eta_i &= \left.\frac{d\ln T_i}{d\ln n}\right|_{r=r_s},\\[0.5ex]
\iota&= \left(\frac{T_e}{T_i}\right)_{r=r_s}\left(\frac{1+\eta_e}{1+\eta_i}\right),\label{e213}
\end{align}
where $n(r)$, $p(r)$, $T_e(r)$, and $T_i(r)$ refer to
number density, pressure, and temperature profiles that are unperturbed by the tearing mode. 

For the sake of simplicity, the perturbed electron and ion temperature profiles are assumed to be functions of
the perturbed electron number density profile in the immediate vicinity of the rational surface. In other words, $T_e=T_e(n)$ and $T_i=T_i(n)$. This
implies that $p=p(n)$. 
The ``MHD velocity'', which is the velocity of a
fictional MHD fluid, is defined ${\bf V}={\bf V}_E + V_{\parallel\,i}\,{\bf b}$, where ${\bf V}_E$ is the
${\bf E}\times{\bf B}$ drift velocity, $V_{\parallel\,i}$ is the parallel component of the ion fluid
velocity, ${\bf b}= {\bf B}/|{\bf B}|$, and ${\bf B}$ is the magnetic field-strength.

\subsection{Fundamental Fields}
The four fundamental fields in our four-field model---namely, $\psi$, $N$, $\phi$, and $V$---have the following
definitions:
\begin{align}\label{e10}
\nabla\psi &= \frac{{\bf n}\times{\bf B}} {r_s\,B_z},\\[0.5ex]
N &=-\hat{d}_i\left(\frac{p-p_0}{B_z^{\,2}/\mu_0}\right),\\[0.5ex]
\nabla\phi &= \frac{{\bf n}\times {\bf V}}{r_s\,V_A},\\[0.5ex]
V &= \hat{d}_i\left(\frac{{\bf n}\cdot {\bf V}}{V_A}\right).\label{e13}
\end{align}
Here,   ${\bf n} = (0,\,\epsilon/q_s,\,1)$, $\epsilon = r/R_0$, $q_s=m/n$, 
$V_A =B_z/\sqrt{\mu_0\,n_0\,m_i}$, 
$d_i = \sqrt{m_i/(n_0\,e^{\,2}\,\mu_0)}$,
and $\hat{d}_i=d_i/r_s$. 
 Our
model also employs the auxiliary field
\begin{align}\label{e16}
J=-\frac{2\,\epsilon_s}{q_s}+\hat{\nabla}^2\psi,
\end{align}
where 
$\epsilon_s=r_s/R_0$, and $\hat{\nabla} = r_s\,\nabla$. Note that $V_A$ is the Alfv\'{e}n speed, whereas $d_i$ is the collisionless ion skin-depth. 

\subsection{Fundamental Equations}
The four-field model takes the form:\,\cite{fw,cole,rf2022}
\begin{align}\label{e12v}
\frac{\partial\psi}{\partial\hat{t}}&= [\phi,\psi] -\iota_e\,[N,\psi]
+\hat{\eta}_\parallel\,J + \hat{E}_\parallel,\\[0.5ex]
\frac{\partial N}{\partial \hat{t}}&= [\phi,N] +\hat{d}_\beta^{\,2}\,[J,\psi]+c_\beta^{\,2}\,[V,\psi] 
+ \hat{D}_\perp\,\hat{\nabla}_\perp^{\,2}N,\\[0.5ex]
\frac{\partial \hat{\nabla}^2\phi}{\partial \hat{t}}&= [\phi,\hat{\nabla}^2\phi] - \frac{\iota_i}{2}\left(\hat{\nabla}^2[\phi,N] + [\hat{\nabla}^2\phi,N] + [\hat{\nabla}^2 N,\phi]\right) + [J,\psi] \nonumber\\[0.5ex]&\phantom{=}+\hat{\chi}_\varphi  \,\hat{\nabla}^4\!\left(\phi + \iota_i\,N\right), \\[0.5ex]
\frac{\partial V}{\partial\hat{t}}&= [\phi,V] +[N,\psi] + \hat{\chi}_\varphi\,\hat{\nabla}^2 V.\label{e21}
\end{align}
Here, $[A,B]\equiv \hat{\nabla} A\times \hat{\nabla} B\cdot {\bf n}$, $\iota_e=\iota/(1+\iota)$, $\iota_i=1/(1+\iota)$, $\hat{t} = t/(r_s/V_A)$, $\hat{\eta}_{\parallel} = \eta_{\parallel}/(\mu_0\,r_s\,V_A)$, $\hat{E}_\parallel = E_\parallel/(B_z\,V_A)$, 
$\hat{\chi}_\varphi= \chi_\varphi/(r_s\,V_A)$, where $\eta_{\parallel}$ is the parallel  plasma electrical
resistivity at the rational surface, $E_\parallel$ the parallel inductive electric field that maintains the equilibrium toroidal
plasma current in the vicinity of the rational surface, $\chi_\varphi$  the anomalous perpendicular ion momentum
diffusivity at the rational surface, and $D_\perp$ the  anomalous perpendicular electron/ion  energy diffusivity in the vicinity of the rational surface. 
Moreover, $d_\beta=c_\beta\,d_i$, and $\hat{d}_\beta=d_\beta/r_s$, where $c_\beta = \sqrt{\beta/(1+\beta)}$, and
$\beta=(5/3)\,\mu_0\,p_0/B_z^{\,2}$. Here, $d_\beta$ is usually referred to as the {\em ion sound radius}.  

\subsection{Matching to Plasma Equilibrium}
The unperturbed plasma equilibrium is such that
${\bf B} = (0,\,B_\theta(r),\,B_z)$,  $p = p(r)$,
${\bf V} = (0,\,V_E(r),\,V_z(r))$,
where 
$V_E(r)\simeq E_r/B_z$
 is the (dominant $\theta$-component of the) ${\bf E}\times {\bf B}$ velocity. Now, the resonant layer is assumed to have a radial thickness that is
much smaller than $r_s$.   Hence, we only need to evaluate plasma equilibrium quantities in the immediate vicinity of the rational
surface. Equations~(\ref{e10})--(\ref{e13}) suggest that 
\begin{align}
\psi(\hat{x})&= \frac{\hat{x}^{\,2}}{2\,\hat{L}_s},\label{e23}\\[0.5ex]
N(\hat{x}) &= -\hat{V}_\ast\,\hat{x},\\[0.5ex]
\phi(\hat{x}) &= - \hat{V}_E\,\hat{x},\\[0.5ex]
V (\hat{x})&= \hat{V}_\parallel,\label{e26}
\end{align}
where 
$\hat{x}=(r-r_s)/r_s$,
 $\hat{L}_s=L_s/r_s$,  $L_s=R_0\,q_s/s_s$, 
  $\hat{V}_E= V_E(r_s)/V_A$,
$\hat{V}_\ast= V_\ast(r_s)/V_A$,
$V_\ast(r) = (dp/dr)/(e\,n_0\,B_z)$ 
is the (dominant $\theta$-component of the) diamagnetic velocity,
  and 
 $\hat{V}_\parallel=\hat{d}_i\, V_z(r_s)/V_A$. Here, $s_s=s(r_s)$ and $s(r)=d\ln q/d\ln r$. We also have
 \begin{align}\label{e28}
 J (\hat{x})&= -\left(\frac{2}{s_s}-1\right)\frac{1}{\hat{L}_s},
\end{align} 
and $ \hat{E}_\parallel(\hat{x}) =(2/s_s-1)\, (\hat{\eta}_\parallel/\hat{L}_s)$.

\section{Linear Resonant Plasma Response Model}\label{layer}
\subsection{Introduction}
The aim of this section is to obtain a set of linear layer equations from the four-field model
introduced in Sect.~\ref{sfour}.

\subsection{Derivation of Linear Layer Equations}
In accordance with Eqs.~(\ref{e23})--(\ref{e28}), let us write
\begin{align}\label{e31}
\psi(\hat{x},\zeta,\hat{t}) &= \frac{\hat{x}^{\,2} }{2\,\hat{L}_s}+ \tilde{\psi}(\hat{x})\,{\rm e}^{\,{\rm i}\,(\zeta-\hat{\omega}\,\hat{t})},\\[0.5ex]
\phi(\hat{x},\zeta,\hat{t}) &=-\hat{V}_E\,\hat{x}+ \tilde{\phi}(\hat{x})\,{\rm e}^{\,{\rm i}\,(\zeta-\hat{\omega}\,\hat{t})},\\[0.5ex]
N(\hat{x},\zeta,\hat{t}) &= -\hat{V}_\ast\,\hat{x}+\iota_e\,\tilde{N}(\hat{x})\,{\rm e}^{\,{\rm i}\,(\zeta-\hat{\omega}\,\hat{t})},\\[0.5ex]
V(\hat{x},\zeta,\hat{t}) &= \hat{V}_\parallel +\iota_e\,\tilde{V}(\hat{x})\,{\rm e}^{\,{\rm i}\,(\zeta-\hat{\omega}\,\hat{t})},\\[0.5ex]
J(\hat{x},\zeta,\hat{t}) &=-\left(\frac{2}{s_s}-1\right)\!\frac{1}{\hat{L}_s}+ \hat{\nabla}^2\tilde{\psi}(\hat{x})\,{\rm e}^{\,{\rm i}\,(\zeta-\hat{\omega}\,\hat{t})},\label{e36}
\end{align}
where $\zeta=m\,\theta-n\,\varphi$, $\hat{\omega}=r_s\,\omega/V_A$, and $\omega$ is the frequency of the tearing mode in the laboratory frame. 
Substituting Eqs.~(\ref{e31})--(\ref{e36}) into Eqs.~(\ref{e16})--(\ref{e21}), 
 and only retaining terms that
are first order in perturbed quantities, we obtain the following set of linear equations:
\begin{align}\label{e37}
-{\rm i}\,(\omega-\omega_E -\omega_{\ast\,e})\,\tau_H\,\tilde{\psi} &= -{\rm i} \,\hat{x}\,(\tilde{\phi}-\tilde{N})+ S^{-1}\,\hat{\nabla}^2\tilde{\psi},\\[0.5ex]
-{\rm i}\,(\omega-\omega_E)\,\tau_H\,\tilde{N} &= -{\rm i}\,\omega_{\ast\,e}\,\tau_H\,\tilde{\phi}- {\rm i}\,\iota_e\,\hat{d}_\beta^{\,2}\,\hat{x}\,\hat{\nabla}^2\tilde{\psi}-{\rm i}\,c_\beta^{\,2}\,\hat{x}\,\tilde{V} \nonumber\\[0.5ex]
&\phantom{=} + S^{-1}\,P_\perp\,\hat{\nabla}_\perp^2 \tilde{N},\\[0.5ex]
-{\rm i}\,(\omega-\omega_E-\omega_{\ast\,i})\,\tau_H\,\hat{\nabla}^2\tilde{\phi} &= -{\rm i}\,\hat{x}\,\hat{\nabla}^2\tilde{\psi}+S^{-1}\,P_\varphi\,\hat{\nabla}^4\!\left(\tilde{\phi} + \frac{\tilde{N}}{\iota}\right),\\[0.5ex]
-{\rm i}\,(\omega-\omega_E)\,\tau_H\,\tilde{V} &= {\rm i}\,\omega_{\ast\,e}\,\tau_H\,\tilde{\psi} - {\rm i}\,\hat{x}\,\tilde{N}
+ S^{-1}\,P_\varphi\,\hat{\nabla}^2\tilde{V}.\label{e40}
\end{align}
Here, 
$\tau_H = L_s/(m\,V_A)$ 
is the  hydromagnetic time, 
$\omega_E =(m/r_s)\,V_E(r_s)$
  the 
 ${\bf E}\times {\bf B}$ frequency, 
$\omega_{\ast\,e} = -\iota_e\,(m/r_s)V_\ast(r_s)$
the electron diamagnetic frequency,
$\omega_{\ast\,i} =\iota_i\,(m/r_s)V_\ast(r_s)$
 the  ion diamagnetic frequency,  $S=\tau_R/\tau_H$ the  Lundquist number, 
$\tau_R = \mu_0\,r_s^{\,2}/\eta_\parallel$
 the
 resistive diffusion time, 
$\tau_\varphi
= r_s^{\,2}/\chi_\varphi$
the  toroidal momentum confinement time, and 
$\tau_\perp = r_s^{\,2}/D_\perp$.
  Furthermore, $P_\varphi = \tau_R/\tau_\varphi$ and $P_\perp = \tau_R/\tau_\perp$ are magnetic Prandtl numbers.

 Let us define the stretched radial variable $X = S^{\,1/3}\,\hat{x}$.
Assuming that $X\sim{\cal O}(1)$ in the layer (i.e., assuming that the layer thickness is roughly of order $S^{-1/3}\,r_s$),
and making use of the fact that $S\gg 1$ in conventional tokamak plasmas,  Eqs.~(\ref{e37})--(\ref{e40}) reduce to the following
set of linear layer equations:\,\cite{cole}
\begin{align}\label{e43xx}
(g+{\rm i}\,Q_e)\,\tilde{\psi}&= - {\rm i}\,X\left(\tilde{\phi}-\tilde{N}\right)+ \frac{d^{\,2}\tilde{\psi}}{d X^2},\\[0.5ex]
g\,\tilde{N} &= - {\rm i}\,Q_{e}\,\tilde{\phi}   - {\rm i} \,D^{2}\,X\,\frac{d^{\,2}\tilde{\psi}}{dX^{2}}- {\rm i}\,c_\beta^{\,2}\,X\,\tilde{V}
+ P_\perp\,\frac{d^{\,2} \tilde{N}}{dX^{2}},\label{e44xx}\\[0.5ex]
(g+{\rm i}\,Q_i)\,\frac{d^{\,2}\tilde{\phi}}{dX^2}&= - {\rm i}\,X\,\frac{d^{\,2}\tilde{\psi}}{dX^2}+ P_\varphi\,\frac{d^{\,4}}{dX^4}\!\left(\tilde{\phi} + \frac{\tilde{N}}{\iota}\right),\\[0.5ex]
g\,\tilde{V} &= {\rm i}\,Q_{e}\,\tilde{\psi} - {\rm i}\,X\,\tilde{N} + P_\varphi\,\frac{d^{\,2}\tilde{V}}{dX^{2}}.\label{e46}
\end{align}
Here, $g = -{\rm i}\,(Q-Q_E)=-{\rm i}\,S^{\,1/3}\,(\omega-\omega_E)\,\tau_H$, $Q_{e,i} = S^{\,1/3}\,\omega_{\ast\,e,i}\,\tau_H$,
and $D = S^{\,1/3}\,\iota_e^{1/2}\,\hat{d}_\beta$. If we write $P_\perp = c_\beta^{\,2}$
then Eqs.~(\ref{e43xx})--(\ref{e46}) become equivalent to the set of  layer equations solved by Lee et alia.


The previously mentioned low-$\beta$ approximation used in Refs.~\onlinecite{cole} and \onlinecite{rf2022} involves neglecting the term
containing $c_\beta^{\,2}$ in Eq.~(\ref{e44xx}). This approximation decouples Eq.~(\ref{e46}) from the three preceding equations, and effectively
converts a four-field resonant response model into a three-field model. In the following, we shall not use this approximation.

\subsection{Asymptotic Matching}
The  linear layer equations, (\ref{e43xx})--(\ref{e46}), possess tearing parity solutions
characterized by the symmetry $\tilde{\psi}(-X)=\tilde\psi(X)$, 
$\tilde{N}(-X)= - \tilde{N}(X)$, $\tilde{\phi}(-X)=-\tilde{\phi}(X)$,  $\tilde{V}(-X)=\tilde{V}(X)$. If we assume that
the asymptotic behavior of the tearing parity layer solutions is such that 
\begin{equation}\label{e47}
\tilde{\psi}(X)\rightarrow  \psi_0\!\left[1+ \frac{\hat{\mit\Delta}}{2}\,|X| + {\cal O}(X^2)\right]
\end{equation}
as $|X|\rightarrow\infty$, where $\psi_0$ is an arbitrary constant, then asymptotic matching to the outer solution yields
\begin{equation}\label{e48}
{\mit\Delta\Psi}_s= (S^{1/3}\,\hat{\mit\Delta})\,\,{\mit\Psi}_s.
\end{equation}

\section{Fourier-Transformed Four-Field Equations}\label{linear}
\subsection{Fourier Transformation}
Equations~(\ref{e43xx})--(\ref{e46}) are most conveniently solved in Fourier transform space.\cite{cole} 
Let
\begin{equation}
\bar{\phi}(p) = \int_{-\infty}^\infty \tilde{\phi}(X)\,{\rm e}^{-{\rm i}\,p\,X}\,dX,
\end{equation}
et cetera. The Fourier transformed linear layer equations become
\begin{align}\label{e314}
(g+{\rm i}\,Q_e)\,\bar{\psi}&=\frac{d}{dp}\!\left(\bar{\phi}-\bar{N}\right)-p^2\,\bar{\psi},\\[0.5ex]
g\,\bar{N} &= - {\rm i}\,Q_{e}\,\bar{\phi} -D^{\,2}\,\frac{d(p^2\,\bar{\psi})}{dp}+ c_\beta^{\,2}\,\frac{d\bar{V}}{dp}
  - P_\perp\,p^2\bar{N},\label{e316}\\[0.5ex]
(g+{\rm i}\,Q_i)\,p^2\,\bar{\phi}&=  \frac{d(p^2\,\bar{\psi})}{dp}- P_\varphi\,p^4\!\left(\bar{\phi} + \frac{\bar{N}}{\iota}\right),\\[0.5ex]
g\,\bar{V} &= {\rm i}\,Q_{e}\,\bar{\psi} +\frac{d\bar{N}}{dp}- P_\varphi\,p^2\,\bar{V},\label{e317}
\end{align}
where, for a tearing parity solution, 
\begin{equation}\label{e41}
\bar{\phi}(p)\rightarrow \bar{\phi}_0\!\left[\frac{\hat{\mit\Delta}}{\pi\,p} + 1+ {\cal O}(p)\right]
\end{equation}
as $p\rightarrow 0$. 

Finally, if we define 
\begin{align}
\bar{J}(p)&= p^2\,\bar{\psi},\\[0.5ex]
\bar{Y} (p)&= \bar{\phi}-\bar{N},\label{e43}
\end{align}
then we obtain the following set of four coupled first-order differential equations:
\begin{align}\label{e1}
\frac{d\bar{Y}}{dp} &= \left(\frac{g+{\rm i}\,Q_e+p^2}{p^2}\right)\bar{J},\\[0.5ex]
\frac{d\bar{N}}{dp} &= \left(\frac{-{\rm i}\,Q_e}{p^2}\right)\bar{J} + (g + P_\varphi\,p^2)\,\bar{V},\\[0.5ex]
\frac{d\bar{J}}{dp} &= [(g+{\rm i}\,Q_i)\,p^2 + P_\varphi\,p^4]\,\bar{Y}
+ [(g+{\rm i}\,Q_i)\,p^2 + \iota_e^{\,-1}\,P_\varphi\,p^4]\,\bar{N},\\[0.5ex]
c_\beta^{\,2}\,\frac{d\bar{V}}{dp} &= [{\rm i}\,Q_e+ D^{\,2}\,(g+{\rm i}\,Q_i)\,p^2 + D^{\,2}\,P_\varphi\,p^4]\,\bar{Y}
\nonumber\\[0.5ex]&\phantom{=} +(g+{\rm i}\,Q_e + [P_\perp+ D^{\,2}\,(g + {\rm i}\,Q_i)]\,p^2 + \iota_e^{\,-1}\,D^{\,2}\,P_\varphi\,p^4)\,\bar{N},\label{e4}
\end{align}
Note that $\iota_e= -Q_e/(Q_i-Q_e)$. 

\subsection{Small Argument Expansion}
Let us search for power-law solutions of Eqs.~(\ref{e1})--(\ref{e4}) at small values of $p$. Given that we have four coupled first-order differential equations,
we expect to find four independent power-law solutions.
The first solution is
such that
\begin{align}
\bar{Y}(p) &= (g+{\rm i}\,Q_e)\,a_{-1}\,p^{-1} + \left[\frac{1}{2}\,g\,(g+{\rm i}\,Q_e)\,(g+{\rm i}\,Q_i) - 1\right]a_{-1}\,p +{\cal O}(p^3),\\[0.5ex]
\bar{N}(p)&= -{\rm i}\,Q_e\,a_{-1}\, p^{-1} - \frac{1}{2}\,g\,({\rm i}\,Q_e)\,(g+{\rm i}\,Q_i)\,a_{-1}\,p + {\cal O}(p^3),\\[0.5ex]
\bar{J}(p)&= -a_{-1}+ \frac{1}{2}\,g\,(g+{\rm i}\,Q_i)\,a_{-1}\,p^2 +{\cal O}(p^4),\\[0.5ex]
\bar{V}(p)&= \frac{[-{\rm i}\,Q_e\,(1+P_\perp) + g\,(g+{\rm i}\,Q_i)\,D^{\,2}]}{2\,c_\beta^{\,2}}\,a_{-1}\,p^2+{\cal O}(p^4),
\end{align}
where $a_{-1}$ is an arbitrary constant. 
The second solution is such that
\begin{align}
\bar{Y}(p) &= (g+{\rm i}\,Q_e)\,a_0+\frac{1}{6}\,g\,(g+{\rm i}\,Q_e)\,(g+{\rm i}\,Q_i)\,a_0\,p^2 +{ \cal O}(p^4),\\[0.5ex]
\bar{N}(p) &= -{\rm i}\,Q_e\,a_0-\frac{1}{6}\,g\,({\rm i}\,Q_e)\,(g+{\rm i}\,Q_i)\,a_0\,p^2+{\cal O}(p^4),\\[0.5ex]
\bar{J}(p)&= \frac{1}{3}\,g\,(g+{\rm i}\,Q_i)\,a_0\,p^3+{\cal O}(p^5),\\[0.5ex]
\bar{V}(p)&= \frac{1}{3}\,\frac{[-{\rm i}\,Q_e\,P_\perp + g\,(g+{\rm i }\,Q_i)\,D^{\,2}]}{c_\beta^{\,2}}\,a_0\,p^3+{\cal O}(p^5),
\end{align}
where $a_0$ is an arbitrary constant. 
The third solution is such that
\begin{align}
\bar{Y}(p) &= \frac{1}{6}\,(g+{\rm i}\,Q_e)\,(g+{\rm i}\,Q_i)\,a_2\,p^2+{\cal O}(p^4),\\[0.5ex]
\bar{N}(p) &= a_2+\frac{1}{2}\,(g+{\rm i}\,Q_i)\left(-\frac{1}{3}\,{\rm i}\,Q_e\,+ \frac{g}{c_\beta^{\,2}}\right)a_2\,p^2+{\cal O}(p^4),\\[0.5ex]
\bar{J}(p) &=\frac{1}{3}\,(g+{\rm i}\,Q_i)\,a_2\,p^3+{\cal O}(p^5),\\[0.5ex]
\bar{V}(p) &=\frac{(g+{\rm i}\,Q_e)}{c_\beta^{\,2}}\,a_2\,p+{\cal O}(p^3),
\end{align}
where $a_2$ is an arbitrary constant. 
The final solution is such that
\begin{align}
\bar{Y}(p)& = \frac{1}{12}\,g\,(g+{\rm i}\,Q_e)\,(g+{\rm i}\,Q_i)\,a_3\,p^3+{\cal O}(p^5),\\[0.5ex]
\bar{N}(p) &= g\,a_3\,p+{\cal O}(p^3),\\[0.5ex]
\bar{J}(p)&= \frac{1}{4}\,g\,(g+{\rm i}\,Q_i)\,a_3\,p^4+{\cal O}(p^6),\\[0.5ex]
\bar{V}(p) &= \frac{g\,(g+{\rm i}\,Q_e)}{2\,c_\beta^{\,2}}\,a_3\,p^2+{\cal O}(p^4),
\end{align}
where $a_3$ is an arbitrary constant. 

We conclude that, at small values of $p$, the most general solution for $\bar{Y}(p)$ and $\bar{N}(p)$ takes the form 
\begin{align}\label{yy}
\bar{Y}(p) &= (g+{\rm i}\,Q_e)\,a_{-1} \,p^{-1}+(g+{\rm i}\,Q_e) \,a_0+ {\cal O}(p),\\[0.5ex]
\bar{N}(p) &= (-{\rm i}\,Q_e)\,a_{-1}\,p^{-1} + (-{\rm i}\,Q_e)\,a_0 + a_2 + {\cal O}(p).\label{nn}
\end{align}

\subsection{Ricatti Matrix Differential Equation}
Let
\begin{align}
\underline{u}= \left(\begin{array}{c}\bar{Y}\\\bar{N}\end{array}\right),\\[0.5ex]
\underline{v}= \left(\begin{array}{c}\bar{J}\\c_\beta^{\,2}\,\bar{V}\end{array}\right).
\end{align}
Equations~(\ref{e1})--(\ref{e4}) can be written in the form 
\begin{align}
\frac{d\underline{u}}{dp}= \underline{\underline{A}}\,\underline{v},\\[0.5ex]
\frac{d\underline{v}}{dp}= \underline{\underline{B}}\,\underline{u},
\end{align}
where
\begin{align}
A_{11} &=  \frac{g+{\rm i}\,Q_e+p^2}{p^2},\\[0.5ex]
A_{12}&=0,\\[0.5ex]
A_{21} &= \frac{-{\rm i}\,Q_e}{p^2},\\[0.5ex]
A_{22} &= \frac{g + P_\varphi\,p^2}{c_\beta^{\,2}},\\[0.5ex]
B_{11} &= (g+{\rm i}\,Q_i)\,p^2 + P_\varphi\,p^4,\\[0.5ex]
B_{12} &= (g+{\rm i}\,Q_i)\,p^2 + \iota_e^{\,-1}\,P_\varphi\,p^4,\\[0.5ex]
B_{21}&= {\rm i}\,Q_e+ D^{\,2}\,(g+{\rm i}\,Q_i)\,p^2 + D^{\,2}\,P_\varphi\,p^4,\\[0.5ex]
B_{22} & =g+{\rm i}\,Q_e + [P_\perp + D^{\,2}\,(g + {\rm i}\,Q_i)]\,p^2 + \iota_e^{\,-1}\,D^{\,2}\,P_\varphi\,p^4.
\end{align}
Thus, we obtain the following matrix differential equation: 
\begin{equation}\label{mat}
\frac{d}{dp}\!\left(\underline{\underline{A}}^{-1}\,\frac{d\underline{u}}{dp}\right) = \underline{\underline{B}}\,\underline{u}.
\end{equation}

Let
\begin{equation}\label{wdef}
 p\,\frac{d\underline{u}}{dp}=\underline{\underline{W}}\,\underline{u}.
\end{equation}
The previous two equations can be combined to give 
\begin{equation}
\left(p\,\frac{d\underline{\underline{W}}}{dp} - \underline{\underline{W}} 
+ \underline{\underline{W}}\,\underline{\underline{W}} + \underline{\underline{A}}\,p\,\frac{d\underline{\underline{A}}^{-1}}{dp}\,\underline{\underline{W}}- p^2\,\underline{\underline{A}}\,\underline{\underline{B}}\right)\underline{u} = \underline{0},
\end{equation}
which  yields the Riccati matrix differential equation, 
\begin{equation}\label{ricc}
p\,\frac{d\underline{\underline{W}}}{dp} = \underline{\underline{W}} - \underline{\underline{W}}\,\underline{\underline{W}} - \underline{\underline{E}}\,\underline{\underline{W}}
+\underline{\underline{F}},
\end{equation}
where 
\begin{align}
\underline{\underline{E}}(p) &= 
\underline{\underline{A}}\,p\,\frac{d\underline{\underline{A}}^{-1}}{dp},\\[0.5ex]
\underline{\underline{F}}(p)&= p^2\,\underline{\underline{A}}\,\underline{\underline{B}}.
\end{align}
In fact, it is easily demonstrated that
\begin{align}\label{ee1}
E_{11} &= \frac{2\,(g+{\rm i}\,Q_e)}{g+ {\rm i}\,Q_e+p^2},\\[0.5ex]
E_{12}&= 0,\\[0.5ex]
E_{21} &=-\frac{2\,{\rm i}\,Q_e\,(g+2\,P_\varphi\,p^2)}{(g+{\rm i}\,Q_e+p^2)\,(g+P_\varphi\,p^2)},\\[0.5ex]
E_{22} &= -\frac{2\,P_\varphi\,p^2}{g+ P_\varphi\,p^2},\label{ee4}
\end{align}
and
\begin{align}\label{f1}
F_{11} &= p^2\,(g+{\rm i}\,Q_e+p^2)\,(g+{\rm i}\,Q_i + P_\varphi\,p^2),\\[0.5ex]
F_{12} &= p^2\,(g+{\rm i}\,Q_e+p^2)\,(g+{\rm i}\,Q_i + \iota_e^{\,-1}\,P_\varphi\,p^2),\\[0.5ex]
F_{21} &=-{\rm i}\,Q_e\,p^2\,(g+{\rm i}\,Q_i+P_\varphi\,p^2)\nonumber\\[0.5ex]
&\phantom{=} + c_\beta^{\,-2}\,p^2\,(g+P_\varphi\,p^2)\,[\,{\rm i}\,Q_e+ D^{\,2}\,(g+{\rm i}\,Q_i)\,p^2 + D^{\,2}\,P_\varphi\,p^4],\\[0.5ex]
F_{22} &=-{\rm i}\,Q_e\,p^2\,(g+{\rm i}\,Q_i+\iota_e^{\,-1}\,P_\varphi\,p^2)\nonumber\\[0.5ex] 
&\phantom{=}+c_\beta^{\,-2}\,p^2\,(g+P_\varphi\,p^2)\,[g+{\rm i}\,Q_e + [P_\perp +D^{\,2}\,(g + {\rm i}\,Q_i)]\,p^2 + \iota_e^{\,-1}\,D^{\,2}\,P_\varphi\,p^4].\label{f4}
\end{align}

Finally, if 
\begin{equation}
\underline{\underline{W}}(p)= \left(\begin{array}{cc} W_{11},&W_{12}\\W_{21},&W_{22}\end{array}\right)
\end{equation}
then Eq.~(\ref{ricc}) yields 
\begin{align}\label{err1}
p\,\frac{dW_{11}}{dp}  &= W_{11}  - W_{11}\,W_{11}-W_{12}\,W_{21}- E_{11}\,W_{11} + F_{11},\\[0.5ex]
p\,\frac{dW_{12}}{dp} &= W_{12} - W_{11}\,W_{12} - W_{12}\,W_{22} - E_{11}\,W_{12} + F_{12},\\[0.5ex]
p\,\frac{dW_{21}}{dp} &= W_{21} -W_{21}\,W_{11}- W_{22}\,W_{21} - E_{21}\,W_{11} - E_{22}\,W_{21} + F_{21},\\[0.5ex]
p\,\frac{dW_{22}}{dp} &= W_{22} -W_{21}\,W_{12}- W_{22}\,W_{22}- E_{21}\,W_{12} - E_{22}\,W_{22} + F_{22}.\label{err4}
\end{align}
Thus, our final system of equations consists of a set of four coupled nonlinear differential equations. 

\subsection{Small Argument Behavior of Riccati Matrix Differential Equation}
It follows from Eqs.~(\ref{ee1})--(\ref{ee4})  that  $\underline{\underline{E}}(p) = \underline{\underline{E}}^{\,(0)} + {\cal O}(p^2)$ at small values of $p$,
where 
\begin{align}
E_{11}^{\,(0)} &= 2,\\[0.5ex]
E_{12}^{\,(0)}&=0,\\[0.5ex]
E_{21}^{\,(0)} &= - \frac{2\,{\rm i}\,Q_e}{g+{\rm i}\,Q_e},\label{e210}\\[0.5ex]
E_{22}^{\,(0)}&= 0.
\end{align}
Likewise,  Eqs.~(\ref{f1})--(\ref{f4}) imply that $\underline{\underline{F}}(p) ={\cal O}(p^2)$.

Suppose that  $\underline{\underline{W}}(p)= \underline{\underline{W}}^{\,(0)}+ \underline{\underline{W}}^{\,(1)}\,p$ at small values of $p$,
where the elements of  $\underline{\underline{W}}^{\,(0)}$ and $\underline{\underline{W}}^{\,(1)}$ are independent of $p$. 
Equation~(\ref{ricc}) gives
\begin{align}
\underline{\underline{0}}&= \underline{\underline{W}}^{\,(0)}-\underline{\underline{W}}^{\,(0)}\,\underline{\underline{W}}^{\,(0)}- 
\underline{\underline{E}}^{\,(0)}\,\underline{\underline{W}}^{\,(0)},\\[0.5ex]
\underline{\underline{0}}&= -\underline{\underline{W}}^{\,(1)}\,\underline{\underline{W}}^{\,(0)}-\underline{\underline{W}}^{\,(0)}\,\underline{\underline{W}}^{\,(1)}- 
\underline{\underline{E}}^{\,(0)}\,\underline{\underline{W}}^{\,(1)}.
\end{align}
Suitable solutions are
\begin{align}\label{w0def}
\underline{\underline{W}}^{\,(0)} &=  \left(\begin{array}{cc} -1,&0\\-E_{21}^{\,(0)}/2,&0\end{array}\right),\\[0.5ex]
W_{12}^{\,(1)} &= 0,\\[0.5ex]
W_{21}^{\,(1)}& = -\frac{E_{21}^{\,(0)}}{2}\left[W_{11}^{\,(1)}- W_{22}^{\,(1)}\right].\label{w21def}
\end{align}

At small values of $p$, let
\begin{align}
\underline{u}(p) &= \underline{u}_{-1}\,p^{-1} + \underline{u}_0,
\end{align}
where the elements of  $\underline{u}_{-1}$ (which are $y_{-1}$ and $n_{-1}$, respectively) and the elements of $\underline{u}_{0}$ (which are $y_0$ 
and $n_0$, respectively) are all constants.
Equation~(\ref{wdef}) gives 
\begin{align}
\underline{\underline{W}}^{\,(0)}\,\underline{u}_{-1} &= - \underline{u}_{-1},\\[0.5ex]
\underline{\underline{W}}^{\,(0)}\,\underline{u}_{0} + \underline{\underline{W}}^{\,(1)}\,\underline{u}_{-1} &=\underline{0}.\label{w1def}
\end{align}
Thus, making use of Eq.~(\ref{w0def}), we get
\begin{equation}
\left(\begin{array}{cc} -1,&0\\ -E_{21}^{\,(0)}/2,&0\end{array}\right) \left(\begin{array}{c}y_{-1}\\ n_{-1}\end{array}\right)= -\left(\begin{array}{c}y_{-1}\\ n_{-1}\end{array}\right),
\end{equation}
which implies that
\begin{equation}
\frac{E_{21}^{\,(0)}}{2}\,y_{-1}= - \frac{{\rm i}\,Q_e}{g+{\rm i}\,Q_e}\,y_{-1} = n_{-1},
\end{equation}
in accordance with Eqs.~(\ref{yy}) and (\ref{nn}), where use has been made of Eq.~(\ref{e210}). 
Thus, if we write
$y_{-1} = (g+{\rm i}\,Q_e)\,a_{-1}$, 
$n_{-1}= -{\rm i}\,Q_e\,a_{-1}$, 
$y_0= (g+{\rm i}\,Q_e)\,a_0$, and 
$n_0 = -{\rm i}\,Q_e\,a_0 + a_2$,
in accordance with Eqs.~(\ref{yy}) and (\ref{nn}), then we deduce from Eqs.~(\ref{e41}), (\ref{e43}), (\ref{w0def})--(\ref{w21def}),  and (\ref{w1def}) that 
\begin{equation}\label{e111}
\frac{\pi}{\hat{\mit\Delta}}\equiv \frac{a_0}{a_{-1}} = W_{11}^{\,(1)} = \frac{dW_{11}(0)}{dp}.
\end{equation}

\subsection{Large Argument Behavior of Riccati Matrix Differential Equation}
At large values of $p$, it is clear from Eqs.~(\ref{f1})--(\ref{f4}) that  $\underline{\underline{F}}(p)=\underline{\underline{F}}^{\,(6)}\,p^6+ 
\underline{\underline{F}}^{\,(8)}\,p^8$, where the elements of $\underline{\underline{F}}^{\,(6)}$ and $\underline{\underline{F}}^{\,(8)}$ are constants.
On the other hand, Eqs.~(\ref{ee1})--(\ref{ee4}) imply that $\underline{\underline{E}}(p)=\underline{\underline{E}}^{\,(0)}$, 
where the elements of $\underline{\underline{E}}^{\,(0)}$ are constants. 
Thus, if we write $\underline{\underline{W}}(p) =\underline{\underline{W}}^{\,(2)}\,p^2+\underline{\underline{W}}^{\,(4)}\,p^4$,
where the elements of $\underline{\underline{W}}^{\,(2)}$ and  $\underline{\underline{W}}^{\,(4)}$ are constants, then Eq.~(\ref{ricc}) gives 
\begin{align}\label{e138}
\underline{\underline{W}}^{\,(4)}\,\underline{\underline{W}}^{\,(4)}&= \underline{\underline{F}}^{\,(8)},\\[0.5ex]
\underline{\underline{W}}^{\,(2)}\,\underline{\underline{W}}^{\,(4)}+ \underline{\underline{W}}^{\,(4)}\,\underline{\underline{W}}^{\,(2)}&= 
\underline{\underline{F}}^{\,(6)}.\label{e139}
\end{align}
Now, according to Eqs.~(\ref{f1})--(\ref{f4}), 
\begin{align}
F^{\,(8)}_{11} &=0,\\[0.5ex]
F^{\,(8)}_{12} &= 0,\\[0.5ex]
F^{\,(8)}_{21} &= c_\beta^{\,-2}\,D^{\,2}\,P_\varphi^{\,2},\\[0.5ex]
F^{\,(8)}_{22} &= c_\beta^{\,-2}\,\iota_e^{\,-1}\,D^{\,2}\,P_\varphi^{\,2},
\end{align}
so Eq.~(\ref{e138}) yields 
\begin{align}
W^{\,(4)}_{11} &=0,\\[0.5ex]
W^{\,(4)}_{12} &= 0,\\[0.5ex]
W^{\,(4)}_{21} &= -c_\beta^{\,-1}\,\iota_e^{\,1/2}\,D\,P_\varphi,\\[0.5ex]
W^{\,(4)}_{22} &=-c_\beta^{\,-1}\,\iota_e^{\,-1/2}\, D\,P_\varphi,
\end{align}
where we have chosen the sign of the square root that is associated with well-behaved solutions at large values of $p$. Here, we are  assuming that $\iota_e>0$. 
Equations~(\ref{f1})--(\ref{f4}) also give 
\begin{align}
F^{\,(6)}_{11} &=P_\varphi,\\[0.5ex]
F^{\,(6)}_{12} &= \iota_e^{\,-1}\,P_\varphi,\\[0.5ex]
F^{\,(6)}_{21} &=c_\beta^{\,-2}\,D^{\,2}\,g\,P_\varphi + c_\beta^{\,-2}\,D^{\,2}\,(g+{\rm i}\,Q_i)\,P_\varphi,\\[0.5ex]
F^{\,(6)}_{22} &=c_\beta^{\,-2}\,\iota_e^{\,-1}\,D^{\,2}\,g\,P_\varphi + c_\beta^{\,-2}\,[P_\perp+D^{\,2}\,(g+{\rm i}\,Q_i)]\,P_\varphi.
\end{align}
Thus, Eq.~(\ref{e139}) yields
\begin{align}
W_{12}^{\,(2)}\,W_{21}^{\,(4)} &= F_{11}^{\,(6)},\\[0.5ex]
W_{12}^{\,(2)}\,W_{22}^{\,(4)} &= F_{12}^{\,(6)},
\end{align}
which gives 
\begin{align}
W_{12}^{\,(2)} &= - c_\beta\,\iota_e^{\,-1/2}\,D^{-1}.
\end{align}

Now, if
\begin{equation}\label{e155}
\underline{\underline{W}}\,\underline{u}= \lambda(p)\,\underline{u}
\end{equation}
then Eq.~(\ref{wdef}) yields
\begin{equation}
p\,\frac{d\underline{u}}{dp} = \lambda\,\underline{u},
\end{equation}
which implies that
\begin{equation}
\underline{u}(p) = \underline{u}(p_0)\,\exp\left[\int_{p_0}^p\frac{\lambda_r(p')}{p'}\,dp'\right]\exp\left[{\rm i}\int_{p_0}^p\frac{\lambda_i(p')}{p'}\,dp'\right], 
\end{equation}
where $\lambda_r$  and $\lambda_i$ are the real and imaginary parts of $\lambda$, respectively. 
Of course, a solution that is well behaved at large values of $p$ is such that $\lambda_r$ is negative. As we have seen, the large-$p$ limit of
Eq.~(\ref{ricc}) is
\begin{equation}
\underline{\underline{W}}\,\underline{\underline{W}} = \underline{\underline{F}}.
\end{equation}
Hence, if
\begin{equation}\label{e158}
 \underline{\underline{F}}\,\underline{u} = {\mit\Lambda}\,\underline{u}
 \end{equation}
 then Eqs.~(\ref{e155}) and (\ref{e158}) imply that 
 \begin{equation}
 \lambda^2 = {\mit\Lambda}.
 \end{equation}
 The eigenvalue problem for the $F$-matrix reduces to
 \begin{equation}
 {\mit\Lambda}^{\,2}- (F_{11}+ F_{22})\,{\mit\Lambda} + F_{11}\,F_{22} - F_{12}\,F_{21}=0.
 \end{equation}
 Now,
 \begin{align}
 F_{11}+F_{22}&\simeq  F_{22}^{\,(8)}\,p^8 =c_\beta^{\,-2}\, \iota_e^{\,-1}\,D^{\,2}\,P_\varphi^{\,2}\,p^8,\\[0.5ex]
 F_{11}\,F_{22} - F_{12}\,F_{21}&  \simeq \left[F_{11}^{\,(6)}\,F_{22}^{\,(8)} - F_{12}^{\,(6)}\,F_{21}^{\,(8)}\right]p^{14}\nonumber\\[0.5ex]&\phantom{=}+\left[F_{11}^{\,(6)}\,F_{22}^{\,(6)} - F_{12}^{\,(6)}\,F_{21}^{\,(6)}\right]p^{12}
=c_\beta^{\,-2}\,R\,P_\varphi^{\,2}\,p^{12},
 \end{align}
  where 
 \begin{equation}
 R= P_\perp + (1-\iota_e^{\,-1})\,D^{\,2}\,(g+{\rm i}\,Q_i),
 \end{equation}
Hence, the two eigenvalues of the $F$-matrix are
 \begin{align}
 {\mit\Lambda}_1&\simeq F_{22}^{\,(8)}\,p^8=  c_\beta^{\,-2}\,\iota_e^{\,-1}\,D^2\,P_\varphi^{\,2}\,p^8,\\[0.5ex]
 {\mit\Lambda}_2 &\simeq \frac{[F_{11}^{\,(6)}\,F_{22}^{\,(6)} - F_{12}^{\,(6)}\,F_{21}^{\,(6)}]}{F_{22}^{\,(8)}}\,p^4= \iota_e\,D^{-2}\,R\,p^4.
 \end{align}
 Thus, we deduce that the two eigenvalues of the $W$-matrix are 
 \begin{align}
 \lambda_1&=-{\mit\Lambda}_1^{\,1/2}= -c_\beta^{\,-1}\,\iota_e^{\,-1/2}\,D\,P_\varphi\,p^4,\\[0.5ex]
 \lambda_2&=-{\mit\Lambda}_2^{\,1/2}=-\iota_e^{\,1/2}\,D^{-1}\,R^{\,1/2}\,p^2,
 \end{align}
 Here, the square root of $R$ is taken such that the real part of $\lambda_2$ is negative. 
  Now, the eigenvalue problem for the $W$-matrix reduces to 
 \begin{equation}
 \lambda^{2} - W_{22}^{\,(4)}\,p^4\,\lambda + \left[W_{11}^{\,(2)}\,W_{22}^{\,(4)} - W_{12}^{\,(2)}\,W_{21}^{\,(4)}\right]p^6 = 0.
 \end{equation}
which yields
\begin{equation}
\lambda_1\simeq W_{22}^{\,(4)}\,p^4,
\end{equation}
which is satisfied, and
\begin{equation}
\lambda_2 \simeq \left[W_{11}^{\,(2)} - \frac{W_{12}^{\,(2)}\,W_{21}^{\,(4)}}{W_{22}^{\,(4)}}\right]p^2,
\end{equation}
 which implies that
 \begin{equation}
 W_{11}^{\,(2)} = -\iota_e^{\,1/2}\,D^{-1}\,R^{\,1/2}-c_\beta\,\iota_e^{\,1/2}\,D^{-1}.
 \end{equation}
 Hence, the large-$p$ boundary condition for the $W$-matrix
 is
 \begin{equation}\label{e147}
 \underline{\underline{W}}(p) =  \left(\begin{array}{cc} -\iota_e^{\,1/2}\,D^{-1}\,R^{\,1/2}\,p^2-c_\beta\,\iota_e^{\,1/2}\,D^{-1}\,p^2,& - c_\beta\,\iota_e^{\,-1/2}\,D^{-1}\,p^2\\-c_\beta^{\,-1}\,\iota_e^{\,1/2}\,D\,P_\varphi\,p^4,&-c_\beta^{\,-1}\,\iota_e^{\,-1/2}\,D\,P_\varphi\,p^4\end{array}\right).
 \end{equation}

\subsection{Method of Solution}
The method of solution is to launch the well-behaved asymptotic solution (\ref{e147}) of Eqs.~(\ref{err1})--(\ref{err4}) from large $p$, and then
integrate the equations backward to small $p$. The complex layer response parameter, $\hat{\mit\Delta}$, is then determined  from Eq.~(\ref{e111}). 

\section*{Acknowledgements}
This research was directly funded by the U.S.\ Department of Energy, Office of Science, Office of Fusion Energy Sciences,  under  contracts DE-FG02-04ER54742 and DE-SC0021156. 

\section*{Data Availability Statement}
The data that support the findings of this study are available from the corresponding author upon reasonable request.

\section*{References}
\begin{thebibliography}{99}\baselineskip 5ex

\bibitem{wes} J.A.~Wesson, Nucl.\ Fusion {\bf 18}, 87 (1978).

\bibitem{wes1} J.A.~Wesson, R.D.~Gill, M.~Hugon, F.C.~Sch\"{u}ller, J.A.~Snipes, D.J.~Ward, D.V.~Bartlett, 
D.J.~Campbell, P.A.~Duperrex, A.W.~Edwards, et al., Nucl.\ Fusion {\bf 29}, 641 (1989).

\bibitem{vries} P.C.~de\,Vries, M.F.~Johnson, B.~Alper, P.~Buratti, T.C.~Hender, H.R.~Koslowski, V.~Riccardo, and JET-EFDA Contributors, Nucl.\ Fusion {\bf 51}, 053018 (2011).

\bibitem{fkr} H.P.~Furth, J.~Killeen and M.N.~Rosenbluth, Phys.\ Fluids {\bf 6}, 459 (1963).

\bibitem{ara} G.~Ara, B.~Basu, B.~Coppi, G.~Laval, M.N.~Rosenbluth and B.V.~Waddell, Ann.\ Phys.\ (NY) {\bf 112}, 443 (1978).

\bibitem{drake} J.F.~Drake and Y.C.~Lee, Phys.\ Fluids {\bf 20}, 1341 (1977).

\bibitem{wal} F.L.~Waelbroeck, Phys.\ Plasmas {\bf 10}, 4040 (2003).

\bibitem{cole} A.~Cole and R.~Fitzpatrick, Phys.\ Plasmas {\bf 13}, 032503 (2006).

\bibitem{fw} R.~Fitzpatrick and F.L.~Waelbroeck, Phys. Plasmas {\bf 12}, 022307 (2005).

\bibitem{haz} R.D.~Hazeltine, M.~Kotschenreuther and P.G.~Morrison, Phys.\ Fluids {\bf 28}, 2466 (1985).

\bibitem{rf2022} R. Fitzpatrick,  Phys.\ Plasmas {\bf 29}, 032507 (2022).

\bibitem{lee} Y.~Lee, J.-K.~Park and Y.-S.~Na, Nucl.\ Fusion {\bf 64}, 106058 (2024).

\bibitem{ric1} D.P.~Brennan, A.J.~Cole, C.~Akcay and J.M.~Finn, Bul.\ Am.\ Phys.\ Soc.\ {\bf 64}, 249 (2019).

\bibitem{ric2} J.-K.~Park, Phys.\ Plasmas {\bf 29}, 072506 (2022).

\bibitem{paz} C.~Paz-Soldan, R.~Nazikian, L.~Cui, B.C.~Lyons, D.M.~Orlov, A.~Kirk, 
N.C.~Logan, T.H.~Osborne, W.~Suttrop and D.B.~Weisberg, Nucl.\ Fusion {\bf 59},
056012 (2019).

\bibitem{rf1993} R.~Fitzpatrick, Nucl.\ Fusion {\bf 33}, 1049 (1993).

\end{thebibliography}

\end{document}