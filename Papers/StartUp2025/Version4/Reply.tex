\documentclass{article}[12pt]
\usepackage{fullpage}
\usepackage{amsmath}

\begin{document}
\begin{center}
{\em  A  Simple Model of Current Ramp-Up and Ramp-Down in Tokamaks}\\[1ex]
by R.~Fitzpatrick\\[1ex]
{\bf Reply to Referees' Comments}\\[1ex]
~
\end{center}
Let me thank the  referees for their helpful and insightful comments on my paper. Here are my responses to
their comments.

\section*{Referee 1}
\begin{enumerate}
\item I have applied my model to EU-DEMO. There is a clear trend as one goes from JET to ITER to DEMO. Obviously,
the ramp times get longer. However, prospective problems with runaway electron generation during current ramps get progressively
less serious. 

\item I have added a discussion of the L-H transition in Section~2. If it is not true that the forward and backward transitions take place
in the flat-top, as it seems likely to be the case in ITER and DEMO, then the plasma temperature during part of the ramp is going to
be higher than that assumed in my model. Hence, the ramp times will be increased. 
\end{enumerate}

\section*{Referee 2}
\begin{enumerate}
\item I have now mentioned in the Abstract and the Summary that my estimates are for ohmic ramp phases with relatively low flat-top
temperatures. I have also mentioned in Section~2 that the ITER ramp-down will be initiated in a burning plasma.
\item I have mentioned in Section~2 that ITER plasmas may require auxiliary electron heating during the current ramp-down phase
in order to prevent highly-destabilizing hollow temperature and current profiles due to tungsten impurity radiation in the plasma core. 
\item I have referenced the Koechl and Van Mulders papers in the Summary and Discussion, and have indicated that these papers
discuss a range of issues, other than Faraday's law, that limit the ramp-down rate in real tokamaks. 
\item I have mentioned in Section~2 that ITER will maintain the magnetic X-point during a current ramp-down for as long as possible. 
\item I have mentioned in Section~3.7 that peaked current profiles make it difficult to maintain vertical stability.
\item I have fixed the misspelling of Dreicer.
\item In Section~3.8, I make it clear that the rescalings (46) and (51) are the only possible ones that leave Eqs.~(47) and (52)
independent of the explicitly time-varying parameters ${\cal E}$ and $\delta$. The fact that Eqs.~(47) and (52), and the boudary
conditions to which they are subject, are free of explicitly time-varying quantities justifies the assumption that their
solutions, $\bar{T}(x)$ and $\bar{B}_\theta(x)$, respectively, are independent of $\hat{t}$. 
\item I have added more explanation in Sections~4.1.1 and 4.1.2 to make it clear how Eqs.~(62) and (63), and (62) and (67), really
do allow us to determine all other quantities.
\item I calculated the $\chi_\perp$ values predicted by the ITER98(y,2) energy confinement scaling law, assuming that the heating power is
ohmic. With an H-factor of 1, I found $\chi_\perp$ values lying between $1\,{\rm m}^2\,{\rm s}^{-2}$ and $2\,{\rm m}^2\,{\rm s}^{-1}$ 
in all four of the machines considered in the paper. Thus, the estimate $\chi_\perp\simeq 1\,{\rm m}^2\,{\rm s}^{-1}$ is actually a pretty good one. 
I have added a comment to this effect in Section~5.2.
\end{enumerate}

\end{document}