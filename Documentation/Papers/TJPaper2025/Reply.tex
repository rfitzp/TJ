\documentclass{article}[12pt]
\usepackage{fullpage}
\usepackage{amsmath}

\begin{document}
\begin{center}
{\em  Investigation of  Tearing Mode Stability Near Ideal Stability Boundaries Via Asymptotic Matching Techniques}\\[1ex]
by R.~Fitzpatrick\\[1ex]
{\bf Reply to Referees' Comments}\\[1ex]
~
\end{center}
Let me thank the  referees for their helpful and insightful comments on my paper. Here are my responses to
their comments.

\begin{description}

\item \underline{Introduction, 4th paragraph.} I have clarified what I mean by a ``fixed-boundary'' and a ``free-boundary'' calculation. 

\item \underline{Introduction, last paragraph.} I have deleted a sentence that mentions ``internal'' and ``external'' modes, because these
terms will be defined later in the paper. 

\item \underline{Sect.~II.B, penultimate paragraph.}  I have deleted the redundant term ``free boundary''. 

\item \underline{Sect.~II.B, last paragraph.} I have explained why the particular choice of the range of poloidal mode numbers in the TJ code was made.
I have also added Table~I, which shows that the particular choice is sufficient to obtain convergence. 

\item \underline{Sect.~II.C.1.} I have explained that the TEAR code is part of the TJ package. Unfortunately, there is no reference for this code. But, any
interested party can access the code on github. 

\item \underline{Sect.~II.C.2.} I have deleted ``external'', because this term is yet to be defined. 

\item \underline{Sect.~II.E.} I have deleted a repeated ``that''. 

\item \underline{Sect.~III.A, after paragraph 3.} I have added a paragraph that explains more exactly the criterion that has to be satisfied in order for
the linear decoupling of the tearing mode dispersion relation by sheared plasma rotation to be effective. 

\item \underline{Sect.~III.C.}  I have explained that TJ actually performs two tests for ideal stability. The first, which I forgot to mention previously,
is a test for Mercier stability. The second involves a calculation of the eigenvalues of the sum of the plasma and vacuum perturbed energy matrices. I mention that
the second test is a necessary and sufficient test for ideal stability. This implies that if the test is viable then no other test is required. 
It turns out that, in a toroidal plasma, the second test is always viable, because there is no such thing as a truly internal mode. 
I have deleted the reference to the first stability test performed by DCON---here, and in the rest of the paper---because I am now not sure what sort of mode would be
detected by this test. The DCON test does not include the vacuum, so it seems that it could only apply to a mode that was truly fixed boundary. However,
in practice, there is no such mode. 

\item \underline{Sect.~III.D.} I have added two paragraphs to clarify what I mean by an ``external'' and an ``internal'' mode. 

\item \underline{Sect.~III.E.4.} I have tried to explain more clearly how $f_1$ and $f_2$ are defined.

\item \underline{Sect.~III.E.5, paragraph 1.}  I have clarified that you can only see that the 2/1 mode has $\psi_{m=2}\neq 0$ at the $q=2$ surface, and $\psi_{m=3}=0$ at the $q=3$ surface, in Fig.~8. 
Figure 9 plots the sum of all poloidal harmonics, which obscures the fact that the 2/1 mode has  $\psi_{m=3}=0$ at the $q=3$ surface because the non-resonant harmonics are
non-zero at this surface. 

\item \underline{Sect.~III.E.5, paragraph 2.}  I have clarified that, as the ideal stability boundary is approached, the various tearing mode eigenfunctions only
morph into the ideal eigenfunction in the outer region. I have also tried to explain why this morphing occurs. 

\item \underline{Figures 6, 7, 15, 16, and 17.} I have improved many of the figures by adding inserts that show, for instance, that $\gamma_2$ becomes negative in a particular range of $\beta_0$ values in Fig.~7, and that
$\psi_{m=1}$ is non-zero at the $q=1$ surface in the top panel of Fig.~17, but is zero in the bottom panel. 


\end{description}

\end{document}