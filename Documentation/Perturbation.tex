\documentclass[12pt,prb,aps,notitlepage]{revtex4-1}
\usepackage {amsmath}
\usepackage{amssymb}
\pdfoutput = 1 
\usepackage {graphicx}
\newcommand{\bomega}{\mbox{\boldmath$\omega$}}
\allowdisplaybreaks

\begin{document}

\title{Magnetic Perturbations}
\author{R.~Fitzpatrick\,\footnote{rfitzp@utexas.edu}}
\affiliation{Institute for Fusion Studies,  Department of Physics,  University of Texas at Austin,  Austin TX 78712, USA}\begin{abstract}
\end{abstract}
\maketitle

\section{Magnetic Perturbations in Flux Coordinates}
In the $r$, $\theta$, $\phi$ flux coordinate system (where all lengths are normalized to $R_0$, and all magnetic field-strengths to $B_0$), the perturbed magnetic
field is written
\begin{equation}
{\bf b} = b^r\,{\cal J}\,\nabla\theta\times \nabla\phi + b^\theta\,{\cal J}\,\nabla\phi\times\nabla r + b^\phi\,{\cal J}\,\nabla r\times\nabla \theta.
\end{equation}
 T7 Eqs.~(25) and (49) yield
\begin{align}
b^r &=\frac{1}{r\,R^{\,2}}\left(\frac{\partial}{\partial\theta} - {\rm i}\,n\,q\right)y,\\[0.5ex]
b^\phi &= \frac{x}{R^{\,2}},
\end{align}
whereas TJ  Eqs.~(78), (79), (80), (98), and (99) imply that
\begin{align}
y(r,\theta) &= \sum_m\frac{\psi_m(r)}{m-n\,q(r)}\,{\rm e}^{\,{\rm i}\,m\,\theta},\\[0.5ex]
x(r,\theta)&= n\sum_m \frac{Z_m(r)+ k_m(r)\,\psi_m(r)}{m-n\,q(r)}\,{\rm e}^{\,{\rm i}\,m\,\theta}. 
\end{align}

It follows that
\begin{align}
R^{\,2}\,b^r &= \frac{{\rm i}}{r}\sum_m \psi_m\,{\rm e}^{\,{\rm i}\,m \,\theta},\\[0.5ex]
R^{\,2}\,b^\phi &= n\sum_m z_m\,{\rm e}^{\,{\rm i}\,m\,\theta},
\end{align}
where 
\begin{equation}
z_m =  \frac{Z_m+ k_m\,\psi_m}{m-n\,q}.
\end{equation}
Now,  TJ Eqs.~(2) and (A10) yield
\begin{align}
{\cal J}\,\nabla\cdot{\bf b} = \frac{\partial}{\partial r}\left(r\,R^{\,2}\,b^r\right) + \frac{\partial}{\partial \theta}\left(r\,R^{\,2}\,b^\theta\right) - {\rm i}\,n\,r\,R^{\,2}\,b^\phi = 0,
\end{align}
so
\begin{equation}
\frac{\partial}{\partial\theta}\,(r^2\,R^{\,2}\,b^\theta)={\rm i}\sum_m\left[-r\,\frac{d\psi_m}{dr}+ \frac{n^2\,r^2\,(Z_m+k_m\,\psi_m)}{m-n\,q}\right]{\rm e}^{\,{\rm i}\,m\,\theta}.
\end{equation}
But, TJ Eq.~(102) gives
\begin{equation}
r\,\frac{d\psi_m}{dr}=\sum_{m'} \frac{L_{m}^{\,m'}\,Z_{m'} + M_m^{\,m'}\,\psi_{m'}}{m'-n\,q},
\end{equation}
so
\begin{equation}
\frac{\partial}{\partial\theta}\,(r^2\,R^{\,2}\,b^\theta)=-{\rm i}\sum_{m}\,\chi_m\,{\rm e}^{\,{\rm i}\,m\,\theta},
\end{equation}
where
\begin{align}
\chi_m(r)& = \sum_{m'}\frac{\chi_m^{\,m'}}{m'-n\,q},\\[0.5ex]
\chi_m^{\,m'}(r) &= (L_m^{\,m'}- n^2\,r^2\,\delta_m^{\,m'})\,Z_{m'} + (M_m^{\,m'}-n^2\,r^2\,\delta_{m}^{\,m'}\,k_{m'})\,\psi_{m'}
\end{align}
Note from TJ Eqs.~(100), (262), (263), (266), (267), (268), and (269) that $\chi_0^{\,m'}= 0$ for all $m'$, which implies that $\chi_0(r)=0$. 
Thus,
\begin{equation}
r\,R^{\,2}\,b^\theta = -\frac{1}{r}\sum_{m}^{m\neq 0} \hat{\chi}_m\,{\rm e}^{\,{\rm i}\,m\,\theta},
\end{equation}
where 
\begin{equation}
\hat{\chi}_m(r) =\frac{ \chi_m}{m}.
\end{equation} 

According to Eqs.~(4), (26), (79), and (98) of TJ, 
\begin{equation}
\xi^r = \frac{q}{r\,g}\sum_m \hat{\psi}_m\,{\rm e}^{\,{\rm i}\,m\,\theta},
\end{equation}
where
\begin{equation}
\hat{\psi}_m(r) = \frac{\psi_m}{m-n\,q}.
\end{equation}

\section{Regularization}
In evaluating the perturbed magnetic field associated with the unreconnected eigenfunction at the $k$th rational surface, we make the
transformation
\begin{align}
\frac{1}{m_k-n\,q} \rightarrow \frac{m_k-n\,q}{\delta_k^{\,2} + (m_k-n\,q)^2},
\end{align}
where $m_k$ is the resonant poloidal mode number at the $k$th rational surface, and
\begin{equation}
\delta_k = \frac{m_k\,s(r_k)}{2\,\hat{r}_k}\,\frac{W_k}{\epsilon_a}.
\end{equation}
Here, $r_k=\epsilon_a\,\hat{r}_k$ is the minor radius of the $k$th rational surface, $s(r)$ is the magnetic shear, and 
\begin{equation}
W_k = 4\left(\frac{q}{g\,s}\right)_{r_k}^{1/2}{\mit\Psi}_k^{\,1/2}
\end{equation}
is the magnetic island width at the $k$th rational surface. 
Thus, 
\begin{equation}
\frac{{\mit\Psi}_k}{\epsilon_a}= \epsilon_a \left(\frac{W_k}{4\,\epsilon_a}\right)^2\left(\frac{g\,s}{q}\right)_{r_k}.
\end{equation}
For the special case of an $m=1$ mode,
\begin{align}
\delta_k &= \frac{m_k\,s(r_k)}{2\,\hat{r}_k}\,\frac{\xi_k}{\epsilon_a},\\[0.5ex]
\frac{{\mit\Psi}_k}{\epsilon_a} &= \epsilon_a\left(\frac{\hat{r}\,g\,s}{q}\right)_{r_k}\frac{1}{|E_{kk}|} \left(\frac{\xi_k}{\epsilon_a}\right),
\end{align}
where $\xi_k$ is the displacement of the plasma core. 

\section{Magnetic Perturbations in Cylindrical Coordinates}
We can write the perturbed magnetic field associated with the tearing mode that reconnects magnetic flux at the $k$th rational surface as
\begin{align}
b^R &\equiv {\bf b}\cdot\nabla R = b^r\,\frac{\partial R}{\partial r}+ b^\theta\,\frac{\partial R}{\partial\theta},\\[0.5ex]
b^Z &\equiv {\bf b}\cdot\nabla Z = b^r\,\frac{\partial Z}{\partial r}+ b^\theta\,\frac{\partial Z}{\partial\theta},\\[0.5ex]
R\,b^\phi&\equiv R\,{\bf b}\cdot\nabla\phi = R\,b^\phi.
\end{align}
Thus,
\begin{align}
b^R(r,\theta,\phi) &= b^{\,R}_C(r,\theta)\,\cos(n\,\phi) + b_S^{\,R}(r,\theta)\,\sin(n\,\phi),\\[0.5ex]
b^Z(r,\theta,\phi) &= b^{\,Z}_C(r,\theta)\,\cos(n\,\phi) + b_S^{\,Z}(r,\theta)\,\sin(n\,\phi),\\[0.5ex]
R\,b^\phi(r,\theta,\phi) &= b^{\,\phi}_C(r,\theta)\,\cos(n\,\phi) + b_S^{\,\phi}(r,\theta)\,\sin(n\,\phi),\\[0.5ex]\xi^r(r,\theta,\phi)&= \xi^{\,r}_{C}(r,\theta)\,\cos(n\,\phi) + \xi^{\,r}_{S}\,\sin(n\,\phi),
\end{align}
where
\begin{align}
b_C^{\,R}(r,\theta) &= -\frac{{\mit\Psi}_k}{\epsilon_a\,\hat{r}\,R^{\,2}}\left\{\frac{1}{\epsilon_a}\frac{\partial R}{\partial \hat{r}}\sum_m\left[{\rm Re}(\psi_m)\,\sin(m\,\theta) + {\rm Im}(\psi_m)\,\cos(m\,\theta)\right]\right.\\[0.5ex]
&\phantom{=} \left.+ \frac{1}{\epsilon_a\,\hat{r}}\,\frac{\partial R}{\partial\theta}\sum_{m\neq 0}\left[{\rm Re}(\hat{\chi}_m)\,\cos(m\,\theta) - {\rm Im}(\hat{\chi}_m)\,\sin(m\,\theta)\right]\right\},\\[0.5ex]
b_S^{\,R}(r,\theta) &= -\frac{{\mit\Psi}_k}{\epsilon_a\,\hat{r}\,R^{\,2}}\left\{\frac{1}{\epsilon_a}\frac{\partial R}{\partial \hat{r}}\sum_m\left[-{\rm Re}(\psi_m)\,\cos(m\,\theta) + {\rm Im}(\psi_m)\,\sin(m\,\theta)\right]\right.\\[0.5ex]
&\phantom{=} \left.+ \frac{1}{\epsilon_a\,\hat{r}}\,\frac{\partial R}{\partial\theta}\sum_{m\neq 0}\left[{\rm Re}(\hat{\chi}_m)\,\sin(m\,\theta) + {\rm Im}(\hat{\chi}_m)\,\cos(m\,\theta)\right]\right\},\\[0.5ex]
b_C^{\,Z}(r,\theta) &= -\frac{{\mit\Psi}_k}{\epsilon_a\,\hat{r}\,R^{\,2}}\left\{\frac{1}{\epsilon_a}\frac{\partial Z}{\partial r}\sum_m\left[{\rm Re}(\psi_m)\,\sin(m\,\theta) + {\rm Im}(\psi_m)\,\cos(m\,\theta)\right]\right.\\[0.5ex]
&\phantom{=} \left.+ \frac{1}{\epsilon_a\,\hat{r}}\,\frac{\partial Z}{\partial\theta}\sum_{m\neq 0}\left[{\rm Re}(\hat{\chi}_m)\,\cos(m\,\theta) - {\rm Im}(\hat{\chi}_m)\,\sin(m\,\theta)\right]\right\},\\[0.5ex]
b_S^{\,Z}(r,\theta) &= -\frac{{\mit\Psi}_k}{\epsilon_a\,\hat{r}\,R^{\,2}}\left\{\frac{1}{\epsilon_a}\frac{\partial Z}{\partial \hat{r}}\sum_m\left[-{\rm Re}(\psi_m)\,\cos(m\,\theta) + {\rm Im}(\psi_m)\,\sin(m\,\theta)\right]\right.\\[0.5ex]
&\phantom{=} \left.+ \frac{1}{\epsilon_a\,\hat{r}}\,\frac{\partial Z}{\partial\theta}\sum_{m\neq 0}\left[{\rm Re}(\hat{\chi}_m)\,\sin(m\,\theta) +{\rm Im}(\hat{\chi}_m)\,\cos(m\,\theta)\right]\right\},\\[0.5ex]
b_C^{\,\phi}(r,\theta) & =\frac{n\,{\mit\Psi}_k}{R}\sum_{m}\left[{\rm Re}(z_m)\,\cos(m\,\theta) - {\rm Im}(z_m)\,\sin(m\,\theta)\right],\\[0.5ex]
b_S^{\,\phi}(r,\theta) &= \frac{n\,{\mit\Psi}_k}{R}\sum_{m}\left[{\rm Re}(z_m)\,\sin(m\,\theta) + {\rm Im}(z_m)\,\cos(m\,\theta)\right],\\[0.5ex]
\xi_C^{\,r}(r,\theta) & =\frac{{\mit\Psi}_k\,q}{\epsilon_a\,\hat{r}\,g}\sum_{m}\left[{\rm Re}(\hat{\psi}_m)\,\cos(m\,\theta) - {\rm Im}(\hat{\psi}_m)\,\sin(m\,\theta)\right],\\[0.5ex]
\xi_S^{\,r}(r,\theta) &= \frac{{\mit\Psi}_k\,q}{\epsilon_a\,\hat{r}\,g}\sum_{m}\left[{\rm Re}(\hat{\psi}_m)\,\sin(m\,\theta) + {\rm Im}(\hat{\psi}_m)\,\cos(m\,\theta)\right].
\end{align}

\section{Value of $k_m$}
Now, 
\begin{align}
\tilde{k}_m&= -\frac{2-s}{m}
-\frac{\epsilon_a^{\,2}}{m}\left(-\hat{r}\,p_2'+
\frac{3\,\hat{r}^{\,2}}{2}-2\,\hat{r}\,H_1' +S_2\right)\nonumber\\[0.5ex]
&\phantom{=}
+\epsilon_a^{\,2}\,\frac{(2-s)}{m}\left(-\frac{3\,\hat{r}^{\,2}}{4} +\frac{\hat{r}^{\,2}}{q^2}+H_1+S_1\right)\nonumber\\[0.5ex]
&\phantom{=}
+\epsilon_a^{\,2}\,\frac{n\,\hat{r}}{m^2}\left[-q\,p_2' + \frac{\hat{r}}{m\,q}\,(2-s)\,(m-n\,q)\right].
\end{align}
But, for the special case $m=0$, 
\begin{align}
\tilde{k}_0 &= - \frac{q\,p_2'}{n\,\hat{r}} - \frac{2-s}{n\,q} 
\nonumber\\[0.5ex]
&\phantom{=}-\frac{\epsilon_a^{\,2}}{n\,q}\left(
\frac{3\,\hat{r}^{\,2}}{2}-2\,\hat{r}\,H_1+S_2\right)\nonumber\\[0.5ex]
&\phantom{=}
+\epsilon_a^{\,2}\,\frac{(2-s)}{n\,q}\,\left(-\frac{3\,\hat{r}^{\,2}}{4} +\frac{\hat{r}^{\,2}}{q^2}+H_1 +S_1\right)\nonumber\\[0.5ex]
&\phantom{=}
+\epsilon_a^{\,2}\,\frac{q\,p_2'}{n\,\hat{r}}\left(2\,g_2+\frac{\hat{r}^{\,2}}{2}+\frac{\hat{r}^{\,2}}{q^2}-2\,H_1-3\,\hat{r}\,H_1'\right).
\end{align}
It turns out that
\begin{equation}
k_m = \tilde{k}_m - \tilde{k}_0
\end{equation}
for all $m$. 

\end{document}