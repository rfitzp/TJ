\documentclass[12pt,prb,aps,notitlepage]{revtex4-1}
\usepackage {amsmath}
\usepackage{amssymb}
\pdfoutput = 1 
\usepackage {graphicx}
\usepackage{cancel}
\newcommand{\bomega}{\mbox{\boldmath$\omega$}}
\newcommand{\bPi}{\mbox{\boldmath$\Pi$}}
\allowdisplaybreaks
\newcommand{\smalltensor}[1]{\overset{\text{\tiny$\leftrightarrow$}}{#1}}

\begin{document}

\title{Neoclassical Toroidal Viscosity}
\author{R.~Fitzpatrick\,\footnote{rfitzp@utexas.edu}}
\affiliation{Institute for Fusion Studies,  Department of Physics,  University of Texas at Austin,  Austin TX 78712, USA}\begin{abstract}
\end{abstract}
\maketitle

\section{Curvilinear Coordinates}
Let ${\bf x}=(x_1,\,x_2,\,x_3)$ be a position vector, where the $x_i$ are Cartesian coordinates, and $i$ runs from 1 to 3. 
Let the $q_i(x_1,x_2,x_3)$ be curvilinear coordinates. We can define the contravariant basis vectors,
\begin{equation}\label{e1}
{\bf e}^i = \frac{\partial q^i}{\partial {\bf x}},
\end{equation}
and the covariant basis vectors, 
\begin{equation}\label{e2}
{\bf e}_i = \frac{\partial {\bf x}}{\partial q^i}.
\end{equation}
Note that
\begin{equation}\label{e3}
{\bf e}^i\cdot{\bf e}_j = \frac{\partial q^i}{\partial {\bf x}}\,\frac{\partial {\bf x}}{\partial q^j}= \frac{\partial q^i}{\partial x_k}\,\frac{\partial x_k}{\partial q^j}=
\frac{\partial q^i}{\partial q^j} = \delta^i_j.
\end{equation}
Here, use has been made of the Einstein summation convention, as well as the chain rule. 

The Jacobian, ${\cal J}$,  is defined as
\begin{equation}
{\cal J}^{-1} = \frac{\partial q^1}{\partial {\bf x}}\cdot\frac{\partial q^2}{\partial{\bf x}}\times\frac{\partial q^3}{\partial{\bf x}}.
\end{equation}
It is easily seen that
\begin{equation}
{\bf e}_i = {\cal J}\,\frac{\partial q^j}{\partial {\bf x}}\times \frac{\partial{q^k}}{\partial{\bf x}},
\end{equation}
where $i$, $j$, $k$ are cyclic. To demonstrate this we need to show that ${\bf e}_i\cdot{\bf e}^l = \delta_i^l$, 
which implies that 
\begin{equation}
 {\cal J}\,\frac{\partial q^l}{\partial {\bf x}}\cdot\frac{\partial q^j}{\partial {\bf x}}\times \frac{\partial{q^k}}{\partial{\bf x}}= \delta_i^l,
 \end{equation}
 which is obviously satisfied. 
 
 The contravariant components, $a^i$,  of a vector ${\bf a}$ are defined via
 \begin{equation}
 {\bf a} = a^i\,{\bf e}_i.
 \end{equation}
 The covariant components, $a_i$, are defined via
 \begin{equation}
  {\bf a} = a_i\,{\bf e}^i.
  \end{equation}
  Thus,
  \begin{equation}
  {\bf a}\cdot{\bf b} = a^i\,b_j\,{\bf e}_i\cdot{\bf e}^j = a^i\,b_j\,\delta_i^j = a^i\,b_i,
  \end{equation}
  where use has been made of Eq.~(\ref{e3}). 
  Similarly, 
  \begin{equation}
  {\bf a}\cdot{\bf b} = a_i\,b^{\,j}\,{\bf e}^i\cdot{\bf e}_j = a_i\,b^{\,j}\,\delta^i_j = a_i\,b^i,
  \end{equation}
  Note that
  \begin{align}
  a^i &= {\bf a}\cdot{\bf e}^i,\\[0.5ex]
  a_i &= {\bf a}\cdot{\bf e}_i.
    \end{align}
    
 The contravariant components of the metric tensor, $\smalltensor{\bf g}$, are defined
 \begin{equation}
 g^{ij} = {\bf e}^i\cdot{\bf e}^{\,j}.
 \end{equation}
 Note that $g^{ij}=g^{ji}$. 
 Likewise, the covariant components of the metric tensor are  
 \begin{equation}
 g_{ij} = {\bf e}_i\cdot{\bf e}_j.
 \end{equation}
 Note that $g_{ij}= g_{ji}$. 
 Now, 
 \begin{equation}
 {\bf e}^i = ({\bf e}^i\cdot{\bf e}^{\,j})\,{\bf e}_j,
 \end{equation}
so
\begin{equation}
{\bf a}\cdot{\bf e}^i = {\bf a}\cdot[ ({\bf e}^i\cdot{\bf e}^{\,j})\,{\bf e}_j]=( {\bf e}^i\cdot{\bf e}^{\,j})\,{\bf a}\cdot{\bf e}_j,
\end{equation}
which yields
\begin{equation}\label{e17}
a^i = g^{ij}\,a_j.
\end{equation}
Likewise, 
\begin{equation}
 {\bf e}_i = ({\bf e}_i\cdot{\bf e}_j)\,{\bf e}^{\,j},
 \end{equation}
so 
\begin{equation}
{\bf a}\cdot{\bf e}_i = {\bf a}\cdot[ ({\bf e}_i\cdot{\bf e}_j)\,{\bf e}^{\,j}]= ({\bf e}_i\cdot{\bf e}_j)\,{\bf a}\cdot{\bf e}^{\,j},
\end{equation}
which yields
\begin{equation}\label{e20}
a_i = g_{ij}\,a^{\,j}.
\end{equation}
Combining Eqs.~(\ref{e17}) and (\ref{e20}), we get
\begin{equation}
a^i = g^{ij}\,g_{jk}\,a^k = \delta^{i}_k\,a^k,
\end{equation}
which implies that
\begin{equation}
g^{ij}\,g_{jk}= \delta^{\,i}_k.
\end{equation}
Likewise, 
\begin{equation}
a_i = g_{ij}\,g^{\,jk}\,a_k = \delta_{i}^k\,a_k,
\end{equation}
which implies that
\begin{equation}
g_{ij}\,g^{\,jk}= \delta_{i}^k.
\end{equation}

The Christoffel symbol, ${\mit\Gamma}_{k,ij}$ is defined such that 
\begin{equation}\label{e25}
\frac{\partial {\bf e}_i}{\partial q^{\,j}} = {\mit\Gamma}_{k,ij}\,{\bf e}^k.
\end{equation}
It follows that 
\begin{equation}
{\mit\Gamma}_{k,ij} = {\bf e}_k\cdot\,\frac{\partial {\bf e}_i}{\partial q^j}.
\end{equation}
Thus, 
\begin{align}
\frac{\partial g_{ij}}{\partial q^k} = \frac{\partial ({\bf e}_i\cdot{\bf e}_j)}{\partial q^k}= \frac{\partial {\bf e}_i}{\partial q^k}\cdot{\bf e}_j + {\bf e}_i\cdot\frac{\partial {\bf e}_j}{\partial q^k}= {\mit\Gamma}_{j,ik} + {\mit\Gamma}_{i,jk},\\[0.5ex]
\frac{\partial g_{ik}}{\partial q^{\,j}} = \frac{\partial ({\bf e}_i\cdot{\bf e}_k)}{\partial q^{\,j}}= \frac{\partial {\bf e}_i}{\partial q^{\,j}}\cdot{\bf e}_k+ {\bf e}_i\cdot\frac{\partial {\bf e}_k}{\partial q^{\,j}}= {\mit\Gamma}_{k,ij} + {\mit\Gamma}_{i,kj},\\[0.5ex]
\frac{\partial g_{jk}}{\partial q^i} = \frac{\partial ({\bf e}_j\cdot{\bf e}_k)}{\partial q^i}= \frac{\partial {\bf e}_j}{\partial q^i}\cdot{\bf e}_k+ {\bf e}_j\cdot\frac{\partial {\bf e}_k}{\partial q^i}= {\mit\Gamma}_{k,ji} + {\mit\Gamma}_{j,ki}.
\end{align}
However,
\begin{equation}
\frac{\partial{\bf e}_i}{\partial q^{\,j}} = \frac{\partial^2{\bf x}}{\partial q^i\,\partial q^{\,j}}= \frac{\partial^2{\bf x}}{\partial q^{\,j}\,\partial q^{i}}
= \frac{\partial{\bf e}_j}{\partial q^i},
\end{equation}
where use has been made of Eq.~(\ref{e2}). Hence, we deduce that
\begin{equation}
{\mit\Gamma}_{k,ij}= {\mit\Gamma}_{k,ji}.
\end{equation}
It follows that
\begin{align}
\frac{\partial g_{ij}}{\partial q^k} + \frac{\partial g_{jk}}{\partial q^i} -\frac{\partial g_{ik}}{\partial q^{\,j}}& = {\mit\Gamma}_{j,ik} + {\mit\Gamma}_{i,jk}+ {\mit\Gamma}_{k,ji} + {\mit\Gamma}_{j,ki}-{\mit\Gamma}_{k,ij} - {\mit\Gamma}_{i,kj}\nonumber\\[0.5ex]
&= {\mit\Gamma}_{j,ik} + \cancel{{\mit\Gamma}_{i,jk}}+ \cancel{{\mit\Gamma}_{k,ji}} + {\mit\Gamma}_{j,ik}-\cancel{{\mit\Gamma}_{k,ji}} - \cancel{{\mit\Gamma}_{i,jk}},
\end{align}
which implies that
\begin{equation}
 {\mit\Gamma}_{j,ik}=\frac{1}{2}\left(\frac{\partial g_{ij}}{\partial q^k} + \frac{\partial g_{jk}}{\partial q^i} -\frac{\partial g_{ik}}{\partial q^{\,j}}\right),
 \end{equation}
 or 
 \begin{equation}
 {\mit\Gamma}_{i,jk}=\frac{1}{2}\left(\frac{\partial g_{ij}}{\partial q^k} + \frac{\partial g_{ik}}{\partial q^{\,j}} -\frac{\partial g_{jk}}{\partial q^{i}}\right).
 \end{equation}
 The previous two equations yield
 \begin{equation}\label{e35}
 \frac{\partial g_{ij}}{\partial q^k} = {\mit\Gamma}_{i,jk} + {\mit\Gamma}_{j,ik}.
 \end{equation}
 
 Let us define the Jacobian matrix
 \begin{equation}
 J_{ij} = \frac{\partial x_j}{\partial q^i}.
 \end{equation}
 The Jacobian is the determinant of the Jacobian matrix
 \begin{equation}
 {\cal J} = ||J_{ij}||= \frac{\partial{\bf x}}{\partial q^1}\cdot\frac{\partial {\bf x}}{\partial q^2}\times\frac{\partial{\bf x}}{\partial q^3}.
 \end{equation}
 Now,
 \begin{equation}
 g_{ij} = {\bf e}_i\cdot{\bf e}_j= \frac{\partial{\bf x}}{\partial q^i}\cdot\frac{{\partial {\bf x}}}{\partial q^{\,j}}= \frac{\partial x_k}{\partial q^i}\,\frac{{\partial x_k}}{\partial q^{\,j}}= J_{ik}\,J_{jk} = J_{ij}\,J^{\,T}_{kj},
 \end{equation}
 where $J^{\,T}_{ij}= J_{ji}$. However, $||A\,B|| = ||A||\,||B||$ and $||A^{\,T}|| = ||A||$. Hence, we deduce that
 \begin{equation}\label{e39}
 ||g_{ij}|| = {\cal J}^{\,2}.
 \end{equation}
 
 Jacobi's formula states that
 \begin{equation}
 \frac{\partial ||A||}{\partial q^k} = ||A||\,{\rm tr}\left(A^{-1}\,\frac{\partial A}{\partial q^k}\right)
 \end{equation}
 for any square matrix, $A_{ij}(q_1,q_2,q_3)$. 
 Let $A_{ij}= g_{ij}$. Now, the inverse of $g_{ij}$ is $g^{ij}$. Hence, making use of Eq.~(\ref{e39}), we deduce that
 \begin{equation}
 \frac{\partial{\cal J}^{\,2}}{\partial q^k} = {\cal J}^{\,2}\,g^{ij}\,\frac{\partial g_{ji}}{\partial q^k},
 \end{equation}
 which reduces to
 \begin{equation}\label{e42}
 \frac{\partial g_{ij}}{\partial q^k}\,g^{ij} = \frac{2}{\cal J}\,\frac{\partial{\cal J}}{\partial q^k}.
 \end{equation}
 
\section{Equilibrium Magnetic Field}
Let $\psi$, $\theta$, $\alpha$ be magnetic coordinates, where $\psi$ is a flux-surface label, $\theta$ a poloidal angle, and $\alpha$
a helical angle. Let ${\cal  J}^{-1}= \nabla\psi\cdot\nabla\theta\times \nabla\alpha$. Suppose that the geometric toroidal angle is $\varphi = \alpha+ q(\psi)\,\theta$. Let the equilibrium magnetic field
take the form
\begin{equation}\label{e44}
{\bf B} =  \nabla\alpha\times\nabla\psi={\cal J}^{-1}\,{\bf e}_\theta.
\end{equation}
It follows that 
\begin{equation}
{\bf b} \equiv \frac{{\bf B}}{B}= ({\cal J}\,B)^{-1}\,{\bf e}_\theta.
\end{equation}
Thus,
\begin{equation}
({\bf b}\,{\bf b})^{ij}= ({\cal J}\,B)^{-2} \,({\bf e}_\theta\cdot{\bf e}^i)\,({\bf e}_\theta\cdot{\bf e}^{\,j})=  ({\cal J}\,B)^{-2} \,\delta_\theta^{\,i}\,\delta_\theta^{\,j}.
\end{equation}

\section{Plasma Viscosity Tensor}
Let ${\bf 1}$ be the identity tensor. It follows that
\begin{equation}
1^{ij}= {\bf e}^i\cdot{\bf e}^{\,j} = g^{ij}.
\end{equation}
Thus, if 
\begin{equation}
\smalltensor{\bPi}= (\delta p_\parallel-\delta p_\perp)\,{\bf b}\,{\bf b} + \delta p_\perp\,\smalltensor{\bf 1}
\end{equation}
is the plasma viscosity tensor then
\begin{equation}\label{e48}
{\mit\Pi}^{\,ij} = (\delta p_\parallel-\delta p_\perp)\,({\cal J}\,B)^{-2} \,\delta_\theta^{\,i}\,\delta_\theta^{\,j} +\delta p_\perp\,g^{ij}.
\end{equation}

 Consider
 \begin{equation}\label{e43}
 \nabla\cdot\left(\smalltensor{\bPi}\cdot {\bf e}_k\right)= \nabla\cdot\smalltensor{\bPi}\cdot{\bf e}_k+ \smalltensor{\bPi}:\frac{\partial {\bf e}_k}{\partial{\bf x}}.
 \end{equation}
 Now,
  \begin{equation}
 \frac{\partial {\bf e}_k}{\partial{\bf x}}= \frac{\partial{\bf e}_k}{\partial q^{\,j}}\,\frac{\partial q^{\,j}}{\partial {\bf x}}= {\mit\Gamma}_{i,kj}\,{\bf e}^i\,{\bf e}^{\,j}
 = {\mit\Gamma}_{i,jk}\,{\bf e}^i\,{\bf e}^{\,j},
 \end{equation}
 where use has been made of Eqs.~(\ref{e1}) and (\ref{e25}). 
 Thus,
 \begin{equation}
 \smalltensor{\bPi}:\frac{\partial {\bf e}_k}{\partial{\bf x}}= {\mit\Pi}^{\,ij}\,{\mit\Gamma}_{i,jk}.
 \end{equation}
 Now,  ${\mit\Pi}^{\,ij}={\mit\Pi}^{\,ji}$, so  
 \begin{equation}\label{e53}
 \smalltensor{\bPi}:\frac{\partial {\bf e}_k}{\partial{\bf x}}= \frac{1}{2}\,{\mit\Pi}^{\,ij}\left({\mit\Gamma}_{i,jk}+{\mit\Gamma}_{j,ik}\right)= \frac{1}{2}\,{\mit\Pi}^{\,ij}\,\frac{\partial g_{ij}}{\partial q^k},
 \end{equation}
 where use has been made of Eq.~(\ref{e35}).
 
 If we integrate Eq.~(\ref{e43}) over all space, and neglect surface terms, then we find that
 \begin{equation}
 \int \nabla\cdot\smalltensor{\bPi}\cdot{\bf e}_k\,d^3{\bf x}= - \frac{1}{2}\int {\mit\Pi}^{\,ij}\,\frac{\partial g_{ij}}{\partial q^k}\,d^3{\bf x},
 \end{equation}
 where use has been made of Eq.~(\ref{e53}). It follows from Eq.~(\ref{e48}) that
 \begin{equation}
 \int \nabla\cdot\smalltensor{\bPi}\cdot{\bf e}_k\,d^3{\bf x}= - \frac{1}{2}\int \left[\frac{(\delta p_\parallel-\delta p_\perp)}{({\cal J}\,B)^2}\,\frac{\partial g_{\theta\theta}}{\partial q^k}+\delta p_\perp\,g^{ij}\,\frac{\partial g_{ij}}{\partial q^k}\right]d^3{\bf x}.
 \end{equation}
 However, Eq.~(\ref{e44}) implies that
 \begin{equation}
 g_{\theta\theta}= {\bf e}_\theta\cdot{\bf e}_\theta = ({\cal J}\,B)^2.
 \end{equation}
 Finally, making use of Eq.~(\ref{e42}), we get
 \begin{align}
 \int \nabla\cdot\smalltensor{\bPi}\cdot{\bf e}_k\,d^3{\bf x}&= - \frac{1}{2}\int \left[\frac{(\delta p_\parallel-\delta p_\perp)}{({\cal J}\,B)^2}\,\frac{\partial ({\cal J}\,B)^2}{\partial q^k}+\delta p_\perp\,\frac{2}{{\cal J}}\,\frac{\partial{\cal J}}{\partial q^k}\right]d^3{\bf x}\nonumber\\[0.5ex]
 &= -\int \left[(\delta p_\parallel-\delta p_\perp)\,\frac{1}{B}\,\frac{\partial B}{\partial q^k}+\delta p_\parallel\,\frac{1}{{\cal J}}\,\frac{\partial{\cal J}}{\partial q^k}\right]d^3{\bf x}.
 \end{align}
 
\end{document}