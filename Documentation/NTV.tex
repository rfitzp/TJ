\documentclass[12pt,prb,aps,notitlepage]{revtex4-1}
\usepackage {amsmath}
\usepackage{amssymb}
\pdfoutput = 1 
\usepackage {graphicx}
\usepackage{cancel}
\newcommand{\bomega}{\mbox{\boldmath$\omega$}}
\newcommand{\bxi}{\mbox{\boldmath$\xi$}}
\newcommand{\bPi}{\mbox{\boldmath$\Pi$}}
\allowdisplaybreaks
\newcommand{\smalltensor}[1]{\overset{\text{\tiny$\leftrightarrow$}}{#1}}

\begin{document}

\title{Neoclassical Toroidal Viscosity}
\author{R.~Fitzpatrick\,\footnote{rfitzp@utexas.edu}}
\affiliation{Institute for Fusion Studies,  Department of Physics,  University of Texas at Austin,  Austin TX 78712, USA}\begin{abstract}
\end{abstract}
\maketitle

\section{Curvilinear Coordinates}
\subsection{Basis Vectors} 
Let ${\bf x}=(x_1,\,x_2,\,x_3)$ be a position vector, where the $x_i$ are Cartesian coordinates.  (In the following, all latin indices, $i$, $j$, $k$, $l$ are assumed to run from 1 to 3. Moreover,
use is made of the Einstein summation convention.)
Let the $q_i(x_1,x_2,x_3)$ be curvilinear coordinates. We can define the contravariant basis vectors,
\begin{equation}\label{e1}
{\bf e}^i = \frac{\partial q^i}{\partial {\bf x}},
\end{equation}
and the covariant basis vectors, 
\begin{equation}\label{e2}
{\bf e}_i = \frac{\partial {\bf x}}{\partial q^i}.
\end{equation}
Note that
\begin{equation}\label{e3}
{\bf e}^i\cdot{\bf e}_j = \frac{\partial q^i}{\partial {\bf x}}\,\frac{\partial {\bf x}}{\partial q^j}= \frac{\partial q^i}{\partial x_k}\,\frac{\partial x_k}{\partial q^j}=
\frac{\partial q^i}{\partial q^j} = \delta^i_j,
\end{equation}
where  $\delta^i_j$ is the Kronecker delta symbol, and use has been made of the chain rule. 

The Jacobian, ${\cal J}$,  of the curvilinear coordinate system is defined as
\begin{align}\label{e4}
{\cal J} &= \frac{\partial{\bf x}}{\partial q^1}\cdot\frac{\partial {\bf x}}{\partial q^2}\times\frac{\partial{\bf x}}{\partial q^3}={\bf e}_1\cdot{\bf e}_2\times{\bf e}_3.
\end{align}
Note that
\begin{align}
{\cal J}^{\,-1} &= \frac{\partial q^1}{\partial {\bf x}}\cdot\frac{\partial q^2}{\partial{\bf x}}\times\frac{\partial q^3}{\partial{\bf x}}= {\bf e}^1\cdot{\bf e}^2\times{\bf e}^3.
\end{align}
It is easily seen that
\begin{align}
\epsilon_{ijk}\equiv {\bf e}_i\cdot{\bf e}_j\times{\bf e}_k&= {\cal J}\,\varepsilon_{ijk},\label{edown}\\[0.5ex]
\epsilon^{ijk}\equiv {\bf e}^i\cdot{\bf e}^{\,j}\times{\bf e}^k&= {\cal J}^{\,-1}\,\varepsilon_{ijk},\label{eup}
\end{align}
where $\varepsilon_{ijk}$ is the Levi-Civita symbol, and the $\epsilon_{ijk}$  and the $\epsilon^{ijk}$ are the covariant and
contravaiant components of the  Levi-Civita tensor, respectively. 

The covariant basis vectors can be expressed in terms of the contravariant basis vectors, as follows
\begin{equation}\label{e5}
{\bf e}_i = {\cal J}\,\frac{\partial q^j}{\partial {\bf x}}\times \frac{\partial{q^k}}{\partial{\bf x}}= {\cal J}\,{\bf e}^{\,j}\times{\bf e}^k,
\end{equation}
where $i$, $j$, $k$ are cyclic. To demonstrate the validity of this expression we need to show that ${\bf e}^l\cdot{\bf e}_i= \delta_i^l$, 
which implies that 
\begin{equation}
 {\cal J}\,{\bf e}^l\cdot{\bf e}^{\,j}\times {\bf e}^k= \varepsilon_{ljk}=\delta_i^l,
 \end{equation}
 which is obviously satisfied. The contravariant basis vectors can also be expressed in terms of the covariant basis vectors as follows, 
 \begin{equation}
 {\bf e}^i = {\cal J}^{\,-1}\,{\bf e}_j\times{\bf e}_k,
 \end{equation}
 where $i$, $j$, $k$ are cyclic. 
 
\subsection{Vectors}
 The contravariant components, $a^i$,  of a vector ${\bf a}$ are defined via
 \begin{equation}
 {\bf a} = a^i\,{\bf e}_i.
 \end{equation}
 The covariant components, $a_i$, are defined via
 \begin{equation}
  {\bf a} = a_i\,{\bf e}^i.
  \end{equation}
  Thus,
  \begin{equation}
  {\bf a}\cdot{\bf b} = a^i\,b_j\,{\bf e}_i\cdot{\bf e}^{\,j} = a^i\,b_j\,\delta_i^{\,j} = a^i\,b_i,
  \end{equation}
  where use has been made of Eq.~(\ref{e3}). 
  Similarly, 
  \begin{equation}
  {\bf a}\cdot{\bf b} = a_i\,b^{\,j}\,{\bf e}^i\cdot{\bf e}_j = a_i\,b^{\,j}\,\delta^i_j = a_i\,b^i,
  \end{equation}
  Note that
  \begin{align}
  a^i &= {\bf a}\cdot{\bf e}^i,\\[0.5ex]
  a_i &= {\bf a}\cdot{\bf e}_i.
    \end{align}
    
 Now,
 \begin{equation}
 ({\bf a}\times{\bf b})^i = {\bf a}\times {\bf b}\cdot{\bf e}^i = a_j\,b_k\,{\bf e}^{\,j}\times{\bf e}^k\cdot{\bf e}^i ={\cal J}^{-1}\,\varepsilon^{ijk}\,a_j\,b_k,
 \end{equation}
 where use has been made of Eq.~(\ref{eup}). 
 Likewise, 
 \begin{equation}
 ({\bf a}\times{\bf b})_i = {\bf a}\times {\bf b}\cdot{\bf e}_i = a^{\,j}\,b^k\,{\bf e}_{j}\times{\bf e}_k\cdot{\bf e}_i={\cal J}\,\varepsilon_{ijk}\,a^{\,j}\,b^k,
 \end{equation}
 where use has been made of Eq.~(\ref{edown}). 

\subsection{Metric Tensor}  
 The contravariant components of the metric tensor, $\smalltensor{\bf g}$, are defined
 \begin{equation}
 g^{ij} = {\bf e}^i\cdot{\bf e}^{\,j}.
 \end{equation}
 Note that $g^{ji}=g^{ij}$. 
 Likewise, the covariant components of the metric tensor are  
 \begin{equation}
 g_{ij} = {\bf e}_i\cdot{\bf e}_j.
 \end{equation}
 Note that $g_{ji}= g_{ij}$. Furthermore,
 \begin{equation}
 dx^{\,2}= \frac{\partial {\bf x}}{\partial q^i}\cdot \frac{\partial {\bf x}}{\partial q^{\,j}}\,dq^i\,dq^{\,j} = {\bf e}_i\cdot{\bf e}_j\,dq^i\,dq^{\,j} = g_{ij}\,dq^i\,dq^{\,j},
 \end{equation}
 where use has been made of Eq.~(\ref{e2}). 
 
 Now, 
 \begin{equation}
 {\bf e}^i = ({\bf e}^i\cdot{\bf e}^{\,j})\,{\bf e}_j,
 \end{equation}
so
\begin{equation}
{\bf a}\cdot{\bf e}^i = {\bf a}\cdot[ ({\bf e}^i\cdot{\bf e}^{\,j})\,{\bf e}_j]=( {\bf e}^i\cdot{\bf e}^{\,j})\,({\bf a}\cdot{\bf e}_j),
\end{equation}
which yields
\begin{equation}\label{e17}
a^i = g^{ij}\,a_j.
\end{equation}
Likewise, 
\begin{equation}
 {\bf e}_i = ({\bf e}_i\cdot{\bf e}_j)\,{\bf e}^{\,j},
 \end{equation}
so 
\begin{equation}
{\bf a}\cdot{\bf e}_i = {\bf a}\cdot[ ({\bf e}_i\cdot{\bf e}_j)\,{\bf e}^{\,j}]= ({\bf e}_i\cdot{\bf e}_j)\,({\bf a}\cdot{\bf e}^{\,j}),
\end{equation}
which yields
\begin{equation}\label{e20}
a_i = g_{ij}\,a^{\,j}.
\end{equation}
Combining Eqs.~(\ref{e17}) and (\ref{e20}), we get
\begin{equation}
a^i = g^{ij}\,g_{jk}\,a^k = \delta^{i}_k\,a^k,
\end{equation}
which implies that
\begin{equation}
g^{ik}\,g_{kj}= \delta^{i}_j.
\end{equation}
Likewise, 
\begin{equation}
a_i = g_{ij}\,g^{\,jk}\,a_k = \delta_{i}^k\,a_k,
\end{equation}
which implies that
\begin{equation}
g_{ik}\,g^{\,kj}= \delta_{i}^j.
\end{equation}

\subsection{Chistoffel Symbol}
The Christoffel symbols, ${\mit\Gamma}_{kij}$ and ${\mit\Gamma}^k_{\,\,ij}$, are defined such that 
\begin{equation}\label{e25}
\frac{\partial {\bf e}_i}{\partial q^{\,j}} = {\mit\Gamma}_{kij}\,{\bf e}^k= {\mit\Gamma}^k_{\,\,ij}\,{\bf e}_k.
\end{equation}
It follows that 
\begin{align}
{\mit\Gamma}_{kij} &= {\bf e}_k\cdot\,\frac{\partial {\bf e}_i}{\partial q^{\,j}},\\[0.5ex]
{\mit\Gamma}^k_{\,\,ij} &= {\bf e}^k\cdot\,\frac{\partial {\bf e}_i}{\partial q^{\,j}}.
\end{align}
However,
\begin{equation}
\frac{\partial ({\bf e}^k\cdot{\bf e}_i)}{\partial q^j} ={\bf e}^k \cdot\frac{\partial {\bf e}_i}{\partial q_k} + {\bf e}_i\cdot\frac{\partial{\bf e}^k}{\partial q^j}
= 0,
\end{equation}
which implies that
\begin{equation}\label{e27}
{\bf e}_i\cdot\frac{\partial{\bf e}^k}{\partial q^j}= - {\mit\Gamma}^k_{\,\,ij}.
\end{equation}

Now, 
\begin{align}\label{e28}
\frac{\partial g_{ij}}{\partial q^k} = \frac{\partial ({\bf e}_i\cdot{\bf e}_j)}{\partial q^k}= \frac{\partial {\bf e}_i}{\partial q^k}\cdot{\bf e}_j + {\bf e}_i\cdot\frac{\partial {\bf e}_j}{\partial q^k}= {\mit\Gamma}_{jik} + {\mit\Gamma}_{ijk},\\[0.5ex]
\frac{\partial g_{ik}}{\partial q^{\,j}} = \frac{\partial ({\bf e}_i\cdot{\bf e}_k)}{\partial q^{\,j}}= \frac{\partial {\bf e}_i}{\partial q^{\,j}}\cdot{\bf e}_k+ {\bf e}_i\cdot\frac{\partial {\bf e}_k}{\partial q^{\,j}}= {\mit\Gamma}_{kij} + {\mit\Gamma}_{ikj},\\[0.5ex]
\frac{\partial g_{jk}}{\partial q^i} = \frac{\partial ({\bf e}_j\cdot{\bf e}_k)}{\partial q^i}= \frac{\partial {\bf e}_j}{\partial q^i}\cdot{\bf e}_k+ {\bf e}_j\cdot\frac{\partial {\bf e}_k}{\partial q^i}= {\mit\Gamma}_{kji} + {\mit\Gamma}_{jki}.\label{e30}
\end{align}
However,
\begin{equation}
\frac{\partial{\bf e}_i}{\partial q^{\,j}} = \frac{\partial^2{\bf x}}{\partial q^{\,j}\,\partial q^{i}}= \frac{\partial^2{\bf x}}{\partial q^{i}\,\partial q^{\,j}}
= \frac{\partial{\bf e}_j}{\partial q^i},
\end{equation}
where use has been made of Eq.~(\ref{e2}). Hence, we deduce  that
\begin{equation}\label{e32}
{\mit\Gamma}_{kji}= {\mit\Gamma}_{kij}.
\end{equation}
It follows  from Eqs.~(\ref{e28})--(\ref{e30}) and (\ref{e32}) that
\begin{align}
\frac{\partial g_{ij}}{\partial q^k} + \frac{\partial g_{jk}}{\partial q^i} -\frac{\partial g_{ik}}{\partial q^{\,j}}& = {\mit\Gamma}_{jik} + {\mit\Gamma}_{ijk}+ {\mit\Gamma}_{k,ji} + {\mit\Gamma}_{jki}-{\mit\Gamma}_{kij} - {\mit\Gamma}_{ikj}\nonumber\\[0.5ex]
&= {\mit\Gamma}_{jik} + \cancel{{\mit\Gamma}_{ijk}}+ \cancel{{\mit\Gamma}_{kji}} + {\mit\Gamma}_{jik}-\cancel{{\mit\Gamma}_{kji}} - \cancel{{\mit\Gamma}_{i,jk}},
\end{align}
which implies that
\begin{equation}
 {\mit\Gamma}_{jik}=\frac{1}{2}\left(\frac{\partial g_{ij}}{\partial q^k} + \frac{\partial g_{jk}}{\partial q^i} -\frac{\partial g_{ik}}{\partial q^{\,j}}\right),
 \end{equation}
 or 
 \begin{equation}\label{e34}
 {\mit\Gamma}_{ijk}=\frac{1}{2}\left(\frac{\partial g_{ij}}{\partial q^k} + \frac{\partial g_{ik}}{\partial q^{\,j}} -\frac{\partial g_{jk}}{\partial q^{i}}\right).
 \end{equation}
 The previous two equations yield
 \begin{equation}\label{e35}
 \frac{\partial g_{ij}}{\partial q^k} = {\mit\Gamma}_{ijk} + {\mit\Gamma}_{jik}.
 \end{equation}
 
\subsection{Covariant Derivative}
Let ${\bf A}({\bf x})$ be a vector field. 
We can write 
\begin{equation}
\frac{\partial {\bf A}}{\partial q^{\,j}} = \frac{\partial\,(A^k\,{\bf e}_k)}{\partial q^{\,j}} = \frac{\partial A^k}{\partial q^{\,j}}\,{\bf e}_k + A^k\,\frac{\partial {\bf e}_k}{\partial q^{\,j}}= 
 \frac{\partial A^k}{\partial q^{\,j}}\,{\bf e}_k +A^k\, {\mit\Gamma}^l_{\,kj}\,{\bf e}_l,
 \end{equation}
 where use has been made of Eq.~(\ref{e25}). Let
 \begin{equation}
 \partial_j A^i \equiv  {\bf e}^i\cdot\frac{\partial{\bf A}}{\partial q^{\,j}},
 \end{equation}
 where $\partial_j$ is the covariant derivative. 
 It follows that
 \begin{equation}\label{cov1}
 \partial_j A^i = \frac{\partial A^i}{\partial q^{\,j}} + {\mit\Gamma}^i_{\,jk}\,A^k.
 \end{equation}
 We can also write 
\begin{equation}
\frac{\partial {\bf A}}{\partial q^{\,j}} = \frac{\partial\,(A_k\,{\bf e}^k)}{\partial q^{\,j}} = \frac{\partial A_k}{\partial q^{\,j}}\,{\bf e}^k + A_k\,\frac{\partial {\bf e}^k}{\partial q^{\,j}}= 
 \frac{\partial A_k}{\partial q^{\,j}}\,{\bf e}^k -A_k\, {\mit\Gamma}^k_{\,jl}\,{\bf e}_l,
 \end{equation}
 where use has been made of Eq.~(\ref{e27}). Let
 \begin{equation}
 \partial_j A_i \equiv  {\bf e}_i\cdot\frac{\partial{\bf A}}{\partial q^{\,j}}.
 \end{equation}
 It follows that
 \begin{equation}\label{cov2}
 \partial_j A_i = \frac{\partial A_i}{\partial q_j} - {\mit\Gamma}^k_{\,ij}\,A_k.
 \end{equation}
 
\subsection{Jacobian Matrix}
 Let us define the Jacobian matrix
 \begin{equation}
 J_{ij} = \frac{\partial x_j}{\partial q^i}.
 \end{equation}
 Of course, the Jacobian is the determinant of the Jacobian matrix
 \begin{equation}
 {\cal J} = ||J_{ij}||= \frac{\partial{\bf x}}{\partial q^1}\cdot\frac{\partial {\bf x}}{\partial q^2}\times\frac{\partial{\bf x}}{\partial q^3}.
 \end{equation}
 See Eq.~(\ref{e4}). 
 Now,
 \begin{equation}
 g_{ij} = {\bf e}_i\cdot{\bf e}_j= \frac{\partial{\bf x}}{\partial q^i}\cdot\frac{{\partial {\bf x}}}{\partial q^{\,j}}= \frac{\partial x_k}{\partial q^i}\,\frac{{\partial x_k}}{\partial q^{\,j}}= J_{ik}\,J_{jk} = J_{ij}\,J^{\,T}_{kj},
 \end{equation}
 where use has been made of Eq.~(\ref{e2}), and $J^{\,T}_{ij}= J_{ji}$. However, $||A\,B|| = ||A||\,||B||$ and $||A^{\,T}|| = ||A||$. Hence, we deduce that
 \begin{equation}\label{e39}
 ||g_{ij}|| = {\cal J}^{\,2}.
 \end{equation}
 
 Jacobi's formula states that
 \begin{equation}
 \frac{\partial ||A||}{\partial q^k} = ||A||\,{\rm Tr}\!\left(A^{-1}\,\frac{\partial A}{\partial q^k}\right)
 \end{equation}
 for any square differentiable matrix, $A_{ij}(q_1,q_2,q_3)$. 
 Let $A_{ij}= g_{ij}$. Now, the inverse of $g_{ij}$ is $g^{ij}$. Hence, making use of Eq.~(\ref{e39}), we deduce that
 \begin{equation}
 \frac{\partial{\cal J}^{\,2}}{\partial q^k} = {\cal J}^{\,2}\,g^{ij}\,\frac{\partial g_{ji}}{\partial q^k},
 \end{equation}
 which reduces to
 \begin{equation}\label{e42}
 \frac{\partial g_{ij}}{\partial q^k}\,g^{ij} = \frac{2}{\cal J}\,\frac{\partial{\cal J}}{\partial q^k}.
 \end{equation}
 
\subsection{Scalar Fields}
Let $\phi({\bf x})$ be a scalar field. It follows that
\begin{equation}
\nabla \phi \equiv \frac{\partial \phi}{\partial{\bf x}}  = \frac{\partial\phi}{\partial q^i}\,\frac{\partial q^i}{\partial{\bf x}}= \frac{\partial\phi}{\partial q^i}\,{\bf e}^i,
\end{equation}
where use has been made of Eq.~(\ref{e1}). Thus,
\begin{equation}
(\nabla\phi)_i = \frac{\partial\phi}{\partial q^i}.
\end{equation}

\subsection{Vector Fields}
Let ${\bf A}({\bf x})$ be a vector field. Now,
\begin{align}\label{div1}
\nabla\cdot{\bf A}& = \frac{\partial {\bf A}}{\partial {\bf x}} = \frac{\partial q^i}{\partial{\bf x}}\cdot\,\frac{\partial}{\partial q^i}\left(A^{\,j}\,{\bf e}_j\right)
= {\bf e}^i\cdot\left(\frac{\partial A^{\,j}}{\partial q^i}\,{\bf e}_j + A^{\,j}\,\frac{\partial {\bf e}_j}{\partial q^i}\right)\nonumber\\[0.5ex]
&= {\bf e}^i\cdot\left(\frac{\partial A^{\,j}}{\partial q^i}\,{\bf e}_j + A^j\,{\mit\Gamma}^k_{\,\,ji}\,{\bf e}_k\right)= \frac{\partial A^i}{\partial q^i} + 
A^{\,j}\,{\mit\Gamma}^i_{\,\,ij},
\end{align}
where use has been made of Eqs.~(\ref{e1}), (\ref{e3}),  (\ref{e25}), and (\ref{e32}). 
Now,
\begin{equation}
{\mit\Gamma}^i_{\,\,ij} = g^{ik}\,{\mit\Gamma}_{kij} = \frac{1}{2}\,g^{ik}\left(\frac{\partial g_{ki}}{\partial q^j}+\cancel{\frac{\partial g_{kj}}{\partial q^i}}-
\cancel{\frac{\partial g_{ij}}{\partial q_k}}\right)= \frac{1}{2}\,g^{ik}\,\frac{\partial g_{ik}}{\partial q^j} = \frac{1}{\cal J}\,\frac{\partial{\cal J}}{\partial q^j},
\end{equation}
where use has been made of Eqs.~(\ref{e34}) and (\ref{e42}). Thus,
\begin{equation}
\nabla\cdot{\bf A} = \frac{\partial A^i}{\partial q^i} + \frac{A^i}{\cal J}\,\frac{\partial {\cal J}}{\partial q^i} = \frac{1}{\cal J}\,\frac{\partial ({\cal J}\,A^i)}{\partial q^i}.
\end{equation}
It is clear from Eqs.~(\ref{cov1}) and (\ref{div1}) that
\begin{equation}
\nabla\cdot{\bf A} = \partial_i A^i.
\end{equation}

Let us assume that
\begin{equation}
(\nabla\times{\bf A})^i = \epsilon^{ijk}\,\partial_j A_k ={\cal J}^{\,-1}\,\varepsilon_{ijk}\left(\frac{\partial A_k}{\partial q_j} - {\mit\Gamma}^l_{\,jk}\,A_l\right),
\end{equation}
where use has been made of Eqs.~(\ref{eup}) and (\ref{cov1}). However, given that ${\mit\Gamma}^l_{\,\,jk}={\mit\Gamma}^l_{\,\,kj}$, it
is clear that $\varepsilon_{ijk}\,{\mit\Gamma}^l_{\,\,jk} =0$. Thus, we obtain
\begin{equation}\label{ecurl}
(\nabla\times{\bf A})^i = {\cal J}^{\,-1}\,\varepsilon_{ijk}\,\frac{\partial A_k}{\partial q_j}. 
\end{equation}
 
\section{Equilibrium Magnetic Field}
Let $\psi$, $\theta$, $\alpha$ be magnetic coordinates, where $\psi$ is a flux-surface label, $\theta$ a poloidal angle, and $\alpha$
a helical angle. Suppose that the geometric toroidal angle is $\varphi = \alpha+ q(\psi)\,\theta$. 
 Let ${\cal  J}^{-1}= \nabla\psi\cdot\nabla\theta\times \nabla\alpha$ be the Jacobian of the coordinate system. The equilibrium magnetic field
is written 
\begin{equation}\label{e44}
{\bf B} =  \nabla\alpha\times\nabla\psi={\cal J}^{-1}\,{\bf e}_\theta,
\end{equation}
where use has been made of Eq.~(\ref{e5}). 
It follows that 
\begin{equation}\label{e45}
{\bf b} \equiv \frac{{\bf B}}{B}= ({\cal J}\,B)^{-1}\,{\bf e}_\theta.
\end{equation}
Thus,
\begin{equation}\label{e46}
({\bf b}\,{\bf b})^{ij}= ({\cal J}\,B)^{-2} \,({\bf e}_\theta\cdot{\bf e}^i)\,({\bf e}_\theta\cdot{\bf e}^{\,j})=  ({\cal J}\,B)^{-2} \,\delta_\theta^{\,i}\,\delta_\theta^{\,j}.
\end{equation}
Furthermore,
\begin{equation}\label{e48a}
{\bf b}\cdot\nabla f = \frac{1}{{\cal J}\,B}\,\frac{\partial f}{\partial\theta}.
\end{equation}

\section{Plasma Viscosity Tensor}
Let $\smalltensor{\bf 1}$ be the identity tensor. It follows that
\begin{equation}\label{e47}
1^{ij}= {\bf e}^i\cdot{\bf e}^{\,j} = g^{ij}.
\end{equation}
Thus, if 
\begin{equation}
\smalltensor{\bPi}= (\delta p_\parallel-\delta p_\perp)\,{\bf b}\,{\bf b} + \delta p_\perp\,\smalltensor{\bf 1}
\end{equation}
is the perturbed plasma viscosity tensor then
\begin{equation}\label{e48}
{\mit\Pi}^{\,ij} = (\delta p_\parallel-\delta p_\perp)\,({\cal J}\,B)^{-2} \,\delta_\theta^{\,i}\,\delta_\theta^{\,j} +\delta p_\perp\,g^{ij},
\end{equation}
where use has been made of Eqs.~(\ref{e46}) and (\ref{e47}). 

 Consider
 \begin{equation}\label{e43}
 \nabla\cdot\left(\smalltensor{\bPi}\cdot {\bf e}_k\right)= \nabla\cdot\smalltensor{\bPi}\cdot{\bf e}_k+ \smalltensor{\bPi}:\frac{\partial {\bf e}_k}{\partial{\bf x}}.
 \end{equation}
 Now,
  \begin{equation}
 \frac{\partial {\bf e}_k}{\partial{\bf x}}= \frac{\partial{\bf e}_k}{\partial q^{\,j}}\,\frac{\partial q^{\,j}}{\partial {\bf x}}= {\mit\Gamma}_{ikj}\,{\bf e}^i\,{\bf e}^{\,j}
 = {\mit\Gamma}_{ijk}\,{\bf e}^i\,{\bf e}^{\,j},
 \end{equation}
 where use has been made of Eqs.~(\ref{e1}), (\ref{e25}), and (\ref{e32}). 
 Thus,
 \begin{equation}
 \smalltensor{\bPi}:\frac{\partial {\bf e}_k}{\partial{\bf x}}= {\mit\Pi}^{\,ij}\,{\mit\Gamma}_{ijk}.
 \end{equation}
 Now,  ${\mit\Pi}^{\,ij}={\mit\Pi}^{\,ji}$, so  
 \begin{equation}\label{e53}
 \smalltensor{\bPi}:\frac{\partial {\bf e}_k}{\partial{\bf x}}= \frac{1}{2}\,{\mit\Pi}^{\,ij}\left({\mit\Gamma}_{ijk}+{\mit\Gamma}_{jik}\right)= \frac{1}{2}\,{\mit\Pi}^{\,ij}\,\frac{\partial g_{ij}}{\partial q^k},
 \end{equation}
 where use has been made of Eq.~(\ref{e35}).
 
 If we integrate Eq.~(\ref{e43}) over all space, and neglect surface terms, then we find that
 \begin{equation}
 \int \nabla\cdot\smalltensor{\bPi}\cdot{\bf e}_k\,d^3{\bf x}= - \frac{1}{2}\int {\mit\Pi}^{\,ij}\,\frac{\partial g_{ij}}{\partial q^k}\,d^3{\bf x},
 \end{equation}
 where use has been made of Eq.~(\ref{e53}). It follows from Eq.~(\ref{e48}) that
 \begin{equation}
 \int \nabla\cdot\smalltensor{\bPi}\cdot{\bf e}_k\,d^3{\bf x}= - \frac{1}{2}\int \left[\frac{(\delta p_\parallel-\delta p_\perp)}{({\cal J}\,B)^2}\,\frac{\partial g_{\theta\theta}}{\partial q^k}+\delta p_\perp\,g^{ij}\,\frac{\partial g_{ij}}{\partial q^k}\right]d^3{\bf x}.
 \end{equation}
 However, Eq.~(\ref{e45}) implies that
 \begin{equation}\label{gtt}
 g_{\theta\theta}= {\bf e}_\theta\cdot{\bf e}_\theta = ({\cal J}\,B)^2.
 \end{equation}
 Finally, making use of Eq.~(\ref{e42}), we get
 \begin{align}\label{e57}
 \int \nabla\cdot\smalltensor{\bPi}\cdot{\bf e}_k\,d^3{\bf x}&= - \frac{1}{2}\int \left[\frac{(\delta p_\parallel-\delta p_\perp)}{({\cal J}\,B)^2}\,\frac{\partial ({\cal J}\,B)^2}{\partial q^k}+\delta p_\perp\,\frac{2}{{\cal J}}\,\frac{\partial{\cal J}}{\partial q^k}\right]d^3{\bf x}\nonumber\\[0.5ex]
 &= -\int \left[(\delta p_\parallel-\delta p_\perp)\,\frac{1}{B}\,\frac{\partial B}{\partial q^k}+\delta p_\parallel\,\frac{1}{{\cal J}}\,\frac{\partial{\cal J}}{\partial q^k}\right]d^3{\bf x}.
 \end{align}
 
\section{Toroidal Viscous Torque}
 Let $R$, $\varphi$, $Z$ be cylindrical coordinates. It follows that $x_1=R\,\cos\varphi$, $x_2=R\,\sin\varphi$, $x_3=Z$. Thus,
 \begin{equation}
 \frac{\partial {\bf x}}{\partial\varphi} = -R\,\sin\varphi\,\hat{\bf e}_x+R\,\cos\varphi\,\hat{\bf e}_y= R\,\hat{\bf e}_\varphi,
 \end{equation}
 where $\hat{\bf e}_x=\nabla x/|\nabla x|$, $\hat{\bf e}_y=\nabla y/|\nabla y|$, and $\hat{\bf e}_\varphi=\nabla \varphi/|\nabla \varphi|$. 
 It follows that the net toroidal viscous torque acting on the plasma is
 \begin{equation}\label{e59}
 T_\varphi = \int  \frac{\partial {\bf x}}{\partial\varphi} \cdot\nabla\cdot \smalltensor{\bPi}\,d^3{\bf x}= \int  \frac{\partial {\bf x}}{\partial\alpha} \cdot\nabla\cdot \smalltensor{\bPi}\,d^3{\bf x} = \int  {\bf e}_\alpha \cdot\nabla\cdot \smalltensor{\bPi}\,d^3{\bf x}.
 \end{equation}
 Here, use has been made of Eq.~(\ref{e2}), as well as 
 \begin{equation}
 \left.\frac{\partial}{\partial\varphi}\right|_{\psi,\theta} = \left.\frac{\partial}{\partial\alpha}\right|_{\psi,\theta}.
 \end{equation}
 Thus, Eqs.~(\ref{e57}) and (\ref{e59}) imply that
 \begin{equation}
 T_\varphi = -\int \left[(\delta p_\parallel-\delta p_\perp)\,\frac{1}{B}\,\frac{\partial B}{\partial \alpha}+\delta p_\parallel\,\frac{1}{{\cal J}}\,\frac{\partial{\cal J}}{\partial \alpha}\right]d^3{\bf x}.
 \end{equation}
 
 The unperturbed equilibrium is such that $\partial B/\partial \alpha =\partial{\cal J}/\partial\alpha=0$. In the perturbed equilibrium, 
 $B\rightarrow B+\delta B$ and 
 $dV= {\cal J}\,d\psi\,d\theta\,d\alpha\rightarrow {\cal J}\,(1+\nabla\cdot\bxi)\,d\psi\,d\theta\,d\alpha$, where $\bxi({\bf x})$ is the plasma displacement. Hence,
 to lowest order in perturbed quantities, 
 \begin{equation}\label{e62}
 T_\varphi = -\int \left[(\delta p_\parallel-\delta p_\perp)\,\frac{\partial}{\partial \alpha}\!\left(\frac{\delta B}{B}\right)+\delta p_\parallel\,\frac{\partial(\nabla\cdot\bxi)}{\partial \alpha}\right]d^3{\bf x}.
 \end{equation}
 
 Let $\hat{\bf e}_1$, $\hat{\bf e}_2$, $\hat{\bf e}_3$ be right-handed orthogonal unit vectors such that $\hat{\bf e}_3={\bf b}$ at a given
 point in space. The ion velocity is written
 \begin{equation}
 {\bf v} = v_1\,\hat{\bf e}_1 + v_2\,\hat{\bf e}_2+v_3\,\hat{\bf e}_3= v_\perp\,\cos\zeta\,\hat{\bf e}_1+ v_\perp\,\sin\zeta\,\hat{\bf e}_2+ 
 v_\parallel\,\hat{\bf e}_3,
 \end{equation}
 where $\zeta$ is the gyro-angle. 
 Now, 
 \begin{align}\label{e64}
 \delta p_\parallel({\bf x}) &= \int m\,v_\parallel^{\,2}\,f_1\,d^3{\bf v},\\[0.5ex]
 \delta p_\perp({\bf x}) &= \frac{1}{2}\int m\,v_\perp^{\,2}\,f_1\,d^3{\bf v},\label{e65}
 \end{align}
 where $f_1({\bf x},{\bf v})$ is the perturbed, gyro-averaged, ion distribution function, and $m$  the ion mass. Let
 $E=m\,(v_\perp^{\,2}+v_\parallel^{\,2})/2$ be the ion kinetic energy,  and $\mu=m\,v_\perp^{\,2}/(2\,B)$ the ion magnetic moment. 
 It is easily demonstrated that
 \begin{equation}\label{e66}
 \frac{\partial (E,\mu,\zeta)}{\partial(v_1,v_2,v_3)} = \frac{m^2\,v_\parallel}{B}.
 \end{equation}
 Equations~(\ref{e62}), (\ref{e64}), and (\ref{e65}) yield
 \begin{equation}
 T_\varphi =-\int\!\!\!\int \left[(2\,E-3\,\mu\,B)\,\frac{\partial}{\partial\alpha}\!\left(\frac{\delta B}{B}\right) + (2\,E-\,2\,\mu\,B)\,\frac{\partial(\nabla\cdot\bxi)}{\partial\alpha}\right]f_1\,d^3{\bf x}\,d^3{\bf v}.
 \end{equation}
 Finally, making use of Eq.~(\ref{e66}), we get
 \begin{equation}\label{e92x}
 T_\varphi = -\frac{2\pi}{m^2}\int \frac{{\cal J}\,B}{v_\parallel}\left[(2\,E-3\,\mu\,B)\,\frac{\partial}{\partial\alpha}\!\left(\frac{\delta B}{B}\right) + (2\,E-\,2\,\mu\,B)\,\frac{\partial(\nabla\cdot\bxi)}{\partial\alpha}\right]f_1\,dE\,d\mu\,d\psi\,d\theta\,d\alpha.
 \end{equation}
 
\section{Ion Drift-Kinetic Equation}
 The ion drift-kinetic equation is written
 \begin{equation}
 v_\parallel\,{\bf b}\cdot\nabla f_1 + v_d^{\,\alpha}\,\frac{\partial f_1}{\partial\alpha} - C_i(f_1)=- v_d^{\,\psi}\,\frac{\partial f_0}{\partial\psi},
 \end{equation}
 where ${\bf v}_d$ is the ion drift-velocity, $C_i$ the ion collision operator, and $f_0(\psi,E,\mu)$ the unperturbed ion distribution function. Let us
 employ a Krook collision operator, such that $C_i(f_1)= - \nu_i\,f_1$, where $\nu_i$ is the ion collision frequency. It follows that
 \begin{equation}\label{e71}
 \frac{v_\parallel}{{\cal J}\,B}\,\frac{\partial f_1}{\partial\theta} + v_d^{\,\alpha}\,\frac{\partial f_1}{\partial\alpha} + \nu_i\,f_1= - v_d^{\,\psi}\,\frac{\partial f_0}{\partial\psi},
 \end{equation}
 where use has been made of Eq.~(\ref{e48a}).
 Note that we are neglecting $v_d^{\,\theta}$ with respect to $v_\parallel$. This is reasonable because the grad-B and curvature drift velocities are
 order $\rho/L$ smaller than the thermal velocity, where $\rho$ is the ion gyro-radius, and $L$ a typical variation length-scale. We also
 expect the ${\bf E}\times{\bf B}$ drift velocity to be much smaller than the thermal velocity (otherwise ions would be drifting supersonically). 
 
 Now, motion at constant $\psi$ and $\alpha$ corresponds to motion along magnetic field-lines. An element of length, $ds$, along
 a field-line is such that
 \begin{equation}
 (ds)^2 = g_{\theta\theta}\,(d\theta)^2= ({\cal J}\,B)^2\,(d\theta)^2,
 \end{equation}
 where use has been made of Eq.~(\ref{gtt}). Thus, $ds = {\cal J}\,B\,d\theta$. The time interval required for an ion to move along a
 magnetic field-line between points $\theta$ and $\theta+d\theta$ is
 \begin{equation}\label{dt}
 dt = \frac{ds}{v_\parallel} =\frac{{\cal J}\,B}{v_\parallel}\,d\theta.
 \end{equation}
 During this time interval, the ion drifts perpendicular to the field-line (within the magnetic flux-surface) such that its $\alpha$ coordinate
 increases by
 \begin{equation}
 d\alpha =v_d^{\,\alpha}\,dt =\frac{{\cal J}\,B\,v_d^{\,\alpha}}{v_\parallel}\,d\theta.
 \end{equation}
 It follows that
 \begin{equation}
 \alpha(\theta)= \alpha(0) + \int_0^\theta\frac{{\cal J}\,B\,v_d^{\,\alpha}}{v_\parallel}\,d\theta'.
 \end{equation}
  In one poloidal transit, $\alpha$ increases by
 \begin{equation}
 {\mit\Delta}\alpha = \int_{-\theta_t}^{\theta_t}\frac{{\cal J}\,B\,v_d^{\,\alpha}}{v_\parallel}\,d\theta,
 \end{equation}
 where $\theta_t$ is the angular position of a turning point for a trapped ion, and $\theta_t=\pi$ for a passing ion 
 The corresponding mean increase in the toroidal angle is
 \begin{equation}
 {\mit\Delta}\varphi = {\mit\Delta}\alpha
 \end{equation}
 for a trapped ion, 
 \begin{equation}
 {\mit\Delta}\varphi = {\mit\Delta}\alpha+ 2\pi\,q
 \end{equation}
 for a passing ion. 
 
 Suppose that perturbed quantities vary with $\alpha$ as $\exp(-{\rm i}\,n\,\alpha)$, where the positive integer $n$ is the toroidal
 mode number of the magnetic perturbation. It follows that perturbed quantities are periodic in the toroidal angle with periodicity interval $2\pi/n$. If
 \begin{equation}
 {\mit\Delta}\varphi = \frac{2\pi\,\ell}{n},
 \end{equation}
 where $\ell$ is an integer, then trapped ions experience the same perturbed field during each  bounce orbit, whereas passing
 ions experience the same perturbed field during each poloidal transit.  We would expect ions that have this property to resonate with the
 magnetic perturbation. Such resonant ions have
 \begin{equation}
 ({\mit\Delta}\alpha)_{res} = \frac{2\pi\,\ell}{n}
 \end{equation}
 if they are trapped, and 
 \begin{equation}
 ({\mit\Delta}\alpha)_{res} = \frac{2\pi\,(\ell-n\,q)}{n}
 \end{equation}
 if they are passing.
Thus, we can write
  \begin{equation}
 ({\mit\Delta}\alpha)_{res} = \frac{2\pi\,(\ell-\sigma\,n\,q)}{n},
 \end{equation}
 where $\sigma=0$ for trapped ions, and $\sigma=1$ for passing ions. Let us suppose that
 \begin{equation}\label{e83}
 f_1({\bf x},{\bf v})= \delta f_{\ell}(\psi,E,\mu)\,{\cal P}_{\ell}(\psi,\theta,\alpha,E,\mu),
 \end{equation}
 where
 \begin{align}
 {\cal P}_\ell(\psi,\theta,\alpha,E,\mu)& = \exp\left[-{\rm i}\,n\left(\alpha + \frac{({\mit\Delta}\alpha)_{res}}{{\mit\Delta}\alpha}\int_0^\theta\frac{{\cal J}\,B\,v_d^{\,\alpha}}{v_\parallel}\, d\theta'\right)\right]\nonumber\\[0.5ex]
 & = \exp\left[-{\rm i}\,n\,\alpha -{\rm i}\,(\ell - \sigma\,n\,q)\,h\right],
 \end{align}
 and
 \begin{equation}\label{e85}
 h(\psi,\theta,E,\mu)=2\pi \left.\int_0^\theta\frac{{\cal J}\,B\,v_d^{\,\alpha}}{v_\parallel}\, d\theta' \right/
 \int_{-\theta_t}^{\theta_t}\frac{{\cal J}\,B\,v_d^{\,\alpha}}{v_\parallel}\,d\theta.
 \end{equation}
 The previous three equations ensure that ${\mit\Delta}\alpha= ({\mit\Delta}\alpha)_{res}$. Note that, for passing ions, $f_1$ is periodic in both $\theta$ and
 $\varphi$. For trapped ions, $f_1$ is periodic at $\theta=\pm \theta_t$ on a given magnetic field-line.  
 
 Combining Eqs.~(\ref{e71}) and (\ref{e83})--(\ref{e85}), we get 
 \begin{equation}\label{e86}
 \left[-{\rm i}\,(\ell-\sigma\,n\,q)\,2\pi\,v_d^{\,\alpha}\left/\int_{-\theta_t}^{\theta_t}\frac{{\cal J}\,B\,v_d^{\,\alpha}}{v_\parallel}\,d\theta\right.-
 {\rm i}\,n\,v_d^{\,\alpha}+\nu_i\right]\delta f_{\ell}\,{\cal P}_{\ell} = - v_d^{\,\psi}\,\frac{\partial f_0}{\partial\psi}.
 \end{equation}
 The bounce average operator, $\langle \cdots\rangle_b$, is defined such that
 \begin{equation}\label{bounce}
 \langle A\rangle_b =\left.\int_{-\theta_t}^{\theta_t}\frac{{\cal J}\,B\,A}{v_\parallel}\,d\theta\right/ \int_{-\theta_t}^{\theta_t}\frac{{\cal J}\,B}{v_\parallel}\,d\theta.
 \end{equation}
 Note, from Eq.~(\ref{dt}), that the bounce average is merely a time average over the bounce motion.  Moreover,  $\langle 1\rangle_b =1$. 
 The bounce average of Eq.~(\ref{e86}) divided by ${\cal P}_{\ell}$ gives
 \begin{equation}\label{e91}
 \left[-{\rm i}\,(\ell-\sigma\,n\,q)\,\omega_b -{\rm i}\,n\,\langle v_d^{\,\alpha}\rangle_b+\nu_i\right]\delta f_{\ell} = -
 \left\langle v_d^{\,\psi}\,{\cal P}_{\ell}^{\,-1}\right\rangle_b\frac{\partial f_0}{\partial\psi},
 \end{equation}
 where
 \begin{equation}
 \omega_b(\psi,E,\mu) = 2\pi\left/\int_{-\theta_t}^{\theta_t}\frac{{\cal J}\,B}{v_\parallel}\,d\theta\right.
 \end{equation}
 is the bounce frequency. 
 
\section{Ion Drift-Velocity}
The ion drift-velocity takes the form
\begin{equation}
{\bf v}_d =\frac{{\bf b}}{m\,{\mit\Omega}}\times\left[e\,\nabla{\mit\Phi}+ \mu\,\nabla B + m\,v_\parallel^{\,2}\,({\bf b}\cdot\nabla)\,{\bf b}\right],
\end{equation}
where $e$ is the magnitude of the electron charge, ${\mit\Omega}=e\,B/m$  the ion gyro-frequency, and ${\mit\Phi}(\psi)$  the
electrostatic potential. If we assume ion motion takes place at constant magnetic moment, $\mu$, and total energy,
 ${\cal E}= (1/2)\,m\,v_\parallel^{\,2}+ \mu\,B+e\,{\mit\Phi}$, then it is easily
 demonstrated that
 \begin{equation}
 {\bf v}_d = \frac{v_\parallel}{{\mit\Omega}}\,\nabla\times (v_\parallel\,{\bf b}) - \frac{v_\parallel^{\,2}}{{\mit\Omega}}\,[{\bf b}\cdot(\nabla\times {\bf b})]\,{\bf b}.
 \end{equation}
 
 Now, from Eq.~(\ref{ecurl}), 
 \begin{equation}
 v_d^{\,\alpha}= \frac{v_\parallel}{{\cal J}\,{\mit\Omega}}\left[\frac{\partial (v_\parallel\,b_\theta)}{\partial\psi} - \frac{\partial (v_\parallel\,b_\psi)}{\partial \theta}\right],
 \end{equation}
 so
 \begin{equation}
 \langle v_d^{\,\alpha}\rangle_b = \frac{\omega_b}{2\pi}\,\frac{B}{{\mit\Omega}}\int_{-\theta_t}^{\theta_t}\frac{\partial (v_\parallel\,{\cal J}\,B)}{\partial\psi}\,d\theta,
 \end{equation}
 where use has been made of Eq.~(\ref{bounce}), as well as 
 \begin{equation}
 b^\theta = ({\cal J}\,B)^{-1}\,{\bf e}_\theta\cdot{\bf e}_\theta= {\cal J}\,B.
 \end{equation}
 The latter equation follows from Eqs.~(\ref{e45}) and (\ref{gtt}). Thus, we obtain 
 \begin{equation}\label{e98}
  \langle v_d^{\,\alpha}\rangle_b=\omega_E +\omega_D
 \end{equation}
 where 
 \begin{equation}
 \omega_E(\psi)= -\frac{d{\mit\Phi}}{d\psi}
 \end{equation}
 is the  electric precession frequency, whereas 
 \begin{align}
 \omega_D(\psi,E,\mu)&=\left\langle- \frac{\mu}{e}\,\frac{\partial B}{\partial\psi} +\frac{(2\,E-2\,\mu\,B)}{e}\,\frac{\partial \ln({\cal J}\,B)}{\partial\psi}
 \right\rangle_b\nonumber\\[0.5ex]
 &=\left\langle\frac{(2\,E-3\,\mu\,B)}{e}\,\frac{1}{B}\,\frac{\partial B}{\partial\psi} +\frac{(2\,E-2\,\mu\,B)}{e}\,\frac{1}{\cal J}\,\frac{\partial{\cal J}}{\partial\psi}
 \right\rangle_b
 \end{align}
 is the magnetic precession frequency. 
 
 Now, from Eq.~(\ref{ecurl}), 
 \begin{equation}\label{e101}
 v_d^{\,\psi}= \frac{v_\parallel}{{\cal J}\,{\mit\Omega}}\left[\frac{\partial (v_\parallel\,b_\alpha)}{\partial\theta} - \frac{\partial (v_\parallel\,b_\theta)}{\partial \alpha}\right],
 \end{equation}
 which implies that 
 \begin{equation}
 \left\langle v_d^{\,\psi}\,{\cal P}_{\ell}^{\,-1}\right\rangle_b = -\left\langle{\cal P}_{\ell}^{\,-1}\, \frac{v_\parallel}{{\cal J}\,{\mit\Omega}}\,\frac{\partial \,(v_\parallel\,{\cal J}\,B)}{\partial\alpha}\right\rangle_b,
 \end{equation}
 where we have ignored the term involving $b_\alpha$ on the assumption that it will average to zero when integrated in $\alpha$. 
 Thus,
 \begin{equation}\label{e103}
 \left\langle v_d^{\,\psi}\,{\cal P}_{\ell}^{\,-1}\right\rangle_b =-\frac{\omega_b}{2\pi\,e}\,J_{\ell}^{\,\ast}
 \end{equation}
 where
 \begin{equation}
 J_\ell(\psi,\alpha,E,\mu)= \int_{-\theta_t}^{\theta_t}
 \frac{{\cal J}\,B}{v_\parallel}\,{\cal P}_{\ell}
 \left[(2\,E-3\,\mu\,B)\,\frac{1}{B}\,\frac{\partial B}{\partial \alpha} + (2\,E-2\,\mu\,B)\,\frac{1}{{\cal J}}\,\frac{\partial
 {\cal J}}{\partial\alpha}\right]d\theta,
 \end{equation}
 which reduces to 
 \begin{equation}\label{e126}
 J_\ell(\psi,\alpha,E,\mu)= \int_{-\theta_t}^{\theta_t}
 \frac{{\cal J}\,B}{v_\parallel}\,{\cal P}_{\ell}
 \left[(2\,E-3\,\mu\,B)\,\frac{\partial }{\partial \alpha} \!\left(\frac{\delta B}{B}\right)+ (2\,E-2\,\mu\,B)\,\frac{\partial
(\nabla\cdot\bxi)}{\partial\alpha}\right]d\theta
 \end{equation}
 to lowest order in perturbed quantities. 
 
 Finally, Eqs.~(\ref{e91}), (\ref{e98}), and (\ref{e103}) yield
 \begin{equation}\label{e128}
 \left[-{\rm i}\,(\ell - \sigma\,n\,q)\,\omega_b -{\rm i}\,n\,(\omega_E+\omega_D) + \nu_i\right]\delta f_\ell = \frac{\omega_b}{2\pi\,e}\,J_\ell^{\,\ast}\,\frac{\partial f_0}{\partial \psi}.
 \end{equation}
 
\section{Neoclassical Toroidal Viscous Torque}
The equilibrium ion distribution function is assumed to take the form 
\begin{equation}
f_0(\psi,E,\mu) = \frac{N(\psi)}{[2\pi\,T(\psi)/m]^{3/2}}\,\exp\left[-\frac{{\cal E}-e\,{\mit\Phi}(\psi)}{T(\psi)}\right],
\end{equation} 
where $N(\psi)$ is the ion number density, and $T(\psi)$ the ion temperature. The derivative of this distribution function with respect to $\psi$ is
taken at constant ${\cal E}$. Thus, 
\begin{equation}\label{e128a}
\left.\frac{\partial f_0}{\partial\psi}\right|_{\cal E}= -\frac{e\,f_0}{T}\left[\omega_E +\omega_{\ast N}+ \left(\frac{E}{T}-\frac{3}{2}\right)\omega_{\ast T}\right],
\end{equation}
where
\begin{align}
\omega_{\ast N}(\psi)&= - \frac{T}{N\,e}\,\frac{dN}{d\psi},\\[0.5ex]
\omega_{\ast T}(\psi)&=-\frac{1}{e}\,\frac{dT}{d\psi}
\end{align}
are the density gradient and temperature gradient contributions to the ion diamagnetic frequency, respectively. 

Equations~(\ref{e92x}), (\ref{e83}), (\ref{e126}), (\ref{e128}), and (\ref{e128a}) give the following expression for the neoclassical toroidal viscous torque,
\begin{equation}
T_\varphi =- \frac{1}{m^2}\int\int\int K_l\,\frac{\omega_b\,f_0}{T}\,\frac{[\omega_E + \omega_{\ast N}+(E/T-3/2)\,\omega_{\ast T}]}{
{\rm i}\,(\ell - \sigma\,n\,q)\,\omega_b +{\rm i}\,n\,(\omega_E+\omega_D)-\nu_i}\,d\psi\,dE\,d\mu,
\end{equation}
where
\begin{equation}
K_\ell = \oint| J_\ell|^2\,d\alpha.
\end{equation}
Let $x=E/T$, ${\mit\Lambda}=\mu\,B_0/E$, $\bar{\omega}_b(\psi,{\mit\Lambda}) = R_0\,\omega_b/(2\,x\,T\,)^{1/2}$, $\bar{J}_\ell(\psi,{\mit\Lambda}) = J_\ell/(2\,x\,T\,m\,R_0^{\,2})^{1/2}$,
and $\bar{K}_\ell(\psi, {\mit\Lambda}) = \oint |\bar{J}_\ell|^2\,d\alpha$, where all quantities are dimensionless. We find that
\begin{equation}
T_\varphi = - \frac{R_0}{\pi^{3/2}\,B_0}\int d\psi\,N\,T\int d{\mit\Lambda}\,\bar{\omega}_b\,\bar{K}_\ell\int dx\,{\cal R}_\ell,
\end{equation}
where
\begin{equation}
{\cal R}_\ell(\psi,{\mit\Lambda},x) =\frac{[\omega_E + \omega_{\ast N}+(x-3/2)\,\omega_{\ast T}]\,x^{5/2}\,{\rm e}^{-x}}{
{\rm i}\,(\ell - \sigma\,n\,q)\,\omega_b +{\rm i}\,n\,(\omega_E+\omega_D)-\nu_i},
\end{equation}
  \end{document}