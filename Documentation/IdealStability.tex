\documentclass[12pt,prb,aps,notitlepage]{revtex4-1}
\usepackage {amsmath}
\usepackage{amssymb}
\pdfoutput = 1 
\usepackage {graphicx}
\newcommand{\bomega}{\mbox{\boldmath$\omega$}}
\newcommand{\bxi}{\mbox{\boldmath$\xi$}}
\newcommand{\bbeta}{\mbox{\boldmath$\beta$}}
\allowdisplaybreaks

\begin{document}

\title{Calculation of Ideal Stability}
\author{R.~Fitzpatrick\,\footnote{rfitzp@utexas.edu}}
\affiliation{Institute for Fusion Studies,  Department of Physics,  University of Texas at Austin,  Austin TX 78712, USA}\begin{abstract}
\end{abstract}
\maketitle

\section{Analysis}
\subsection{Reference Document}
The document ``{\em Calculation of Tearing Mode Stability in an Inverse Aspect-Ratio Expanded Tokamak Equilibrium}", by R.~Fitzpatrick,  is, henceforth,
referred to as TJ. The document ``{\em An Extended Variational Method for the Resistive Wall Mode in Toroidal Confinement Devices}",
by R.~Fitzpatrick,  is, henceforth,
referred to as RWM. 

\subsection{Normalization}\label{coords}
All lengths are normalized to  the major radius of the plasma magnetic axis, $R_0$. All magnetic field-strengths
are normalized to the  toroidal field-strength at the magnetic axis, $B_0$. All currents are normalized to $B_0\,R_0/\mu_0$. All current densities are normalized to $B_0/(\mu_0\,R_0)$.  All plasma pressures are normalized to $B_0^{\,2}/\mu_0$.
All toroidal electromagnetic torques are normalized to $B_0^{\,2}\,R_0^{\,3}/\mu_0$. 

\subsection{Plasma Potential Energy}
The perturbed magnetic field in the plasma is written 
\begin{equation}
{\bf b} = \nabla\times (\bxi\times{\bf B}).
\end{equation}
The force operator in the plasma takes the form 
\begin{equation}
{\bf F}(\bxi)= \nabla({\mit\Gamma}\,P\,\nabla\cdot\bxi) - {\bf B}\times(\nabla\times {\bf b})+\nabla(\bxi\cdot\nabla P)+{\bf J}\times  {\bf b}.
\end{equation}
Finally, the plasma potential energy is written 
\begin{equation}
\delta W_p = \frac{1}{2}\int\left[{\mit\Gamma}\,P\,(\nabla\cdot\bxi^\ast)\,(\nabla\cdot\bxi)+ \nabla\times (\bxi^\ast\times {\bf B})\cdot {\bf b}
+(\nabla\cdot\bxi^\ast)\,(\bxi\cdot\nabla P)+
{\bf J}\times \bxi^\ast\,\cdot{\bf b}\right]d\tau,
\end{equation}
where $d\tau$ denotes a volume element, and the integral is over the plasma volume. 

Now,
\begin{align}
{\mit\Gamma}\,P\,(\nabla\cdot\bxi^\ast)\,(\nabla\cdot\bxi)&=\nabla\cdot[{\mit\Gamma}\,P\,\bxi^\ast\,\nabla\cdot\bxi)]-\bxi^\ast\cdot\nabla(
{\mit\Gamma}\,P\,\nabla\cdot\bxi),\\[0.5ex]
 \nabla\times (\bxi^\ast\times {\bf B})\cdot {\bf b}&= \nabla\cdot[(\bxi^\ast\times {\bf B})\times {\bf b}] + \bxi^\ast\times {\bf B}\cdot\nabla\times {\bf b},\\[0.5ex]
 (\nabla\cdot\bxi^\ast)\,(\bxi\cdot\nabla P)&=\nabla\cdot(\bxi^\ast\,\bxi\cdot\nabla P) -\bxi^\ast\cdot\nabla(\bxi\cdot\nabla P),
\end{align}
so
\begin{align}
\delta W_p &= \frac{1}{2}\int\left\{\nabla\cdot[{\mit\Gamma}\,P\,\bxi^\ast\,\nabla\cdot\bxi+ (\bxi^\ast\times {\bf B})\times {\bf b}+\bxi^\ast\,\bxi\cdot\nabla P]\right.\nonumber\\[0.5ex]&\left.-\bxi^\ast\cdot[ \nabla({\mit\Gamma}\,P\,\nabla\cdot\bxi) - {\bf B}\times(\nabla\times {\bf b})+\nabla(\bxi\cdot\nabla P)+{\bf J}\times  {\bf b}]\right\}d\tau,
\end{align}
which yields
\begin{align}
\delta W_p &= \frac{1}{2}\left(\oint\oint{\cal J}\,\nabla r\cdot[{\mit\Gamma}\,P\,\bxi^\ast\,\nabla\cdot\bxi+ (\bxi^\ast\times {\bf B})\times {\bf b}+\bxi^\ast\,\bxi\cdot\nabla P]\,d\theta\,d\phi\right)_{\hat{r}=1-}\nonumber\\[0.5ex]
&-\frac{1}{2}\int\bxi^\ast\cdot{\bf F}(\bxi)\,d\tau,
\end{align}
or
\begin{align}
\delta W_p &= \frac{1}{2}\left(\oint\oint{\cal J}\,\xi^{r\,\ast}\,[{\mit\Gamma}\,P\,\nabla\cdot\bxi -{\bf B}\cdot{\bf b} + \xi^r\,P']\,d\theta\,d\phi\right)_{\hat{r}=1_{-}}-\frac{1}{2}\int\bxi^\ast\cdot{\bf F}(\bxi)\,d\tau,
\end{align}
where use has been made of the fact that ${\bf B}\cdot\nabla r = 0$. However, the trial function is calculated with ${\bf F}(\bxi)={\bf 0}$ and
$\nabla\cdot\bxi=0$  in the plasma. Moreover, $P'=0$ at the plasma boundary. Thus,
\begin{equation}
\delta W_p = -\frac{1}{2}\left(\oint\oint {\cal J}\,\xi^{r\,\ast}\,{\bf B}\cdot{\bf b}\,d\theta\,d\phi\right)_{\hat{r}=1_{-}}.
\end{equation}

Now, 
\begin{equation}
{\bf B}\cdot{\bf b} = B^\theta\,b_\theta+B^\phi\,b_\phi = \frac{f}{{\cal J}}\left({\rm i}\,\frac{\partial z}{\partial \theta} + n\,q\,z\right),
\end{equation}
where use has been made of Eqs.~(4), (6), (7), (47), (48), and (78) of TJ.
But, according to Eq.~(26) of TJ,
\begin{equation}
\xi^r = \frac{y}{f}.
\end{equation}
Thus, we obtain
\begin{equation}
\delta W_p = -\frac{1}{2}\left(\oint\oint y^\ast\left[{\rm i}\,\frac{\partial z}{\partial \theta}+q\,n\,z\right]d\theta\,d\phi\right)_{\hat{r}=1_{-}}=
\pi^2\left[\sum_m y_m^{\,\ast}\,(m-n\,q)\,z_m\right]_{\hat{r}=1_{-}},
\end{equation}
which yields 
\begin{equation}
\delta W_p  = \pi^2\left(\sum_m \frac{\psi_m^{\,\ast}\,Z_m}{m-n\,q}\right)_{\hat{r}=1_{-}},
\end{equation}
where use has been made of Eqs.~(98) and (99) of TJ.

\subsection{Electromagnetic Torque and Plasma Energy}
According to Eq.~(117) of TJ, the net toroidal electromagnetic torque acting on a plasma is given by
\begin{align}
T_\phi&={\rm i}\,n\,\pi^2\left(\sum_m\frac{Z_m^\ast\,\psi_m-\psi_m^\ast\,Z_m}{m-n\,q}\right)_{\hat{r}=1_{-}}
=2\,n\,\pi^2\,{\rm Im}\left(\sum_m\frac{\psi_m^\ast\,Z_m}{m-n\,q}\right)_{\hat{r}=1_{-}}.
\end{align}
The previous two equations imply that
\begin{equation}
T_\phi = 2\,n\,{\rm Im}(\delta W_p).
\end{equation}
However, an ideal plasma has zero torque acting on it, so 
\begin{equation}
\left(\sum_m\frac{Z_m^\ast\,\psi_m-\psi_m^\ast\,Z_m}{m-n\,q}\right)_{\hat{r}=1_{-}}=0,
\end{equation}
which implies that $\delta W_p$ is real. 

\subsection{Boundary Current Sheet}
According to RWM, the trial solutions used to calculate $\delta W$ are such that  $\psi_m(\hat{r})$ is continuous at the edge of the plasma, $\hat{r}=1$, 
but $Z_m(\hat{r})$ is, in general, discontinuous. The discontinuity in $Z_m$ at the plasma boundary is due to the fact that the pressure balance
matching condition is not satisfied. The discontinuity is also associated with a (fictitious) current sheet flowing around the plasma boundary.

Now, $\alpha_p=\alpha_g=0$ at $\hat{r}=1$, assuming that there are no edge equilibrium currents. It follows from Eqs.~ (78) and
(99) of TJ that 
\begin{equation}
x_m= \frac{n\,Z_m}{m-n\,q}
\end{equation}
at the plasma boundary. Thus, if ${\bf K}$ is the current sheet density then Eqs.~(66) and (67) of TJ suggest that
\begin{align}
{\cal J}\,K_m^{\,\theta} &= - \frac{n}{m-n\,q(1)}\,[Z_m]_{1_{-}}^{1_{+}},\\[0.5ex]
{\cal J}\,K_m^{\,\phi} &=-\frac{m}{m-n\,q(1)}\,[Z_m]_{1_{-}}^{1_{+}}.
\end{align}
Let us write
\begin{equation}
{\bf K} = {\rm i}\,\nabla J\times \nabla r,
\end{equation}
which ensures that $\nabla\cdot{\bf K}=0$. It follows from (A8) and (A9) of TJ that
\begin{align}
{\cal J}\,K^\theta&= {\rm i}\,\frac{\partial J}{\partial \phi},\\[0.5ex]
{\cal J}\,K^\phi &=-{\rm i}\,\frac{\partial J}{\partial\theta}.
\end{align}
Hence, we deduce that 
\begin{equation}
J_m =-\frac{[Z_m]_{1_-}^{1_+}}{m-n\,q(1)}.
\end{equation}

\subsection{Vacuum Potential Energy}
In the vacuum region, we can write
\begin{equation}
{\bf b} = {\rm i}\,\nabla V,
\end{equation}
where
\begin{equation}
\nabla^2 V =0.
\end{equation}
Hence, the vacuum potential energy is
\begin{align}
\delta W_v &= \frac{1}{2}\int \nabla V\,\cdot\nabla V^\ast\,d\tau= \frac{1}{2}\int  \nabla\cdot(V\,\nabla V^\ast)\,d\tau\nonumber\\[0.5ex]
&=-\frac{1}{2}\left(\oint\oint {\cal J}\,\nabla r\cdot\nabla V^\ast\,V\,d\theta\,d\phi\right)_{1_+} .
\end{align}
But, Eq.~(209) of TJ implies that 
\begin{equation}
{\cal J}\,\nabla V\cdot\nabla r = \psi,
\end{equation}
so we deduce that
\begin{equation}
\delta W_v = -\frac{1}{2}\left(\oint\oint \psi^\ast\,V\,d\theta\,d\phi\right)_{1_{+}} =- \pi^2\left(\sum \psi_m^\ast\,V_m\right)_{1_{+}}.
\end{equation}
However, making use of Eq.~(214) of TJ, we get
\begin{equation}
\delta W_v = - \pi^2\left(\sum \frac{\psi_m^\ast\,Z_m}{m-n\,q}\right)_{1_{+}}.
\end{equation}

\subsection{Potential Energy}
The total potential energy associated with the trial function is
\begin{equation}
\delta W = \delta W_p + \delta W_v = -\pi^2\sum_m \frac{\psi_m^\ast(1)\,[Z_m]_{1_{-}}^{1_{+}}}{m-n\,q(1)},
\end{equation}
which implies that
\begin{equation}
\delta W =\pi^2\sum_m\,\psi_m^\ast(1)\,J_m.
\end{equation}

\subsection{Ideal Eigenfunctions}
The complete set of ideal solutions of the outer region o.d.e.s is written
\begin{align}
\psi_{m\,m'}^i(\hat{r})&= \psi_{m\,m'}^a(\hat{r})-\sum_{k}\psi_{m\,k}^u(\hat{r})\,{\mit\Pi}_{k\,m'}^a,\\[0.5ex]
Z_{m\,m'}^i(\hat{r})&= Z_{m\,m'}^a(\hat{r})-\sum_{k}Z_{m\,k}^u(\hat{r})\,{\mit\Pi}_{k\,m'}^a,
\end{align}
where $m=m_j$, $m'=m_{j'}$, and all other quantities are defined in Sects.~VIII.B and VIII.D of TJ. 
Here, $m$ is the poloidal harmonic, and $m'$ is the dominant poloidal harmonic close to the magnetic axis. It follows
from the definitions of the various quantities that the reconnected fluxes at the rational surfaces in the plasma
associated with these eigenfunctions are all zero. 

\subsection{Energy Matrix}
We can write a general ideal solution as
\begin{align}
\psi_m(\hat{r}) &= \sum_{m'}\psi_{m\,m'}^i(\hat{r})\,\alpha_{m'},\\[0.5ex]
Z_m(\hat{r}) &= \sum_{m'}Z_{m\,m'}^i(\hat{r})\,\alpha_{m'}.
\end{align}
According to Eq,~(215) of TJ, the boundary condition at the plasma/vacuum interface is
\begin{equation}
\frac{Z_m(1_+)}{m-n\,q(1)} = \sum_{m'}H_{m\,m'}\,\psi_{m'}(1).
\end{equation}
It follows that
\begin{equation}
J_m = \frac{Z_m(1_-)}{m-n\,q(1)}-\sum_{m'} H_{m\,m'}\,\psi_{m'}(1).
\end{equation}
Thus, we can write 
\begin{equation}
J_m = \sum_{m'}\,J_{m\,m'}^i\,\alpha_{m'},
\end{equation}
where
\begin{equation}
J_{m\,m'}^i = \frac{Z_{m\,m'}^i(1_-)}{m-n\,q(1)}-\sum_{m''} H_{m\,m''}\,\psi_{m''\,m'}^i(1).
\end{equation}

It follows that
\begin{equation}
\delta W = \sum_{m,m'} \alpha_m^{\,\ast}\,W_{m\,m'}\,\alpha_{m'},
\end{equation}
where
\begin{equation}
W_{m\,m'} =\pi^2\sum_{m''} \psi_{m''\,m}^{i\,\ast}(1)\,J_{m''\,m'}^i.
\end{equation}
Moreover, 
\begin{equation}
W_{m\,m'} =\pi^2\left[\sum_{m''}\frac{\psi_{m''\,m}^{i\,\ast}(1)\,Z_{m''\,m'}^i(1_-)}{m''-n\,q(1)}
-\sum_{m'',m'''}\psi_{m''\,m}^{i\,\ast}(1)\,H_{m''\,m'''}\,\psi^i_{m'''\,m'}(1)\right]
\end{equation}
Now,
\begin{equation}
W_{m'\,m}^\ast =\pi^2\left[\sum_{m''}\frac{Z_{m''\,m}^{i\,\ast}(1_-)\,\psi_{m''\,m'}^i(1)}{m''-n\,q(1)}
-\sum_{m'',m'''}\psi_{m''\,m'}^{i\,\ast}(1)\,H^\ast_{m'''\,m''}\,\psi^i_{m'''\,m'}(1)\right]
\end{equation}
However, $H_{m\,m'}$ is Hermitian, so
\begin{equation}
W_{m\,m'}-W_{m'\,m}^\ast =-\pi^2\left(\sum_{m''} \frac{Z_{m''\,m}^{i\,\ast}\,\psi_{m''\,m'}^i -\psi_{m''\,m}^{i\,\ast}\,Z_{m''\,m'}^i}{m''-n\,q}\right)_{\hat{r}=1_-}.
\end{equation}
But, the toroidal electromagnetic torque acting on the plasma is
\begin{equation}
T_\phi={\rm i}\,n\,\pi^2\left(\sum_m\frac{Z_m^\ast\,\psi_m-\psi_m^\ast\,Z_m}{m-n\,q}\right)_{\hat{r}=1_{-}}=\sum_{m,m'}\alpha_m^{\,\ast}\,T_{m\,m'}\,\alpha_{m'},
\end{equation}
where
\begin{equation}
T_{m\,m'} = {\rm i}\,n\,\pi^2\left(\sum_{m''} \frac{Z_{m''\,m}^{i\,\ast}\,\psi_{m''\,m'}^i -\psi_{m''\,m}^{i\,\ast}\,Z_{m''\,m'}^i}{m''-n\,q}\right)_{\hat{r}=1_-}
\end{equation}
However, $T_\phi=0$ for an ideal plasma, irrespective of the values of the $\alpha_m$. Hence, we deduce that $T_{m\,m'}=0$ for all
$m$ and $m'$. It immediately follows that the energy matrix, $W_{m\,m'}$, is Hermitian. 

\subsection{Diagonalization of Energy Matrix}
Given that the energy matrix, $W_{m\,m'}$, is Hermitian, it possesses real eigenvalues, $\delta W_m$, and orthonormal
eigenvectors $\bbeta_m$. Let $(\bbeta_m)_{m'} = \beta_{m'\,m}$. 
We have
\begin{align}
\sum_{m''} W_{m\,m''}\,\beta_{m''\,m'} &= \delta W_{m'}\,\beta_{m\,m'},\\[0.5ex]
\sum_{m''} \beta_{m''\,m}^\ast\,\beta_{m''\,m'} &= \delta_{mm'}.
\end{align}
Now,
\begin{align}
\sum_{m'',m'''}\beta^\ast_{m''\,m}\,W_{m''\,m'''}\,\beta_{m'''\,m'} = \sum_{m''} \beta^\ast_{m''\,m}\,\beta_{m''\,m'}\,\delta W_{m'}=\delta W_{m'}\,\delta_{mm'}.
\end{align}
It follows that
\begin{equation}
\pi^2\sum_{m'',m''',m''''}\beta^\ast_{m''\,m}\,\psi^{i\,\ast}_{m''''\,m''}(1)\,J^i_{m'''',m'''}\,\beta_{m'''\,m'} = \delta W_{m'}\,\delta_{mm'}.
\end{equation}
Let,
\begin{align}
\hat{\psi}^i_{m\,m'}(\hat{r}) &= \sum_{m''}\psi^i_{m\,m''}(\hat{r})\,\beta_{m''\,m'},\\[0.5ex]
\hat{Z}^i_{m\,m'}(\hat{r}) &= \sum_{m''}Z^i_{m\,m''}(\hat{r})\,\beta_{m''\,m'},\\[0.5ex]
\hat{J}^i_{m\,m'} &= \sum_{m''}J^i_{m\,m''}\,\beta_{m''\,m'}.
\end{align}
It follows that
\begin{equation}
\sum_{m''} \hat{\psi}^{i\,\ast}_{m''\,m}(1)\,\hat{J}^i_{m''\,m'} = \delta\hat{W}_{m'}\,\delta_{mm'},
\end{equation}
where
\begin{equation}
\delta\hat{W}_m = \frac{\delta W_m}{\pi^2}.
\end{equation}
Thus, we can write
\begin{align}
\hat{J}^i_{m'\,m} &= \delta\hat{W}_m\,\hat{\psi}^i_{m'\,m}(1),\\[0.5ex]
\sum_{m'' }\hat{\psi}^{i\,\ast}_{m''\,m}(1)\,\hat{\psi}^i_{m''\,m'}(1)&=\delta_{mm'}.
\end{align}
We can also write
\begin{align}
\psi^i_{m\,m'}(\hat{r}) &= \sum_{m''}\,\hat{\psi}^i_{m\,m''}(\hat{r})\,\beta_{m'\,m''}^\ast\,\\[0.5ex]
Z^i_{m\,m'}(\hat{r}) &= \sum_{m''}\hat{Z}^i_{m\,m''}(\hat{r})\,\beta_{m'\,m''}^\ast\,. 
\end{align}

\subsection{Response to RMP}
We have
\begin{align}
\psi_m^{rmp}(\hat{r})&=-\sum \psi^i_{m\,m'}(\hat{r})\,{\mit\Upsilon}_{m'},\\[0.5ex]
Z_m^{rmp}(\hat{r})&=-\sum Z^i_{m\,m'}(\hat{r})\,{\mit\Upsilon}_{m'}.
\end{align}
Thus, we can write 
\begin{align}
\psi_m^{rmp}(\hat{r})&=\sum \hat{\psi}^i_{m\,m'}(\hat{r})\,\hat{\mit\Upsilon}_{m'},\\[0.5ex]
Z_m^{rmp}(\hat{r})&=\sum \hat{Z}^i_{m\,m'}(\hat{r})\,\hat{\mit\Upsilon}_{m'},
\end{align}
where
\begin{align}
\hat{\mit\Upsilon}_{m} =- \sum_{m'}\beta_{m'\,m}^\ast\,{\mit\Upsilon}_{m'},
\end{align}
It is easily seen that
\begin{equation}
\psi^x_m = \sum_{m'}\,\psi^i_{m\,m'}(1)\,\hat{\mit\Upsilon}_{m'}^x,
\end{equation}
where
\begin{equation}
\hat{\mit\Upsilon}_{m'}^x= \sum_{m'} \psi_{m'}^x\,\hat{\psi}_{m'\,m}^{i\,\ast}(1).
\end{equation}

\end{document}