\documentclass[12pt,prb,aps,notitlepage]{revtex4-1}
\usepackage {amsmath}
\usepackage{amssymb}
\pdfoutput = 1 
\usepackage {graphicx}
\newcommand{\bomega}{\mbox{\boldmath$\omega$}}
\allowdisplaybreaks

\begin{document}

\title{Pressure Flattening due to Asymmetric Magnetic Island}
\author{R.~Fitzpatrick\,\footnote{rfitzp@utexas.edu}}
\affiliation{Institute for Fusion Studies,  Department of Physics,  University of Texas at Austin,  Austin TX 78712, USA}\begin{abstract}
\end{abstract}
\maketitle

\section{Magnetic Island}
Let $x=r-r_s$, $X=x/W$, and $\zeta= m\,\theta-n\,\phi$, where $W$ is the island width. The magnetic flux-surfaces of the magnetic island are contours of
\begin{equation}
{\mit\Omega}(X,\zeta) = 8\,X^2 + \cos(\zeta-\delta^2\,\sin\zeta) - 2\sqrt{8}\,\delta\,X\,\cos\zeta+\delta^2\,\cos^2\zeta,
\end{equation}
where $|\delta|<1$. 
The X-points lie at $X=\delta/\sqrt{8}$ and $\zeta = 0$, $2\pi$, whereas the O-point lies at
$X=-\delta/\sqrt{8}$ and $\zeta=\pi$. The O-point corresponds to ${\mit\Omega}=-1$, whereas the magnetic separatrix corresponds to ${\mit\Omega}=1$.  

Let
\begin{align}
Y &= X -\frac{\delta}{\sqrt{8}}\,\cos\zeta,\\[0.5ex]
\xi &= \zeta-\delta^2\,\sin\zeta.
\end{align}
It follows that
\begin{equation}
{\mit\Omega}(Y,\zeta) = 8\,Y^2 +\cos\xi,
\end{equation}
The X-points lie at $Y=0$ and $\xi = 0$, $2\pi$, whereas the O-point lies at
$Y=0$ and $\zeta=\pi$. 
Moreover, 
\begin{align}
\zeta &= \xi+2\sum_{n=1,\infty} \left[\frac{J_n(n\,\delta^2)}{n}\right]\sin(n\,\xi),\\[0.5ex]
\cos\zeta&=-\frac{\delta^2}{2}+\sum_{n=1,\infty}\left[\frac{J_{n-1}(n\,\delta^2)-J_{n+1}(n\,\delta^2)}{n}\right]\cos(n\,\xi),\\[0.5ex]
\sin\zeta &= \frac{2}{\delta^2}\sum_{n=1,\infty} \left[\frac{J_n(n\,\delta^2)}{n}\right]\sin(n\,\xi),\\[0.5ex]
\cos(m\,\zeta)&= m\sum_{n=1,\infty}\left[\frac{J_{n-m}(n\,\delta^2)-J_{n+m}(n\,\delta^2)}{n}\right]\cos(n\,\xi),\\[0.5ex]
\sin(m\,\zeta) &= m\sum_{n=1,\infty}\left[\frac{J_{n-m}(n\,\delta^2)+J_{n+m}(n\,\delta^2)}{n}\right]\sin(n\,\xi),
\end{align}
for $m>1$. 

\section{Plasma Displacement}
Outside the separatrix, we can write
\begin{equation}
{\mit\Omega}(X,\zeta) = 8\,(X-{\mit\Xi})^2,
\end{equation}
where ${\mit\Xi}= \xi^r/W$. It follows that
\begin{align}
{\mit\Xi}(X,\zeta)&\simeq -\frac{[{\mit\Omega}(X,\zeta)-8\,X^2]}{16\,X}\nonumber\\[0.5ex]
&=- \frac{\cos(\zeta-\delta^2\,\sin\zeta) +\delta^2\,\cos^2\zeta}{16\,X}+\frac{\delta}{\sqrt{8}}\,\cos\zeta.
\end{align}
Note that ${\mit\Xi}$ is an even function of $\zeta$. 
Let us write
\begin{equation}
{\mit\Xi}(X,\zeta)= \sum_{n=0,\infty} {\mit\Xi}_n(X)\,\cos(n\,\zeta).
\end{equation}
Thus,
\begin{align}
{\mit\Xi}_1(X) =2 \oint {\mit\Xi}(X,\zeta)\,\cos(\zeta)\,\frac{d\zeta}{2\pi} &= -\frac{1}{8\,X}\oint\cos(\zeta-\delta^2\,\sin\zeta)\,\cos\zeta\,\frac{d\zeta}{2\pi}+\frac{\delta}{\sqrt{8}} \\[0.5ex]
&=-\frac{1}{16\,X}\oint\cos(-\delta^2\,\sin\zeta)\,\cos\zeta\,\frac{d\zeta}{2\pi}\nonumber\\[0.5ex]
&\phantom{=}- \frac{1}{16\,X}\oint\cos(2\,\zeta-\delta^2\,\sin\zeta)\,\cos\zeta\,\frac{d\zeta}{2\pi}+\frac{\delta}{\sqrt{8}} .
\end{align}
But,
\begin{equation}
J_n(\delta^2) = \oint\cos(n\,\zeta-\delta^2\,\sin\zeta)\,\frac{d\zeta}{2\pi},
\end{equation}
so
\begin{equation}
{\mit\Xi}_1(X) =- \frac{J_0(\delta^2) + J_2(\delta^2)}{16\,X}+ \frac{\delta}{\sqrt{8}} ,
\end{equation}
and
\begin{equation}
\xi^r_1(x) =- \frac{W^2}{16\,x}\,[J_0(\delta^2)+ J_2(\delta^2)]+ \frac{W\,\delta}{\sqrt{8}} .
\end{equation}
Thus,
\begin{align}
\psi_1(x) &= \frac{r\,g}{q}\,(m-n\,q)\,\xi_1^r = -(n\,s\,g)_{r_s}\,x\,\xi_1^r\nonumber\\[0.5ex]
&= (n\,s\,g)_{r_k}\,\frac{W^2}{16}\,[J_0(\delta^2)+ J_2(\delta^2)] - (n\,s\,g)_{r_s}\,\frac{W\,\delta}{\sqrt{8}}\,x,
\end{align}
so
\begin{equation}
\frac{\psi_m(x)}{m} = J_0(\delta^2)+ J_2(\delta^2) - 2\sqrt{8}\,\frac{\delta}{W}\,x.
\end{equation}
It follows that
\begin{align}
\delta &= -\frac{W}{2\sqrt{8}\,m}\,\frac{d\psi_m(0)}{dr}\\[0.5ex]
&\simeq -\left[\frac{\psi_m(r_s+W)-\psi_m(r_s-W)}{4\sqrt{8}\,m}\right].
\end{align}

\section{Flux-Surface Average Operator}
Now,
\begin{equation}
[A,B] \equiv \left.\frac{\partial A}{\partial X}\right|_\zeta\left.\frac{\partial B}{\partial\zeta}\right|_X- \left.\frac{\partial B}{\partial X}\right|_\zeta\left.\frac{\partial A}{\partial\zeta}\right|_X.
\end{equation}
But,
\begin{equation}
\left.\frac{\partial}{\partial X}\right|_\zeta= \left.\frac{\partial{\mit\Omega}}{\partial X}\right|_\zeta\left.\frac{\partial}{\partial{\mit\Omega}}\right|_{\xi}+ \left.\frac{\partial\xi}{\partial X}\right|_\zeta\left.\frac{\partial}{\partial\xi}\right|_{\mit\Omega}=16\,Y\left.\frac{\partial}{\partial{\mit\Omega}}\right|_{\xi},
\end{equation}
and
\begin{equation}
\left.\frac{\partial}{\partial \zeta}\right|_X= \left.\frac{\partial{\mit\Omega}}{\partial \zeta}\right|_X\left.\frac{\partial}{\partial{\mit\Omega}}\right|_{\xi}+ \left.\frac{\partial\xi}{\partial \zeta}\right|_X\left.\frac{\partial}{\partial\xi}\right|_{\mit\Omega},
\end{equation}
so
\begin{equation}
[A,B] \equiv \frac{16\,Y}{\sigma}\left(\left.\frac{\partial A}{\partial{\mit\Omega}}\right|_\xi\left.\frac{\partial B}{\partial\xi}\right|_{\mit\Omega}-\left.\frac{\partial B}{\partial{\mit\Omega}}\right|_\xi\left.\frac{\partial A}{\partial\xi}\right|_{\mit\Omega}\right),
\end{equation}
where
\begin{equation}
\sigma(\xi) \equiv\frac{d\zeta}{d\xi}=  1+2\sum_{n=1,\infty} J_n(n\,\delta^2)\,\cos(n\,\xi).
\end{equation}
In particular,
\begin{equation}
[A,{\mit\Omega}] = -\frac{16\,Y}{\sigma}\left.\frac{\partial A}{\partial\xi}\right|_{\mit\Omega}.
\end{equation}

The flux-surface average operator, $\langle\cdots\rangle$, is the annihilator of $[A,{\mit\Omega}]$ for arbitrary $A(s,{\mit\Omega},\xi)$. Here, $s=+1$ for $Y>0$ and $s=-1$ for
$Y<0$. It follows that
\begin{equation}
\langle A\rangle = \int_{\zeta_0}^{2\pi-\zeta_0}\frac{\sigma(\xi)\,A_+({\mit\Omega},\xi)}{\sqrt{2\,({\mit\Omega}-\cos\xi)}}\,\frac{d\xi}{2\pi}
\end{equation}
for $-1\leq {\mit\Omega}\leq 1$, and
\begin{equation}
\langle A\rangle = \int_0^{2\pi}\frac{\sigma(\xi)\,A(s,{\mit\Omega},\xi)}{\sqrt{2\,({\mit\Omega}-\cos\xi)}}\,\frac{d\xi}{2\pi}
\end{equation}
for ${\mit\Omega}>1$. Here, $\xi_0=\cos^{-1}({\mit\Omega})$, and
\begin{equation}
A_+({\mit\Omega},\xi)= \frac{1}{2}\left[A(+1,{\mit\Omega},\xi) + A(-1,{\mit\Omega},\xi)\right].
\end{equation}

\section{Temperature Perturbation}
The electron temperature in the vicinity of the island can be written
\begin{equation}
T_e(X,\zeta) = T_{e\,s} + s\,W\,T_{e\,s}'\,\tilde{T}({\mit\Omega}).
\end{equation}
Here, $\tilde{T}({\mit\Omega})$ satisfies
\begin{equation}\label{e30}
\left\langle \left.\frac{\partial ^2\tilde{T}}{\partial X^2}\right|_\zeta \right\rangle =0,
\end{equation}
subject to the boundary condition that
\begin{equation}
\tilde{T}({\mit\Omega})\rightarrow |X|
\end{equation}
as $|X|\rightarrow \infty$. It
follows that
\begin{equation}\label{e34}
\frac{d}{d{\mit\Omega}}\!\left(\langle Y^2\rangle\,\frac{d\tilde{T}}{d{\mit\Omega}}\right)=0
\end{equation}
subject to the boundary condition that
\begin{equation}
\tilde{T}({\mit\Omega})\rightarrow \frac{{\mit\Omega}^{1/2}}{\sqrt{8}}
\end{equation}
as ${\mit\Omega}\rightarrow\infty$. 

Outside the separatrix,
\begin{equation}
\langle Y^2\rangle({\mit\Omega}) = \frac{1}{16}\int_0^{2\pi}\sigma(\xi)\sqrt{2\,({\mit\Omega}-\cos\xi)}\,\frac{d\xi}{2\pi}.
\end{equation}
Let 
\begin{equation}
k = \left(\frac{1+{\mit\Omega}}{2}\right)^{1/2}.
\end{equation}
Thus, the O-point corresponds to $k=0$ and the separatrix to $k=1$. 
It follows that 
\begin{equation}
\langle Y^2\rangle(k) = \frac{k}{4\pi}\int_0^{\pi/2}\sigma(2\,\theta-\pi)\,(1-\sin^2\theta/k^2)^{1/2}\,d\theta.
\end{equation}
Thus,
\begin{equation}
\langle Y^2\rangle(k) = \frac{k}{4\pi}\,G(1/k),
\end{equation}
where
\begin{align}
G(p) &=E_0(p) +2\,\cos(n\,\pi)\sum_{n=1,\infty}J_n(n\,\delta^2)\,E_n(p),\\[0.5ex]
E_n(p) &= \int_0^{\pi/2} \cos(2\,n\,\theta)\,(1-p^2\,\sin^2\theta)^{1/2}\,d\theta.
\end{align}

Equation~(\ref{e34}) yields
\begin{equation}
\tilde{T}(k) = 0
\end{equation}
for $0\leq k\leq 1$, and 
\begin{equation}
\frac{d}{dk}\!\left[G(1/k)\,\frac{d\tilde{T}}{dk}\right]=0
\end{equation}
for $k>1$. Thus,
\begin{equation}
\frac{d\tilde{T}}{dk} = \frac{c}{G(1/k)}
\end{equation}
for $k>1$, subject to the boundary condition that
\begin{equation}
\tilde{T}(k)\rightarrow \frac{k}{2}
\end{equation}
as $k\rightarrow \infty$. In the limit that $p\rightarrow 0$, 
\begin{align}
E_0(p)&=\frac{\pi}{2},\\[0.5ex]
E_{n>0}(p) &= 0,
\end{align}
which implies that $c=\pi/4$. So
\begin{align}
\frac{d\tilde{T}}{dk} &= \frac{\pi}{4}\,\frac{1}{G(1/k)},\\[0.5ex]
\tilde{T}(k) &= F(k),\\[0.5ex]
F(k) &= \frac{\pi}{4}\int_1^k\frac{dk'}{G(1/k')}
\end{align}
for $k>1$. 

\section{Harmonics of Temperature Perturbation}
We can write
\begin{equation}
\tilde{T}(X,\zeta)=\sum_{\nu=0,\infty}\delta T_\nu(X)\,\cos(\nu\,\zeta).
\end{equation}
Now,
\begin{equation}
\delta T_0(X) = \oint \tilde{T}(X,\zeta)\,\frac{d\zeta}{2\pi},
\end{equation}
where the integral is at constant $X$. It follows that
\begin{equation}
\delta T_0(X) = \int_0^{\xi_c}F(k)\,\sigma(\xi)\,\frac{d\xi}{\pi},
\end{equation}
where 
\begin{equation}
\xi_c = \cos^{-1}(1-8\,Y^2)
\end{equation}
for $|Y|<1/2$, and $\xi_c=\pi$ for $|Y|\geq 1/2$. Furthermore,
\begin{equation}
k =\left[4\,Y^2 +\cos^2\left(\frac{\xi}{2}\right)\right]^{1/2}.
\end{equation}
Let 
\begin{equation}
\delta T_{0\,\infty} =\lim_{X\rightarrow \infty}\left[X - \delta T_0(X)\right].
\end{equation}

For $\nu>0$, we have
\begin{equation}
\delta T_\nu(X) = 2\oint\tilde{T}(X,\zeta)\,\cos(\nu\,\zeta)\,\frac{d\zeta}{2\pi}.
\end{equation}
Integrating by parts, we obtain
\begin{equation}
\delta T_\nu(X) = -\frac{2}{\nu}\oint\left.\frac{\partial \tilde{T}}{\partial\zeta}\right|_X\,\sin(\nu\,\zeta)\,\frac{d\zeta}{2\pi}.
\end{equation}
But,
\begin{equation}
\left.\frac{\partial \tilde{T}}{\partial\zeta}\right|_X=\frac{d\tilde{T}}{dk}\left.\frac{\partial k}{\partial\zeta}\right|_{X}
=\frac{1}{4\,k}\frac{d\tilde{T}}{dk}\,\left.\frac{\partial {\mit\Omega}}{\partial\zeta}\right|_{X}=-\frac{1}{4\,k}\,\frac{d\tilde{T}}{dk}\,\kappa(\xi),
\end{equation}
where
\begin{equation}
\kappa(\xi) = \sin\xi\,(1-\delta^2\,\cos\zeta)  -2\sqrt{8}\,\delta\,X\,\sin\zeta +\delta^2\,\sin(2\,\zeta).
\end{equation}
Hence,
\begin{equation}
\delta T_\nu(X) =\frac{1}{8\,\nu}\int_0^{\xi_c}\frac{\sin(\nu\,\zeta)\,\kappa(\xi)\,\sigma(\xi)}{k\,G(1/k)}\,d\xi.
\end{equation}

\section{Asymptotic Matching}
Consider the $k$th rational surface whose radius is $r_k$ and whose resonant poloidal mode number is $m_k$. Let  $x=r-r_k$ and $\zeta_k=m_k\,\theta-n\,\phi$. 

 In the outer region,  we write the total electron temperature as 
\begin{equation}\label{e36}
\tilde{T}_e(r, \theta,\phi)=T_{e\,0}(r) - {\mit\Psi}_k\,\frac{q(r)}{r\,g(r)}\,\frac{T_{e\,0}'(r)\,\psi_{m_k}(r)}{m_k-n\,q(r)}\,{\rm e}^{\,{\rm i}\,\zeta_k},
\end{equation}
where $T_{e\,0}'=dT_{e\,0}/dr$, $T_{e\,0}(r)$ is the equilibrium electron temperature profile, ${\mit\Psi}_k$ is the reconnected flux, and 
\begin{equation}
W_k = 4\left(\frac{q}{g\,s}\right)^{1/2}_{r_k}\,{\mit\Psi}_k^{1/2}\delta T_{0\,\infty}
\end{equation}
is the island width.
 In the limit, $|x|\ll 1$, Eq.~(\ref{e36}) yield 
\begin{equation}
\tilde{T}_e(x, \theta,\phi)=T_{e\,k} + T_{e\,k}'\,x + T_{e\,k}'\,W_k\left(\frac{W_k}{16\,x} - \frac{\delta_k}{\sqrt{8}}\right){\rm e}^{\,{\rm i}\,\zeta_k},
\end{equation}
Here, $T_{e\,k}=T_{e\,0}(r_k)$ and $T_{e\,k}'= (dT_{e\,0}/dr)_{r_k}$, 
\begin{equation}
\delta_k =  -\frac{W_k}{2\sqrt{8}\,m}\,\frac{d\psi_{m_k}(r_k)}{dr},
\end{equation}
and we have made use of the fact that $\psi_{m_k}(r_k)\simeq m_k$. 

In the inner region, we write the total electron temperature as 
\begin{equation}
\tilde{T}_e(x, \theta,\phi)=T_{e\,k} +T_{e\,k}'\,W_k\sum_{\nu=0,\infty}\delta T_\nu(x/W_k)\,{\rm e}^{\,{\rm i}\,\nu\,\zeta_k}+T_{e\,k}'\,W_k\,\delta T_{0\,\infty},
\end{equation}

The asymptotic matching process consists of writing
\begin{equation}
\tilde{T}_e(r,\theta,\phi) = T_{e\,0}(r) + \delta T_{e\,+} - {\mit\Psi}_{k+}\,\frac{q(r)}{r\,g(r)}\,\frac{T_{e\,0}'(r)\,\psi_{m_k}(r)}{m_k-n\,q(r)}\,{\rm e}^{\,{\rm i}\,\zeta_k}
\end{equation}
in the region $r>r_k+W_k$, 
\begin{equation}
\tilde{T}_e(r,\theta,\phi) = T_{e\,0}(r) + \delta T_{e\,-} - {\mit\Psi}_{k-}\,\frac{q(r)}{r\,g(r)}\,\frac{T_{e\,0}'(r)\,\psi_{m_k}(r)}{m_k-n\,q(r)}\,{\rm e}^{\,{\rm i}\,\zeta_k}
\end{equation}
in the region $r< r_k-W_k$, and 
\begin{equation}
\tilde{T}_e(r, \theta,\phi)=T_{e\,k} +T_{e\,k}'\,W_k\sum_{\nu=0,\infty}\delta T_\nu(x/W_k)\,{\rm e}^{\,{\rm i}\,\nu\,\zeta_k}+T_{e\,k}'\,W_k\,\delta T_{0\,\infty}
\end{equation}
in the region $r_k-W_k \leq r\leq r_k+W_k$. Continuity of the solution at $r=r_k\pm W_k$ implies that
\begin{align}
\delta T_{e\,+} &= T_{e\,k}'\,W_k\,\delta T_{e\,0}(1)+T_{e\,k}'\,W_k\,\delta T_{0\,\infty} - T_{e\,k}'\,W_k,\\[0.5ex]
\delta T_{e\,-} &= T_{e\,k}'\,W_k\,\delta T_{e\,0}(-1) +T_{e\,k}'\,W_k\,\delta T_{0\,\infty}+ T_{e\,k}'\,W_k,\\[0.5ex]
{\mit\Psi}_{k\,+} &= - T_{e\,k}'\,W_k\,\delta T_1(1)\left(\frac{r\,g}{q}\,\frac{m_k-n\,q}{T_{e\,0}'\,\psi_{m\,k}}\right)_{r_k+W_k},\\[0.5ex]
{\mit\Psi}_{k\,-} &=- T_{e\,k}'\,W_k\,\delta T_1(-1)\left(\frac{r\,g}{q}\,\frac{m_k-n\,q}{T_{e\,0}'\,\psi_{m\,k}}\right)_{r_k-W_k}.
\end{align}

Finally, for the special case $m=1$, we write
\begin{equation}
\tilde{T}_e(r,\theta,\phi) = - \xi^r(r,\theta,\phi)\,\frac{dT_{e\,0}}{dr}.
\end{equation}

\subsection{Normalized Quantities}
Let $\hat{r}=r/\epsilon_a$, $\hat{r}_k=r_k/\epsilon_a$, $\hat{x}= x/\epsilon_a$, $\hat{T}_{e\,0}' = \epsilon_a\,T_{e\,0}'$, $\hat{T}_{e\,k}'=\epsilon_a\,\hat{T}_{e\,k}'$, 
$\hat{W}_k = W_k/\epsilon_a$, and $\hat{\mit\Psi}_k={\mit\Psi}_k/\epsilon_a^{\,2}$, etc., 
then
\begin{equation}
\tilde{T}_e(\hat{r},\theta,\phi) = T_{e\,0}(\hat{r}) + \delta T_{e\,+} - \hat{\mit\Psi}_{k+}\,\frac{q(\hat{r})}{\hat{r}\,g(\hat{r})}\,\frac{\hat{T}_e'(r)\,\psi_{m_k}(r)}{m_k-n\,q(\hat{r})}\,{\rm e}^{\,{\rm i}\,\zeta_k}
\end{equation}
in the region $\hat{r}>\hat{r}_k+\hat{W}_k$, 
\begin{equation}
\tilde{T}_e(\hat{r},\theta,\phi) = T_{e\,0}(\hat{r}) + \delta T_{e\,-} - \hat{{\mit\Psi}}_{k-}\,\frac{q(\hat{r})}{\hat{r}\,g(\hat{r})}\,\frac{\hat{T}_e'(r)\,\psi_{m_k}(r)}{m_k-n\,q(\hat{r})}\,{\rm e}^{\,{\rm i}\,\zeta_k}
\end{equation}
in the region $\hat{r}< \hat{r}_k-\hat{W}_k$, and 
\begin{equation}
\tilde{T}_e(\hat{r}, \theta,\phi)=T_{e\,k} +\hat{T}_{e\,k}'\,\hat{W}_k\sum_{\nu=0,\infty}\delta T_\nu(\hat{x}/\hat{W}_k)\,{\rm e}^{\,{\rm i}\,\nu\,\zeta_k}+\hat{T}_{e\,k}'\,\hat{W}_k\,\delta T_{0\,\infty}
\end{equation}
in the region $\hat{r}_k-\hat{W}_k \leq \hat{r}\leq \hat{r}_k+\hat{W}_k$. Here, 
\begin{align}
\delta T_{e\,+} &= \hat{T}_{e\,k}'\,\hat{W}_k\,\delta T_0(1)+\hat{T}_{e\,k}'\,\hat{W}_k\,\delta T_{0\,\infty} - \hat{T}_{e\,k}'\,\hat{W}_k,\\[0.5ex]
\delta T_{e\,-} &=\hat{T}_{e\,k}'\,\hat{W}_k\,\delta T_0(-1) +\hat{T}_{e\,k}'\,\hat{W}_k\,\delta T_{0\,\infty} +\hat{T}_{e\,k}'\,\hat{W}_k,\\[0.5ex]
\hat{\mit\Psi}_k&=\left(\frac{\hat{W}_k}{4}\right)^2\left(\frac{g\,s}{q}\right)_{\hat{r}_k},\\[0.5ex]
\hat{\mit\Psi}_{k\,+} &= - \hat{T}_{e\,k}'\,\hat{W}_k\,\delta T_1(1)\left(\frac{\hat{r}\,g}{q}\,\frac{m_k-n\,q}{\hat{T}_{e\,0}'\,\psi_{m\,k}}\right)_{\hat{r}_k+\hat{W}_k},\\[0.5ex]
\hat{\mit\Psi}_{k\,-} &= -\hat{T}_{e\,k}'\,\hat{W}_k\,\delta T_1(-1)\left(\frac{\hat{r}\,g}{q}\,\frac{m_k-n\,q}{\hat{T}_{e\,0}'\,\psi_{m\,k}}\right)_{\hat{r}_k-\hat{W}_k}.
\end{align}
For the special case $m=1$, 
\begin{equation}
\tilde{T}_e(\hat{r},\theta,\phi) = - \frac{\xi^r(\hat{r},\theta,\phi)}{\epsilon_a}\,\frac{ dT_{e\,0}}{d\hat{r}}.
\end{equation}

\section{Relativistic Downshifting and Broadening}
Neglecting doppler broadening, the angular frequency of an $n$th harmonic electron cyclotron emission (ece) signal is
\begin{equation}
\omega = \frac{n\,{\mit\Omega}_0\,R_0}{R}\,\left[1-\left(\frac{v}{c}\right)^2\right]^{1/2},
\end{equation}
where $v$ is the electron speed, and ${\mit\Omega}_0=e\,B_0/R_0$. Here, we are neglecting the poloidal magnetic field-strength, and assuming that the toroidal
field-strength falls off like $1/R$. Let
\begin{equation}
R_\omega(\omega) = \frac{n\,{\mit\Omega}_0\,R_0}{\omega}
\end{equation}
be the major radius from which the ece of frequency $\omega$ is emitted  in the absence of relativistic downshifting and broadening. We can write
\begin{align}
\frac{v}{c} =\left\{\begin{array}{ccc}\left[1-\left(\frac{R}{R_\omega}\right)^2\right]^{1/2} &~~~~~&R\leq R_\omega\\[0.5ex]
0&&R>R_\omega
\end{array}\right..
\end{align}
Now, the distribution of electron speeds is
\begin{equation}
f(v)= A\,v^2\left(-\frac{1}{\theta_\omega}\left[1-\left(\frac{v}{c}\right)^2\right]^{-1/2}\right),
\end{equation}
where
\begin{equation}
\theta_\omega(\omega) = \frac{T_e(R_\omega)}{m_e\,c^{\,2}}.
\end{equation}
Thus, we can define
\begin{equation}
F(R,R_\omega) = \left[1-\left(\frac{R}{R_\omega}\right)^2\right]\exp\!\left(-\frac{1}{\theta_\omega}\,\frac{R_\omega}{R}\right).
\end{equation}
The electron temperature measured by the ece diagnostic is
\begin{equation}
T_e(R_\omega) = \frac{\int_{R_{\rm min}}^{R_\omega} T_e(R)\,F(R,R_\omega)\,dR}
                                     {\int_{R_{\rm min}}^{R_\omega}             F(R,R_\omega)\,dR}.
\end{equation}

\end{document}