\documentclass[12pt,prb,aps,notitlepage]{revtex4-1}
\usepackage {amsmath}
\usepackage{amssymb}
\pdfoutput = 1 
\usepackage {graphicx}
\newcommand{\bomega}{\mbox{\boldmath$\omega$}}
\allowdisplaybreaks

\begin{document}

\title{Pressure Flattening due to Magnetic Island}
\author{R.~Fitzpatrick\,\footnote{rfitzp@utexas.edu}}
\affiliation{Institute for Fusion Studies,  Department of Physics,  University of Texas at Austin,  Austin TX 78712, USA}\begin{abstract}
\end{abstract}
\maketitle

\section{Magnetic Island}
Let $x=r-r_s$, $X=x/W$, and $\zeta= m\,\theta-n\,\phi$, where $W$ is the island width. The magnetic flux-surfaces of the magnetic island are contours of
\begin{equation}
{\mit\Omega}(X,\zeta) = 8\,X^2 + \cos\zeta.
\end{equation}
The X-points lie at $X=0$ and $\zeta = 0$, $2\pi$, whereas the X-point lies at
$X=0$ and $\zeta=\pi$. The O-point corresponds to ${\mit\Omega}=-1$, whereas the magnetic separatrix corresponds to ${\mit\Omega}=1$.  Note that ${\mit\Omega}\simeq
8\,X^2$ in the limit $|X|\gg 1$. 

\section{Temperature Perturbation in Inner Region}
Let $T_{e\,0}(r)$ be the unperturbed electron temperature profile. Let 
\begin{equation}\label{e2}
T_e(X,\zeta) = T_{e\,s} + {\rm sgn}(X)\,W\,T_{e\,s}'\,\tilde{T}({\mit\Omega})
\end{equation}
 be the temperature profile in the presence of the island,
where $T_{e\,s}=T_{e\,0}(r_s)$, and  $T_{e\,s}'=(dT_{e\,0}/dr)_{r=r_s}$. 
The normalized perturbed electron temperature profile, $\tilde{T}({\mit\Omega})$,  satisfies the energy conservation equation
\begin{equation}\label{e3}
\frac{d}{d{\mit\Omega}}\!\left[\oint({\mit\Omega}-\cos\zeta)^{1/2}\,\frac{d\zeta}{2\pi}\,\frac{d\tilde{T}}{d
{\mit\Omega}} \right]= 0,
\end{equation}
subject to the boundary condition that
\begin{equation}\label{e4}
\tilde{T}({\mit\Omega}) \rightarrow |X|
\end{equation}
as $|X|\rightarrow\infty$. Note that $T(X,\zeta)-T_{e\,s}$ is an odd function of $X$. 

Equations~(\ref{e2}) and (\ref{e3})
imply that
\begin{equation}
\tilde{T}({\mit\Omega}) = 0
\end{equation}
for $-1\leq {\mit\Omega}<1$, and
\begin{equation}\label{e6}
\frac{d\tilde{T}}{d{\mit\Omega}} = \frac{c}{\oint({\mit\Omega}-\cos\zeta)^{1/2}\,d\zeta/2\pi}
\end{equation}
for ${\mit\Omega}\geq 1$, where $c$ is a constant. 
Let
\begin{equation}
k = \left(\frac{1+{\mit\Omega}}{2}\right)^{1/2}.
\end{equation}
The island O-point corresponds to $k=0$, whereas the magnetic separatrix corresponds to $k=1$. Note that $k\rightarrow 2\,|X|$ as $|X| \rightarrow\infty$. 

Equation~(\ref{e6}) yields
\begin{equation}
\frac{d\tilde{T}}{dk} = \frac{\sqrt{2}\,\pi\,c}{E(1/k)},
\end{equation}
where
\begin{equation}
E(p) \equiv \int_0^{\pi/2}(1-p^2\,\sin^2\theta)^{1/2}\,d\theta
\end{equation}
is a complete elliptic integral. 
The boundary condition (\ref{e4}) implies that
\begin{equation}
c = \frac{1}{4\sqrt{2}}.
\end{equation}
Hence, we conclude that
\begin{equation}
\frac{d\tilde{T}}{dk} =\frac{\pi}{4}\,\frac{1}{E(1/k)}
\end{equation}
for $k\geq 1$. 
Thus, 
\begin{equation}
\tilde{T}(k)= 0
\end{equation}
for $0\leq k< 1$, and
\begin{equation}
\tilde{T}(k)=F(k)
\end{equation}
for $k\geq 1$,
where
\begin{equation}
F(k) =\frac{\pi}{4}\int_1^k \frac{dk'}{E(1/k')}.
\end{equation}

\section{Harmonics of Temperature Perturbation}
We can write
\begin{equation}
\tilde{T}(|X|,\zeta) = \sum_{\nu=0,\infty}\delta T_\nu(|X|)\,\cos(\nu\,\zeta).
\end{equation}

Now,
\begin{equation}
\delta T_{0}(|X|) = \oint\tilde{T}(|X|,\zeta)\,\frac{d\zeta}{2\pi},
\end{equation}
where the integral is at constant $|X|$. It follows that
\begin{equation}
\delta T_0(|X|) =  \int_0^{\zeta_c}\,F(k)\,\frac{d\zeta}{\pi},
\end{equation}
where 
\begin{equation}
\zeta_c = \cos^{-1}\left(1-8\,X^2\right)
\end{equation}
for $|X|< 1/2$, and $\zeta_c=\pi$ for $|X|\geq 1/2$. Furthermore, 
\begin{equation}
k = \left[4\,|X|^2+\cos^2\left(\frac{\zeta}{2}\right)\right]^{1/2}.
\end{equation}

For $\nu>0$, we have
\begin{equation}
\delta T_\nu(|X|) = 2\oint \tilde{T}(|X|,\zeta)\,\cos(\nu\,\zeta)\,\frac{d\zeta}{2\pi},
\end{equation}
where the integral is at constant $|X|$. Integrating by parts, we obtain
\begin{equation}
\delta T_\nu(|X|) =-\frac{2}{\nu}\oint\frac{\partial\tilde{T}}{\partial \zeta}\,\sin(\nu\,\zeta)\,\frac{d\zeta}{2\pi}.
\end{equation}
But,
\begin{equation}
\frac{\partial\tilde{T}}{\partial \zeta}=\frac{d\tilde{T}}{dk}\,\frac{\partial k}{\partial\zeta}=-\frac{d\tilde{T}}{dk}\,\frac{\sin\zeta}{4\,k} = -\frac{\pi}{16}\,\frac{\sin\zeta}{k\,E(1/k)},
\end{equation}
so
\begin{equation}
\delta T_\nu(X) =\frac{1}{16\,\nu} \int_0^{\zeta_c}\frac{\cos[(\nu-1)\,\zeta]-\cos[(\nu+1)\,\zeta]}{k\,E(1/k)}\,d\zeta.
\end{equation}

\section{Asymptotic Behavior}
In the limit $|X| \ll 1$, we have
\begin{align}
\zeta_c &\simeq 4\,|X|,\\[0.5ex]
k &\simeq 1 + \frac{\zeta_c^{\,2}-\zeta^2}{8},\\[0.5ex]
E(1/k)&\simeq 1,\\[0.5ex]
F(k)&\simeq \frac{\pi}{4}\,(k-1).
\end{align}
It follows that
\begin{align}
\delta T_0(|X|) \simeq \frac{4}{3}\,|X|^3,\\[0.5ex]
\delta T_{\nu>0}(|X|) \simeq \frac{8}{3}\,|X|^3
\end{align}

In the limit $|x|/W\gg 1$, we have
\begin{align}
k&\simeq 2\,|X|,\\[0.5ex]
E(1/k) &\simeq \frac{\pi}{2}.
\end{align}
It follows that 
\begin{align}
F(k) &\simeq \frac{k}{2} - F_\infty,\\[0.5ex]
\delta T_0(|X|)&\simeq |X| - F_\infty,\\[0.5ex]
\delta T_1(|X|) &\simeq \frac{1}{16\,|X|},\\[0.5ex]
\delta T_{\nu>1}(|X|)& \sim {\cal O}\left(\frac{1}{|X|^3}\right),
\end{align}
where
\begin{equation}
F_\infty = 0.3447.
\end{equation}

\section{Asymptotic Matching}
Consider the $k$th rational surface whose radius is $r_k$ and whose resonant poloidal mode number is $m_k$. Let  $x=r-r_k$ and $\zeta_k=m_k\,\theta-n\,\phi$. 

 In the outer region,  we write the total electron temperature as 
\begin{equation}\label{e36}
\tilde{T}_e(r, \theta,\phi)=T_{e\,0}(r) - {\mit\Psi}_k\,\frac{q(r)}{r\,g(r)}\,\frac{T_{e\,0}'(r)\,\psi_{m_k}(r)}{m_k-n\,q(r)}\,{\rm e}^{\,{\rm i}\,\zeta_k},
\end{equation}
where $T_{e\,0}'=dT_{e\,0}/dr$, $T_{e\,0}(r)$ is the equilibrium electron temperature profile, ${\mit\Psi}_k$ is the reconnected flux, and 
\begin{equation}
W_k = 4\left(\frac{q}{g\,s}\right)^{1/2}_{r_k}\,{\mit\Psi}_k^{1/2}
\end{equation}
is the island width.
 In the limit, $|x|\ll 1$, Eq.~(\ref{e36}) yield 
\begin{equation}
\tilde{T}_e(x, \theta,\phi)=T_{e\,k} + T_{e\,k}'\,x + \frac{T_{e\,k}'\,W_k^{\,2}}{16\,x}\,{\rm e}^{\,{\rm i}\,\zeta_k},
\end{equation}
Here, $T_{e\,k}=T_{e\,0}(r_k)$ and $T_{e\,k}'= (dT_{e\,0}/dr)_{r_k}$, and we have made use of the fact that $\psi_{m_k}(r_k)\simeq m_k$. 

In the inner region, we write the total electron temperature as 
\begin{equation}
\tilde{T}_e(x, \theta,\phi)=T_{e\,k} +{\rm sgn}(x)\,T_{e\,k}'\,W_k\sum_{\nu=0,\infty}\delta T_\nu(|x|/W_k)\,{\rm e}^{\,{\rm i}\,\nu\,\zeta_k}+T_{e\,k}'\,W_k\,F_\infty,
\end{equation}
In the limit $x\gg W_k$, the previous equation yields 
\begin{equation}
\tilde{T}_e(x,\theta,\phi) \simeq T_{e\,k} + T_{e\,k}'\,x + \frac{T_{e\,k}'\,W_k^{\,2}}{16\,x}\,{\rm e}^{\,{\rm i}\,\zeta_k},
\end{equation}
On the other hand, in the limit $x\ll -W_k$, we get 
\begin{equation}
\tilde{T}_e(x,\theta,\phi) \simeq T_{e\,k} + T_{e\,k}'\,x  + 2\,T_{e\,k}'\,W_k\,F_\infty + \frac{T_{e\,k}'\,W_k^{\,2}}{16\,x}\,{\rm e}^{\,{\rm i}\,\zeta_k}.
\end{equation}

The asymptotic matching process consists of writing
\begin{equation}
\tilde{T}_e(r,\theta,\phi) = T_{e\,0}(r) + \delta T_{e\,+} - {\mit\Psi}_{k+}\,\frac{q(r)}{r\,g(r)}\,\frac{T_{e\,0}'(r)\,\psi_{m_k}(r)}{m_k-n\,q(r)}\,{\rm e}^{\,{\rm i}\,\zeta_k}
\end{equation}
in the region $r>r_k+W_k$, 
\begin{equation}
\tilde{T}_e(r,\theta,\phi) = T_{e\,0}(r) + \delta T_{e\,-} - {\mit\Psi}_{k-}\,\frac{q(r)}{r\,g(r)}\,\frac{T_{e\,0}'(r)\,\psi_{m_k}(r)}{m_k-n\,q(r)}\,{\rm e}^{\,{\rm i}\,\zeta_k}
\end{equation}
in the region $r< r_k-W_k$, and 
\begin{equation}
\tilde{T}_e(r, \theta,\phi)=T_{e\,k} +{\rm sgn}(x)\,T_{e\,k}'\,W_k\sum_{\nu=0,\infty}\delta T_\nu(|x|/W_k)\,{\rm e}^{\,{\rm i}\,\nu\,\zeta_k}+T_{e\,k}'\,W_k\,F_\infty
\end{equation}
in the region $r_k-W_k \leq r\leq r_k+W_k$. Continuity of the solution at $r=r_k\pm W_k$ implies that
\begin{align}
\delta T_{e\,+} &= T_{e\,k}'\,W_k\,\delta T_{e\,0}(1)+T_{e\,k}'\,W_k\,F_\infty - T_{e\,k}'\,W_k,\\[0.5ex]
\delta T_{e\,-} &= - T_{e\,k}'\,W_k\,\delta T_{e\,0}(1) +T_{e\,k}'\,W_k\,F_\infty+ T_{e\,k}'\,W_k,\\[0.5ex]
{\mit\Psi}_{k\,+} &= - T_{e\,k}'\,W_k\,\delta T_1(1)\left(\frac{r\,g}{q}\,\frac{m_k-n\,q}{T_{e\,0}'\,\psi_{m\,k}}\right)_{r_k+W_k},\\[0.5ex]
{\mit\Psi}_{k\,-} &= T_{e\,k}'\,W_k\,\delta T_1(1)\left(\frac{r\,g}{q}\,\frac{m_k-n\,q}{T_{e\,0}'\,\psi_{m\,k}}\right)_{r_k-W_k}.
\end{align}

Finally, for the special case $m=1$, we write
\begin{equation}
\tilde{T}_e(r,\theta,\phi) = - \xi^r(r,\theta,\phi)\,\frac{dT_{e\,0}}{dr}.
\end{equation}

\subsection{Normalized Quantities}
Let $\hat{r}=r/\epsilon_a$, $\hat{r}_k=r_k/\epsilon_a$, $\hat{x}= x/\epsilon_a$, $\hat{T}_{e\,0}' = \epsilon_a\,T_{e\,0}'$, $\hat{T}_{e\,k}'=\epsilon_a\,\hat{T}_{e\,k}'$, 
$\hat{W}_k = W_k/\epsilon_a$, and $\hat{\mit\Psi}_k={\mit\Psi}_k/\epsilon_a^{\,2}$, etc., 
then
\begin{equation}
\tilde{T}_e(\hat{r},\theta,\phi) = T_{e\,0}(\hat{r}) + \delta T_{e\,+} - \hat{\mit\Psi}_{k+}\,\frac{q(\hat{r})}{\hat{r}\,g(\hat{r})}\,\frac{\hat{T}_e'(r)\,\psi_{m_k}(r)}{m_k-n\,q(\hat{r})}\,{\rm e}^{\,{\rm i}\,\zeta_k}
\end{equation}
in the region $\hat{r}>\hat{r}_k+\hat{W}_k$, 
\begin{equation}
\tilde{T}_e(\hat{r},\theta,\phi) = T_{e\,0}(\hat{r}) + \delta T_{e\,-} - \hat{{\mit\Psi}}_{k-}\,\frac{q(\hat{r})}{\hat{r}\,g(\hat{r})}\,\frac{\hat{T}_e'(r)\,\psi_{m_k}(r)}{m_k-n\,q(\hat{r})}\,{\rm e}^{\,{\rm i}\,\zeta_k}
\end{equation}
in the region $\hat{r}< \hat{r}_k-\hat{W}_k$, and 
\begin{equation}
\tilde{T}_e(\hat{r}, \theta,\phi)=T_{e\,k} +{\rm sgn}(\hat{x})\,\hat{T}_{e\,k}'\,\hat{W}_k\sum_{\nu=0,\infty}\delta T_\nu(|\hat{x}|/\hat{W}_k)\,{\rm e}^{\,{\rm i}\,\nu\,\zeta_k}+\hat{T}_{e\,k}'\,\hat{W}_k\,F_\infty
\end{equation}
in the region $\hat{r}_k-\hat{W}_k \leq \hat{r}\leq \hat{r}_k+\hat{W}_k$. Here, 
\begin{align}
\delta T_{e\,+} &= \hat{T}_{e\,k}'\,\hat{W}_k\,\delta T_0(1)+\hat{T}_{e\,k}'\,\hat{W}_k\,F_\infty - \hat{T}_{e\,k}'\,\hat{W}_k,\\[0.5ex]
\delta T_{e\,-} &=- \hat{T}_{e\,k}'\,\hat{W}_k\,\delta T_0(1) +\hat{T}_{e\,k}'\,\hat{W}_k\,F_\infty +\hat{T}_{e\,k}'\,\hat{W}_k,\\[0.5ex]
\hat{\mit\Psi}_k&=\left(\frac{\hat{W}_k}{4}\right)^2\left(\frac{g\,s}{q}\right)_{\hat{r}_k},\\[0.5ex]
\hat{\mit\Psi}_{k\,+} &= - \hat{T}_{e\,k}'\,\hat{W}_k\,\delta T_1(1)\left(\frac{\hat{r}\,g}{q}\,\frac{m_k-n\,q}{\hat{T}_{e\,0}'\,\psi_{m\,k}}\right)_{\hat{r}_k+\hat{W}_k},\\[0.5ex]
\hat{\mit\Psi}_{k\,-} &= \hat{T}_{e\,k}'\,\hat{W}_k\,\delta T_1(1)\left(\frac{\hat{r}\,g}{q}\,\frac{m_k-n\,q}{\hat{T}_{e\,0}'\,\psi_{m\,k}}\right)_{\hat{r}_k-\hat{W}_k}.
\end{align}
For the special case $m=1$, 
\begin{equation}
\tilde{T}_e(\hat{r},\theta,\phi) = - \frac{\xi^r(\hat{r},\theta,\phi)}{\epsilon_a}\,\frac{ dT_{e\,0}}{d\hat{r}}.
\end{equation}


\end{document}
