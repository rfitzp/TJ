\documentclass[12pt,prb,aps,notitlepage]{revtex4-1}
\usepackage {amsmath}
\usepackage{amssymb}
\pdfoutput = 1 
\usepackage {graphicx}
\newcommand{\bomega}{\mbox{\boldmath$\omega$}}
\allowdisplaybreaks

\begin{document}

\title{Resistive Wall}
\author{R.~Fitzpatrick\,\footnote{rfitzp@utexas.edu}}
\affiliation{Institute for Fusion Studies,  Department of Physics,  University of Texas at Austin,  Austin TX 78712, USA}\begin{abstract}
\end{abstract}
\maketitle

\section{Vacuum Solution}

\subsection{Normalization}
Let all lengths be normalized to the major radius of the  axisymmetric plasma equilibrium's magnetic axis, $R_0$. Let all
magnetic field-strengths be normalized to the toroidal magnetic field-strength at the magnetic axis, $B_0$.

\subsection{Toroidal Coordinates}
Let $\mu$, $\eta$, $\phi$ be right-handed toroidal coordinates defined such that
\begin{align}
R &= \frac{\sinh\mu}{\cosh\mu-\cos\eta},\\[0.5ex]
Z&=\frac{\sin\eta}{\cosh\mu-\cos\eta},
\end{align}
where $R$, $\phi$, $Z$ are right-handed cylindrical coordinates whose symmetry axis corresponds to that of the plasma equilibrium. 
Note that $(\nabla R\times \nabla\phi\cdot\nabla Z)^{-1}=R$. 
The scale-factors of the toroidal coordinate system are
\begin{align}
h_\mu&=h_\eta= \frac{1}{\cosh\mu-\cos\eta}\equiv h,\\[0.5ex]
h_\phi &= \frac{\sinh\mu}{\cosh\mu-\cos\eta} = h\,\sinh\mu.
\end{align}
Moreover, 
\begin{equation}
{\cal J}' \equiv (\nabla\mu\times\nabla\eta\cdot\nabla\phi)= h^3\,\sinh\mu.
\end{equation}

\subsection{Perturbed Magnetic Field}
The curl-free perturbed magnetic field in the vacuum region is written
\begin{equation}
{\bf b} = {\rm i}\,\nabla \left[V(\mu,\eta)\,{\rm e}^{-{\rm i}\,n\,\phi}\right],
\end{equation}
where where the toroidal mode number, $n$, is a positive integer. Given that $\nabla\cdot{\bf b}=0$, we deduce that 
\begin{align}\label{e6}
\nabla^2 V &\equiv (z-\cos\eta)^3\left\{\frac{\partial}{\partial z}\!\left[\frac{z^2-1}{z-\cos\eta}\,\frac{\partial V}{\partial z}\right]\right.\nonumber\\[0.5ex]
&\left.\phantom{=}+\frac{\partial}{\partial \eta}\!\left[\frac{1}{z-\cos\eta}\,\frac{\partial V}{\partial\eta}\right]
-\frac{n^2\,V}{(z^2-1)\,(z-\cos\eta)}\right\}=0.
\end{align}
Here, $z=\cosh\mu$. 

Let
\begin{align}
f_z &= z^2-1,\\[0.5ex]
f_\eta &= (z-\cos\eta)^{1/2},
\end{align}
which implies that
\begin{align}
\frac{df_z}{dz} &= 2\,z,\\[0.5ex]
\frac{\partial f_\eta}{\partial z}&= \frac{1}{2\,f_\eta},\\[0.5ex]
\frac{\partial f_\eta}{\partial\eta} &= \frac{\sin\eta}{2\,f_\eta}
\end{align}
Suppose that
\begin{equation}
V(z,\eta)= \sum_m (z-\cos\eta)^{1/2}\,U_m(z)\,{\rm e}^{-{\rm i}\,m\,\eta}.
\end{equation}
Taking the sum and eikonal as read, and letting $'=d/dz$, we get 
\begin{align}
\frac{\partial V}{\partial z} &=\frac{U_m}{2\,f_\eta} + f_\eta\,U_m',\\[0.5ex]
\frac{\partial}{\partial z}\!\left(\frac{f_z}{f_\eta^{\,2}}\,\frac{\partial V}{\partial z}\right)&=
\frac{\partial}{\partial z}\!\left(\frac{f_z\,U_m}{2\,f_\eta^{\,3}}+ \frac{f_z\,U_m'}{f_\eta}\right)
= \frac{z\,U_m}{f_\eta^{\,3}} -\frac{3\,f_z\,U_m}{4\,f_\eta^{\,5}}+\frac{f_z\,U_m'}{2\,f_\eta^{\,3}}
+\frac{2\,z\,U_m'}{f_\eta}- \frac{f_z\,U_m'}{2\,f_\eta^{\,3}}
+\frac{f_z\,U_m''}{f_\eta}\nonumber\\[0.5ex]
&=\frac{z\,U_m}{f_\eta^{\,3}} -\frac{3\,(z^2-1)\,U_m}{4\,f_\eta^{\,5}}
+\frac{2\,z\,U_m'}{f_\eta}
+\frac{(z^2-1)\,U_m''}{f_\eta},\\[0.5ex]
\frac{\partial V}{\partial \eta} &=\frac{\sin\eta\,U_m}{2\,f_\eta} -{\rm i}\,m f_\eta\,U_m,\\[0.5ex]
\frac{\partial}{\partial \eta}\!\left(\frac{1}{f_\eta^{\,2}}\,\frac{\partial V}{\partial \eta}\right)&=
\frac{\partial}{\partial \eta}\!\left(\frac{\sin\eta\,U_m}{2\,f_\eta^{\,3}}-\frac{{\rm i}\,m\,U_m}{f_\eta}\right)=
\frac{\cos\eta\,U_m}{2\,f_\eta^{\,3}}- \frac{3\,\sin^2\eta\,U_m}{4\,f_\eta^{\,5}}-\frac{{\rm i}\,m\,\sin\eta\,U_m}{2\,f_\eta^{\,3}}\nonumber\\[0.5ex]
&\phantom{=} + \frac{{\rm i}\,m\,\sin\eta\,U_m}{2\,f_\eta^{\,3}} - \frac{m^2\,U_m}{f_\eta}\nonumber\\[0.5ex]
&=\frac{\cos\eta\,U_m}{2\,f_\eta^{\,3}}- \frac{3\,\sin^2\eta\,U_m}{4\,f_\eta^{\,5}} - \frac{m^2\,U_m}{f_\eta},\\[0.5ex]
-\frac{n^2\,V}{f_z\,f_\eta^{\,2}}&= -\frac{n^2\,U_m}{(z^2-1)\,f_\eta}.
\end{align}
Thus, Eq.~(\ref{e6}) becomes 
\begin{align}
0&=\frac{\partial}{\partial z}\!\left(\frac{f_z}{f_\eta^{\,2}}\,\frac{\partial V}{\partial z}\right)+ \frac{\partial}{\partial \eta}\!\left(\frac{1}{f_\eta^{\,2}}\,\frac{\partial V}{\partial \eta}\right)
-\frac{n^2\,V}{f_z\,f_\eta^{\,2}}\nonumber\\[0.5ex]&
= \frac{z\,U_m}{f_\eta^{\,3}} -\frac{3\,(z^2-1)\,U_m}{4\,f_\eta^{\,5}}
+\frac{2\,z\,U_m'}{f_\eta}
+\frac{(z^2-1)\,U_m''}{f_\eta}\nonumber\\[0.5ex]
&\phantom{=}+\frac{\cos\eta\,U_m}{2\,f_\eta^{\,3}}- \frac{3\,\sin^2\eta\,U_m}{4\,f_\eta^{\,5}} - \frac{m^2\,U_m}{f_\eta} -\frac{n^2\,U_m}{(z^2-1)\,f_\eta}\nonumber\\[0.5ex]
&=\frac{1}{f_\eta}\left[(z^2-1)\,U_m'' + 2\,z\,U_m' + \left(\frac{1}{4}-m^2\right)U_m - \frac{n^2\,U_m}{z^2-1}\right].
\end{align}
The most general solution of the previous equation is
\begin{equation}
U_m(z) = p_m\,\hat{P}_{|m|-1/2}^{\,n}(z)+q_m\,\hat{Q}_{m-1/2}^{\,n}(z),
\end{equation}
where
\begin{align}
\hat{P}_{|m|-1/2}^{\,n}(z) &= \cos(|m|\,\pi)\,\frac{\sqrt{\pi}\,\Gamma(|m|+1/2-n)\,\epsilon^{\,|m|}}{2^{\,|m|-1/2}\,|m|!}\,P_{|m|-1/2}^{\,n}(z),\\[0.5ex]
\hat{Q}_{|m|-1/2}^{\,n}(z)&= \cos(n\,\pi)\,\cos(|m|\,\pi)\,\frac{2^{\,|m|-1/2}\,|m|!\,\epsilon^{-|m|}}{\sqrt{\pi}\,\Gamma(|m|+1/2+n)}\,Q_{|m|-1/2}^{\,n}.
\end{align}
Here,  $\epsilon$ is the inverse-aspect ratio of the plasma equilibrium, and  $p_m$ and $q_m$ are arbitrary complex coefficients. 
Moreover, we have made use of the fact that 
\begin{align}
P_{-m-1/2}^{\,n}(z) &= P_{m-1/2}^{\,n}(z),\\[0.5ex]
Q_{-m-1/2}^{\,n}(z) &= Q_{m-1/2}^{\,n}(z).
\end{align}

\subsection{Toroidal Electromagnetic Angular Momentum Flux}
The outward flux of toroidal angular momentum across a constant-$z$ surface is
\begin{align}
T_\phi(z) = -\oint\oint {\cal J}' \,b_\phi\,b^{\,\mu}\,d\eta\,d\phi.
\end{align}
Now,
\begin{align}
b^{\,\mu} &\equiv {\bf b}\cdot\nabla\mu = {\rm i}\,\frac{\partial V}{\partial \mu}\,|\nabla\mu|^2 = {\rm i}\,\frac{\sinh\mu}{h^2}\,\frac{\partial V}{\partial z},\\[0.5ex]
b^{\,\phi}&\equiv {\cal J}'\,\nabla\mu\times\nabla \eta\cdot\nabla V = n\,V,
\end{align}
so
\begin{align}
T_\phi(z) &= -\frac{{\rm i}\,n\,\pi}{2}\oint \frac{z^2-1}{z-\cos\eta}\left(\frac{\partial V}{\partial z}\,V^\ast -\frac{\partial V^{\ast}}{\partial z}\,V\right)d\eta\nonumber\\[0.5ex]
&= - {\rm i}\,n\,\pi^2\sum_{m}\, (z^2-1)\left(\frac{dU_m}{dz}\,U_{m}^\ast -\frac{dU_m^\ast}{dz}\,U_{m}\right)\nonumber\\[0.5ex]
&= -{\rm i}\,n\,\pi^2\sum_{m}(p_m\,q_m^\ast-q_m\,p_m^{\ast})\,(z^2-1)\left(\frac{d\hat{P}|_{|m|-1/2}^{\,n}}{dz}\,\hat{Q}_{|m|-1/2}^{\,n}- \frac{d\hat{Q}_{|m|-1/2}^{\,n}}{dz}\,\hat{P}_{|m|-1/2}^{\,n}\right)\nonumber\\[0.5ex]
& ={\rm i}\,n\,\pi^2\sum_{m}(p_m\,q_m^\ast-q_m\,p_m^{\ast})\,(z^2-1)\,{\cal W}(\hat{P}_{|m|-1/2}^{\,n},\hat{Q}_{|m|-1/2}^{\,n}).
\end{align}
But,
\begin{align}
{\cal W}(\hat{P}_{|m|-1/2}^{\,n},\hat{Q}_{|m|-1/2}^{\,n})&= 
\cos(n\,\pi) \,\frac{\Gamma(|m|+1/2-n)}{\Gamma(|m|+1/2+n)}\,{\cal W}(P_{|m|-1/2}^{\,n},Q_{|m|-1/2}^{\,n})\nonumber\\[0.5ex]
&= \cos(n\,\pi) \,\frac{\Gamma(|m|+1/2-n)}{\Gamma(|m|+1/2+n)}
\,\frac{\cos(n\,\pi)}{1-z^2}\,\frac{\Gamma(|m|+1/2+n)}{\Gamma(|m|+1/2-n)}\nonumber\\[0.5ex]
&= \frac{1}{1-z^2}, 
\end{align}
so
\begin{equation}
T_\phi(z) = 2\pi^2\,n\sum_m {\rm Im}(q_m^\ast\,p_m).
\end{equation}

\section{Resistive Wall Physics}
\subsection{Resistive Wall}
Let the inner surface of the resistive wall surrounding the plasma lie at $\mu=\mu_w$,
and let the outer surface lie at $\mu=\mu_w-\bar{d}_w\,\sinh\mu_w$,  where $\bar{d}_w\ll 1$ is a positive constant. The physical wall thickness is
\begin{equation}
d(\eta)= \frac{\bar{d}_w\,\sinh\mu_w}{|\nabla \mu|} = h_w(\eta)\,\sinh \mu_w\,\bar{d}_w,
\end{equation}
where
\begin{equation}
h_w(\eta) = \frac{1}{z_w-\cos\eta},
\end{equation}
and $z_w=\cosh\mu_w$. 
Let the electrical conductivity of the wall material vary as
\begin{equation}
\sigma(\eta) = \frac{\bar{\sigma}_w}{h_w^{\,2}(\eta)\,\sinh^2 \mu_w},
\end{equation}
where $\bar{\sigma}_w$ is a positive constant. It follows that $\sigma\,d^{\,2}=\bar{\sigma}_w\,\bar{d}_w^{\,2}$. 

\subsection{Wall Matching Conditions}
If we write
\begin{equation}
{\bf b} = \nabla\times{\bf A}
\end{equation}
in the vacuum region then the boundary conditions at the wall are 
\begin{align}
\left.{\bf n}_w\times {\bf A}\right|_{z_{w-}} &= \frac{1}{\cosh\lambda}\left.{\bf n}_w\times {\bf A}\right|_{z_{w+}} \\[0.5ex]
\left.{\bf n}_w\times (\nabla\times{\bf A})\right|_{z_{w+}}&= -\frac{\lambda\,\tanh\lambda}{\bar{d}_w\,h_w\,\sinh\mu_w}\,\left.{\bf n}_w\times({\bf n}_w\times{\bf A})\right|_{z_{w+}}+
\frac{\left.{\bf n}_w\times (\nabla\times{\bf A})\right|_{z_{w-}}}{\cosh\lambda},\\[0.5ex]
\lambda&= \sqrt{\hat{\gamma}\,\bar{d}_w},\\[0.5ex]
\hat\gamma&= \gamma\,\bar{\tau}_w,\\[0.5ex]
\bar{\tau}_w &=\mu_0\,R_0^{\,2}\,\bar{\sigma}_w\,\bar{d}_w,
\end{align}
where $\gamma$ is the growth-rate of the magnetic perturbation, and $\bar{\tau}_w$ is the effective L/R time of the wall. Here, ${\bf n}_w= -{\bf e}_\mu$ is an outward unit normal vector to the wall. Now,
\begin{align}
\nabla\times{\bf A} &= \frac{1}{h^2\,\sinh\mu}\left(\frac{\partial \hat{A}_\phi}{\partial \eta}-\frac{\partial\hat{A}_\eta}{\partial\phi}\right){\bf e}_\mu
+  \frac{1}{h^2\,\sinh\mu}\left(\frac{\partial \hat{A}_\mu}{\partial \phi}-\frac{\partial\hat{A}_\phi}{\partial\mu}\right){\bf e}_\eta\nonumber\\[0.5ex]
&\phantom{=}
+\frac{1}{h^2}\left(\frac{\partial \hat{A}_\eta}{\partial \mu}-\frac{\partial\hat{A}_\mu}{\partial\eta}\right){\bf e}_\phi,
\end{align}
where
\begin{align}
\hat{A}_\mu &= h\,A_\mu,\\[0.5ex]
\hat{A}_\eta&= h\,A_\eta,\\[0.5ex]
\hat{A}_\phi &= h\,\sinh\mu\,A_\phi.
\end{align}
Furthermore,
\begin{align}
{\bf n}_w\times {\bf A} &= -{\bf e}_\mu\times {\bf A} = A_\phi\,{\bf e}_\eta - A_\eta\,{\bf e}_\phi,\\[0.5ex]
{\bf n}_w\times ({\bf n}_w\times {\bf A})&= -{\bf e}_\mu\times ({\bf n}_w\times {\bf A} )= -A_\eta\,{\bf e}_\eta-A_\phi\,{\bf e}_\phi,\\[0.5ex]
{\bf n}_w\times (\nabla\times {\bf A}) &= -{\bf e}_\mu\times(\nabla\times {\bf A}) = \frac{1}{h^2}\left(\frac{\partial \hat{A}_\eta}{\partial \mu}-\frac{\partial\hat{A}_\mu}{\partial\eta}\right)\,{\bf e}_\eta- \frac{1}{h^2\,\sinh\mu}\left(\frac{\partial \hat{A}_\mu}{\partial \phi}-\frac{\partial\hat{A}_\phi}{\partial\mu}\right){\bf e}_\phi.
\end{align}
Thus, the wall matching conditions become
\begin{align}
\left.\hat{A}_\eta\right|_{z_{w-}}&= \frac{1}{\cosh\lambda}\left.\hat{A}_\eta\right|_{z_{w+}},\\[0.5ex]
\left.\hat{A}_\phi\right|_{z_{w+}}&= \frac{1}{\cosh\lambda}\,\left.\hat{A}_\phi\right|_{z_{w-}},\\[0.5ex]
\left(\frac{\partial \hat{A}_\eta}{\partial \mu}-\frac{\partial\hat{A}_\mu}{\partial\eta}\right)_{z_{w+}}
&=\frac{\lambda\,\tanh\lambda}{\bar{d}_w\,\sinh\mu_w}\left.\hat{A}_\eta\right|_{z_{w+}} + \frac{1}{\cosh\lambda}\left(\frac{\partial \hat{A}_\eta}{\partial \mu}-\frac{\partial\hat{A}_\mu}{\partial\eta}\right)_{z_{w-}},\\[0.5ex]
\left(\frac{\partial \hat{A}_\mu}{\partial \phi}-\frac{\partial\hat{A}_\phi}{\partial\mu}\right)_{z_{w+}}&= -\frac{\lambda\,\tanh\lambda}{\bar{d}_w\,\sinh\mu_w}\left.\hat{A}_\phi\right|_{\mu_{z+}}+\frac{1}{\cosh\lambda} \left(\frac{\partial \hat{A}_\mu}{\partial \phi}-\frac{\partial\hat{A}_\phi}{\partial\mu}\right)_{z_{w-}}.
\end{align}

Let
\begin{equation}
C(z,\eta,\phi) = \frac{\partial \hat{A}_\eta}{\partial \phi}- \frac{\partial \hat{A}_\phi}{\partial \eta}.
\end{equation}
The wall matching conditions reduce to 
\begin{align}
C(z_{w-},\eta,\phi) &= \frac{1}{\cosh\lambda}\,C(z_{w+},\eta,\phi),\\[0.5ex]
\frac{\partial C(z_{w+},\eta,\phi) }{\partial z}& = \frac{\lambda\,\tanh\lambda}{\bar{d}_w\,\sinh^2\mu_w}\,C(z_{w+},\eta,\phi) +\frac{1}{\cosh\lambda}\,\frac{\partial C(z_{w-},\eta,\phi) }{\partial z}.
\end{align}
However, if
\begin{equation}
{\bf b} = {\rm i}\,\nabla V =\nabla\times {\bf A}
\end{equation}
then
\begin{equation}
C = -{\rm i}\,h\,\sinh\mu\,\frac{\partial V}{\partial \mu} = -{\rm i}\,h\,(z^2-1)\,\frac{\partial V}{\partial z}.
\end{equation}
Thus,
\begin{align}
C &= -{\rm i}\,\frac{z^2-1}{z-\cos\eta}\sum_m\left[\frac{U_m}{2\,(z-\cos\eta)^{1/2}} + (z-\cos\eta)^{1/2}\,\frac{dU_m}{dz}\right]{\rm e}^{-{\rm i}\,(m\,\eta+n\,\phi)},\\[0.5ex]
\frac{\partial C}{\partial z} &= -{\rm i}\sum_m\left[\frac{(3/4)\,\sin^2\eta}{(z-\cos\eta)^{5/2}}-\frac{(1/2)\,\cos\eta}{(z-\cos\eta)^{3/2}}+ \frac{m^2+n^2/(z^2-1)}{(z-\cos\eta)^{1/2}}\right]
U_m\,{\rm e}^{-{\rm i}\,(m\,\eta+n\,\phi)}.
\end{align}
It follows that
\begin{align}
\sum_m\left[\frac{U_m}{2}+(z-\cos\eta)\,\frac{dU_m}{dz}\right]_{z_{w-}}{\rm e}^{-{\rm i}\,m\,\eta}= \frac{1}{\cosh\lambda}\sum_m\left[\frac{U_m}{2}+(z-\cos\eta)\,\frac{dU_m}{dz}\right]_{z_{w+}}{\rm e}^{-{\rm i}\,m\,\eta},\\[0.5ex]
\left.\sum_m\left[\frac{3}{4}\,\sin^2\eta-\frac{1}{2}\,(z-\cos\eta)\,\cos\eta+ (z-\cos\eta)^2\left(m^2+\frac{n^2}{z^2-1}\right)\right]
U_m\,{\rm e}^{-{\rm i}\,m\,\eta}\right|_{z_{w+}}=\nonumber\\[0.5ex]
f_w \sum_m(z-\cos\eta)\left[\frac{U_m}{2}+(z-\cos\eta)\,\frac{dU_m}{dz}\right]_{z_{w+}}{\rm e}^{-{\rm i}\,m\,\eta}\nonumber\\[0.5ex]
+\frac{1}{\cosh\lambda}\left.\sum_m\left[\frac{3}{4}\,\sin^2\eta-\frac{1}{2}\,(z-\cos\eta)\,\cos\eta+ (z-\cos\eta)^2\left(m^2+\frac{n^2}{z^2-1}\right)\right]
U_m\,{\rm e}^{-{\rm i}\,m\,\eta}\right|_{z_{w-}},
\end{align}
where
\begin{equation}
f_w= \frac{\lambda\,\tanh\lambda}{\bar{d}_w}.
\end{equation}
Thus, we can write
\begin{align}
\sum_{m'}I_{mm'}\,U_{m'}(z_{w-}) &= \frac{1}{\cosh\lambda}\sum_{m'}I_{mm'}\,U_{m'}(z_{w+}),\\[0.5ex]
\sum_{m'}J_{mm'}\,U_{m'}(z_{w+})&= f_w\sum_{m',m''}k_{mm''}\,I_{m''m'}\,U_{m'}(z_{w+}) + \frac{1}{\cosh\lambda}\sum_{m'}J_{mm'}\,U_{m'}(z_{w-}),
\end{align}
where
\begin{align}
I_{mm'}&= \left(\frac{1}{2}+z\,\frac{d}{dz}\right)\delta_{mm'}-\frac{1}{2}\,\frac{d}{dz}\,(\delta_{m\,m'+1}+\delta_{m\,m'-1}),\\[0.5ex]
J_{mm'}&= \left[\frac{5}{8} + \left(\frac{1}{2}+z^2\right)\left(m^2+\frac{n^2}{z^2-1}\right)\right]\delta_{mm'}
-z\left[\frac{1}{4}+ \left(m^2+\frac{n^2}{z^2-1}\right)\right](\delta_{m\,m'+1}+\delta_{m\,m'-1})\nonumber\\[0.5ex]
&\phantom{=}+\left[-\frac{1}{16}+\frac{1}{4} \left(m^2+\frac{n^2}{z^2-1}\right)\right](\delta_{m\,m'+2}+\delta_{m\,m'-2}),\\[0.5ex]
k_{mm'} &= z\,\delta_{mm'} - \frac{1}{2}\,(\delta_{m\,m'+1}+\delta_{m\,m'-1}).
\end{align}

\subsection{Vacuum Solution}
Now, 
\begin{equation}
U_m(z) = p_{m-}\,\hat{P}_{|m|-1/2}^{\,n}(z)
\end{equation}
in the region $z<z_w$, whereas 
\begin{equation}
U_m(z) =p_{m+}\,\hat{P}_{|m|-1/2}^{\,n}(z) + q_{m+}\,\hat{Q}_{|m|-1/2}^{\,n}(z) 
\end{equation}
in the region $z>z_w$. 
Let $\underline{\underline{I}}_{\,p}$ be the matrix of the
\begin{equation}
\left\{\left[\left(\frac{1}{2}+z\,\frac{d}{dz}\right)\delta_{mm'}-\frac{1}{2}\,\frac{d}{dz}\,(\delta_{m\,m'+1}+\delta_{m\,m'-1})\right]\hat{P}_{|m'|-1/2}^{\,n}(z)\right\}_{z_w}
\end{equation}
values.
Let $\underline{\underline{I}}_{\,q}$ be the matrix of the
\begin{equation}
\left\{\left[\left(\frac{1}{2}+z\,\frac{d}{dz}\right)\delta_{mm'}-\frac{1}{2}\,\frac{d}{dz}\,(\delta_{m\,m'+1}+\delta_{m\,m'-1})\right]\hat{Q}_{|m'|-1/2}^{\,n}(z)\right\}_{z_w}
\end{equation}
values. 
Let $\underline{\underline{J}}_{\,p}$ be the matrix of the 
\begin{align}
 &\left\{\left[\frac{5}{8} + \left(\frac{1}{2}+z^2\right)\left(m^2+\frac{n^2}{z^2-1}\right)\right]\delta_{mm'}
-z\left[\frac{1}{4}+ \left(m^2+\frac{n^2}{z^2-1}\right)\right](\delta_{m\,m'+1}+\delta_{m\,m'-1})\right.\nonumber\\[0.5ex]
&\phantom{=}\left.+\left[-\frac{1}{16}+\frac{1}{4} \left(m^2+\frac{n^2}{z^2-1}\right)\right](\delta_{m\,m'+2}+\delta_{m\,m'-2})\right\}\hat{P}_{|m'|-1/2}^{\,n}(z_w)
\end{align}
values. Let $\underline{\underline{J}}_{\,q}$ be the matrix of the 
\begin{align}
 &\left\{\left[\frac{5}{8} + \left(\frac{1}{2}+z^2\right)\left(m^2+\frac{n^2}{z^2-1}\right)\right]\delta_{mm'}
-z\left[\frac{1}{4}+ \left(m^2+\frac{n^2}{z^2-1}\right)\right](\delta_{m\,m'+1}+\delta_{m\,m'-1})\right.\nonumber\\[0.5ex]
&\phantom{=}\left.+\left[-\frac{1}{16}+\frac{1}{4} \left(m^2+\frac{n^2}{z^2-1}\right)\right](\delta_{m\,m'+2}+\delta_{m\,m'-2})\right\}\hat{Q}_{|m'|-1/2}^{\,n}(z_w)
\end{align}
values. 
Let $\underline{\underline{k}}$ be the matrix of the $k_{mm'}$ values. 
 Finally, let $\underline{p}_{\,+}$ be the vector of the $p_{m+}$ values, et cetera. Thus, we
obtain
\begin{align}
\underline{\underline{I}}_{\,p}\,\underline{p}_{\,-} &= \frac{1}{\cosh\lambda}\left(\underline{\underline{I}}_{\,p}\,\underline{p}_{\,+} + \underline{\underline{I}}_{\,q}\,\underline{q}_{\,+}\right),\\[0.5ex]
\underline{\underline{J}}_{\,p}\,\underline{p}_{\,+}+ \underline{\underline{J}}_{\,q}\,\underline{q}_{\,+} &=
f_w\,\underline{\underline{k}}\left(\underline{\underline{I}}_{\,p}\,\underline{p}_{\,+}+ \underline{\underline{I}}_{\,q}\,\underline{q}_{\,+} \right)
+\frac{1}{\cosh\lambda} \,\underline{\underline{J}}_{\,p}\,\underline{p}_{\,-},
\end{align}
which can be rearranged to give 
\begin{equation}\label{e73}
\left(\tanh^2\lambda\,\underline{\underline{J}}_{\,p}-f_w\,\underline{\underline{\hat{I}}}_{\,p}\right)\underline{p}_{\,+}+
\left(\underline{\underline{J}}_{\,pq}+\tanh^2\lambda\,\underline{\underline{J}}_{\,qp} - f_w\,\underline{\underline{\hat{I}}}_{\,q}\right)\underline{q}_{\,+},
\end{equation}
where
\begin{align}
\underline{\underline{\hat I}}_p &= \underline{\underline{k}}\,\underline{\underline{I}}_p,\\[0.5ex]
\underline{\underline{\hat I}}_q &= \underline{\underline{k}}\,\underline{\underline{I}}_q,\\[0.5ex]
\underline{\underline{J}}_{\,pq} &=\underline{\underline{J}}_{\,q}- \underline{\underline{J}}_{\,p}\,\underline{\underline{\hat{I}}}_{\,p}^{-1}\,\underline{\underline{\hat{I}}}_{\,q},\\[0.5ex]
\underline{\underline{J}}_{\,qp} &= \underline{\underline{J}}_{\,p}\,\underline{\underline{\hat{I}}}_{\,p}^{-1}\,\underline{\underline{\hat{I}}}_{\,q}.
\end{align}

Now, $z_w\sim 1/\bar{b}_w$, where $\bar{b}_w$ is the mean wall minor radius. 
In the large aspect-ratio limit, $b_w\ll 1$, we have $\underline{\underline{I}}_{\,p}\sim{\cal O}(1)$, $\underline{\underline{I}}_{\,q}\sim{\cal O}(1)$, 
$\underline{\underline{J}}_{\,p}\sim{\cal O}(1/\bar{b}_w^{\,2})$, $\underline{\underline{J}}_{\,q}\sim{\cal O}(1/\bar{b}_w^{\,2})$, 
$\underline{\underline{K}}_{\,p}\sim{\cal O}(1/\bar{b}_w)$, $\underline{\underline{K}}_{\,q}\sim{\cal O}(1/\bar{b}_w)$,
and $\underline{\underline{k}}\sim{ \cal O}(1/\bar{b}_w)$  It follows that
$\underline{\underline{\hat I}}_p\sim{\cal O}(1/\bar{b}_w)$, $\underline{\underline{\hat I}}_q\sim{\cal O}(1/\bar{b}_w)$, $\underline{\underline{J}}_{\,pq} \sim{\cal O}(1/\bar{b}_w^{\,2})$ and $\underline{\underline{J}}_{\,qp} \sim{\cal O}(1/\bar{b}_w^{\,2})$. 
Thus, the ratio of the first to the second term multiplying $\underline{p}_{\,+}$ in Eq.~(\ref{e73}) is 
\begin{equation}
\tanh\lambda \,\frac{\bar{d}_w}{\lambda\,\bar{b}_w}.
\end{equation}
However, the wall analysis is premised on the assumption that
\begin{equation}
\frac{\bar{d}_w}{\lambda\,\bar{b}_w}\ll 1.
\end{equation}
Hence, the first term is negligible with respect to the second, irrespective of the value of $\lambda$. The
ratios of the three terms multiplying $\underline{q}_{\,+}$ in Eq.~(\ref{e73}) are
\begin{equation}
\frac{\bar{d}_w}{\lambda\,\bar{b}_w},~ \tanh^2\lambda\,\frac{\bar{d}_w}{\lambda\,\bar{b}_w},~ \tanh\lambda.
\end{equation}
Thus, in the thin-shell limit, $\lambda\ll 1$, the second term is negligible with respect to the first. In the thick-shell limit, $\lambda\gg 1$, the third term is
dominant. Thus, we can neglect the second term. Hence, we deduce that
\begin{equation}\label{e88}
\underline{q}_{\,+} = \underline{\underline{\cal F}}\,\,\underline{p}_{\,+},
\end{equation}
where
\begin{align}
\underline{\underline{\cal F}}&= 
f_w\,\underline{\underline{I}}\,(\underline{\underline{J}}+ f_w\,\underline{\underline{1}})^{-1},\\[0.5ex]
\underline{\underline{I}}&=  -\underline{\underline{\hat{I}}}_{\,q}^{\,-1}\,\underline{\underline{\hat{I}}}_{\,p},\\[0.5ex]
\underline{\underline{J}}&= \underline{\underline{\hat{I}}}_{\,p}^{\,-1}\,(\underline{\underline{J}}_{\,q}\,\underline{\underline{I}} + \underline{\underline{J}}_{\,p}).
\end{align}
Note that $\underline{\underline{I}}\sim {\cal O}(1)$  and $\underline{\underline{J}}\sim {\cal O}(1/\bar{b}_w)$. 

\subsection{Toroidal Electromagnetic Torque}
The net toroidal electromagnetic torque acting on the plasma is
\begin{equation}\label{e85}
T_\phi=-2\pi^2\,n\, {\rm Im}(\underline{p}_{\,+}^\dag\,\underline{q}_{\,+})= -2\pi^2\,n\, {\rm Im}(\underline{p}_{\,+}^\dag\,\underline{\underline{\cal F}}\,\,\underline{p}_{\,+})
=-\pi^2\,n\,{\rm Im}[\underline{p}_{\,+}^\dag\,(\underline{\underline{\cal F}}-\underline{\underline{\cal F}}^\dag)\,\underline{p}_{\,+}].
\end{equation}
However, we expect this torque to be zero if $f_w$ is real, which implies that $\underline{\underline{\cal F}}=\underline{\underline{\cal F}}^\dag$
when $f_w$ is real. In other words,
\begin{align}
f_w\,\underline{\underline{I}}\,(\underline{\underline{J}}+ f_w\,\underline{\underline{1}})^{-1} =f_w\,(\underline{\underline{J}}^\dag+ f_w\,\underline{\underline{1}})^{-1}\,
\underline{\underline{I}}^\dag,
\end{align}
which implies that
\begin{equation}
f_w\,(\underline{\underline{J}}^\dag+ f_w\,\underline{\underline{1}})\,\underline{\underline{I}}=
f_w\, \underline{\underline{I}}^{\dag}\,(\underline{\underline{J}}+ f_w\,\underline{\underline{1}}).
\end{equation}
However, the previous equation holds for arbitrary real $f_w$, so we can separately equate the coefficients of $f_w$ and $f_w^{\,2}$
to give
\begin{align}\label{e90a}
\underline{\underline{J}}^\dag\,\underline{\underline{I}}&= \underline{\underline{I}}^{\dag}\underline{\underline{J}}\\[0.5ex]
\underline{\underline{I}}&=\,\underline{\underline{I}}^{\dag}.\label{e91a}
\end{align}
It follows that $\underline{\underline{I}}$ and 
\begin{equation}
\underline{\underline{K}} =  \underline{\underline{I}}\,\underline{\underline{J}}
\end{equation}
 are both real symmetric matrices. In general,
\begin{align}
\underline{\underline{\cal F}}-\underline{\underline{\cal F}}^{\,\dag}&= (f_w-f_w^\ast)\,[(\underline{\underline{J}}+ f_w\,\underline{\underline{1}})^{-1}]^\dag\,\underline{\underline{K}}\,
(\underline{\underline{J}}+ f_w\,\underline{\underline{1}})^{-1},\\[0.5ex]
T_\phi &= -2\pi^2\,n\,{\rm Im}(f_w)\,[(\underline{\underline{J}}+ f_w\,\underline{\underline{1}})^{-1}\,\underline{p}_{\,+}]^\dag\,\underline{\underline{K}}\,
[(\underline{\underline{J}}+ f_w\,\underline{\underline{1}})^{-1}\,\underline{p}_{\,+}].
\end{align}
Thus, $\underline{\underline{\cal F}}$ is clearly Hermitian if $f_w$ is real. 

\section{Matching at Plasma/Vacuum Interface}
\subsection{Matching Condition}
Let $r$, $\theta$, $\phi$ be right-handed flux coordinates, where $r$ is a flux-surface label,  $\theta$ is a poloidal angle that is zero on the inboard mid-plane, and
\begin{equation}
{\cal J} \equiv (\nabla r\times\nabla\theta\cdot\nabla\phi)^{-1}= r\,R^{\,2}.
\end{equation}
The plasma/vacuum interface lies at $r=\epsilon$. In the vacuum region between the interface and the wall, 
\begin{equation}
V(z,\eta) =\sum_m (z-\cos\eta)^{1/2}\left[p_{m+}\,\hat{P}_{|m|-1/2}^{\,n}(z)+q_{m+}\,\hat{Q}_{|m|-1/2}^{\,n}(z)\right]{\rm e}^{-{\rm i}\,m\,\eta}. 
\end{equation}
Thus, if we write 
\begin{align}\label{e90}
V(r,\theta) &= \sum_m V_m(r)\,{\rm e}^{\,{\rm i}\,m\,\theta},\\[0.5ex]
\psi(r,\theta) & = \sum_m \psi_m(r)\,{\rm e}^{\,{\rm i}\,m\,\theta}\label{e91}
\end{align}
in the same region,
where
\begin{equation}
\psi(r,\theta)  = {\cal J}\,\nabla V\cdot\nabla r,
\end{equation}
then
\begin{align}
\underline{V}&= \underline{\underline{{\cal P}}}\,\,\underline{p}_{\,+}+ \underline{\underline{{\cal Q}}}\,\,\underline{q}_{\,+},\\[0.5ex]
\underline{\psi}&= \underline{\underline{{\cal R}}}\,\,\underline{p}_{\,+}+ \underline{\underline{{\cal S}}}\,\,\underline{q}_{\,+},
\end{align}
where $\underline{V}$ is the vector of the $V_m(\epsilon)$ values, $\underline{\psi}$ is the vector of the $\psi_m(\epsilon)$ values, $\underline{\underline{{\cal P}}}$ is the
matrix of the
\begin{equation}
{\cal P}_{mm'}=\oint_{r=\epsilon}(z-\cos\eta)^{1/2}\,\hat{P}_{|m'|-1/2}^{\,n}(z)\,\exp[-{\rm i}\,(m\,\theta+m'\,\eta)]\,\frac{d\theta}{2\pi}
\end{equation}
values, 
$\underline{\underline{{\cal Q}}}$ is the
matrix of the
\begin{equation}
{\cal Q}_{mm'}=\oint_{r=\epsilon}(z-\cos\eta)^{1/2}\,\hat{Q}_{|m'|-1/2}^{\,n}(z)\,\exp[-{\rm i}\,(m\,\theta+m'\,\eta)]\,\frac{d\theta}{2\pi}
\end{equation}
values, $\underline{\underline{{\cal R}}}$ is the matrix of the 
\begin{align}\label{e354}
{\cal R}_{mm'} &=\oint_{r=\epsilon}
\left\{\left[\frac{1}{2}\,(z-\cos\eta)^{-1/2}\,\hat{P}_{|m'|-1/2}^{\,n}(z)+(z-\cos\eta)^{1/2}\,\frac{d\hat{P}_{|m'|-1/2}^{\,n}}{dz}\right]{\cal J}\,\nabla r\cdot \nabla z
\right.\nonumber\\[0.5ex]&
\left.\phantom{=}+\left[\frac{1}{2}\,(z-\cos\eta)^{-1/2}\,\sin\eta-{\rm i}\,m'\,(z-\cos\eta)^{1/2}\right]\hat{P}_{|m'|-1/2}^{\,n}(z)\,{\cal J}\,\nabla r\cdot \nabla \eta
\right\}\nonumber\\[0.5ex] &
\phantom{=}\times\exp[-{\rm i}\,(m\,\theta+m'\,\eta)]\,\frac{d\theta}{2\pi}
\end{align}
values, and 
$\underline{\underline{{\cal S}}}$ is the matrix of the 
\begin{align}
{\cal S}_{mm'} &=\oint_{r=\epsilon}
\left\{\left[\frac{1}{2}\,(z-\cos\eta)^{-1/2}\,\hat{Q}_{|m'|-1/2}^{\,n}(z)+(z-\cos\eta)^{1/2}\,\frac{d\hat{Q}_{|m'|-1/2}^{\,n}}{dz}\right]{\cal J}\,\nabla r\cdot \nabla z
\right.\nonumber\\[0.5ex]&
\left.\phantom{=}+\left[\frac{1}{2}\,(z-\cos\eta)^{-1/2}\,\sin\eta-{\rm i}\,m'\,(z-\cos\eta)^{1/2}\right]\hat{Q}_{|m'|-1/2}^{\,n}(z)\,{\cal J}\,\nabla r\cdot \nabla \eta
\right\}\nonumber\\[0.5ex] &
\phantom{=}\times\exp[-{\rm i}\,(m\,\theta+m'\,\eta)]\,\frac{d\theta}{2\pi}
\end{align}

Equations~(\ref{e88}), (\ref{e90}), and (\ref{e91}) imply that
\begin{align}
\underline{V} &= ( \underline{\underline{\cal P}}+\underline{\underline{\cal Q}}\,\underline{\underline{\cal F}})\,\underline{p}_{\,+},\\[0.5ex]
\underline{\psi} &= ( \underline{\underline{\cal R}}+\underline{\underline{\cal S}}\,\underline{\underline{\cal F}})\,\underline{p}_{\,+},
\end{align}
which yields 
\begin{equation}\label{e105}
\underline{V}= \underline{\underline{H}}\,\,\underline{\psi},
\end{equation}
where
\begin{equation}\label{e106}
\underline{\underline{H}}= (\underline{\underline{\cal P}}+\underline{\underline{\cal Q}}\,\underline{\underline{\cal F}})\,(\underline{\underline{\cal R}}+\underline{\underline{\cal S}}\,\underline{\underline{\cal F}})^{-1}.
\end{equation}

\subsection{Toroidal Electromagnetic Torque}
The net toroidal electromagnetic torque acting on the plasma is
\begin{align}
T_\phi &= -2\pi^2\,n\,{\rm Im}(\underline{V}^\dag\,\underline{\psi})\nonumber\\[0.5ex]
&
= -2\pi^2\,n\,{\rm Im}[\underline{p}_{\,+}^\dag\,( \underline{\underline{\cal P}}^\dag+\underline{\underline{\cal F}}^\dag\,\underline{\underline{\cal Q}}^\dag)\,
( \underline{\underline{\cal R}}+\underline{\underline{\cal S}}\,\underline{\underline{\cal F}})\,\underline{p}_{\,+}]\nonumber\\[0.5ex]
&= -2\pi^2\,n\,{\rm Im}[\underline{p}_{\,+}^\dag\,(\underline{\underline{\cal P}}^\dag\,\underline{\underline{\cal R}}
+ \underline{\underline{\cal F}}^\dag\,\underline{\underline{\cal Q}}^\dag\,\underline{\underline{\cal R}} +\underline{\underline{\cal P}}^\dag\,\underline{\underline{\cal S}}\,
\underline{\underline{\cal F}}+
 \underline{\underline{\cal F}}^\dag\,\underline{\underline{\cal Q}}^\dag\,\underline{\underline{\cal S}}\,\underline{\underline{\cal F}})\,
\underline{p}_{\,+}]\nonumber\\[0.5ex]
&= - \pi^2\,n\,{\rm Im}[\underline{p}_{\,+}^\dag\,(\underline{\underline{\cal P}}^\dag\,\underline{\underline{\cal R}}- \underline{\underline{\cal R}}^\dag\,\underline{\underline{\cal P}})\,
\underline{p}_{\,+}]  
- \pi^2\,n\,{\rm Im}[\underline{p}_{\,+}^\dag\,(\underline{\underline{\cal P}}^\dag\,\underline{\underline{\cal S}}- \underline{\underline{\cal R}}^\dag\,\underline{\underline{\cal Q}})\,\underline{\underline{\cal F}}\,\,\underline{p}_{\,+}]\nonumber\\[0.5ex]
&\phantom{=}
+ \pi^2\,n\,{\rm Im}[\underline{p}_{\,+}^\dag\,\underline{\underline{\cal F}}^\dag\,(\underline{\underline{\cal S}}^\dag\,\underline{\underline{\cal P}}- \underline{\underline{\cal Q}}^\dag\,\underline{\underline{\cal R}})\,\underline{p}_{\,+}]
- \pi^2\,n\,{\rm Im}[\underline{p}_{\,+}^\dag\,\underline{\underline{\cal F}}^\dag\,(\underline{\underline{\cal Q}}^\dag\,\underline{\underline{\cal S}}- \underline{\underline{\cal S}}^\dag\,\underline{\underline{\cal Q}})\,\underline{\underline{\cal F}}\,\underline{p}_{\,+}].
\end{align}
The previous equation is consistent with Eq.~(\ref{e85}) provided that
\begin{align}
\underline{\underline{\cal P}}^\dag\,\underline{\underline{\cal R}}&= \underline{\underline{\cal R}}^\dag\,\underline{\underline{\cal P}},\label{e100}\\[0.5ex]
\underline{\underline{\cal Q}}^\dag\,\underline{\underline{\cal S}}&= \underline{\underline{\cal S}}^\dag\,\underline{\underline{\cal Q}},\\[0.5ex]
\underline{\underline{\cal P}}^\dag\,\underline{\underline{\cal S}}- \underline{\underline{\cal R}}^\dag\,\underline{\underline{\cal Q}}&=\underline{\underline{1}}.\label{e103}
\end{align}
Making use of the previous three equations, we can show that
\begin{equation}\label{herm}
\underline{\underline{H}} -\underline{\underline{H}}^{\,\dag} = 
-[(\underline{\underline{\cal R}}+\underline{\underline{\cal S}}\,\underline{\underline{\cal F}})^{-1}]^{\dag}\,(\underline{\underline{\cal F}}-
\underline{\underline{\cal F}}^{\,\dag})\,(\underline{\underline{\cal R}}+\underline{\underline{\cal S}}\,\underline{\underline{\cal F}})^{-1}.
\end{equation}
Thus, $\underline{\underline{H}}$ is Hermitian if $\underline{\underline{\cal F}}$ is Hermitian, which implies that
$\underline{\underline{H}}$ is Hermitian if $f_w$ is real. 

\section{Vacuum Matrix}
\subsection{No-Wall and Perfect-Wall Boundary Conditions}
In the no-wall limit, $f_w=0$, and $\underline{\underline{\cal F}}=\underline{\underline{0}}$. Hence, the boundary condition at the
plasma/vacuum interface becomes
\begin{equation}
\underline{V}= \underline{\underline{H}}_{\,nw}\,\,\underline{\psi},
\end{equation}
where
\begin{equation}
\underline{\underline{H}}_{\,nw}= \underline{\underline{\cal P}}\,\underline{\underline{\cal R}}^{-1}.
\end{equation}
Equation~(\ref{e100}) implies that $\underline{\underline{H}}_{\,nw}$ is Hermitian. 

In the perfect-wall limit, $f(\lambda)\rightarrow\infty$, and $\underline{\underline{\cal F}}= \underline{\underline{I}}$. 
Hence, the boundary condition at the interface becomes 
\begin{equation}
\underline{V}= \underline{\underline{H}}_{\,pw}\,\,\underline{\psi},
\end{equation}
where
\begin{equation}
\underline{\underline{H}}_{\,pw} = (\underline{\underline{\cal P}}+\underline{\underline{\cal Q}}\,\underline{\underline{I}})\,(\underline{\underline{\cal R}}+\underline{\underline{\cal S}}\,\underline{\underline{I}})^{-1}.
\end{equation}
Comparison to Eqs.~(\ref{e106}) and (\ref{herm}) reveals that $\underline{\underline{H}}_{\,pw}$ is Hermitian, given that that $\underline{\underline{I}}$ is Hermitian.

\subsection{Symmetric form of Vacuum Matrix}
It is possible to demonstrate that
\begin{equation}\label{e130}
\underline{\underline{H}} = \underline{\underline{H}}_{\,nw} + f_w\,( \underline{\underline{H}}_{\,pw} - \underline{\underline{H}}_{\,nw} )\,(\underline{\underline{B}}+f_w\,\underline{\underline{1}})^{-1},
\end{equation}
where
\begin{equation}
\underline{\underline{B}}= \underline{\underline{\cal R}}\,\underline{\underline{J}}\,(\underline{\underline{\cal R}} + \underline{\underline{\cal S}}\,\underline{\underline{I}})^{-1}.
\end{equation}
Note that $\underline{\underline{B}}\sim{\cal O}(1/\bar{b}_w)$. 
Suppose that $f_w$ is real. It follows that
\begin{equation}
\underline{\underline{H}} -\underline{\underline{H}}^{\,\dag} = 
f_w\left[( \underline{\underline{H}}_{\,pw} - \underline{\underline{H}}_{\,nw} )\,(\underline{\underline{B}}+f_w\,\underline{\underline{1}})^{-1}- (\underline{\underline{B}}^{\,\dag}+f_w\,
\underline{\underline{1}})^{-1}\,( \underline{\underline{H}}_{\,pw} - \underline{\underline{H}}_{\,nw} )
\right].
\end{equation}
However, we know that $\underline{\underline{H}}$ is Hermitian when $f_w$ is real. Hence, we deduce that
\begin{equation}\label{e133}
 (\underline{\underline{H}}_{\,pw} - \underline{\underline{H}}_{\,nw} )\,(\underline{\underline{B}}+f_w\,\underline{\underline{1}})^{-1}= (\underline{\underline{B}}^{\,\dag}+f_w\,
\underline{\underline{1}})^{-1}\,( \underline{\underline{H}}_{\,pw} - \underline{\underline{H}}_{\,nw} ),
\end{equation}
which implies that
\begin{equation}\label{e122}
\underline{\underline{B}}^{\,\dag}\,( \underline{\underline{H}}_{\,pw} - \underline{\underline{H}}_{\,nw} )= ( \underline{\underline{H}}_{\,pw} - \underline{\underline{H}}_{\,nw} )\,\underline{\underline{B}}.
\end{equation}
The previous equation yields
\begin{align}
&(\underline{\underline{\cal R}}^{\,\dag}+\underline{\underline{I}}\,\underline{\underline{\cal S}}^{\,\dag})\,\underline{\underline{J}}^{\,\dag}\,\underline{\underline{\cal R}}^{\,\dag}
\left[(\underline{\underline{\cal P}}+\underline{\underline{\cal Q}}\,\underline{\underline{I}})\,(\underline{\underline{\cal R}}+\underline{\underline{\cal S}}\,\underline{\underline{I}})^{-1}
 -\underline{\underline{\cal R}}^{\,-1\,\dag}\,\underline{\underline{P}}^{\,\dag}\right]\nonumber\\[0.5ex]
 &=
  \left[(\underline{\underline{\cal R}}^{\,\dag}+
 \underline{\underline{I}}\,\underline{\underline{\cal S}}^{\,\dag})^{-1}\,(\underline{\underline{\cal P}}^{\,\dag}
 +\underline{\underline{I}}\,\underline{\underline{\cal Q}}^{\,\dag})
 -\underline{\underline{\cal P}}\,\underline{\underline{\cal R}}^{-1}\right]\underline{\underline{\cal R}}\,\underline{\underline{J}}\,(\underline{\underline{\cal R}} + \underline{\underline{\cal S}}\,\underline{\underline{I}})^{-1}.
\end{align}
where use has been made of the fact that $\underline{\underline{H}}_{\,pw}$ and $\underline{\underline{H}}_{\,nw}$ are Hermitian. 
Making use of Eqs.~(\ref{e100})--(\ref{e103}), the previous equation reduces to 
\begin{equation}
-\underline{\underline{J}}^\dag\,\underline{\underline{I}}= -\underline{\underline{I}}\,\underline{\underline{J}},
\end{equation}
which  Eqs.~(\ref{e90a}) and (\ref{e91a}) ensure is satisfied. Equations~(\ref{e130}) and (\ref{e133}) can
be combined to give
\begin{align}\label{e128x}
\underline{\underline{H}} &= \underline{\underline{H}}_{\,nw} 
+\frac{ f_w}{2}\left[ 
(\underline{\underline{B}}^{\,\dag}+f_w\,\underline{\underline{1}})^{-1}\,( \underline{\underline{H}}_{\,pw} - \underline{\underline{H}}_{\,nw} )
+(\underline{\underline{H}}_{\,pw} - \underline{\underline{H}}_{\,nw} )\,(\underline{\underline{B}}+f_w\,\underline{\underline{1}})^{-1} \right]\nonumber\\[0.5ex]
&= \underline{\underline{H}}_{\,nw} +f_w\,(\underline{\underline{B}}^{\,\dag}+f_w\,\underline{\underline{1}})^{-1}\,( \underline{\underline{H}}_{\,pw} - \underline{\underline{H}}_{\,nw} ),\nonumber\\[0.5ex]
&= \underline{\underline{H}}_{\,nw} +f_w\,(\underline{\underline{H}}_{\,pw} - \underline{\underline{H}}_{\,nw} )\,(\underline{\underline{B}}+f_w\,\underline{\underline{1}})^{-1}.
\end{align}
Note that $\underline{\underline{H}}$ is clearly Hermitian if $f_w$ is real. 

\section{Wall Torque}
Consider an unreconnected tearing mode, resonant at the $k$th rational surface, and rotating at the angular phase velocity $\omega_k$. It follows that
$\gamma=-{\rm i}\,\omega_k$. Thus, 
\begin{equation}
f_w = \frac{\zeta}{2\,\bar{d}_w}\left[\left(\frac{\sinh\zeta - \sin\zeta}{\cosh\zeta+\cos\zeta}\right)-{\rm i}\left(\frac{\sinh\zeta+ \sin\zeta}{\cosh\zeta+\cos\zeta}\right)\right],
\end{equation}
where
\begin{equation}
\zeta = (2\,\hat{\omega}_k\,\bar{d}_w)^{1/2},
\end{equation}
and $\hat{\omega}_k=\omega_k\,\bar{\tau}_w$. 
The matching condition at the plasma vacuum interface becomes 
\begin{equation}
\underline{V}= \underline{\underline{H}}(\zeta)\,\,\underline{\psi}.
\end{equation}
Note that $\underline{\underline{H}}(\zeta)$ is not generally Hermitian, because $f_w$ is complex,  which implies that
the $E$-matrix is not Hermitian. The toroidal electromagnetic torque acting at the $k$th rational surface is
\begin{equation}
\delta T_k = 2\pi^2\,n\,{\rm Im}(E_{kk})\,|{\mit\Psi}_k|^2.
\end{equation}

\section{Resistive Wall Mode}
We can write
\begin{align}
\underline{V}&= \underline{\underline{V}}_{\,i}\,\underline{\alpha},\\[0.5ex]
\underline{\psi}&= \underline{\underline{\psi}}_{\,i}\,\underline{\alpha},
\end{align}
where the $\underline{\underline{V}}_{\,i}$ and $\underline{\underline{\psi}}_{\,i}$ are ideal solutions.
The net toroidal electromagnetic torque acting on the plasma is
\begin{equation}
T_\phi= -2\pi^2\,n\,{\rm Im}(\underline{V}^\dag\,\underline{\psi})=-2\pi^2\,n\,{\rm Im}(\underline{\alpha}^\dag\,\underline{\underline{V}}_{\,i}^\dag
\,\underline{\underline{\psi}}_{\,i}\,\underline{\alpha}).
\end{equation}
However, the net torque acting on an ideal plasma is zero, so
\begin{equation}\label{e121}
\underline{\underline{V}}_{\,i}^\dag \,\underline{\underline{\psi}}_{\,i}=\underline{\underline{\psi}}_{\,i}^\dag \,\underline{\underline{V}}_{\,i}.
\end{equation}

Equation~(\ref{e105}) implies that 
\begin{equation}
\underline{\underline{V}}_{\,i}\,\underline{\alpha}=\underline{\underline{H}}\,\underline{\underline{\psi}}_{\,i}\,\underline{\alpha}.
\end{equation}
Writing
\begin{equation}
\underline{\underline{\psi}}_{\,i}\,\underline{\alpha} = \underline{x},
\end{equation}
we obtain
\begin{equation}
\underline{\underline{W}}_{\,p}\,\underline{x}
= \underline{\underline{H}}\,\underline{x},
\end{equation}
where
\begin{equation}\label{e128}
\underline{\underline{W}}_{\,p} = \underline{\underline{V}}_{\,i}\,\underline{\underline{\psi}}_{\,i}^{\,-1}.
\end{equation}
Equation~(\ref{e121}) ensures that $\underline{\underline{W}}_{\,p}$ is Hermitian. 

Equation~(\ref{e128x}) can be combined with the previous equation to give
\begin{equation}
\underline{\underline{W}}_{\,nw}\,\underline{\beta} = -\frac{f_w}{2}
\left[ (\underline{\underline{W}}_{\,pw} - \underline{\underline{W}}_{\,nw} )\,(\underline{\underline{B}}+f_w\,\underline{\underline{1}})^{-1} + (\underline{\underline{B}}^{\,\dag}+f_w\,
\underline{\underline{1}})^{-1}\,( \underline{\underline{W}}_{\,pw} - \underline{\underline{W}}_{\,nw} )\right]\underline{\beta},
\end{equation}
where
\begin{align}
\underline{\underline{W}}_{\,nw}  &= \underline{\underline{W}}_{\,p}-\underline{\underline{H}}_{\,nw},\\[0.5ex]
\underline{\underline{W}}_{\,pw}  &= \underline{\underline{W}}_{\,p}-\underline{\underline{H}}_{\,pw}.
\end{align}
Note that $\underline{\underline{W}}_{\,nw}$ and $\underline{\underline{W}}_{\,pw}$ are Hermitian.
Writing
\begin{equation}
(\underline{\underline{B}}+f_w\,\underline{\underline{1}})^{-1} \,\underline{x}= \underline{y},
\end{equation}
we obtain
\begin{align}\label{e141}
\left(f_w\,\underline{\underline{W}}_{\,pw}
+\underline{\underline{B}}^{\,\dag}\,\underline{\underline{W}}_{\,nw}\right)\underline{x}&=\underline{0},\\[0.5ex]
\left(f_w\,\underline{\underline{W}}_{\,pw}
+\underline{\underline{W}}_{\,nw}\,\underline{\underline{B}}\right)\underline{y} &=\underline{0}.\label{e142}
\end{align}
Note that the $\underline{x}$ are the left-eigenvectors of Eq.~(\ref{e142}), whereas the $\underline{y}$ are the left-eigenvectors of Eq.~(\ref{e141}). 
The previous two equations can be combined to give
\begin{equation}
(f_w-f_w^{\,\ast})\,\underline{y}^{\,\dag}\,\underline{\underline{W}}_{\,pw}\,\underline{x} = 0.
\end{equation}
Hence, we deduce that the eigenvalues, $f_w$,  are real, and that the eigenvectors, $\underline{x}$ and $\underline{y}$,  are orthonormal, in the sense that
\begin{equation}
\underline{y}_i^{\,\dag}\,
\underline{\underline{W}}_{\,pw}\,\underline{x}_j= \delta_{ij}.
\end{equation} 
Once we have determined the eigenvectors then
\begin{align}
\underline{\psi} &=\underline{x},\\[0.5ex]
\underline{V} &= \underline{\underline{W}}_{\,p} \,\underline{x}.
\end{align}

We can write
\begin{align}
\underline{\psi} &= \underline{\underline{Q}}\,\underline{\mit\Xi},\\[0.5ex]
\underline{Z} &= \underline{\underline{Q}}\,\underline{V},
\end{align}
where $\underline{\underline{Q}}$ is the diagonal matrix of the $m-n\,q(\epsilon)$ values. 
Let
\begin{align}
\underline{\underline{\widetilde{W}}}_{\,nw} &= \underline{\underline{Q}}\,\underline{\underline{W}}_{\,nw}\,\underline{\underline{Q}},\\[0.5ex]
\underline{\underline{\widetilde{W}}}_{\,pw} &= \underline{\underline{Q}}\,\underline{\underline{W}}_{\,pw}\,\underline{\underline{Q}},\\[0.5ex]
\underline{\underline{\widetilde{B}}} &= \underline{\underline{Q}}^{-1}\,\underline{\underline{B}}\,\underline{\underline{Q}},\\[0.5ex]
\underline{\widetilde{x}}&=  \underline{\underline{Q}}^{-1}\,\underline{x},\\[0.5ex]
\underline{\widetilde{y}}&=  \underline{\underline{Q}}^{-1}\,\underline{x}.
\end{align}
It follows that 
\begin{align}
\left(f_w\,\underline{\underline{\widetilde{W}}}_{\,pw}
+\underline{\underline{\widetilde{B}}}^{\,\dag}\,\underline{\underline{\widetilde{W}}}_{\,nw}\right)\underline{\widetilde{x}}&=\underline{0},\\[0.5ex]
\left(f_w\,\underline{\underline{\widetilde{W}}}_{\,pw}
+\underline{\underline{\widetilde{W}}}_{\,nw}\,\underline{\underline{\widetilde{B}}}\right)\underline{\widetilde{y}} &=\underline{0}.\label{e142}
\end{align}
Hence, we again deduce that 
\begin{equation}
(f_w-f_w^{\,\ast})\,\underline{\widetilde{y}}^{\,\dag}\,\underline{\underline{\widetilde{W}}}_{\,pw}\,\underline{\widetilde{x}} = 0.
\end{equation}
Furthermore,
\begin{align}
\underline{\mit\Xi} &=\underline{\widetilde{x}},\\[0.5ex]
\underline{Z} &= \underline{\underline{\widetilde{W}}}_{\,p} \,\underline{\widetilde{x}}.
\end{align}

\end{document}