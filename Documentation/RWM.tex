\documentclass[12pt,prb,aps,notitlepage]{revtex4-1}
\usepackage {amsmath}
\usepackage{amssymb}
\pdfoutput = 1 
\usepackage {graphicx}
\newcommand{\bomega}{\mbox{\boldmath$\omega$}}
\allowdisplaybreaks

\begin{document}

\title{Resistive Wall}
\author{R.~Fitzpatrick\,\footnote{rfitzp@utexas.edu}}
\affiliation{Institute for Fusion Studies,  Department of Physics,  University of Texas at Austin,  Austin TX 78712, USA}\begin{abstract}
\end{abstract}
\maketitle

\section{Vacuum Solution}

\subsection{Normalization}
Let all lengths be normalized to the major radius of the  axisymmetric plasma equilibrium's magnetic axis, $R_0$. Let all
magnetic field-strengths be normalized to the toroidal magnetic field-strength at the magnetic axis, $B_0$.

\subsection{Toroidal Coordinates}
Let $\mu$, $\eta$, $\phi$ be right-handed toroidal coordinates defined such that
\begin{align}
R &= \frac{\sinh\mu}{\cosh\mu-\cos\eta},\\[0.5ex]
Z&=\frac{\sin\eta}{\cosh\mu-\cos\eta},
\end{align}
where $R$, $\phi$, $Z$ are right-handed cylindrical coordinates whose symmetry axis corresponds to that of the plasma equilibrium. 
Note that $(\nabla R\times \nabla\phi\cdot\nabla Z)^{-1}=R$. 
The scale-factors of the toroidal coordinate system are
\begin{align}
h_\mu&=h_\eta= \frac{1}{\cosh\mu-\cos\eta}\equiv h,\\[0.5ex]
h_\phi &= \frac{\sinh\mu}{\cosh\mu-\cos\eta} = h\,\sinh\mu.
\end{align}
Moreover, 
\begin{equation}
{\cal J}' \equiv (\nabla\mu\times\nabla\eta\cdot\nabla\phi)= h^3\,\sinh\mu.
\end{equation}

\subsection{Perturbed Magnetic Field}
The curl-free perturbed magnetic field in the vacuum region is written
\begin{equation}
{\bf b} = {\rm i}\,\nabla \left[V(\mu,\eta)\,{\rm e}^{-{\rm i}\,n\,\phi}\right],
\end{equation}
where the toroidal mode number, $n$, is a positive integer. Given that $\nabla\cdot{\bf b}=0$, we deduce that 
\begin{align}\label{e6}
\nabla^2 V &\equiv (z-\cos\eta)^3\left\{\frac{\partial}{\partial z}\!\left[\frac{z^2-1}{z-\cos\eta}\,\frac{\partial V}{\partial z}\right]\right.\nonumber\\[0.5ex]
&\left.\phantom{=}+\frac{\partial}{\partial \eta}\!\left[\frac{1}{z-\cos\eta}\,\frac{\partial V}{\partial\eta}\right]
-\frac{n^2\,V}{(z^2-1)\,(z-\cos\eta)}\right\}=0.
\end{align}
Here, $z=\cosh\mu$. 

Let
\begin{align}
f_z &= z^2-1,\\[0.5ex]
f_\eta &= (z-\cos\eta)^{1/2},
\end{align}
which implies that
\begin{align}
\frac{df_z}{dz} &= 2\,z,\\[0.5ex]
\frac{\partial f_\eta}{\partial z}&= \frac{1}{2\,f_\eta},\\[0.5ex]
\frac{\partial f_\eta}{\partial\eta} &= \frac{\sin\eta}{2\,f_\eta}
\end{align}
Suppose that
\begin{equation}\label{e13}
V(z,\eta)= \sum_m (z-\cos\eta)^{1/2}\,U_m(z)\,{\rm e}^{-{\rm i}\,m\,\eta}.
\end{equation}
Taking the sum and eikonal as read, and letting $'=d/dz$, we get 
\begin{align}
\frac{\partial V}{\partial z} &=\frac{U_m}{2\,f_\eta} + f_\eta\,U_m',\\[0.5ex]
\frac{\partial}{\partial z}\!\left(\frac{f_z}{f_\eta^{\,2}}\,\frac{\partial V}{\partial z}\right)&=
\frac{\partial}{\partial z}\!\left(\frac{f_z\,U_m}{2\,f_\eta^{\,3}}+ \frac{f_z\,U_m'}{f_\eta}\right)
= \frac{z\,U_m}{f_\eta^{\,3}} -\frac{3\,f_z\,U_m}{4\,f_\eta^{\,5}}+\frac{f_z\,U_m'}{2\,f_\eta^{\,3}}
+\frac{2\,z\,U_m'}{f_\eta}- \frac{f_z\,U_m'}{2\,f_\eta^{\,3}}
+\frac{f_z\,U_m''}{f_\eta}\nonumber\\[0.5ex]
&=\frac{z\,U_m}{f_\eta^{\,3}} -\frac{3\,(z^2-1)\,U_m}{4\,f_\eta^{\,5}}
+\frac{2\,z\,U_m'}{f_\eta}
+\frac{(z^2-1)\,U_m''}{f_\eta},\\[0.5ex]
\frac{\partial V}{\partial \eta} &=\frac{\sin\eta\,U_m}{2\,f_\eta} -{\rm i}\,m f_\eta\,U_m,\\[0.5ex]
\frac{\partial}{\partial \eta}\!\left(\frac{1}{f_\eta^{\,2}}\,\frac{\partial V}{\partial \eta}\right)&=
\frac{\partial}{\partial \eta}\!\left(\frac{\sin\eta\,U_m}{2\,f_\eta^{\,3}}-\frac{{\rm i}\,m\,U_m}{f_\eta}\right)=
\frac{\cos\eta\,U_m}{2\,f_\eta^{\,3}}- \frac{3\,\sin^2\eta\,U_m}{4\,f_\eta^{\,5}}-\frac{{\rm i}\,m\,\sin\eta\,U_m}{2\,f_\eta^{\,3}}\nonumber\\[0.5ex]
&\phantom{=} + \frac{{\rm i}\,m\,\sin\eta\,U_m}{2\,f_\eta^{\,3}} - \frac{m^2\,U_m}{f_\eta}\nonumber\\[0.5ex]
&=\frac{\cos\eta\,U_m}{2\,f_\eta^{\,3}}- \frac{3\,\sin^2\eta\,U_m}{4\,f_\eta^{\,5}} - \frac{m^2\,U_m}{f_\eta},\\[0.5ex]
-\frac{n^2\,V}{f_z\,f_\eta^{\,2}}&= -\frac{n^2\,U_m}{(z^2-1)\,f_\eta}.
\end{align}
Thus, Eq.~(\ref{e6}) becomes 
\begin{align}
0&=\frac{\partial}{\partial z}\!\left(\frac{f_z}{f_\eta^{\,2}}\,\frac{\partial V}{\partial z}\right)+ \frac{\partial}{\partial \eta}\!\left(\frac{1}{f_\eta^{\,2}}\,\frac{\partial V}{\partial \eta}\right)
-\frac{n^2\,V}{f_z\,f_\eta^{\,2}}\nonumber\\[0.5ex]&
= \frac{z\,U_m}{f_\eta^{\,3}} -\frac{3\,(z^2-1)\,U_m}{4\,f_\eta^{\,5}}
+\frac{2\,z\,U_m'}{f_\eta}
+\frac{(z^2-1)\,U_m''}{f_\eta}\nonumber\\[0.5ex]
&\phantom{=}+\frac{\cos\eta\,U_m}{2\,f_\eta^{\,3}}- \frac{3\,\sin^2\eta\,U_m}{4\,f_\eta^{\,5}} - \frac{m^2\,U_m}{f_\eta} -\frac{n^2\,U_m}{(z^2-1)\,f_\eta}\nonumber\\[0.5ex]
&=\frac{1}{f_\eta}\left[(z^2-1)\,U_m'' + 2\,z\,U_m' + \left(\frac{1}{4}-m^2\right)U_m - \frac{n^2\,U_m}{z^2-1}\right].
\end{align}
The most general solution of the previous equation is
\begin{equation}\label{e20}
U_m(z) = p_m\,\hat{P}_{|m|-1/2}^{\,n}(z)+q_m\,\hat{Q}_{m-1/2}^{\,n}(z),
\end{equation}
where
\begin{align}\label{e21}
\hat{P}_{|m|-1/2}^{\,n}(z) &= \cos(|m|\,\pi)\,\frac{\sqrt{\pi}\,\Gamma(|m|+1/2-n)\,\epsilon^{\,|m|}}{2^{\,|m|-1/2}\,|m|!}\,P_{|m|-1/2}^{\,n}(z),\\[0.5ex]
\hat{Q}_{|m|-1/2}^{\,n}(z)&= \cos(n\,\pi)\,\cos(|m|\,\pi)\,\frac{2^{\,|m|-1/2}\,|m|!}{\sqrt{\pi}\,\Gamma(|m|+1/2+n)\,\epsilon^{\,|m|}}\,Q_{|m|-1/2}^{\,n}.\label{e22}
\end{align}
Here,  $\epsilon$ is the inverse-aspect ratio of the plasma equilibrium, and  $p_m$ and $q_m$ are arbitrary complex coefficients. 
Moreover, we have made use of the fact that 
\begin{align}
P_{-m-1/2}^{\,n}(z) &= P_{m-1/2}^{\,n}(z),\\[0.5ex]
Q_{-m-1/2}^{\,n}(z) &= Q_{m-1/2}^{\,n}(z).
\end{align}


\subsection{Toroidal Electromagnetic Angular Momentum Flux}
The outward flux of toroidal angular momentum across a constant-$z$ surface is
\begin{align}
T_\phi(z) = -\oint\oint {\cal J}' \,b_\phi\,b^{\,\mu}\,d\eta\,d\phi.
\end{align}
Now,
\begin{align}
b^{\,\mu} &\equiv {\bf b}\cdot\nabla\mu = {\rm i}\,\frac{\partial V}{\partial \mu}\,|\nabla\mu|^2 = {\rm i}\,\frac{\sinh\mu}{h^2}\,\frac{\partial V}{\partial z},\\[0.5ex]
b_{\phi}&\equiv {\cal J}'\,\nabla\mu\times\nabla \eta\cdot\nabla V = n\,V,
\end{align}
so
\begin{align}
T_\phi(z) &= -\frac{{\rm i}\,n\,\pi}{2}\oint \frac{z^2-1}{z-\cos\eta}\left(\frac{\partial V}{\partial z}\,V^\ast -\frac{\partial V^{\ast}}{\partial z}\,V\right)d\eta\nonumber\\[0.5ex]
&= - {\rm i}\,n\,\pi^2\sum_{m}\, (z^2-1)\left(\frac{dU_m}{dz}\,U_{m}^\ast -\frac{dU_m^\ast}{dz}\,U_{m}\right)\nonumber\\[0.5ex]
&= -{\rm i}\,n\,\pi^2\sum_{m}(p_m\,q_m^\ast-q_m\,p_m^{\ast})\,(z^2-1)\left(\frac{d\hat{P}|_{|m|-1/2}^{\,n}}{dz}\,\hat{Q}_{|m|-1/2}^{\,n}- \frac{d\hat{Q}_{|m|-1/2}^{\,n}}{dz}\,\hat{P}_{|m|-1/2}^{\,n}\right)\nonumber\\[0.5ex]
& ={\rm i}\,n\,\pi^2\sum_{m}(p_m\,q_m^\ast-q_m\,p_m^{\ast})\,(z^2-1)\,{\cal W}(\hat{P}_{|m|-1/2}^{\,n},\hat{Q}_{|m|-1/2}^{\,n}),
\end{align}
where use has been made of Eqs.~(\ref{e13}) and (\ref{e20}). 
But,
\begin{align}
{\cal W}(\hat{P}_{|m|-1/2}^{\,n},\hat{Q}_{|m|-1/2}^{\,n})&= 
\cos(n\,\pi) \,\frac{\Gamma(|m|+1/2-n)}{\Gamma(|m|+1/2+n)}\,{\cal W}(P_{|m|-1/2}^{\,n},Q_{|m|-1/2}^{\,n})\nonumber\\[0.5ex]
&= \cos(n\,\pi) \,\frac{\Gamma(|m|+1/2-n)}{\Gamma(|m|+1/2+n)}
\,\frac{\cos(n\,\pi)}{1-z^2}\,\frac{\Gamma(|m|+1/2+n)}{\Gamma(|m|+1/2-n)}\nonumber\\[0.5ex]
&= \frac{1}{1-z^2}, 
\end{align}
where use has been made of Eqs.~(\ref{e21}) and (\ref{e22}), 
so
\begin{equation}\label{e30}
T_\phi(z) = 2\pi^2\,n\sum_m {\rm Im}(q_m^\ast\,p_m).
\end{equation}
Note that $T_\phi$ is independent of $z$. 

\section{Matching at Plasma-Vacuum Interface}
\subsection{Solution in Vacuum Region}
Let $r$, $\theta$, $\phi$ be right-handed flux coordinates, where $r$ is a flux-surface label,  $\theta$ is a poloidal angle that is zero on the inboard mid-plane, and
\begin{equation}
{\cal J} \equiv (\nabla r\times\nabla\theta\cdot\nabla\phi)^{-1}= r\,R^{\,2}.
\end{equation}

In the large-aspect ratio limit $r\ll 1$, it can be demonstrated that 
\begin{align}\label{e25t}
z&\simeq \frac{1}{r},\\[0.5ex]
z^{\,1/2}\,\hat{P}^{\,n}_{-1/2}(z) &\simeq \frac{1}{2}\ln\left(\frac{8\,z}{\zeta_n}\right),\\[0.5ex]
z^{1/2}\,\hat{P}^{\,n}_{|m|-1/2}(z) &\simeq \frac{\cos(|m|\,\pi)\,(\epsilon\,z)^{|m|}}{|m|},\label{ety}\\[0.5ex]
z^{1/2}\,\hat{Q}^{\,n}_{|m|-1/2}(z) &\simeq \frac{\cos(|m|\,\pi)\,(\epsilon\,z)^{-|m|}}{2},\\[0.5ex]
\zeta_n &= \exp\left(\sum_{j=1,n}\frac{2}{2\,j-1}\right).\label{e29t}
\end{align}
Note that Eq.~(\ref{ety}) only applies to $|m|>0$. 


The plasma-vacuum interface lies at $r=\epsilon$. The wall lies at $r=b_w\,\epsilon$, where $b_w>1$,  In the
vacuum region, $\epsilon\leq r\leq b_w\,\epsilon$,  lying between the plasma and the wall, we can write
\begin{align}\label{e32a}
\underline{V}(r)&= \underline{\underline{{\cal P}}}(r)\,\underline{p}+ \underline{\underline{{\cal Q}}}(r)\,\underline{q},\\[0.5ex]
\underline{\psi}(r)&= \underline{\underline{{\cal R}}}(r)\,\underline{p}+ \underline{\underline{{\cal S}}}(r)\,\underline{q},\label{e33a}
\end{align}
where $\underline{V}(r)$ is the vector of the $V_m(r)$ values, $\underline{\psi}(r)$ is the vector of the $\psi_m(r)$ values, $\underline{\underline{{\cal P}}}(r)$ is the
matrix of the
\begin{equation}
{\cal P}_{mm'}=\oint_{r}(z-\cos\eta)^{1/2}\,\hat{P}_{|m'|-1/2}^{\,n}(z)\,\exp[-{\rm i}\,(m\,\theta+m'\,\eta)]\,\frac{d\theta}{2\pi}
\end{equation}
values, 
$\underline{\underline{{\cal Q}}}(r)$ is the
matrix of the
\begin{equation}
{\cal Q}_{mm'}=\oint_{r}(z-\cos\eta)^{1/2}\,\hat{Q}_{|m'|-1/2}^{\,n}(z)\,\exp[-{\rm i}\,(m\,\theta+m'\,\eta)]\,\frac{d\theta}{2\pi}
\end{equation}
values, $\underline{\underline{{\cal R}}}(r)$ is the matrix of the 
\begin{align}\label{e354}
{\cal R}_{mm'} &=\oint_{r}
\left\{\left[\frac{1}{2}\,(z-\cos\eta)^{-1/2}\,\hat{P}_{|m'|-1/2}^{\,n}(z)+(z-\cos\eta)^{1/2}\,\frac{d\hat{P}_{|m'|-1/2}^{\,n}}{dz}\right]{\cal J}\,\nabla r\cdot \nabla z
\right.\nonumber\\[0.5ex]&
\left.\phantom{=}+\left[\frac{1}{2}\,(z-\cos\eta)^{-1/2}\,\sin\eta-{\rm i}\,m'\,(z-\cos\eta)^{1/2}\right]\hat{P}_{|m'|-1/2}^{\,n}(z)\,{\cal J}\,\nabla r\cdot \nabla \eta
\right\}\nonumber\\[0.5ex] &
\phantom{=}\times\exp[-{\rm i}\,(m\,\theta+m'\,\eta)]\,\frac{d\theta}{2\pi}
\end{align}
values, 
$\underline{\underline{{\cal S}}}(r)$ is the matrix of the 
\begin{align}\label{e355}
{\cal S}_{mm'} &=\oint_{r}
\left\{\left[\frac{1}{2}\,(z-\cos\eta)^{-1/2}\,\hat{Q}_{|m'|-1/2}^{\,n}(z)+(z-\cos\eta)^{1/2}\,\frac{d\hat{Q}_{|m'|-1/2}^{\,n}}{dz}\right]{\cal J}\,\nabla r\cdot \nabla z
\right.\nonumber\\[0.5ex]&
\left.\phantom{=}+\left[\frac{1}{2}\,(z-\cos\eta)^{-1/2}\,\sin\eta-{\rm i}\,m'\,(z-\cos\eta)^{1/2}\right]\hat{Q}_{|m'|-1/2}^{\,n}(z)\,{\cal J}\,\nabla r\cdot \nabla \eta
\right\}\nonumber\\[0.5ex] &
\phantom{=}\times\exp[-{\rm i}\,(m\,\theta+m'\,\eta)]\,\frac{d\theta}{2\pi}
\end{align}
values, $\underline{p}$ is the vector of the $p_m$ coefficients, and  $\underline{q}$ is the vector of the $q_m$ coefficients. Here, the
subscript $r$ on the integrals indicates that they are taken at constant $r$. 

\subsection{Toroidal Electromagnetic Torque}
The net toroidal electromagnetic torque acting on the plasma is
\begin{align}\label{e38}
T_\phi(r) &= -2\pi^2\,n\,{\rm Im}(\underline{V}^\dag\,\underline{\psi})= -\pi^2\,n\,(\underline{V}^\dag\,\underline{\psi}-\underline{\psi}^\dag\,\underline{V}).
\end{align}
However, Eq.~(\ref{e30}) implies that
\begin{equation}\label{e39}
T_\phi= 2\pi^2\,n\,{\rm Im}(\underline{q}^\dag\,\underline{p})= \pi^2\,n\,(\underline{q}^\dag\,\underline{p}- \underline{p}^\dag\,\underline{q}).
\end{equation}
Now, Eqs.~(\ref{e32a}), (\ref{e33a}), and (\ref{e38}) give 
\begin{align}
T_\phi&= -\pi^2\,n\,\left[
\underline{p}^\dag\,(\underline{\underline{\cal P}}^\dag\,\underline{\underline{\cal R}}
- \underline{\underline{\cal R}}^\dag\,\underline{\underline{\cal P}})\,\underline{p}
+\underline{p}^\dag\,(\underline{\underline{\cal P}}^\dag\,\underline{\underline{\cal S}}
-\underline{\underline{\cal R}}^\dag\,\underline{\underline{\cal Q}})\,\underline{q}\right.\nonumber\\[0.5ex]
&\phantom{=}\left.- \underline{q}^\dag\,(\underline{\underline{\cal S}}^\dag\,\underline{\underline{\cal P}}
- \underline{\underline{\cal Q}}^\dag\,\underline{\underline{\cal R}})\,\underline{p}
+\underline{q}^\dag\,(\underline{\underline{\cal Q}}^\dag\,\underline{\underline{\cal S}}
- \underline{\underline{\cal S}}^\dag\,\underline{\underline{\cal R}})\,\underline{q}
\right]
\end{align}
The previous equation is consistent with Eq.~(\ref{e39}) provided that
\begin{align}\label{e41}
\underline{\underline{\cal P}}^\dag\,\underline{\underline{\cal R}}&= \underline{\underline{\cal R}}^\dag\,\underline{\underline{\cal P}},\\[0.5ex]
\underline{\underline{\cal Q}}^\dag\,\underline{\underline{\cal S}}&= \underline{\underline{\cal S}}^\dag\,\underline{\underline{\cal Q}},\label{e42}\\[0.5ex]
\underline{\underline{\cal P}}^\dag\,\underline{\underline{\cal S}}- \underline{\underline{\cal R}}^\dag\,\underline{\underline{\cal Q}}&=\underline{\underline{1}}.\label{e43}
\end{align}

The previous three equations can be combined with Eqs.~(\ref{e32a}) and (\ref{e33a}) to give 
\begin{align}
\underline{p} &= \underline{\underline{\cal S}}^{\,\dag} \,\underline{V} - \underline{\underline{\cal Q}}^{\,\dag}\,\underline{\psi},\\[0.5ex]
\underline{q} &= -\underline{\underline{\cal R}}^{\,\dag} \,\underline{V} + \underline{\underline{\cal P}}^{\,\dag}\,\underline{\psi}.
\end{align}
However, the previous two equations are only consistent with Eqs.~(\ref{e32a}) and (\ref{e33a}) provided 
\begin{align}\label{e46}
\underline{\underline{\cal Q}}\,\underline{\underline{\cal P}}^\dag&= \underline{\underline{\cal P}}\,\underline{\underline{\cal Q}}^\dag,\\[0.5ex]
\underline{\underline{\cal R}}\,\underline{\underline{\cal S}}^\dag&= \underline{\underline{\cal S}}\,\underline{\underline{\cal R}}^\dag,\label{e47}\\[0.5ex]
\underline{\underline{\cal P}}\,\underline{\underline{\cal S}}^{\dag}- \underline{\underline{\cal Q}}\,\underline{\underline{\cal R}}^{\dag}&=\underline{\underline{1}}.\label{e48}
\end{align}
Note that Eqs.~(\ref{e41})--(\ref{e43}) and (\ref{e46})--(\ref{e48}) hold throughout the vacuum region. 

\subsection{Ideal-Wall Matching Condition}
If the wall is perfectly-conducting then  $\underline{\psi}(b_w\,\epsilon)=0$. 
It follows from Eq.~(\ref{e33a}) that
\begin{equation}
\underline{q} = \underline{\underline{I}}\,\underline{p},
\end{equation}
where
\begin{equation}
 \underline{\underline{I}}=- \underline{\underline{\cal S}}_{\,b}^{-1}\,\underline{\underline{\cal R}}_{\,b}
 \end{equation}
 is termed the wall matrix.
 Here, $\underline{\underline{\cal S}}_{\,b}= \underline{\underline{\cal S}}(b_w\,\epsilon)$, et cetera. Equation~(\ref{e47}) ensures that $ \underline{\underline{I}}$
 is Hermitian. 
  Making use of Eqs.~(\ref{e32a}) and (\ref{e33a}),  the matching condition at the plasma-vacuum interface  for an ideal wall becomes 
 \begin{equation}
 \underline{V}(\epsilon)= \underline{\underline{H}}\,\underline{\psi}(\epsilon),
 \end{equation}
 where 
 \begin{equation}
 \underline{\underline{H}}= (\underline{\underline{\cal P}}_{\,\epsilon}+\underline{\underline{\cal Q}}_{\,\epsilon}\,\underline{\underline{I}})\,(\underline{\underline{\cal R}}_{\,\epsilon}+\underline{\underline{\cal S}}_{\,\epsilon}\,\underline{\underline{I}})^{-1}
 \end{equation}
 is termed the vacuum matrix. 
  Here, $\underline{\underline{\cal P}}_{\,\epsilon}= \underline{\underline{\cal P}}(\epsilon)$, et cetera. 
Making use of Eqs.~(\ref{e41})--(\ref{e43}), it is easily demonstrated that
 \begin{equation}\label{e53}
 \underline{\underline{H}}-\underline{\underline{H}}^\dag =- [(\underline{\underline{\cal R}}_{\,\epsilon}+\underline{\underline{\cal S}}_{\,\epsilon}\,\underline{\underline{I}})^{-1}]^{\dag}\,(\underline{\underline{I}} -\underline{\underline{I}}^\dag)\,  (\underline{\underline{\cal R}}_{\,\epsilon}+\underline{\underline{\cal S}}_{\,\epsilon}\,\underline{\underline{I}})^{-1}
\end{equation}
Thus,  $\underline{\underline{H}}$ is Hermitian because  $\underline{\underline{I}}$ is Hermitian. 
 
\subsection{Model Wall Matrix}
 Equations~(\ref{e25t})--(\ref{e29t}), (\ref{e354}), and (\ref{e355}) suggest that
 \begin{align}
 \underline{\underline{{\cal R}}}_{\,b}& = \underline{\underline{{\cal R}}}_{\,\epsilon} \,\underline{\underline{\rho}}^{\,-1},\\[0.5ex]
 \underline{\underline{{\cal S}}}_{\,b} &= \underline{\underline{{\cal S}}}_{\,\epsilon} \,\underline{\underline{\rho}},
 \end{align}
 where
 \begin{align}
 \rho_{mm'} &= \delta_{mm'}\,\rho_m,\\[0.5ex]
 \rho_0 &= 1+\ln b_w,\\[0.5ex]
 \rho_{m\neq 0} &= b_w^{\,|m|}.
 \end{align}
 Hence,
 \begin{equation}
 \underline{\underline{I}} = - \underline{\underline{\rho}}^{\,-1}\,\underline{\underline{\cal S}}_{\,\epsilon}^{-1}\,\underline{\underline{\cal R}}_{\,\epsilon}\,\underline{\underline{\rho}}^{\,-1}.
 \end{equation}
 Note that $\underline{\underline{I}}$ is Hermitian, given that $\underline{\underline{\cal S}}_{\,\epsilon}^{-1}\,\underline{\underline{\cal R}}_{\,\epsilon}$
 is Hermitian. 
 
\subsection{Resistive-Wall Matching Condition}
 If the wall is resistive then  
 \begin{equation}\label{e55}
\underline{q} =g_w\,\underline{\underline{I}} \,\underline{p},
\end{equation}
where
\begin{align}
g_w &= \frac{f_w}{1+f_w},\\[0.5ex]
f_w &= \frac{\lambda\,\tanh\lambda}{d_w},\\[0.5ex]
\lambda &= (\hat{\gamma}\,d_w)^{1/2},\\[0.5ex]
\hat{\gamma}&= \gamma\,\tau_w,\\[0.5ex]
\tau_w&= \mu_0\,R_0^{\,2}\,\sigma_w\,d_w.
\end{align}
Here, $d_w$ is the wall thickness (normalized to $R_0$), the perturbed magnetic field is assumed to vary in time as $\exp(\gamma\,t)$, $\sigma_w$
is the electrical conductivity of the wall material, and $\tau_w$ is the L/R time of the wall. 

Making use of Eqs.~(\ref{e32a}), (\ref{e33a}), and (\ref{e55}),  the matching condition at the plasma-vacuum interface  for an resistive wall becomes 
\begin{equation}\label{e61}
 \underline{V}(\epsilon)= \underline{\underline{H}}\,\underline{\psi}(\epsilon),
 \end{equation}
 where 
 \begin{equation}\label{e62}
 \underline{\underline{H}}= (\underline{\underline{\cal P}}_{\,\epsilon}+g_w\,\underline{\underline{\cal Q}}_{\,\epsilon}\,\underline{\underline{I}})\,(\underline{\underline{\cal R}}_{\,\epsilon}+g_w\,\underline{\underline{\cal S}}_{\,\epsilon}\,\underline{\underline{I}})^{-1}.
 \end{equation}
 Assuming that $g_w$ is real, and making use of Eqs.~(\ref{e41})--(\ref{e43}),  it is easily demonstrated that
 \begin{equation}
 \underline{\underline{H}}-\underline{\underline{H}}^\dag =-g_w\, [(\underline{\underline{\cal R}}_{\,\epsilon}+\underline{\underline{\cal S}}_{\,\epsilon}\,\underline{\underline{I}})^{-1}]^{\dag}\,(\underline{\underline{I}} -\underline{\underline{I}}^\dag)\,  (\underline{\underline{\cal R}}_{\,\epsilon}+\underline{\underline{\cal S}}_{\,\epsilon}\,\underline{\underline{I}})^{-1}
\end{equation}
Thus, given that $\underline{\underline{I}}$ is Hermitian, we deduce that $\underline{\underline{H}}$ is Hermitian, as long as $g_w$ is real. 
 
\section{Vacuum Matrix}
\subsection{No-Wall Vacuum Matrix}
In the no-wall limit, $g_w=0$, and so Eq.~(\ref{e62}) yields the following expression for the vacuum matrix: 
\begin{equation}\label{e64x}
\underline{\underline{H}}_{\,nw}= \underline{\underline{\cal P}}_{\,\epsilon}\,\underline{\underline{\cal R}}_{\,\epsilon}^{-1}.
\end{equation}
Equation~(\ref{e41}) implies that $\underline{\underline{H}}_{\,nw}$ is Hermitian. 

\subsection{Perfect-Wall Vacuum Matrix}
In the perfect-wall limit, $g_w=1$, and  so Eq.~(\ref{e62}) yields  the following expression for the vacuum matrix: 
\begin{equation}\label{e65x}
\underline{\underline{H}}_{\,pw} = (\underline{\underline{\cal P}}_{\,\epsilon}+\underline{\underline{\cal Q}_{\,\epsilon}}\,\underline{\underline{I}})\,(\underline{\underline{\cal R}}_{\,\epsilon}+\underline{\underline{\cal S}_{\,\epsilon}}\,\underline{\underline{I}})^{-1}.
\end{equation}
Equation~(\ref{e53}) ensures  that $\underline{\underline{H}}_{\,pw}$ is Hermitian, given that that $\underline{\underline{I}}$ is Hermitian.

\subsection{General Vacuum Matrix}
The expression, (\ref{e62}), for the general vacuum matrix can be rewritten in the form 
\begin{equation}\label{e130}
\underline{\underline{H}} = \underline{\underline{H}}_{\,nw} + f_w\,( \underline{\underline{H}}_{\,pw} - \underline{\underline{H}}_{\,nw} )\,(\underline{\underline{B}}+f_w\,\underline{\underline{1}})^{-1},
\end{equation}
where
\begin{equation}
\underline{\underline{B}}= \underline{\underline{\cal R}}_{\,\epsilon}\,(\underline{\underline{\cal R}}_{\,\epsilon} + \underline{\underline{\cal S}}_{\,\epsilon}\,\underline{\underline{I}})^{-1}.
\end{equation}

Suppose that $f_w$ is real. It follows that
\begin{equation}
\underline{\underline{H}} -\underline{\underline{H}}^{\,\dag} = 
f_w\left[( \underline{\underline{H}}_{\,pw} - \underline{\underline{H}}_{\,nw} )\,(\underline{\underline{B}}+f_w\,\underline{\underline{1}})^{-1}- (\underline{\underline{B}}^{\,\dag}+f_w\,
\underline{\underline{1}})^{-1}\,( \underline{\underline{H}}_{\,pw} - \underline{\underline{H}}_{\,nw} )
\right].
\end{equation}
However, we know that $\underline{\underline{H}}$ is Hermitian when $f_w$ is real. Hence, we deduce that
\begin{equation}\label{e133}
 (\underline{\underline{H}}_{\,pw} - \underline{\underline{H}}_{\,nw} )\,(\underline{\underline{B}}+f_w\,\underline{\underline{1}})^{-1}= (\underline{\underline{B}}^{\,\dag}+f_w\,
\underline{\underline{1}})^{-1}\,( \underline{\underline{H}}_{\,pw} - \underline{\underline{H}}_{\,nw} ),
\end{equation}
which implies that
\begin{equation}\label{e122}
\underline{\underline{B}}^{\,\dag}\,( \underline{\underline{H}}_{\,pw} - \underline{\underline{H}}_{\,nw} )= ( \underline{\underline{H}}_{\,pw} - \underline{\underline{H}}_{\,nw} )\,\underline{\underline{B}}.
\end{equation}
The previous equation yields
\begin{align}
&(\underline{\underline{\cal R}}_{\,\epsilon}^{\,\dag}+\underline{\underline{I}}\,\underline{\underline{\cal S}}_{\,\epsilon}^{\,\dag})^{-1}\,\underline{\underline{\cal R}}_{\,\epsilon}^{\,\dag}
\left[(\underline{\underline{\cal P}}_{\,\epsilon}+\underline{\underline{\cal Q}}_{\,\epsilon}\,\underline{\underline{I}})\,(\underline{\underline{\cal R}}_{\,\epsilon}+\underline{\underline{\cal S}}_{\,\epsilon}\,\underline{\underline{I}})^{-1}
 -\underline{\underline{\cal R}}_{\,\epsilon}^{\,-1\,\dag}\,\underline{\underline{P}}_{\,\epsilon}^{\,\dag}\right]\nonumber\\[0.5ex]
 &=
  \left[(\underline{\underline{\cal R}}_{\,\epsilon}^{\,\dag}+
 \underline{\underline{I}}\,\underline{\underline{\cal S}}_{\,\epsilon}^{\,\dag})^{-1}\,(\underline{\underline{\cal P}}_{\,\epsilon}^{\,\dag}
 +\underline{\underline{I}}\,\underline{\underline{\cal Q}}_{\,\epsilon}^{\,\dag})
 -\underline{\underline{\cal P}}_{\,\epsilon}\,\underline{\underline{\cal R}}_{\,\epsilon}^{-1}\right]\underline{\underline{\cal R}}_{\,\epsilon}\,(\underline{\underline{\cal R}}_{\,\epsilon} + \underline{\underline{\cal S}}_{\,\epsilon}\,\underline{\underline{I}})^{-1}.
\end{align}
where use has been made of Eqs.~(\ref{e64x}) and (\ref{e65x}), as well as the fact that $\underline{\underline{H}}_{\,pw}$ and $\underline{\underline{H}}_{\,nw}$ are both Hermitian. 
Making use of Eqs.~(\ref{e41}) and (\ref{e43}), the previous equation reduces to 
\begin{equation}
-\underline{\underline{I}}= -\underline{\underline{I}},
\end{equation}
which is obviously satisfied. Equations~(\ref{e130}) and (\ref{e133}) can
be combined to give
\begin{align}\label{e128x}
\underline{\underline{H}} &= \underline{\underline{H}}_{\,nw} 
+\frac{ f_w}{2}\left[ 
(\underline{\underline{B}}^{\,\dag}+f_w\,\underline{\underline{1}})^{-1}\,( \underline{\underline{H}}_{\,pw} - \underline{\underline{H}}_{\,nw} )
+(\underline{\underline{H}}_{\,pw} - \underline{\underline{H}}_{\,nw} )\,(\underline{\underline{B}}+f_w\,\underline{\underline{1}})^{-1} \right]\nonumber\\[0.5ex]
&= \underline{\underline{H}}_{\,nw} +f_w\,(\underline{\underline{B}}^{\,\dag}+f_w\,\underline{\underline{1}})^{-1}\,( \underline{\underline{H}}_{\,pw} - \underline{\underline{H}}_{\,nw} ),\nonumber\\[0.5ex]
&= \underline{\underline{H}}_{\,nw} +f_w\,(\underline{\underline{H}}_{\,pw} - \underline{\underline{H}}_{\,nw} )\,(\underline{\underline{B}}+f_w\,\underline{\underline{1}})^{-1}.
\end{align}
Note that $\underline{\underline{H}}$ is clearly Hermitian if $f_w$ is real. 

\section{Applications}
\subsection{Wall Torque on Rotating Tearing Mode}
Consider an unreconnected tearing mode, resonant at the $k$th rational surface, and rotating at the angular phase velocity $\omega_k$. It follows that
$\gamma=-{\rm i}\,\omega_k$. Thus, 
\begin{equation}
f_w = \frac{\zeta}{2\,d_w}\left[\left(\frac{\sinh\zeta - \sin\zeta}{\cosh\zeta+\cos\zeta}\right)-{\rm i}\left(\frac{\sinh\zeta+ \sin\zeta}{\cosh\zeta+\cos\zeta}\right)\right],
\end{equation}
where
\begin{equation}
\zeta = (2\,\hat{\omega}_k\,d_w)^{1/2},
\end{equation}
and $\hat{\omega}_k=\omega_k\,\tau_w$. 
The matching condition at the plasma-vacuum interface becomes 
\begin{equation}
\underline{V}(\epsilon)= \underline{\underline{H}}(\zeta)\,\,\underline{\psi}(\epsilon).
\end{equation}
Note that $\underline{\underline{H}}(\zeta)$ is not generally Hermitian, because $f_w$ is complex,  which implies that
the $E$-matrix is not Hermitian. The toroidal electromagnetic torque acting at the $k$th rational surface is
\begin{equation}
\delta T_k = 2\pi^2\,n\,{\rm Im}(E_{kk})\,|{\mit\Psi}_k|^2.
\end{equation}

\subsection{Ideal-Plasma Resistive-Wall Mode}
We can write
\begin{align}\label{e78}
\underline{V}(\epsilon)&= \underline{\underline{V}}_{\,i}\,\underline{\alpha},\\[0.5ex]
\underline{\psi}(\epsilon)&= \underline{\underline{\psi}}_{\,i}\,\underline{\alpha},\label{e79}
\end{align}
where the $\underline{\underline{V}}_{\,i}$ and $\underline{\underline{\psi}}_{\,i}$ are ideal solutions at the plasma-vacuum interface.
The net toroidal electromagnetic torque acting on the plasma is
\begin{equation}
T_\phi= -2\pi^2\,n\,{\rm Im}(\underline{V}^\dag\,\underline{\psi})=-2\pi^2\,n\,{\rm Im}(\underline{\alpha}^\dag\,\underline{\underline{V}}_{\,i}^\dag
\,\underline{\underline{\psi}}_{\,i}\,\underline{\alpha}).
\end{equation}
However, the net torque acting on an ideal plasma is zero, so
\begin{equation}\label{e121}
\underline{\underline{V}}_{\,i}^\dag \,\underline{\underline{\psi}}_{\,i}=\underline{\underline{\psi}}_{\,i}^\dag \,\underline{\underline{V}}_{\,i}.
\end{equation}

Equations~(\ref{e61}), (\ref{e78}), and (\ref{e79}) imply that 
\begin{equation}
\underline{\underline{V}}_{\,i}\,\underline{\alpha}=\underline{\underline{H}}\,\underline{\underline{\psi}}_{\,i}\,\underline{\alpha}.
\end{equation}
Writing
\begin{equation}
\underline{\underline{\psi}}_{\,i}\,\underline{\alpha} = \underline{x},
\end{equation}
we obtain
\begin{equation}
\underline{\underline{W}}_{\,p}\,\underline{x}
= \underline{\underline{H}}\,\underline{x},
\end{equation}
where
\begin{equation}\label{e128}
\underline{\underline{W}}_{\,p} = \underline{\underline{V}}_{\,i}\,\underline{\underline{\psi}}_{\,i}^{\,-1}.
\end{equation}
Equation~(\ref{e121}) ensures that $\underline{\underline{W}}_{\,p}$ is Hermitian. 

Equation~(\ref{e128x}) can be combined with the previous equation to give
\begin{align}
\underline{\underline{W}}_{\,nw}\,\underline{x} &= f_w\,(\underline{\underline{B}}^{\,\dag}+f_w\,\underline{\underline{1}})^{-1}\,( \underline{\underline{W}}_{\,nw} - \underline{\underline{W}}_{\,pw} )\,\underline{x}\nonumber\\[0.5ex]
&=f_w\,(\underline{\underline{W}}_{\,nw} - \underline{\underline{W}}_{\,pw} )\,(\underline{\underline{B}}+f_w\,\underline{\underline{1}})^{-1}\,\underline{x}
\end{align}
where
\begin{align}
\underline{\underline{W}}_{\,nw}  &= \underline{\underline{W}}_{\,p}-\underline{\underline{H}}_{\,nw},\\[0.5ex]
\underline{\underline{W}}_{\,pw}  &= \underline{\underline{W}}_{\,p}-\underline{\underline{H}}_{\,pw}.
\end{align}
Note that $\underline{\underline{W}}_{\,nw}$ and $\underline{\underline{W}}_{\,pw}$ are Hermitian.
Writing
\begin{equation}
(\underline{\underline{B}}+f_w\,\underline{\underline{1}})^{-1} \,\underline{x}= \underline{y},
\end{equation}
we obtain
\begin{align}\label{e141}
\left(f_w\,\underline{\underline{W}}_{\,pw}
+\underline{\underline{B}}^{\,\dag}\,\underline{\underline{W}}_{\,nw}\right)\underline{x}&=\underline{0},\\[0.5ex]
\left(f_w\,\underline{\underline{W}}_{\,pw}
+\underline{\underline{W}}_{\,nw}\,\underline{\underline{B}}\right)\underline{y} &=\underline{0}.\label{e142}
\end{align}
The previous two equations can be combined to give
\begin{equation}
(f_w-f_w^{\,\ast})\,\underline{y}^{\,\dag}\,\underline{\underline{W}}_{\,pw}\,\underline{x} = 0.
\end{equation}
Hence, we deduce that the eigenvalues, $f_w$,  are real, and that the eigenvectors, $\underline{x}$ and $\underline{y}$,  are orthonormal, in the sense that
\begin{equation}
\underline{y}_i^{\,\dag}\,
\underline{\underline{W}}_{\,pw}\,\underline{x}_j= \delta_{ij}.
\end{equation} 
Once we have determined the eigenvectors then
\begin{align}
\underline{\psi}(\epsilon) &=\underline{x},\\[0.5ex]
\underline{V}(\epsilon) &= \underline{\underline{W}}_{\,p} \,\underline{x}.
\end{align}

We can write
\begin{align}
\underline{\psi} &= \underline{\underline{Q}}\,\underline{\mit\Xi},\\[0.5ex]
\underline{Z} &= \underline{\underline{Q}}\,\underline{V},
\end{align}
where $\underline{\underline{Q}}$ is the diagonal matrix of the $m-n\,q(\epsilon)$ values. 
Let
\begin{align}
\underline{\underline{\widetilde{W}}}_{\,nw} &= \underline{\underline{Q}}\,\underline{\underline{W}}_{\,nw}\,\underline{\underline{Q}},\\[0.5ex]
\underline{\underline{\widetilde{W}}}_{\,pw} &= \underline{\underline{Q}}\,\underline{\underline{W}}_{\,pw}\,\underline{\underline{Q}},\\[0.5ex]
\underline{\underline{\widetilde{B}}} &= \underline{\underline{Q}}^{-1}\,\underline{\underline{B}}\,\underline{\underline{Q}},\\[0.5ex]
\underline{\widetilde{x}}&=  \underline{\underline{Q}}^{-1}\,\underline{x},\\[0.5ex]
\underline{\widetilde{y}}&=  \underline{\underline{Q}}^{-1}\,\underline{x}.
\end{align}
It follows that 
\begin{align}
\left(f_w\,\underline{\underline{\widetilde{W}}}_{\,pw}
+\underline{\underline{\widetilde{B}}}^{\,\dag}\,\underline{\underline{\widetilde{W}}}_{\,nw}\right)\underline{\widetilde{x}}&=\underline{0},\\[0.5ex]
\left(f_w\,\underline{\underline{\widetilde{W}}}_{\,pw}
+\underline{\underline{\widetilde{W}}}_{\,nw}\,\underline{\underline{\widetilde{B}}}\right)\underline{\widetilde{y}} &=\underline{0}.\label{e142a}
\end{align}
Hence, we again deduce that 
\begin{equation}
(f_w-f_w^{\,\ast})\,\underline{\widetilde{y}}^{\,\dag}\,\underline{\underline{\widetilde{W}}}_{\,pw}\,\underline{\widetilde{x}} = 0.
\end{equation}
Finally, 
\begin{align}
\underline{\mit\Xi} &=\underline{\widetilde{x}},\\[0.5ex]
\underline{Z} &= \underline{\underline{\widetilde{W}}}_{\,p} \,\underline{\widetilde{x}}.
\end{align}

\subsection{Resistive-Plasma Resistive-Wall Mode}
The $m'$th ideal eigenfunction has the form
\begin{align}
\psi^i_{mm'}(r) &= \psi^a_{mm'}(r) - \sum_{k'}\psi^u_{mk'}(r)\,{\mit\Pi}_{k'm'}^a,\\[0.5ex]
V^i_{mm'}(r) &=  V^a_{mm'}(r) - \sum_{k'} V^u_{mk'}(r)\,{\mit\Pi}_{k'm'}^a.
\end{align}
Now, the $\psi^a_{mm'}(r)$ and the $V^a_{mm'}(r)$ have no current sheets at the various rational surfaces in the plasma. 
On the other hand, the $\psi^u_{mk'}(r)$ and  the $V^u_{mk'}(r)$ are such that ${\mit\Delta\Psi}_{k}= E_{kk'}$. Hence, the current sheet generated at the $k$th rational surface the plasma by the $m'$th ideal eigenfunction is
\begin{equation}
{\mit\Delta\Psi}_{km'} = -\sum_{k'}E_{kk'}\,{\mit\Pi}_{k'm}^a.
\end{equation}
Thus, the reconnected flux driven at the $k$th rational surface is
\begin{equation}
{\mit\Psi}_{km'}= -\sum_{k'}\frac{E_{kk'}\,{\mit\Pi}_{k'm}^a}{{\mit\Delta}_k(0)},
\end{equation}
where the layer response parameter is evaluated at zero frequency, on the assumption that the resistive wall mode evolves in time very slowly. 
To take into account the reconnected flux in the plasma, we write
\begin{align}
\psi_{mm'}(r) &=\psi^i_{mm'}(r) + \sum_k \psi^u_{mk}(r)\,{\mit\Psi}_{km'},\\[0.5ex]
V_{mm'}(r) &=  V^i_{mm'}(r) + \sum_k V^u_{mk}(r)\,{\mit\Psi}_{km'},
\end{align}
which implies that
\begin{align}
\psi_{mm'}(r) &= \psi^a_{mm'}(r) - \sum_{k,k'}\psi^u_{mk}(r)\left[\delta_{kk'}+ \frac{E_{kk'}}{{\mit\Delta}_k(0)}\right]{\mit\Pi}_{k'm'}^a,\\[0.5ex]
V_{mm'}(r) &=  V^a_{mm'}(r) - \sum_{k,k'} V^u_{mk}(r)\left[\delta_{kk'}+ \frac{E_{kk'}}{{\mit\Delta}_k(0)}\right]{\mit\Pi}_{k'm'}^a.
\end{align}

Let $\underline{\underline{\psi}}$ be the matrix of the $\psi_{mm'}(\epsilon)$ values and let $\underline{\underline{V}}$ be the matrix of the
$V_{mm'}(\epsilon)$ values. The plasma energy matrix is
\begin{equation}
\underline{\underline{W}}_{\,p} = \underline{\underline{V}}\,\underline{\underline{\psi}}^{-1}.
\end{equation}
However, the energy matrix is not necessarily  Hermitian. The net toroidal electromagnetic torque acting on the plasma is
\begin{equation}
T_\phi = -2\pi^2\,n\,{\rm Im}(\underline{x}^\dag\,\underline{\underline{W}}_p\,\underline{x}).
\end{equation}
Given that $\underline{\underline{W}}_p$ is not necessarily Hermitian, the torque is not necessarily zero. 
As before, the resistive wall mode dispersion relation is 
\begin{align}
\left(f_w\,\underline{\underline{W}}_{\,pw}
+\underline{\underline{B}}^{\,\dag}\,\underline{\underline{W}}_{\,nw}\right)\underline{x}&=\underline{0},\\[0.5ex]
\left(f_w\,\underline{\underline{W}}_{\,pw}
+\underline{\underline{W}}_{\,nw}\,\underline{\underline{B}}\right)\underline{y} &=\underline{0}.
\end{align}
However, $\underline{\underline{W}}_{\,nw}$ and $\underline{\underline{W}}_{\,pw}$ are not necessarily  Hermitian. Hence, $f_w$ is not necessarily real. 


\end{document}