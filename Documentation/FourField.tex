\documentclass[12pt,prb,aps,notitlepage]{revtex4-1}
\usepackage {amsmath}
\usepackage{amssymb}
\usepackage{color}
\pdfoutput = 1 
\usepackage {graphicx}
\newcommand{\bomega}{\mbox{\boldmath$\omega$}}
\allowdisplaybreaks

\begin{document}

\title{Four-Field Resonant Layer Model}
\author{R.~Fitzpatrick\,\footnote{rfitzp@utexas.edu}}
\affiliation{Institute for Fusion Studies,  Department of Physics,  University of Texas at Austin,  Austin TX 78712, USA}\begin{abstract}
\end{abstract}
\maketitle

\section{Fourier-Transformed Four-Field Equations}
In real space, the four-field resonant layer equations can be reduced to a set of ten coupled first-order differential equations.
As we shall demonstrate, the equations can be reduced to a set of four coupled first-order differential
equations in Fourier space. Clearly, it is advantageous to solve the equations in Fourier space.

The  Fourier-transformed four-field layer equations take the form:
\begin{align}
(g+{\rm i}\,Q_e)\,\bar{\psi} &= \frac{d(\bar{\phi}-\bar{N})}{dp} - p^2\,\bar{\psi},\\[0.5ex]
g\,\bar{N} &= -{\rm i}\,Q_e\,\bar{\phi} - D^{\,2}\,\frac{d(p^2\,\bar{\psi})}{dp} + c_\beta^{\,2}\,\frac{d\bar{V}}{dp} - P_\perp\,p^2\,\bar{N},\\[0.5ex]
(g+ {\rm i}\,Q_i)\,p^2\,\bar{\phi} &= \frac{d(p^2\,\bar{\psi})}{dp} - P_\varphi\,p^4\left(\bar{\phi}+ \frac{\bar{N}}{\iota}\right),\\[0.5ex]
g\,\bar{V} &= {\rm i}\,Q_e\,\bar{\psi} + \frac{d\bar{N}}{dp} - P_\varphi\,p^2\,\bar{V}.\label{eq4}
\end{align}
It follows that
\begin{equation}
\bar{\psi} = \frac{1}{g+{\rm i}\,Q_e+p^2}\,\frac{d(\bar{\phi}-\bar{N})}{dp},
\end{equation}
and
\begin{equation}
\frac{d(p^2\,\bar{\psi})}{dp} = [(g+{\rm i}\,Q_i)\,p^2 + P_\varphi\,p^4]\,\bar{\phi} + \frac{P_\varphi}{\iota}\,p^4\,\bar{N},
\end{equation}
and
\begin{equation}
c_\beta^{\,2}\,\frac{d\bar{V}}{dp} = (g + P_\perp\,p^2+ \iota^{-1}\,D^{\,2}\,P_\varphi\,p^4)\,\bar{N}+ [
{\rm i}\,Q_e + D^{\,2}\,(g+{\rm i}\,Q_i)\,p^2 + D^{\,2}\,P_\varphi\,p^4]\,\bar{\phi}.\label{eq7}
\end{equation}
If we define 
\begin{align}
\bar{J}&= p^2\,\bar{\psi},\\[0.5ex]
\bar{Y} &= \bar{\phi}-\bar{N},
\end{align}
then we can transform Eqs.~(\ref{eq4})--(\ref{eq7}) into the  the following set of four coupled first-order differential equations:
\begin{align}\label{e1}
\frac{d\bar{Y}}{dp} &= \left(\frac{g+{\rm i}\,Q_e+p^2}{p^2}\right)\bar{J},\\[0.5ex]
\frac{d\bar{N}}{dp} &= \left(\frac{-{\rm i}\,Q_e}{p^2}\right)\bar{J} + (g + P_\varphi\,p^2)\,\bar{V},\\[0.5ex]
\frac{d\bar{J}}{dp} &= [(g+{\rm i}\,Q_i)\,p^2 + P_\varphi\,p^4]\,\bar{Y}
+ [(g+{\rm i}\,Q_i)\,p^2 + \iota_e^{\,-1}\,P_\varphi\,p^4]\,\bar{N},\\[0.5ex]
c_\beta^{\,2}\,\frac{d\bar{V}}{dp} &= [{\rm i}\,Q_e+ D^{\,2}\,(g+{\rm i}\,Q_i)\,p^2 + D^{\,2}\,P_\varphi\,p^4]\,\bar{Y}
\nonumber\\[0.5ex]&\phantom{=} +(g+{\rm i}\,Q_e + [P_\perp+ D^{\,2}\,(g + {\rm i}\,Q_i)]\,p^2 + \iota_e^{\,-1}\,D^{\,2}\,P_\varphi\,p^4)\,\bar{N},\label{e4}
\end{align}
where $\iota_e=\iota/(1+\iota)$. 

\section{Small-$p$ Behavior of Fourier-Transformed Four-Field Equations}
\subsection{Introduction}
Let us search for power-law solutions of Eqs.~(\ref{e1})--(\ref{e4}) at small values of $p$. Given that we have four coupled first-order differential equations,
we expect to find four independent power-law solutions.

\subsection{First Solution}
Suppose that 
\begin{align}
\bar{Y}(p) &=y_{-1}\,p^{-1} + y_1\,p + {\cal O}(p^3),\\[0.5ex]
\bar{N}(p) &= n_{-1}\,p^{-1}+n_1\,p + {\cal O}(p^3),\\[0.5ex]
\bar{J}(p) &= j_0 + j_2\,p^2+ {\cal O}(p^4),\\[0.5ex]
\bar{V}(p) &= v_2\,p^2+ {\cal O}(p^4).
\end{align}
Equations~(\ref{e1})--(\ref{e4}) yield 
\begin{align}
-y_{-1}\,p^{-2} + y_1 &= (g+{\rm i}\,Q_e)\,(j_0\,p^{-2}+j_2) + j_0 + {\cal O}(p^2),\\[0.5ex]
-n_{-1}\,p^{-2} + n_1 &= -{\rm i}\,Q_e\,(j_0\,p^{-2}+j_2)+ {\cal O}(p^2),\\[0.5ex]
2\,j_2\,p &= (g+{\rm i}\,Q_i)\,(y_{-1}+n_{-1})\,p+ {\cal O}(p^3),\\[0.5ex]
2\,c_\beta^{\,2}\,v_2\,p &={\rm i}\,Q_e\,(y_{-1}\,p^{-1} +y_1\,p) + (g+{\rm i}\,Q_e)\,(n_{-1}\,p^{-1}+ n_1\,p)\nonumber\\[0.5ex]&\phantom{=}
+ D^{\,2}\,(g+{\rm i}\,Q_i)\,y_{-1}\,p + [P_\perp + D^{\,2}\,(g+{\rm i}\,Q_i)]\,n_{-1}\,p+ {\cal O}(p^3).
\end{align}
It follows that
\begin{align}
-y_{-1}&= (g+{\rm i}\,Q_e)\,j_0,\\[0.5ex]
y_1 &= (g+{\rm i}\,Q_e)\,j_2 + j_0,\\[0.5ex]
-n_{-1}&= -{\rm i}\,Q_e\,j_0,\\[0.5ex]
n_1 &= -{\rm i}\,Q_e\,j_2,\\[0.5ex]
2\,j_2 &= (g+{\rm i}\,Q_i)\,(y_{-1}+n_{-1}),\\[0.5ex]
0 &= {\rm i}\,Q_e\,y_{-1} + (g+{\rm i}\,Q_e)\,n_{-1},\\[0.5ex]
2\,c_\beta^{\,2}\,v_2 &= {\rm i}\,Q_e\,y_1 + (g+{\rm i}\,Q_e)\,n_1\nonumber\\[0.5ex]
&\phantom{=} + D^{\,2}\,(g+{\rm i}\,Q_i)\,y_{-1} + [P_\perp+ D^{\,2}\,(g+{\rm i}\,Q_i)]\,n_{-1},
\end{align}
which gives
\begin{align}
y_{-1} &= (g+{\rm i}\,Q_e)\,a_{-1},\label{ym1}\\[0.5ex]
y_1 &=\left[\frac{1}{2}\,g\,(g+{\rm i}\,Q_e)\,(g+{\rm i}\,Q_i) - 1\right]a_{-1},\\[0.5ex]
n_{-1}&= -{\rm i}\,Q_e\,a_{-1},\label{nm1}\\[0.5ex]
n_1&= - \frac{1}{2}\,g\,({\rm i}\,Q_e)\,(g+{\rm i}\,Q_i)\,a_{-1},\\[0.5ex]
j_0 &= -a_{-1},\\[0.5ex]
j_2&= \frac{1}{2}\,g\,(g+{\rm i}\,Q_i)\,a_{-1},\\[0.5ex]
v_2 &= \frac{[-{\rm i}\,Q_e\,(1+P_\perp) + g\,(g+{\rm i}\,Q_i)\,D^{\,2}]}{2\,c_\beta^{\,2}}\,a_{-1},
\end{align}
where $a_{-1}$ is an arbitrary constant. 

\subsection{Second Solution}
Suppose that 
\begin{align}
\bar{Y}(p) &=y_{0}+ y_2\,p^2+{ \cal O}(p^4),\\[0.5ex]
\bar{N}(p) &= n_{0}+n_2\,p^2+{ \cal O}(p^4),\\[0.5ex]
\bar{J}(p) &= j_3\,p^3+{ \cal O}(p^5),\\[0.5ex]
\bar{V}(p) &= v_3\,p^3+{ \cal O}(p^5).
\end{align}
Equations~(\ref{e1})--(\ref{e4}) give 
\begin{align}
2\, y_2\,p &= (g+{\rm i}\,Q_e)\,j_3\,p + {\cal O}(p^3),\\[0.5ex]
2\, n_2\,p &= -{\rm i}\,Q_e\,j_3\,p+ {\cal O}(p^3),\\[0.5ex]
3\,j_3\,p^2 &= (g+{\rm i}\,Q_i)\,(y_0+n_0)\,p^2+ {\cal O}(p^4),\\[0.5ex]
3\,c_\beta^{\,2}\,v_3\,p^2 &={\rm i}\,Q_e\,(y_{0} +y_2\,p^2) + (g+{\rm i}\,Q_e)\,(n_0+ n_2\,p^2)\nonumber\\[0.5ex]&\phantom{=}
+ D^{\,2}\,(g+{\rm i}\,Q_i)\,y_0\,p^2 + [P_\perp + D^{\,2}\,(g+{\rm i}\,Q_i)]\,n_0\,p^2+ {\cal O}(p^4).
\end{align}
It follows that
\begin{align}
2\,y_2&= (g+{\rm i}\,Q_e)\,j_3,\\[0.5ex]
2\,n_2&= -{\rm i}\,Q_e\,j_3,\\[0.5ex]
3\,j_3&= (g+{\rm i}\,Q_i)\,y_0 + (g+{\rm i}\,Q_i)\,n_0,\\[0.5ex]
0 &= {\rm i}\,Q_e\,y_0 + (g+{\rm i}\,Q_e)\,n_0,\\[0.5ex]
3\,c_\beta^{\,2}\,v_3 &= {\rm i}\,Q_e\,y_2 + (g+{\rm i}\,Q_e)\,n_2 \nonumber\\[0.5ex]
&\phantom{=}+D^{\,2}\,(g+{\rm i}\,Q_e)\,n_0 + [P_\perp + D^{\,2}\,(g+{\rm i}\,Q_i)]\,n_0.
\end{align}
which gives
\begin{align}
y_0 &= (g+{\rm i}\,Q_e)\,a_0,\\[0.5ex]
y_2 &= \frac{1}{6}\,g\,(g+{\rm i}\,Q_e)\,(g+{\rm i}\,Q_i)\,a_0,\\[0.5ex]
n_0 &= -{\rm i}\,Q_e\,a_0,\\[0.5ex]
n_2 &= -\frac{1}{6}\,g\,({\rm i}\,Q_e)\,(g+{\rm i}\,Q_i)\,a_0,\\[0.5ex]
j_3 &= \frac{1}{3}\,g\,(g+{\rm i}\,Q_i)\,a_0,\\[0.5ex]
v_3 &= \frac{1}{3}\,\frac{[-{\rm i}\,Q_e\,P_\perp + g\,(g+{\rm i }\,Q_i)\,D^{\,2}]}{c_\beta^{\,2}}\,a_0,
\end{align}
where $a_0$ is an arbitrary constant. 

\subsection{Third Solution}
Suppose that
\begin{align}
\bar{Y}(p) &= y_2\,p^2 + {\cal O}(p^4),\\[0.5ex]
\bar{N}(p) &=n_0+n_2\,p^2 + {\cal O}(p^4),\\[0.5ex]
\bar{J}(p) &= j_3\,p^3 + {\cal O}(p^5),\\[0.5ex]
\bar{V} (p)&= v_1\,p +{\cal O}(p^3).
\end{align}
Equations~(\ref{e1})--(\ref{e4}) give 
\begin{align}
2\,y_2\,p &= (g+{\rm i}\,Q_e)\,j_3\,p + {\cal O}(p^3),\\[0.5ex]
2\,n_2\,p& = -{\rm i}\,Q_e\,j_3\,p + g\,v_1\,p +{\cal O}(p^3),\\[0.5ex]
3\,j_3 \,p^2&= (g+{\rm i}\,Q_i)\,n_0\,p^2+{\cal O}(p^4),\\[0.5ex]
c_\beta^{\,2}\,v_1&= (g+{\rm i}\,Q_e)\,n_0 +{ \cal O}(p^2).
\end{align}
It follows that 
\begin{align}
y_2 &= \frac{1}{6}\,(g+{\rm i}\,Q_e)\,(g+{\rm i}\,Q_i)\,a_2,\\[0.5ex]
n_0 &= a_2,\\[0.5ex]
n_2 &= \frac{1}{2}\,(g+{\rm i}\,Q_i)\left(-\frac{1}{3}\,{\rm i}\,Q_e\,+ \frac{g}{c_\beta^{\,2}}\right)a_2,\\[0.5ex]
j_3 &=\frac{1}{3}\,(g+{\rm i}\,Q_i)\,a_2,\\[0.5ex]
v_1 &=\frac{(g+{\rm i}\,Q_e)}{c_\beta^{\,2}}\,a_2,
\end{align}
where $a_2$ is an arbitrary constant. 

\subsection{Fourth Solution}
Suppose that
\begin{align}
\bar{Y}(p) &= y_3\,p^3 +{ \cal O}(p^5),\\[0.5ex]
\bar{N}(p) &=n_1\,p + {\cal O}(p^3),\\[0.5ex]
\bar{J}(p) &= j_4\,p^4 + {\cal O}(p^6),\\[0.5ex]
\bar{V}(p) &= v_0 + v_2\,p^2 +{ \cal O}(p^4).
\end{align}
Equations~(\ref{e1})--(\ref{e4}) give 
\begin{align}
3\,y_3 \,p^2&= (g+{\rm i}\,Q_e)\,j_4\,p^2 + {\cal O}(p^4),\\[0.5ex]
n_1&= g\,v_0 +{\cal O}(p^2),\\[0.5ex]
4\,j_4\,p^3 &= (g+{\rm i}\,Q_i)\,n_1\,p^3 + {\cal O}(p^5),\\[0.5ex]
2\,c_\beta^{\,2}\,v_2\,p& = (g+{\rm i}\,Q_e)\,n_1\,p + {\cal O}(p^3).
\end{align}
It follows that 
\begin{align}
y_3& = \frac{1}{12}\,g\,(g+{\rm i}\,Q_e)\,(g+{\rm i}\,Q_i)\,a_3,\\[0.5ex]
n_1 &= g\,a_3,\\[0.5ex]
j_4 &= \frac{1}{4}\,g\,(g+{\rm i}\,Q_i)\,a_3,\\[0.5ex]
v_2 &= \frac{g\,(g+{\rm i}\,Q_e)}{2\,c_\beta^{\,2}}\,a_3,
\end{align}
where $a_3$ is an arbitrary constant. 

\subsection{General Solution}
We conclude that, at small values of $p$, the most general solution for $\bar{Y}(p)$ and $\bar{N}(p)$ takes the form 
\begin{align}\label{yy}
\bar{Y}(p) &= (g+{\rm i}\,Q_e)\,p^{-1}\,a_{-1} +(g+{\rm i}\,Q_e) \,a_0+ {\cal O}(p),\\[0.5ex]
\bar{N}(p) &= (-{\rm i}\,Q_e)\,p^{-1}\,a_{-1} + (-{\rm i}\,Q_e)\,a_0 + a_2 + {\cal O}(p).\label{nn}
\end{align}

\section{Matrix Differential Equation}
Let
\begin{align}
\underline{u}= \left(\begin{array}{c}\bar{Y}\\\bar{N}\end{array}\right),\\[0.5ex]
\underline{v}= \left(\begin{array}{c}\bar{J}\\c_\beta^{\,2}\,\bar{V}\end{array}\right).
\end{align}
Equations~(\ref{e1})--(\ref{e4}) can be written in the form 
\begin{align}
\frac{d\underline{u}}{dp}= \underline{\underline{A}}\,\underline{v},\\[0.5ex]
\frac{d\underline{v}}{dp}= \underline{\underline{B}}\,\underline{u},
\end{align}
where
\begin{align}
A_{11} &=  \frac{g+{\rm i}\,Q_e+p^2}{p^2},\\[0.5ex]
A_{21} &= \frac{-{\rm i}\,Q_e}{p^2},\\[0.5ex]
A_{22} &= \frac{g + P_\varphi\,p^2}{c_\beta^{\,2}},\\[0.5ex]
B_{11} &= (g+{\rm i}\,Q_i)\,p^2 + P_\varphi\,p^4,\\[0.5ex]
B_{12} &= (g+{\rm i}\,Q_i)\,p^2 + \iota_e^{\,-1}\,P_\varphi\,p^4,\\[0.5ex]
B_{21}&= {\rm i}\,Q_e+ D^{\,2}\,(g+{\rm i}\,Q_i)\,p^2 + D^{\,2}\,P_\varphi\,p^4,\\[0.5ex]
B_{22} & =g+{\rm i}\,Q_e + [P_\perp + D^{\,2}\,(g + {\rm i}\,Q_i)]\,p^2 + \iota_e^{\,-1}\,D^{\,2}\,P_\varphi\,p^4.
\end{align}
Thus, we obtain the following matrix differential equation: 
\begin{equation}\label{mat}
\frac{d}{dp}\!\left(\underline{\underline{A}}^{-1}\,\frac{d\underline{u}}{dp}\right) = \underline{\underline{B}}\,\underline{u}.
\end{equation}

\section{Riccati Matrix Differential Equation}
Let
\begin{equation}\label{wdef}
 p\,\frac{d\underline{u}}{dp}=\underline{\underline{W}}\,\underline{u}.
\end{equation}
The previous equation can be combined with Eq.~(\ref{mat}) to give 
\begin{equation}
\left(p\,\frac{d\underline{\underline{W}}}{dp} - \underline{\underline{W}} 
+ \underline{\underline{W}}\,\underline{\underline{W}} + \underline{\underline{A}}\,p\,\frac{d\underline{\underline{A}}^{-1}}{dp}\,\underline{\underline{W}}- p^2\,\underline{\underline{A}}\,\underline{\underline{B}}\right)\underline{u} = \underline{0},
\end{equation}
which implies that 
\begin{equation}\label{eq96}
p\,\frac{d\underline{\underline{W}}}{dp} = \underline{\underline{W}} - \underline{\underline{W}}\,\underline{\underline{W}} - \underline{\underline{A}}\,p\,\frac{d\underline{\underline{A}}^{-1}}{dp}\,\underline{\underline{W}}
+ p^2\,\underline{\underline{A}}\,\underline{\underline{B}}.
\end{equation}

Now,
\begin{equation}
\underline{\underline{A}}^{-1}= \left(\begin{array}{cc} C_{11}&0\\C_{21}&C_{22}\end{array}\right),
\end{equation}
where
\begin{align}
C_{11} &= \frac{p^2}{g+{\rm i}\,Q_e+p^2},\\[0.5ex]
C_{21}&=\frac{{\rm i}\,c_\beta^{\,2}\,Q_e}{(g+{\rm i}\,Q_e+p^2)\,(g+P_\varphi\,p^2)},\\[0.5ex]
C_{22} &= \frac{c_\beta^{\,2}}{g+P_\varphi\,p^2}.
\end{align}
So, if
\begin{equation}
p\,\frac{d\underline{\underline{A}}^{-1}}{dp}= \left(\begin{array}{cc} D_{11}&0\\D_{21}&D_{22}\end{array}\right)
\end{equation}
then 
\begin{align}
D_{11} &= \frac{2\,p^2\,(g+{\rm i}\,Q_e)}{(g+{\rm i}\,Q_e+p^2)^2},\\[0.5ex]
D_{21} &= - \frac{2\,{\rm i}\,c_\beta^{\,2}\,Q_e\,p^2\,[g+ P_\varphi\,(g+{\rm i}\,Q_e) + 2\,P_\varphi\,p^2]}{(g+{\rm i}\,Q_e+p^2)^2\,(g+P_\varphi\,p^2)^2 },\\[0.5ex]
D_{22} &= -\frac{2\,c_\beta^{\,2}\,P_\varphi\,p^2}{(g+P_\varphi\,p^2)^2}.
\end{align}
Furthermore, if
\begin{equation}\label{ee1}
\underline{\underline{A}}\,p\,\frac{d\underline{\underline{A}}^{-1}}{dp}= \left(\begin{array}{cc} E_{11}&0\\E_{21}&E_{22}\end{array}\right)
\end{equation}
then 
\begin{align}
E_{11} &= \frac{2\,(g+{\rm i}\,Q_e)}{g+ {\rm i}\,Q_e+p^2},\\[0.5ex]
E_{21} &=-\frac{2\,{\rm i}\,Q_e\,(g+2\,P_\varphi\,p^2)}{(g+{\rm i}\,Q_e+p^2)\,(g+P_\varphi\,p^2)},\\[0.5ex]
E_{22} &= -\frac{2\,P_\varphi\,p^2}{g+ P_\varphi\,p^2}.\label{ee4}
\end{align}
Finally,  if
\begin{equation}
p^2\,\underline{\underline{A}}\,\underline{\underline{B}}= \left(\begin{array}{cc} F_{11}&F_{12}\\F_{21}&F_{22}\end{array}\right)
\end{equation}
then
\begin{align}\label{f1}
F_{11} &= p^2\,(g+{\rm i}\,Q_e+p^2)\,(g+{\rm i}\,Q_i + P_\varphi\,p^2),\\[0.5ex]
F_{12} &= p^2\,(g+{\rm i}\,Q_e+p^2)\,(g+{\rm i}\,Q_i + \iota_e^{\,-1}\,P_\varphi\,p^2),\\[0.5ex]
F_{21} &=-{\rm i}\,Q_e\,p^2\,(g+{\rm i}\,Q_i+P_\varphi\,p^2)\nonumber\\[0.5ex]
&\phantom{=} + c_\beta^{\,-2}\,p^2\,(g+P_\varphi\,p^2)\,[\,{\rm i}\,Q_e+ D^{\,2}\,(g+{\rm i}\,Q_i)\,p^2 + D^{\,2}\,P_\varphi\,p^4],\\[0.5ex]
F_{22} &=-{\rm i}\,Q_e\,p^2\,(g+{\rm i}\,Q_i+\iota_e^{\,-1}\,P_\varphi\,p^2)\nonumber\\[0.5ex] 
&\phantom{=}+c_\beta^{\,-2}\,p^2\,(g+P_\varphi\,p^2)\,[g+{\rm i}\,Q_e + [P_\perp +D^{\,2}\,(g + {\rm i}\,Q_i)]\,p^2 + \iota_e^{\,-1}\,D^{\,2}\,P_\varphi\,p^4].\label{f4}
\end{align}

Hence, Eq.~(\ref{eq96}) can be written as the  following Riccati matrix differential equation:  
\begin{equation}\label{ricc}
p\,\frac{d\underline{\underline{W}}}{dp} = \underline{\underline{W}} - \underline{\underline{W}}\,\underline{\underline{W}} - \underline{\underline{E}}\,\underline{\underline{W}}
+\underline{\underline{F}}.
\end{equation}
Furthermore, if 
\begin{equation}
\underline{\underline{W}}= \left(\begin{array}{cc} W_{11}&W_{12}\\W_{21}&W_{22}\end{array}\right)
\end{equation}
then
\begin{align}
p\,\frac{dW_{11}}{dp}  &= W_{11}  - W_{11}\,W_{11}-W_{12}\,W_{21}- E_{11}\,W_{11} + F_{11},\\[0.5ex]
p\,\frac{dW_{12}}{dp} &= W_{12} - W_{11}\,W_{12} - W_{12}\,W_{22} - E_{11}\,W_{12} + F_{12},\\[0.5ex]
p\,\frac{dW_{21}}{dp} &= W_{21} -W_{21}\,W_{11}- W_{22}\,W_{21} - E_{21}\,W_{11} - E_{22}\,W_{21} + F_{21},\\[0.5ex]
p\,\frac{dW_{22}}{dp} &= W_{22} -W_{21}\,W_{12}- W_{22}\,W_{22}- E_{21}\,W_{12} - E_{22}\,W_{22} + F_{22}.
\end{align}

\section{Small-$p$ Behavior of Riccati Matrix Differential Equation}
Let $\underline{\underline{E}} = \underline{\underline{E}}^{\,(0)}$ at $p=0$.
It follows from Eqs.~(\ref{ee1})--(\ref{ee4})  that
\begin{align}
E_{11}^{\,(0)} &= 2,\\[0.5ex]
E_{12}^{\,(0)}&=0,\\[0.5ex]
E_{21}^{\,(0)} &= - \frac{2\,{\rm i}\,Q_e}{g+{\rm i}\,Q_e},\label{e210}\\[0.5ex]
E_{22}^{\,(0)}&= 0.
\end{align}
Likewise, at small values of $p$, we can write $\underline{\underline{F}} =p^2\, \underline{\underline{F}}^{\,(2)}$,
where the elements of  $\underline{\underline{F}}^{\,(2)}$ are constants, and where use has been made of Eqs.~(\ref{f1})--(\ref{f4}). 

Suppose that  $\underline{\underline{W}}= \underline{\underline{W}}^{\,(0)}$ at $p=0$.
Equation~(\ref{ricc}) gives
\begin{equation}
\underline{\underline{0}}= \underline{\underline{W}}^{\,(0)}-\underline{\underline{W}}^{\,(0)}\,\underline{\underline{W}}^{\,(0)}- 
\underline{\underline{E}}^{\,(0)}\,\underline{\underline{W}}^{\,(0)},
\end{equation}
at $p=0$, which yields
\begin{align}
( \underline{\underline{1}}-\underline{\underline{W}}^{\,(0)}-\underline{\underline{E}}^{\,(0)})\,\underline{\underline{W}}^{\,(0)} = \underline{\underline{0}}.
\end{align}
Hence, we deduce that 
\begin{equation}\label{w0def}
\underline{\underline{W}}^{\,(0)} =  \underline{\underline{1}}-\underline{\underline{E}}^{\,(0)} =  \left(\begin{array}{cc} -1&0\\-E_{21}^{\,(0)}&1\end{array}\right).
\end{equation}

At small values of $p$, let
\begin{align}
\underline{u}(p) &= \underline{u}_{-1}\,p^{-1} + \underline{u}_0,\\[0.5ex]
\underline{\underline{W}}(p) &= \underline{\underline{W}}^{\,(0)} + p\,\underline{\underline{W}}^{\,(1)},
\end{align}
where the elements of  $\underline{u}_{-1}$ (which are $y_{-1}$ and $n_{-1}$, respectively), the elements of $\underline{u}_{0}$ (which are $y_0$ 
and $n_0$, respectively), and the elements of $\underline{\underline{W}}^{\,(1)}$, are all constants.
Equation~(\ref{wdef}) gives 
\begin{align}
\underline{\underline{W}}^{\,(0)}\,\underline{u}_{-1} &= - \underline{u}_{-1},\\[0.5ex]
\underline{\underline{W}}^{\,(0)}\,\underline{u}_{0} + \underline{\underline{W}}^{\,(1)}\,\underline{u}_{-1} &=\underline{0}.\label{w1def}
\end{align}
Thus, making use of Eq.~(\ref{w0def}), we get
\begin{equation}
\left(\begin{array}{cc} -1&0\\ -E_{21}^{\,(0)}&1\end{array}\right) \left(\begin{array}{c}y_{-1}\\ n_{-1}\end{array}\right)= -\left(\begin{array}{c}y_{-1}\\ n_{-1}\end{array}\right),
\end{equation}
which implies that
\begin{equation}
E_{21}^{\,(0)}\,y_{-1}= - \frac{2\,{\rm i}\,Q_e}{g+{\rm i}\,Q_e}\,y_{-1} = 2\,n_{-1},
\end{equation}
in accordance with Eqs.~(\ref{ym1}) and (\ref{nm1}), where use has been made of Eq.~(\ref{e210}). 
Thus, if we write
\begin{align}
y_{-1} &= (g+{\rm i}\,Q_e)\,a_{-1},\\[0.5ex]
n_{-1}&= -{\rm i}\,Q_e\,a_{-1},\\[0.5ex]
y_0 &= (g+{\rm i}\,Q_e)\,a_0,\\[0.5ex]
n_0 &= -{\rm i}\,Q_e\,a_0 + a_2,
\end{align}
in accordance with Eqs.~(\ref{yy}) and (\ref{nn}), then we deduce from Eqs.~(\ref{w0def}) and (\ref{w1def}) that 
\begin{equation}
\frac{\pi}{\hat{\mit\Delta}_s}\equiv \frac{a_0}{a_{-1}} = W_{11}^{\,(1)} - W_{12}^{\,(1)}\,\frac{(\,{\rm i}\,Q_e)}{g+{\rm i}\, Q_e},
\end{equation}
and
\begin{equation} 
\frac{a_2}{a_{-1}} = {\rm i}\,Q_e\left[W_{22}^{\,(1)}-W_{11}^{(1)}\right] + \frac{({\rm i}\,Q_e)^2}{g+{\rm i}\,Q_e}\,W_{12}^{\,(1)} - 
(g+{\rm i}\,Q_e)\,W_{21}^{\,(1)}.
\end{equation}

\section{Large-$p$ Behavior of Riccati Matrix Differential Equation}
At large values of $p$, it is clear from Eqs.~(\ref{f1})--(\ref{f4}) that  $\underline{\underline{F}}(p)=p^6\,\underline{\underline{F}}^{\,(6)}+ p^8\,\underline{\underline{F}}^{\,(8)}$, where the elements of $\underline{\underline{F}}^{\,(6)}$ and $\underline{\underline{F}}^{\,(8)}$ are constants.
On the other hand, Eqs.~(\ref{ee1})--(\ref{ee4}) imply that $\underline{\underline{E}}(p)=\underline{\underline{E}}^{\,(0)}$, 
where the elements of $\underline{\underline{E}}^{\,(0)}$ are constants. 
Thus, if we write $\underline{\underline{W}}(p) = p^2\,\underline{\underline{W}}^{\,(2)}+p^4\,\underline{\underline{W}}^{\,(4)}$,
where the elements of $\underline{\underline{W}}^{\,(2)}$ and  $\underline{\underline{W}}^{\,(4)}$ are constants, then Eq.~(\ref{ricc}) gives 
\begin{align}\label{e138}
\underline{\underline{W}}^{\,(4)}\,\underline{\underline{W}}^{\,(4)}&= \underline{\underline{F}}^{\,(8)},\\[0.5ex]
\underline{\underline{W}}^{\,(2)}\,\underline{\underline{W}}^{\,(4)}+ \underline{\underline{W}}^{\,(4)}\,\underline{\underline{W}}^{\,(2)}&= 
\underline{\underline{F}}^{\,(6)}.\label{e139}
\end{align}
Now, according to Eqs.~(\ref{f1})--(\ref{f4}), 
\begin{align}
F^{\,(8)}_{11} &=0,\\[0.5ex]
F^{\,(8)}_{12} &= 0,\\[0.5ex]
F^{\,(8)}_{21} &= c_\beta^{\,-2}\,D^{\,2}\,P_\varphi^{\,2},\\[0.5ex]
F^{\,(8)}_{22} &= c_\beta^{\,-2}\,\iota_e^{\,-1}\,D^{\,2}\,P_\varphi^{\,2},
\end{align}
so Eq.~(\ref{e138}) yields 
\begin{align}
W^{\,(4)}_{11} &=0,\\[0.5ex]
W^{\,(4)}_{12} &= 0,\\[0.5ex]
W^{\,(4)}_{21} &= -c_\beta^{\,-1}\,\iota_e^{\,1/2}\,D\,P_\varphi,\\[0.5ex]
W^{\,(4)}_{22} &=-c_\beta^{\,-1}\,\iota_e^{\,-1/2}\, D\,P_\varphi,
\end{align}
where we have chosen the sign of the square root that is associated with well-behaved solutions at large values of $p$. Here, we are  assuming that $\iota_e>0$. 
Equations~(\ref{f1})--(\ref{f4}) also give 
\begin{align}
F^{\,(6)}_{11} &=P_\varphi,\\[0.5ex]
F^{\,(6)}_{12} &= \iota_e^{\,-1}\,P_\varphi,\\[0.5ex]
F^{\,(6)}_{21} &=c_\beta^{\,-2}\,D^{\,2}\,g\,P_\varphi + c_\beta^{\,-2}\,D^{\,2}\,(g+{\rm i}\,Q_i)\,P_\varphi,\\[0.5ex]
F^{\,(6)}_{22} &=c_\beta^{\,-2}\,\iota_e^{\,-1}\,D^{\,2}\,g\,P_\varphi + c_\beta^{\,-2}\,[P_\perp+D^{\,2}\,(g+{\rm i}\,Q_i)]\,P_\varphi.
\end{align}
Thus, Eq.~(\ref{e139}) yields
\begin{align}
W_{12}^{\,(2)}\,W_{21}^{\,(4)} &= F_{11}^{\,(6)},\\[0.5ex]
W_{12}^{\,(2)}\,W_{22}^{\,(4)} &= F_{12}^{\,(6)},
\end{align}
which gives 
\begin{align}
W_{12}^{\,(2)} &= - c_\beta\,\iota_e^{\,-1/2}\,D^{-1}.
\end{align}

Now, if
\begin{equation}\label{e155}
\underline{\underline{W}}\,\underline{u}= \lambda(p)\,\underline{u}
\end{equation}
then Eq.~(\ref{wdef}) yields
\begin{equation}
p\,\frac{d\underline{u}}{dp} = \lambda\,\underline{u},
\end{equation}
which implies that
\begin{equation}
\underline{u}(p) = \underline{u}(p_0)\,\exp\left[\int_{p_0}^p\frac{\lambda_r(p')}{p'}\,dp'\right]\exp\left[{\rm i}\int_{p_0}^p\frac{\lambda_i(p')}{p'}\,dp'\right], 
\end{equation}
where $\lambda_r$  and $\lambda_i$ are the real and imaginary parts of $\lambda$, respectively. 
Of course, a solution that is well behaved at large values of $p$ is such that $\lambda_r$ is negative. As we have seen, the large-$p$ limit of
Eq.~(\ref{ricc}) is
\begin{equation}
\underline{\underline{W}}\,\underline{\underline{W}} = \underline{\underline{F}}.
\end{equation}
Hence, if
\begin{equation}\label{e158}
 \underline{\underline{F}}\,\underline{u} = {\mit\Lambda}\,\underline{u}
 \end{equation}
 then Eqs.~(\ref{e155}) and (\ref{e158}) imply that 
 \begin{equation}
 \lambda^2 = {\mit\Lambda}.
 \end{equation}
 The eigenvalue problem for the $F$-matrix reduces to
 \begin{equation}
 {\mit\Lambda}^{\,2}- (F_{11}+ F_{22})\,{\mit\Lambda} + F_{11}\,F_{22} - F_{12}\,F_{21}=0.
 \end{equation}
 Now,
 \begin{align}
 F_{11}+F_{22}&\simeq  F_{22}^{\,(8)}\,p^8 =c_\beta^{\,-2}\, \iota_e^{\,-1}\,D^{\,2}\,P_\varphi^{\,2}\,p^8,\\[0.5ex]
 F_{11}\,F_{22} - F_{12}\,F_{21}&  \simeq \left[F_{11}^{\,(6)}\,F_{22}^{\,(8)} - F_{12}^{\,(6)}\,F_{21}^{\,(8)}\right]p^{14}\nonumber\\[0.5ex]&\phantom{=}+\left[F_{11}^{\,(6)}\,F_{22}^{\,(6)} - F_{12}^{\,(6)}\,F_{21}^{\,(6)}\right]p^{12}
=c_\beta^{\,-2}\,R\,P_\varphi^{\,2}\,p^{12},
 \end{align}
  where 
 \begin{equation}
 R= P_\perp + (1-\iota_e^{\,-1})\,D^{\,2}\,(g+{\rm i}\,Q_i),
 \end{equation}
Hence, the two eigenvalues of the $F$-matrix are
 \begin{align}
 {\mit\Lambda}_1&\simeq F_{22}^{\,(8)}\,p^8=  c_\beta^{\,-2}\,\iota_e^{\,-1}\,D^2\,P_\varphi^{\,2}\,p^8,\\[0.5ex]
 {\mit\Lambda}_2 &\simeq \frac{[F_{11}^{\,(6)}\,F_{22}^{\,(6)} - F_{12}^{\,(6)}\,F_{21}^{\,(6)}]}{F_{22}^{\,(8)}}\,p^4= \iota_e\,D^{-2}\,R\,p^4.
 \end{align}
 Thus, we deduce that the two eigenvalues of the $W$-matrix are 
 \begin{align}
 \lambda_1&=-{\mit\Lambda}_1^{\,1/2}= -c_\beta^{\,-1}\,\iota_e^{\,-1/2}\,D\,P_\varphi\,p^4,\\[0.5ex]
 \lambda_2&=-{\mit\Lambda}_2^{\,1/2}=-\iota_e^{\,1/2}\,D^{-1}\,R^{\,1/2}\,p^2,
 \end{align}
 Here, the square root of $R$ is taken such that the real part of $\lambda_2$ is negative. 
  Now, the eigenvalue problem for the $W$-matrix reduces to 
 \begin{equation}
 \lambda^{2} - W_{22}^{\,(4)}\,p^4\,\lambda + \left[W_{11}^{\,(2)}\,W_{22}^{\,(4)} - W_{12}^{\,(2)}\,W_{21}^{\,(4)}\right]p^6 = 0.
 \end{equation}
which yields
\begin{equation}
\lambda_1\simeq W_{22}^{\,(4)}\,p^4,
\end{equation}
which is satisfied, and
\begin{equation}
\lambda_2 \simeq \left[W_{11}^{\,(2)} - \frac{W_{12}^{\,(2)}\,W_{21}^{\,(4)}}{W_{22}^{\,(4)}}\right]p^2,
\end{equation}
 which implies that
 \begin{equation}
 W_{11}^{\,(2)} = -\iota_e^{\,1/2}\,D^{-1}\,R^{\,1/2}-c_\beta\,\iota_e^{\,1/2}\,D^{-1}.
 \end{equation}
 Hence, the large-$p$ boundary condition for the $W$-matrix
 is
 \begin{equation}
 \underline{\underline{W}}(p) =  \left(\begin{array}{cc} -\iota_e^{\,1/2}\,D^{-1}\,R^{\,1/2}\,p^2-c_\beta\,\iota_e^{\,1/2}\,D^{-1}\,p^2,& - c_\beta\,\iota_e^{\,-1/2}\,D^{-1}\,p^2\\-c_\beta^{\,-1}\,\iota_e^{\,1/2}\,D\,P_\varphi\,p^4,&-c_\beta^{\,-1}\,\iota_e^{\,-1/2}\,D\,P_\varphi\,p^4\end{array}\right).
 \end{equation}
\end{document}