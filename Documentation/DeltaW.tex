\documentclass[12pt,prb,aps,notitlepage]{revtex4-1}
\usepackage {amsmath}
\usepackage{amssymb}
\pdfoutput = 1 
\usepackage {graphicx}
\newcommand{\bomega}{\mbox{\boldmath$\omega$}}
\newcommand{\bxi}{\mbox{\boldmath$\xi$}}
\newcommand{\bta}{\mbox{\boldmath$\eta$}}
\allowdisplaybreaks

\begin{document}

\title{Ideal-MHD Energy Principle Analysis}
\maketitle

\section{Introduction}
\subsection{Fundamental Equations}
Our fundamental equations are
\begin{align}
\frac{\partial\rho}{\partial t} + \nabla\cdot(\rho\,{\bf v}) & =0,\label{e1}\\[0.5ex]
\rho\left[\frac{\partial{\bf v}}{\partial t} + ({\bf v}\cdot\nabla)\,{\bf v}\right]+\nabla p - {\bf j}\times {\bf B}&=0,\label{e2}\\[0.5ex]
\frac{\partial p}{\partial t}+{\bf v}\cdot\nabla p + \gamma\,p\,\nabla\cdot{\bf v} &=0,\label{e3}\\[0.5ex]
\frac{\partial {\bf B}}{\partial t} &= \nabla\times({\bf v}\times {\bf B}),\label{e4}\\[0.5ex]
\mu_0\,{\bf j} &= \nabla\times {\bf B}.\label{e5}
\end{align}
Note that Eq.~(\ref{e4}) ensures that 
\begin{equation}\label{e6}
\nabla\cdot{\bf B} = 0
\end{equation}
is automatically satisfied, provided that it is satisfied initially. 

\section{Stability Analysis}
\subsection{Equilibrium}
The equilibrium is such that
\begin{align}
\rho({\bf r},t) &= \rho_0({\bf r}),\\[0.5ex]
{\bf v}({\bf r},t) &= {\bf 0},\\[0.5ex]
p({\bf r},t) &= p_0({\bf r}),\\[0.5ex]
{\bf B}({\bf r},t) &= {\bf B}_0({\bf r}),\\[0.5ex]
{\bf j}({\bf r},t) &= {\bf j}_0({\bf r}),
\end{align}
where
\begin{align}\label{e11}
\nabla p_0 &= {\bf j}_0\times {\bf B}_0,\\[0.5ex]
\mu_0\,{\bf j}_0 &= \nabla\times {\bf B}_0.\label{e12}
\end{align}

\subsection{Perturbed Quantities}
The perturbation is such that
\begin{align}
\rho_1({\bf r},t) &= \rho_1({\bf r})\,{\rm e}^{-{\rm i}\,\omega\,t},\\[0.5ex]
{\bf v}_1({\bf r},t) &= {-{\rm i}\,\omega}\,\bxi({\bf r})\,{\rm e}^{-{\rm i}\,\omega\,t},\\[0.5ex]
p({\bf r},t) &= p_1({\bf r})\,{\rm e}^{-{\rm i}\,\omega\,t},\\[0.5ex]
{\bf B}({\bf r},t) &= {\bf B}_1({\bf r})\,{\rm e}^{-{\rm i}\,\omega\,t},\\[0.5ex]
{\bf j}({\bf r},t) &= {\bf j}_1({\bf r})\,{\rm e}^{-{\rm i}\,\omega\,t},
\end{align}

The linearized perturbed versions of  Eqs.~(\ref{e1})--(\ref{e5}) are
\begin{align}
\rho_1 &=-\nabla\cdot(\rho_0\,\bxi),\\[0.5ex]
-\omega^2\,\rho_0\,\bxi &=  {\bf j}_1\times {\bf B}_0 + {\bf j}_0\times {\bf B}_1-\nabla p_1,\\[0.5ex]
p_1&= - \bxi\cdot\nabla p_0-\gamma\,p_0\,\nabla\cdot\bxi,\\[0.5ex]
{\bf B}_1 &=\nabla\times (\bxi\times {\bf B}_0),\\[0.5ex]
\mu_0\,{\bf j}_1&= \nabla\times {\bf B}_1.
\end{align}
Combining the previous four equations, we obtain
\begin{equation}
-\omega^2\,\rho_0\,\bxi= {\bf F}(\bxi),
\end{equation}
where
\begin{align}\label{e24}
{\bf F}(\bxi) = \mu_0^{-1}\,(\nabla\times {\bf Q})\times{\bf B}_0+ \mu_0^{-1}\, (\nabla\times {\bf B}_0)\times {\bf Q} + 
\nabla(\bxi\cdot \nabla p_0 + \gamma\,p_0\,\nabla\cdot\bxi),
\end{align}
and
\begin{equation}\label{e25}
{\bf Q} = \nabla\times (\bxi\times {\bf B}_0).
\end{equation}

\subsection{Self-Adjointness of Force Operator}
Suppose that the plasma is surrounded by a perfectly conducting wall whose outward unit normal is ${\bf n}$.
We wish to demonstrate that
\begin{equation}\label{e26}
\int \bta\cdot{\bf F}(\bxi)\,d{\bf r} = \int \bxi\cdot {\bf F}(\bta)\,d{\bf r},
\end{equation}
where $\bxi({\bf r})$ and $\bta({\bf r})$ are two arbitrary vector fields that satisfy the boundary conditions
\begin{equation}\label{e27}
{\bf n}\cdot\bxi={\bf n}\cdot\bta=0
\end{equation}
at the wall  

According to Eq.~(\ref{e24}), the integrand of the left-hand side of Eq.~(\ref{e26}) takes the form 
\begin{equation}
\bta\cdot {\bf F}(\xi)= \bta\cdot\left[ \mu_0^{-1}\,(\nabla\times {\bf Q})\times{\bf B}_0+ \mu_0^{-1}\, (\nabla\times {\bf B}_0)\times {\bf Q}
+\nabla(\bxi\cdot\nabla p_0) + \nabla(\gamma\,p_0\,\nabla\cdot\bxi)\right].
\end{equation}
The final term can be written
\begin{equation}
\bta\cdot \nabla(\gamma\,p_0\,\nabla\cdot\bxi)= \nabla\cdot (\bta\,\gamma\,p_0\,\nabla\cdot\bxi)-\gamma\,p_0\,(\nabla\cdot\bta)\,(\nabla\cdot\bxi).
\end{equation}
However, according to Eq.~(\ref{e27}), the divergence term integrates to zero.
Hence, we obtain
\begin{equation}
\bta\cdot {\bf F}(\xi)= \bta\cdot\left[ \mu_0^{-1}\,(\nabla\times {\bf Q})\times{\bf B}_0+ \mu_0^{-1}\, (\nabla\times {\bf B}_0)\times {\bf Q}
+\nabla(\bxi\cdot\nabla p_0)\right] -\gamma\,p_0\,(\nabla\cdot\bta)\,(\nabla\cdot\bxi).
\end{equation}

Let us write
\begin{align}
\bxi &=\bxi_\perp + \xi_\parallel\,{\bf b},\\[0.5ex]
\bta&=\bta_\perp+\eta_\parallel\,{\bf b},
\end{align}
where 
\begin{align}
{\bf b} &=\frac{{\bf B}_0}{B_0},\\[0.5ex]
{\bf b} \cdot\bxi_\perp &= {\bf b}\cdot\bta_\perp = 0.
\end{align}
It follows from Eq.~(\ref{e25}) that
\begin{equation}\label{e36}
{\bf Q} = \nabla\times (\bxi_\perp\times {\bf B}_0).
\end{equation}
Moreover, Eq.~(\ref{e11}) implies that
\begin{equation}
\bxi\cdot\nabla p_0 = \bxi\cdot{\bf j}_0\times {\bf B}_0= \bxi_\perp\cdot{\bf j}_0\times {\bf B}_0= \bxi_\perp\cdot\nabla p_0.
\end{equation}

Now,
\begin{equation}
{\bf B}_0 \cdot \left[\mu_0^{-1}\,(\nabla\times {\bf Q})\times{\bf B}_0\right] =0,
\end{equation}
and
\begin{align}
{\bf B}_0\cdot\left[\mu_0^{-1}\, (\nabla\times {\bf B}_0)\times {\bf Q}\right]= {\bf B}_0\cdot{\bf j}_0\times {\bf Q} =-{\bf j}_0\times {\bf B}_0\cdot {\bf Q}
=-\nabla p_0\cdot{\bf Q},
\end{align}
where use has been made of Eqs.~(\ref{e11}) and (\ref{e12}). However, according to Eq.~(\ref{e36}), 
\begin{align}
-\nabla p_0\cdot{\bf Q} &= -\nabla p_0\cdot\nabla\times(\bxi_\perp\times {\bf B}_0)=\nabla\cdot\left[\nabla p_0\times(\bxi_\perp\times {\bf B}_0)\right]
=-\nabla\cdot[(\bxi_\perp\cdot\nabla p_0)\, {\bf B}_0]\nonumber\\[0.5ex]
&= - {\bf B}_0\cdot\nabla (\bxi_\perp\cdot\nabla p_0),
\end{align}
where use has been made of Eqs.~(\ref{e6}) and (\ref{e11}). The previous two equations imply that
\begin{equation}
{\bf B}_0\cdot\left[\mu_0^{-1}\, (\nabla\times {\bf B}_0)\times {\bf Q}+\nabla(\bxi_\perp\cdot\nabla p_0)\right] =0.
\end{equation}

\end{document}