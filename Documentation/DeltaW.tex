\documentclass[12pt,prb,aps,notitlepage]{revtex4-1}

\usepackage {amsmath}
\usepackage{amssymb}
\usepackage{stmaryrd}
\usepackage {graphicx}

\pdfoutput = 1 

\newcommand{\bomega}{\mbox{\boldmath$\omega$}}
\newcommand{\bxi}{\mbox{\boldmath$\xi$}}
\newcommand{\bta}{\mbox{\boldmath$\eta$}}
\newcommand{\bkappa}{\mbox{\boldmath$\kappa$}}
\newcommand{\blambda}{\mbox{\boldmath$\lambda$}}

\allowdisplaybreaks

\begin{document}

\title{Ideal-MHD Energy Principle Analysis}
\maketitle

\section{Ideal-MHD Stability Analayis}
\subsection{Ideal-MHD Equations}
The fundamental  equations of ideal magnetohydrodynamics (ideal-MHD) are
\begin{align}
\frac{\partial\rho}{\partial t} + \nabla\cdot(\rho\,{\bf v}) & =0,\label{e1}\\[0.5ex]
\rho\left[\frac{\partial{\bf v}}{\partial t} + ({\bf v}\cdot\nabla)\,{\bf v}\right]+\nabla p - {\bf j}\times {\bf B}&=0,\label{e2}\\[0.5ex]
\frac{\partial p}{\partial t}+{\bf v}\cdot\nabla p + \gamma\,p\,\nabla\cdot{\bf v} &=0,\label{e3}\\[0.5ex]
\nabla\cdot{\bf B} &=0,\\[0.5ex]
\frac{\partial {\bf B}}{\partial t} &= \nabla\times({\bf v}\times {\bf B}),\label{e4}\\[0.5ex]
\mu_0\,{\bf j} &= \nabla\times {\bf B},\label{e5}
\end{align}
where $\rho$ is the plasma mass density, ${\bf v}$ the plasma velocity, $p$ the (scalar) plasma pressure, $\gamma=5/3$ the ratio of specific heats,
${\bf B}$ the magnetic field-strength, and ${\bf j}$ the electric current density. 

\subsection{Plasma Equilibrium}
The plasma equilibrium is such that
\begin{align}
\rho({\bf r},t) &= \rho_0({\bf r}),\\[0.5ex]
{\bf v}({\bf r},t) &= {\bf 0},\label{e7}\\[0.5ex]
p({\bf r},t) &= p_0({\bf r}),\\[0.5ex]
{\bf B}({\bf r},t) &= {\bf B}_0({\bf r}),\\[0.5ex]
{\bf j}({\bf r},t) &= {\bf j}_0({\bf r}),
\end{align}
where
\begin{align}
\nabla p_0 &= {\bf j}_0\times {\bf B}_0,\label{e11}\\[0.5ex]
\nabla\cdot{\bf B}_0&= 0,\label{e6}\\[0.5ex]
\mu_0\,{\bf j}_0 &= \nabla\times {\bf B}_0.\label{e12}
\end{align}
Note that we are neglecting equilibrium plasma flows, because these generally have little influence on the ideal stability of fusion plasmas, provided that they remain sub-sonic and sub-Alfv\'{e}nic. 

\subsection{Plasma Boundary}
The plasma is assumed to be confined on a set of nested toroidal magnetic flux-surfaces. 
Let the plasma occupy the toroidal volume $V$, and let $S$ be its bounding surface. Furthermore, let ${\bf n}$ be
a unit outward-directed normal to $S$. Because $S$ corresponds to  an equilibrium  magnetic flux-surface, it follows that
\begin{equation}\label{e31x}
{\bf n}\cdot {\bf B}_0=0
\end{equation}
on $S$. 

\subsection{Plasma Equation of Motion}
Consider a perturbation to the plasma equilibrium. 
Let us write
\begin{align}
\rho({\bf r},t) &= \rho_0({\bf r})+ \tilde{\rho}_1({\bf r},t),\\[0.5ex]
{\bf v}({\bf r},t) &= \tilde{\bf v}_1({\bf r},t),\\[0.5ex]
p({\bf r},t) &= p_0({\bf r})+ \tilde{p}_1({\bf r},t),\\[0.5ex]
{\bf B}({\bf r},t) &= {\bf B}_0({\bf r})+ \widetilde{\bf B}_1({\bf r},t),\\[0.5ex]
{\bf j}({\bf r},t) &= {\bf j}_0({\bf r})+ \tilde{\bf j}({\bf r},t),
\end{align}
where the perturbed quantities (denoted by the subscript $1$) are all assumed to be much smaller than
the corresponding equilibrium quantities (denoted by the subscript $0$).

The linearized perturbed versions of  Eqs.~(\ref{e1})--(\ref{e5}) are
\begin{align}
\frac{\partial\tilde{\rho}_1}{\partial t} + \nabla\cdot(\rho_0\,\tilde{\bf v}_1) & =0,\label{e19f}\\[0.5ex]
\rho_0\,\frac{\partial\tilde{\bf v}_1}{\partial t} +\nabla \tilde{p}_1 - \tilde{\bf j}_1\times {\bf B}_0-{\bf j}_0\times \widetilde{\bf B}_1&=0,\label{e20f}\\[0.5ex]
\frac{\partial \tilde{p}_1}{\partial t}+\tilde{\bf v}_1\cdot\nabla p_0 + \gamma\,p_0\,\nabla\cdot\tilde{\bf v}_1 &=0,\label{e21f}\\[0.5ex]
\nabla\cdot\widetilde{\bf B}_1&=0,\label{e23g}\\[0.5ex]
\frac{\partial \widetilde{\bf B}_1}{\partial t} &= \nabla\times(\tilde{\bf v}_1\times {\bf B}_0),\label{e22f}\\[0.5ex]
\mu_0\,\tilde{\bf j}_1 &= \nabla\times \widetilde{\bf B}_1,\label{e23f}
\end{align}
respectively. 
Let us write
\begin{equation}\label{e15x}
\tilde{\bf v}_1 = \frac{\partial\tilde{\bxi}}{\partial t},
\end{equation}
where $\tilde{\bxi}({\bf r},t)$ represents the {\em displacement}\/ of the plasma from its equilibrium position. If
we assume that $\tilde{\rho}_1({\bf r},0)= \tilde{p}_1({\bf r},0)=\tilde{\bf B}_1({\bf r},0) = 0$ then Eqs.~(\ref{e19f}), (\ref{e21f}),
and (\ref{e22f}) can be integrated to give
\begin{align}
\tilde{\rho}_1&= -\nabla\cdot(\rho_0\,\tilde{\bxi}),\label{e25f}\\[0.5ex]
\tilde{p}_1&= -\tilde{\bxi}\cdot\nabla p_0 -\gamma\,p_0\,\nabla\cdot\tilde{\bxi},\label{e26f}\\[0.5ex]
\tilde{\bf B}_1&= \nabla\times (\tilde{\bxi}\times {\bf B}_0),
\end{align}
respectively. 
Substitution of these equations into Eq.~(\ref{e20f}), making use of Eq.~(\ref{e23f}),  yields the linearized perturbed plasma {\em equation of motion},
\begin{equation}\label{e28f}
\rho_0\,\frac{\partial^2\tilde{\bxi}}{\partial t^2} = {\bf F}(\tilde{\bxi}),
\end{equation}
where 
\begin{align}
{\bf F}(\tilde{\bxi}) = \mu_0^{-1}\,(\nabla\times \widetilde{\bf Q})\times{\bf B}_0+ \mu_0^{-1}\, (\nabla\times {\bf B}_0)\times \widetilde{\bf Q} + 
\nabla(\tilde{\bxi}\cdot \nabla p_0 + \gamma\,p_0\,\nabla\cdot\tilde{\bxi})
\end{align}
is known as the {\em force operator}, 
and
\begin{equation}
\widetilde{\bf Q}\equiv  \widetilde{\bf B}_1= \nabla\times (\tilde{\bxi}\times {\bf B}_0)
\end{equation}
is the perturbed magnetic field. Note that the previous equation automatically satisfies Eq.~(\ref{e23g}). 
 The force operator specifies the perturbed force density that develops in the plasma in
response to the displacement $\tilde{\bxi}$. Note that Eq.~(\ref{e25f}) does not actually contribute to the equation of motion. 

\subsection{Normal Mode Analysis}
The most efficient way to investigate linear stability is to formulate 
 the problem in terms of normal modes. This goal is
achieved by letting all perturbed quantities vary in time as $\exp(-{\rm i}\,\omega\,t)$. 
Thus,
\begin{align}
\tilde{\bf v}_1({\bf r},t) &= {\bf v}_1({\bf r})\, \exp(-{\rm i}\,\omega\,t),\label{e33q}\\[0.5ex]
\tilde{\bxi}({\bf r},t)&= \bxi({\bf r})\, \exp(-{\rm i}\,\omega\,t),\\[0.5ex]
\widetilde{\bf Q}({\bf r},t) &= {\bf Q}({\bf r})\, \exp(-{\rm i}\,\omega\,t),
\end{align}
et cetera. 
Equations~(\ref{e15x}) and (\ref{e26f}) transform to give
\begin{align}\label{e15}
{\bf v}_1 &=-{\rm i}\,\omega\,\bxi,\\[0.5ex]
p_1&= -{\bxi}\cdot\nabla p_0 -\gamma\,p_0\,\nabla\cdot{\bxi},\label{e15t}
\end{align}
respectively, 
whereas the linearized perturbed plasma equation of motion, (\ref{e28f}), yields
\begin{equation}\label{e24a}
-\omega^2\,\rho_0\,\bxi= {\bf F}(\bxi),
\end{equation}
where
\begin{align}\label{e24}
{\bf F}(\bxi) = \mu_0^{-1}\,(\nabla\times {\bf Q})\times{\bf B}_0+ \mu_0^{-1}\, (\nabla\times {\bf B}_0)\times {\bf Q} + 
\nabla(\bxi\cdot \nabla p_0 + \gamma\,p_0\,\nabla\cdot\bxi),
\end{align} 
and
\begin{equation}\label{e25}
{\bf Q} = \nabla\times (\bxi\times {\bf B}_0).
\end{equation}
We can think of Eq.~(\ref{e24a}) as a sort of eigenmode equation in which $\xi({\bf r})$ is the eigenvector and $\omega^2$ the eigenvalue. 

\subsection{Useful Analysis}
Consider the integral
 \begin{equation}\label{intg}
\int_V \bta\cdot{\bf F}(\bxi)\,d{\bf r},
\end{equation}
where $\bxi({\bf r})$ and $\bta({\bf r})$ are two arbitrary vector fields. 
According to Eq.~(\ref{e24}), the integrand  takes the form 
\begin{equation}
\bta\cdot {\bf F}(\bxi)= \bta\cdot\left[ \mu_0^{-1}\,(\nabla\times {\bf Q})\times{\bf B}_0+ \mu_0^{-1}\, (\nabla\times {\bf B}_0)\times {\bf Q}
+\nabla(\bxi\cdot\nabla p_0) + \nabla(\gamma\,p_0\,\nabla\cdot\bxi)\right].
\end{equation}
The final term can be written
\begin{equation}
\bta\cdot \nabla(\gamma\,p_0\,\nabla\cdot\bxi)= \nabla\cdot [\gamma\,p_0\,(\nabla\cdot\bxi)\,\bta]-\gamma\,p_0\,(\nabla\cdot\bta)\,(\nabla\cdot\bxi).
\end{equation}
 Hence,
\begin{align}
\bta\cdot {\bf F}(\xi)&= \nabla\cdot [\gamma\,p_0\,(\nabla\cdot\bxi)\,\bta]+\bta\cdot\left[ \mu_0^{-1}\,(\nabla\times {\bf Q})\times{\bf B}_0+ \mu_0^{-1}\, (\nabla\times {\bf B}_0)\times {\bf Q}
+\nabla(\bxi\cdot\nabla p_0)\right]\nonumber\\[0.5ex]
&\phantom{=} -\gamma\,p_0\,(\nabla\cdot\bta)\,(\nabla\cdot\bxi).\label{e31}
\end{align}

Let us write
\begin{align}\label{e32}
\bxi &=\bxi_\perp + \xi_\parallel\,{\bf b},\\[0.5ex]
\bta&=\bta_\perp+\eta_\parallel\,{\bf b},\label{e33x}
\end{align}
where 
\begin{align}
{\bf b} &=\frac{{\bf B}_0}{B_0},\\[0.5ex]
{\bf b} \cdot\bxi_\perp &= {\bf b}\cdot\bta_\perp = 0.
\end{align}
Thus, $\xi_\parallel\,{\bf b}$ and $\bxi_\perp$ are the components of $\bxi$ that are, respectively, parallel to and perpendicular to the equilibrium magnetic field, et cetera.
It follows from Eq.~(\ref{e25}) that
\begin{equation}\label{e36}
{\bf Q} = \nabla\times (\bxi_\perp\times {\bf B}_0).
\end{equation}
Moreover, Eq.~(\ref{e11}) implies that
\begin{equation}\label{e37}
\bxi\cdot\nabla p_0 = \bxi\cdot{\bf j}_0\times {\bf B}_0= \bxi_\perp\cdot{\bf j}_0\times {\bf B}_0= \bxi_\perp\cdot\nabla p_0.
\end{equation}

Now,
\begin{equation}\label{e38}
{\bf B}_0 \cdot \left[\mu_0^{-1}\,(\nabla\times {\bf Q})\times{\bf B}_0\right] =0,
\end{equation}
and
\begin{align}
{\bf B}_0\cdot\left[\mu_0^{-1}\, (\nabla\times {\bf B}_0)\times {\bf Q}\right]= {\bf B}_0\cdot{\bf j}_0\times {\bf Q} =-{\bf j}_0\times {\bf B}_0\cdot {\bf Q}
=-\nabla p_0\cdot{\bf Q},
\end{align}
where use has been made of Eqs.~(\ref{e11}) and (\ref{e12}). However, according to Eq.~(\ref{e36}), 
\begin{align}
-\nabla p_0\cdot{\bf Q} &= -\nabla p_0\cdot\nabla\times(\bxi_\perp\times {\bf B}_0)=\nabla\cdot\left[\nabla p_0\times(\bxi_\perp\times {\bf B}_0)\right]
=-\nabla\cdot[(\bxi_\perp\cdot\nabla p_0)\, {\bf B}_0]\nonumber\\[0.5ex]
&= - {\bf B}_0\cdot\nabla (\bxi_\perp\cdot\nabla p_0),
\end{align}
where use has been made of Eqs.~(\ref{e11}) and (\ref{e6}). The previous two equations, combined with Eq.~(\ref{e37}), imply that
\begin{equation}\label{e41}
{\bf B}_0\cdot\left[\mu_0^{-1}\, (\nabla\times {\bf B}_0)\times {\bf Q}+\nabla(\bxi\cdot\nabla p_0)\right] =0.
\end{equation}
Thus, Eqs.~(\ref{e31}), (\ref{e33x}), (\ref{e37}), (\ref{e38}), and (\ref{e41}) yield 
\begin{align}\label{e52x}
\bta\cdot {\bf F}(\bxi)&= \nabla\cdot [\gamma\,p_0\,(\nabla\cdot\bxi)\,\bta]+\bta_\perp\cdot\left[ \mu_0^{-1}\,(\nabla\times {\bf Q})\times{\bf B}_0+ \mu_0^{-1}\, (\nabla\times {\bf B}_0)\times {\bf Q}
+\nabla(\bxi_\perp\cdot\nabla p_0)\right] \nonumber\\[0.5ex]
&\phantom{=}-\gamma\,p_0\,(\nabla\cdot\bta)\,(\nabla\cdot\bxi).
\end{align}

Let 
\begin{equation}\label{e43}
I= \bta_\perp\cdot\left[ \mu_0^{-1}\,(\nabla\times {\bf Q})\times{\bf B}_0+ \mu_0^{-1}\, (\nabla\times {\bf B}_0)\times {\bf Q}
+\nabla(\bxi_\perp\cdot\nabla p_0)\right].
\end{equation}
Now,
\begin{equation}
\bta_\perp\cdot\nabla (\bxi_\perp\cdot\nabla p_0)= \nabla\cdot[(\bxi_\perp\cdot \nabla p_0)\,\bta_\perp] - (\bxi_\perp\cdot\nabla p_0)\,(\nabla\cdot
\bta_\perp).
\end{equation}
 Furthermore, 
\begin{equation}
(\nabla\times {\bf Q})\times{\bf B}_0+(\nabla\times {\bf B}_0)\times {\bf Q}= {\bf Q}\cdot\nabla{\bf B}_0
+ {\bf B}_0\cdot\nabla{\bf Q} - \nabla({\bf B}_0\cdot{\bf Q}).
\end{equation}
Hence,
\begin{equation}\label{e46}
I =\nabla\cdot[(\bxi_\perp\cdot \nabla p_0)\,\bta_\perp]+  \mu_0^{-1}\,\bta_\perp\cdot\left[ {\bf Q}\cdot\nabla{\bf B}_0
+ {\bf B}_0\cdot\nabla{\bf Q} - \nabla({\bf B}_0\cdot{\bf Q})\right]  - (\bxi_\perp\cdot\nabla p_0)\,(\nabla\cdot
\bta_\perp).
\end{equation}

Note that
\begin{equation}\label{e47}
Q_i = \frac{\partial}{\partial x_j}\,(\xi_{\perp\,i}\,B_{0\,j} - \xi_{\perp\,j}\,B_{0\,i}).
\end{equation}
It follows that
\begin{align}
({\bf Q}\cdot\nabla {\bf B}_0)_i&= Q_j\,\frac{\partial B_{0\,i}}{\partial x_j} 
= \frac{\partial}{\partial x_k}\,(\xi_{\perp\,j}\,B_{0\,k} - \xi_{\perp\,k}\,B_{0\,j})\,\frac{\partial B_{0\,i}}{\partial x_j}\nonumber\\[0.5ex]
&= B_{0\,k}\,\frac{\partial\xi_{\perp\,j}}{\partial x_k}\,\frac{\partial B_{0\,i}}{\partial x_j}-\xi_{\perp\,k}\,\frac{\partial B_{0\,j}}{\partial x_k}\,\frac{\partial B_{0\,i}}{\partial x_j}
-B_{0\,j}\,\frac{\partial B_{0\,i}}{\partial x_j}\,\frac{\partial \xi_{\perp\,k}}{\partial x_k} \nonumber\\[0.5ex]
&= \left[ ({\bf B}_0\cdot
\nabla\bxi_\perp)\cdot\nabla {\bf B}_0 - (\bxi_\perp\cdot \nabla{\bf B}_0)\cdot\nabla {\bf B}_0- ({\bf B}_0\cdot\nabla {\bf B}_0)\,(\nabla\cdot\bxi_\perp)\right]_i,
\end{align}
where use has been made of Eq.~(\ref{e6}). However,
\begin{equation}\label{e49}
({\bf B}_0\cdot\nabla {\bf B}_0) = B_0^{\,2}\,\bkappa+ ({\bf B}_0\cdot\nabla B_0)\,{\bf b},
\end{equation}
where
\begin{equation}
\bkappa \equiv {\bf b}\cdot\nabla {\bf b} = \frac{{\bf R}_c}{R_c^{\,2}}
\end{equation}
is the {\em curvature vector}\/ of the equilibrium magnetic field. To be more exact, ${\bf R}_c$ is the local radius of curvature,
and is directed toward the local center of curvature, of equilibrium magnetic field-lines.
Hence, we deduce that
\begin{equation}\label{e51}
\bta_\perp \cdot({\bf Q}\cdot\nabla {\bf B}_0)= \bta_\perp\cdot\left[({\bf B}_0\cdot
\nabla\bxi_\perp)\cdot\nabla {\bf B}_0 - (\bxi_\perp\cdot \nabla{\bf B}_0)\cdot\nabla {\bf B}_0- B_0^{\,2}\,(\nabla\cdot\bxi_\perp)\,\bkappa\right].
\end{equation}

Now,
\begin{align}
\bta_\perp\cdot({\bf B}_0\cdot\nabla{\bf Q}) 
= \eta_{\perp\,i}\,B_{0\,j}\,\frac{\partial Q_i}{\partial x_j}=
\frac{\partial}{\partial x_j}\,(\eta_{\perp i}\,Q_i\,B_{0\,j})
- Q_i\,B_{0\,j}\,\frac{\partial\eta_{\perp\,i}}{\partial x_j}\nonumber\\[0.5ex]
= \nabla\cdot[(\bta_\perp\cdot {\bf Q})\,{\bf B}_0] - {\bf Q}\cdot ({\bf B}_0\cdot\nabla \bta_\perp)=- {\bf Q}\cdot ({\bf B}_0\cdot\nabla \bta_\perp),
\end{align}
where we have  made use of  Eq.~(\ref{e6}). Note that the divergence term would integrate to zero in Eq.~(\ref{intg}),  because of the boundary condition Eq.~(\ref{e31x}). 
Hence, this term has been neglected. 
However, from Eqs.~(\ref{e6}) and (\ref{e47}),
\begin{equation}\label{e53}
{\bf Q}  ={\bf B}_0\cdot\nabla \bxi_\perp - \bxi_\perp\cdot\nabla {\bf B}_0 - (\nabla\cdot\bxi_\perp)\,{\bf B}_0.
\end{equation}
Thus,
\begin{align}
\bta_\perp \cdot({\bf B}_0\cdot\nabla {\bf Q})&=- ({\bf B}_0\cdot\nabla\bxi_\perp)\cdot({\bf B}_0\cdot\nabla \bta_\perp)+ (\bxi_\perp\cdot\nabla {\bf B}_0)\cdot({\bf B}_0\cdot\nabla \bta_\perp) \nonumber\\[0.5ex]
&\phantom{=}+ (\nabla\cdot\bxi_\perp)\,{\bf B}_0\cdot({\bf B}_0\cdot\nabla \bta_\perp). 
\end{align}
But,
\begin{align}
{\bf B}_0\cdot({\bf B}_0\cdot\nabla \bta_\perp)&= B_{0\,i}\,B_{0\,j}\,\frac{\partial \eta_{\perp\,i}}{\partial x_j}= \frac{\partial}{\partial x_j}\,( B_{0\,i}\,B_{0\,j}\, \eta_{\perp\,i}) - B_{0\,i}\,\frac{\partial B_{0\,j}}{\partial x_j}\,\eta_{\perp\,i}\nonumber\\[0.5ex]
&= \nabla\cdot [({\bf B}_0\cdot\bta_\perp)\,{\bf B}_0] -B_0^{\,2}\,\bkappa\cdot \bta_\perp=-B_0^{\,2}\,\bkappa\cdot \bta_\perp,
\end{align}
where we have made use of Eq.~(\ref{e6}), as well as Eq.~(\ref{e49}), and the divergence term has integrated to zero because of Eq.~(\ref{e31x}). Thus, we arrive at
\begin{equation}\label{e56}
\bta_\perp \cdot({\bf B}_0\cdot\nabla {\bf Q})=- ({\bf B}_0\cdot\nabla\bxi_\perp)\cdot({\bf B}_0\cdot\nabla \bta_\perp)+ (\bxi_\perp\cdot\nabla {\bf B}_0)\cdot({\bf B}_0\cdot\nabla \bta_\perp) - B_0^{\,2}\,(\bta_\perp\cdot\bkappa)\,(\nabla\cdot\bxi_\perp).
\end{equation}

Now,
\begin{align}
-\bta_\perp\cdot\nabla ({\bf B}_0\cdot{\bf Q}) &= -\eta_{\perp i}\,\frac{\partial}{\partial x_i}\,(B_{0\,j}\,Q_j) = - \frac{\partial}{\partial x_i}\,(\eta_{\perp\,i}\,B_{0\,j}\,Q_j)+ B_{0\,j}\,Q_j\,\frac{\partial \eta_{\perp i}}{\partial x_i}\nonumber\\[0.5ex]
&= -\nabla\cdot[({\bf B}_0\cdot {\bf Q})\,\bta_\perp]  +({\bf B}_0\cdot{\bf Q})\,(\nabla\cdot\bta_\perp).
\end{align}
But, from Eq.~(\ref{e53}), 
\begin{align}
{\bf B}_0\cdot{\bf Q} &= - B_0^{\,2}\,\nabla\cdot\bxi_\perp + {\bf B}_0\cdot ({\bf B}_0\cdot\nabla\bxi_\perp) - {\bf B}_0\cdot
(\bxi_\perp\cdot\nabla {\bf B}_0)\nonumber\\[0.5ex]
&=- B_0^{\,2}\,\nabla\cdot\bxi_\perp - \bxi_\perp\cdot\nabla(B_0^{\,2}/2) + {\bf B}_0\cdot ({\bf B}_0\cdot\nabla\bxi_\perp).
\end{align}
However,
\begin{align}
 {\bf B}_0\cdot ({\bf B}_0\cdot\nabla\bxi_\perp) &= 
 B_{0\,i}\,B_{0\,j}\,\frac{\partial \xi_{\perp\,i}}{\partial x_j}= \frac{\partial}{\partial x_j}\,(B_{0\,i}\,B_{0\,j}\,\xi_{\perp\,i})- B_{0\,j} \,\frac{\partial B_{0\,i}}{\partial x_j}\,\xi_{\perp\,i}\nonumber\\[0.5ex]
 &= \nabla\cdot[({\bf B}_0\cdot\bxi_\perp)\,{\bf B}_0] - \bxi_\perp\cdot({\bf B}_0\cdot\nabla{\bf B}_0) = -B_0^{\,2}\,(\bxi_\perp\cdot\bkappa),
 \end{align}
 where we have made  use of Eqs.~(\ref{e6}) and (\ref{e49}), and the divergence term has  integrated to zero because of Eq.~(\ref{e31x}). 
 Thus, we deduce that
 \begin{align}\label{e60}
 -\bta_\perp\cdot\nabla ({\bf B}_0\cdot{\bf Q})& = -\nabla\cdot[({\bf B}_0\cdot {\bf Q})\,\bta_\perp]   -B_0^{\,2}\,(\nabla\cdot\bxi_\perp)\,(\nabla\cdot\bta_\perp)
 \nonumber\\[0.5ex]
 &\phantom{=}-[\bxi_\perp\cdot\nabla(B_0^{\,2}/2) + B_0^{\,2}\,(\bxi_\perp\cdot\bkappa)]\,(\nabla\cdot\bta_\perp).
 \end{align}
 
 It follows from Eqs.~(\ref{e52x}), (\ref{e43}), (\ref{e46}),  (\ref{e51}), (\ref{e56}), and (\ref{e60}) that
 \begin{align}\label{e61}
\bta\cdot {\bf F}(\bxi) &=\nabla\cdot\left[\gamma\,p_0\,(\nabla\cdot\bxi)\,\bta+ (\bxi_\perp\cdot\nabla p_0-{\bf B}_0\cdot{\bf Q})\,\bta_\perp\right]\nonumber\\[0.5ex]
&\phantom{=}+\frac{B_0^{\,2}}{\mu_0}\,(\nabla\cdot\bxi_\perp)\,(\nabla\cdot\bta_\perp) 
 -\mu_0^{-1}\,({\bf B}_0\cdot\nabla\bxi_\perp)\cdot
 ({\bf B}_0\cdot\nabla\bta_\perp)-\gamma\,p_0\,(\nabla\cdot\bxi)\,(\nabla\cdot\bta)\nonumber\\[0.5ex]
 &\phantom{=}-\left[\bxi_\perp\cdot\nabla\left(p_0+\frac{B_0^{\,2}}{2\,\mu_0}\right)+ \frac{B_0^{\,2}}{\mu_0}\,\bxi_\perp\cdot\bkappa\right]\nabla\cdot\bta_\perp\nonumber
\\[0.5ex]&
\phantom{=}-\frac{2\,B_0^{\,2}}{\mu_0}\,(\bta_\perp\cdot\bkappa)\,(\nabla\cdot\bxi_\perp) +R,
 \end{align}
 where
 \begin{equation}
 \mu_0\,R= \bta_\perp\cdot\left[({\bf B}_0\cdot\nabla\bxi_\perp)\cdot\nabla {\bf B}_0 - (\bxi_\perp\cdot\nabla {\bf B}_0)\cdot\nabla {\bf B}_0\right]
 + (\bxi_\perp\cdot\nabla {\bf B}_0)\cdot({\bf B}_0\cdot\nabla\bta_\perp).
 \end{equation}
 
 However, from Eqs.~(\ref{e11}) and (\ref{e12}), 
 \begin{equation}\label{e64}
 \nabla p_0 = \mu_0^{-1}\,(\nabla\times{\bf B}_0)\times {\bf B}_0= \mu_0^{-1}\left[({\bf B}_0\cdot\nabla)\,{\bf B}_0-\nabla(B_0^{\,2}/2)\right].
 \end{equation}
 Thus, Eq.~(\ref{e49}) yields
 \begin{equation}
 \bxi_\perp\cdot\nabla\left(p_0 +\frac{B_0^{\,2}}{2}\right) = \frac{B_0^{\,2}}{\mu_0}\,\bxi_\perp\cdot\bkappa.
 \end{equation}
 Hence, Eq.~(\ref{e61}) simplifies to give 
 \begin{align}\label{e65}
\bta\cdot {\bf F}(\bxi) &=\nabla\cdot\left[\gamma\,p_0\,(\nabla\cdot\bxi)\,\bta+ (\bxi_\perp\cdot\nabla p_0-{\bf B}_0\cdot{\bf Q})\,\bta_\perp\right]\nonumber\\[0.5ex]
&\phantom{=}+\frac{B_0^{\,2}}{\mu_0}\,(\nabla\cdot\bxi_\perp)\,(\nabla\cdot\bta_\perp) 
 - \mu_0^{-1}\,({\bf B}_0\cdot\nabla\bxi)\cdot
 ({\bf B}_0\cdot\nabla\bta_\perp)-\gamma\,p_0\,(\nabla\cdot\bxi)\,(\nabla\cdot\bta)\nonumber\\[0.5ex]
 &\phantom{=}-\frac{2\,B_0^{\,2}}{\mu_0}\,(\bxi_\perp\cdot\bkappa)\, (\nabla\cdot\bta_\perp)-\frac{2\,B_0^{\,2}}{\mu_0}\,(\bta_\perp\cdot\bkappa)\,(\nabla\cdot\bxi_\perp) +R.
 \end{align}

Now, 
\begin{align}
\mu_0\,R &= \eta_{\perp\,i}\,B_{0\,k}\,\frac{\partial\xi_{\perp\,j}}{\partial x_k}\,\frac{\partial B_{0\,i}}{\partial x_j}
- \eta_{\perp\,i}\,\xi_{\perp\,k}\,\frac{\partial B_{0\,j}}{\partial x_k}\,\frac{\partial B_{0\,i}}{\partial x_j}
+\xi_{\perp\,j}\,\frac{\partial B_{0\,i}}{\partial x_j}\,B_{0\,k}\,\frac{\partial\eta_{\perp\,i}}{\partial x_k}\nonumber\\[0.5ex]
&= \frac{\partial}{\partial x_k}\!\left(\eta_{\perp\,i}\,B_{0\,k}\,\xi_{\perp\,j}\,\frac{\partial B_{0\,i}}{\partial x_j}\right)
-\xi_{\perp\,j}\,\frac{\partial}{\partial x_k}\!\left(\eta_{\perp\,i}\,B_{0\,k}\,\frac{\partial B_{0\,i}}{\partial x_j}\right)\nonumber\\[0.5ex]
&\phantom{=}- \eta_{\perp\,i}\,\xi_{\perp\,k}\,\frac{\partial B_{0\,j}}{\partial x_k}\,\frac{\partial B_{0\,i}}{\partial x_j}
+\xi_{\perp\,j}\,\frac{\partial B_{0\,i}}{\partial x_j}\,B_{0\,k}\,\frac{\partial\eta_{\perp\,i}}{\partial x_k}\nonumber\\[0.5ex]
&=-\eta_{\perp\,i}\,\xi_{\perp\,j}\,\frac{\partial}{\partial x_k}\!\left(B_{0\,k}\,\frac{\partial B_{0\,i}}{\partial x_j}\right)- \eta_{\perp\,i}\,\xi_{\perp\,j}\,\frac{\partial B_{0\,k}}{\partial x_j}\,\frac{\partial B_{0\,i}}{\partial x_k}\nonumber\\[0.5ex]
&=-\eta_{\perp\,i}\,\xi_{\perp\,j}\,B_{0\,k}\,\frac{\partial^2 B_{0\,i}}{\partial x_j\,\partial x_k}- \eta_{\perp\,i}\,\xi_{\perp\,j}\,\frac{\partial B_{0\,k}}{\partial x_j}\,\frac{\partial B_{0\,i}}{\partial x_k} \nonumber\\[0.5ex]
&=- \eta_{\perp\,i}\,\xi_{\perp\,j}\,\frac{\partial}{\partial x_j}\!\left(B_{0\,k}\,\frac{\partial B_{0\,i}}{\partial x_k}\right),
\end{align}
where the divergence term has integrated to zero because of Eq.~(\ref{e31x}), and use has been made of Eq.~(\ref{e6}). 
However, from Eq.~(\ref{e64}), 
\begin{equation}
\mu_0^{-1}\,B_{0\,k}\,\frac{\partial B_{0\,i}}{\partial x_k} = \frac{\partial}{\partial x_i}\!\left(p_0 + \frac{B_0^{\,2}}{2\,\mu_0}\right).
\end{equation}
Thus,
\begin{equation}
R=  -\eta_{\perp\,i}\,\xi_{\perp\,j}\,\frac{\partial^2}{\partial x_i\,\partial x_j}\!\left(p_0 + \frac{B_0^{\,2}}{2\,\mu_0}\right)=-
(\bta_\perp\,\bxi_\perp : \nabla\nabla)\left(p_0 + \frac{B_0^{\,2}}{2\,\mu_0}\right).
\end{equation}

It follows from Eq.~(\ref{e65}), and the previous equation, that 
 \begin{align}
 \bta\cdot {\bf F}(\bxi) &=\nabla\cdot\left[\gamma\,p_0\,(\nabla\cdot\bxi)\,\bta+ (\bxi_\perp\cdot\nabla p_0-{\bf B}_0\cdot{\bf Q})\,\bta_\perp\right]\nonumber\\[0.5ex]
&\phantom{=}
  - \mu_0^{-1}\,({\bf B}_0\cdot\nabla\bxi_\perp)\cdot
 ({\bf B}_0\cdot\nabla\bta_\perp)-\gamma\,p_0\,(\nabla\cdot\bxi)\,(\nabla\cdot\bta)+\frac{B_0^{\,2}}{\mu_0}\,(\nabla\cdot\bxi_\perp)\,(\nabla\cdot\bta_\perp) 
\nonumber\\[0.5ex]
 &\phantom{=}-\frac{2\,B_0^{\,2}}{\mu_0}\,(\bxi_\perp\cdot\bkappa)\, (\nabla\cdot\bta_\perp)-\frac{2\,B_0^{\,2}}{\mu_0}\,(\bta_\perp\cdot\bkappa)\,(\nabla\cdot\bxi_\perp)\nonumber\\[0.5ex]
 &\phantom{=}
 -(\bta_\perp\,\bxi_\perp : \nabla\nabla)\left(p_0 + \frac{B_0^{\,2}}{2\,\mu_0}\right).
\end{align}
Finally, Eq.~(\ref{intg}) gives 
 \begin{align}\label{e83d}
\int_V \bta\cdot{\bf F}(\bxi)\,d{\bf r} &=\int_V \biggl[- \mu_0^{-1}\,({\bf B}_0\cdot\nabla\bxi_\perp)\cdot
 ({\bf B}_0\cdot\nabla\bta_\perp)-\gamma\,p_0\,(\nabla\cdot\bxi)\,(\nabla\cdot\bta)
\nonumber\\[0.5ex]
 &\phantom{=}+\frac{B_0^{\,2}}{\mu_0}\,(\nabla\cdot\bxi_\perp)\,(\nabla\cdot\bta_\perp) -\frac{2\,B_0^{\,2}}{\mu_0}\,(\bxi_\perp\cdot\bkappa)\, (\nabla\cdot\bta_\perp)\nonumber\\[0.5ex]
 &\phantom{=}
 -\frac{2\,B_0^{\,2}}{\mu_0}\,(\bta_\perp\cdot\bkappa)\,(\nabla\cdot\bxi_\perp)-(\bta_\perp\,\bxi_\perp : \nabla\nabla)\left(p_0 + \frac{B_0^{\,2}}{2\,\mu_0}\right)\biggr]d{\bf r}\nonumber\\[0.5ex]
 &\phantom{=}+ \int_S {\bf n}\cdot\bta_\perp\left(\gamma\,p_0\,\nabla\cdot\bxi + \bxi_\perp\cdot\nabla p_0 - {\bf B}_0\cdot{\bf Q}\right)dS,
\end{align}
where use has been made of Eq.~(\ref{e31x}).

\subsection{Perfectly Conducting Wall}
Suppose, for the sake of simplicity, that the plasma is immediately surrounded by a perfectly conducting wall. In other words, suppose that the plasma's bounding surface, $S$, corresponds to the inner
boundary of the wall. Of course, ${\bf B}= {\bf E}=0$ inside the wall, where ${\bf E}$ is the electric field. 
Standard electromagnetic boundary conditions 
require that
\begin{align}\label{e27a}
{\bf n}\times {\bf E}&= 0,\\[0.5ex]
{\bf n} \cdot {\bf B}&=0,\label{e27b}
\end{align}
on $S$. 

The plasma velocity is assumed to be dominated by the ${\bf E}\times {\bf B}$ drift velocity. In other words,
\begin{equation}
{\bf v} = \frac{{\bf E}\times {\bf B}}{B^2}.
\end{equation}
If we write ${\bf E}({\bf r},t)= {\bf E}_0({\bf r})+ {\bf E}_1({\bf r})\,\exp(-{\rm i}\,\omega\,t)$
then it is clear from Eq.~(\ref{e7}) that ${\bf E}_0={\bf 0}$. It follows from Eq.~(\ref{e33q}), and the previous equation, that
\begin{equation}\label{e30a}
{\bf v}_1 = \frac{{\bf E}_1\times {\bf B}_0}{B_0^{\,2}}.
\end{equation}
Now, Eqs.~(\ref{e27a}) and (\ref{e27b}) imply that
\begin{equation}\label{e31xx}
{\bf n} \cdot {\bf B}_0=0
\end{equation}
on $S$, 
in accordance with Eq.~(\ref{e31x}), 
as well as 
\begin{align}
{\bf n}\times {\bf E}_1&= 0,\label{e31a}\\[0.5ex]
{\bf n} \cdot {\bf B}_1&=0.\label{e33}
\end{align}
Equations~(\ref{e30a}) and (\ref{e31a}) yield
\begin{equation}
{\bf n}\cdot{\bf v}_1=0
\end{equation}
on $S$. Hence, it follows from Eqs.~(\ref{e15}) and (\ref{e31xx}) that
\begin{equation}\label{e35}
{\bf n}\cdot\bxi = {\bf n}\cdot\bxi_\perp = 0
\end{equation}
on $S$. In other words, the local plasma displacement normal to the wall is constrained to be zero. 

 Making use of Eqs.~(\ref{e25}), (\ref{e36}) and (\ref{e33}),
we also require 
\begin{equation}\label{e81}
{\bf n}\cdot\nabla\times(\bxi_\perp\times {\bf B}_0) = 0
\end{equation}
on $S$. Now, Eqs.~(\ref{e31xx}) and (\ref{e35})  imply that
\begin{equation}
\bxi_\perp\times {\bf B}_0= f\,{\bf n}
\end{equation}
on $S$, where $f$ is some scalar.
Thus, the boundary condition (\ref{e81}) becomes
\begin{equation}
{\bf n}\cdot \nabla\times (f\,{\bf n}) = {\bf n}\cdot[\nabla f\times {\bf n} + f\,\nabla\times {\bf n})= f\,{\bf n}\cdot\nabla\times {\bf n} = 0.
\end{equation}
However, because the bounding surface of the plasma, $S$, is also an equilibrium magnetic flux-surface, it must correspond to a contour of the equilibrium poloidal
magnetic flux,  $\psi({\bf r})$. 
It follows that
\begin{equation}
{\bf n} = \frac{\nabla\psi}{|\nabla\psi|}.
\end{equation}
Thus,
\begin{equation}\label{e98p}
{\bf n}\cdot\nabla\times {\bf n} = \frac{\nabla\psi}{|\nabla\psi|}\cdot \left[\nabla\left(\frac{1}{|\nabla\psi|}\right)\times \nabla\psi\right]=0.
\end{equation}
Hence, we deduce that the boundary condition (\ref{e81}) is automatically satisfied provided that the boundary condition (\ref{e35}) is satisfied. 

\subsection{Self-Adjoint Property of Force Operator}
We wish to demonstrate that the force operator is {\em self-adjoint}. In other words, we wish to prove that 
\begin{equation}\label{e26}
\int_V \bta\cdot{\bf F}(\bxi)\,d{\bf r} = \int_V \bxi\cdot {\bf F}(\bta)\,d{\bf r},
\end{equation}
where  $\bxi({\bf r})$ and $\bta({\bf r})$ are two arbitrary vector fields that satisfy the 
physical boundary conditions
\begin{equation}\label{e27}
{\bf n}\cdot\bxi_\perp={\bf n}\cdot\bta_\perp=0
\end{equation}
on $S$.
However, it follows from Eq.~(\ref{e83d}), and the previous boundary conditions, that
 \begin{align}
\int_V \bta\cdot{\bf F}(\bxi)\,d{\bf r} &=\int_V \bigg[- \mu_0^{-1}\,({\bf B}_0\cdot\nabla\bxi_\perp)\cdot
 ({\bf B}_0\cdot\nabla\bta_\perp)-\gamma\,p_0\,(\nabla\cdot\bxi)\,(\nabla\cdot\bta)
\nonumber\\[0.5ex]
 &\phantom{=}+\frac{B_0^{\,2}}{\mu_0}\,(\nabla\cdot\bxi_\perp)\,(\nabla\cdot\bta_\perp) -\frac{2\,B_0^{\,2}}{\mu_0}\,(\bxi_\perp\cdot\bkappa)\, (\nabla\cdot\bta_\perp)-\frac{2\,B_0^{\,2}}{\mu_0}\,(\bta_\perp\cdot\bkappa)\,(\nabla\cdot\bxi_\perp)\nonumber\\[0.5ex]
 &\phantom{=}
 -(\bta_\perp\,\bxi_\perp : \nabla\nabla)\left(p_0 + \frac{B_0^{\,2}}{2\,\mu_0}\right)\biggr]d{\bf r}=\int_V \bxi\cdot{\bf F}(\bta)\,d{\bf r},
\end{align}
which immediately proves the result, because we can make the substitution $\bta\leftrightarrow \bxi$ in the middle integral without
changing its value. 

\subsection{Reality of $\omega^2$}
Consider a discrete normal mode with eigenvalue $\omega^2$ and eigenvector $\bxi({\bf r})$. 
Taking the scalar product of the eigenmode equation, Eq.~(\ref{e24a}), with $\bxi^\ast({\bf r})$, and integrating over the plasma volume, we obtain
\begin{equation}
\omega^2\int_V\rho_0\,|\bxi|^2\,d{\bf r} = - \int_V\bxi^\ast\cdot {\bf F}(\bxi)\,d{\bf r}.
\end{equation}
Likewise, taking the scalar product of the complex conjugate of Eq.~(\ref{e24a}) with $\bxi({\bf r})$, and integrating over the plasma
volume, we get
\begin{equation}
(\omega^2)^\ast\int_V\rho_0\,|\bxi|^2\,d{\bf r} = - \int_V\bxi\cdot {\bf F}(\bxi^\ast)\,d{\bf r}.
\end{equation}
Here, we have made use of the fact that $[{\bf F}(\bxi)]^\ast={\bf F}(\bxi^\ast)$, because ${\bf F}(\bxi)$ is a real linear differential operator
that acts on $\bxi$. 
Taking the difference between the previous two equations, we obtain
\begin{equation}
\left[\omega^2 - (\omega^2)^\ast\right]\int_V\rho_0\,|\bxi|^2\,d{\bf r}=  - \int_V\bxi^\ast\cdot {\bf F}(\bxi)\,d{\bf r}+ \int\bxi\cdot {\bf F}(\bxi^\ast)\,d{\bf r}.
\end{equation}
However, the self-adjoint property of the force operator, (\ref{e26}), which is valid as long as the eigenvector satisfies the physical
boundary condition ${\bf n}\cdot\bxi_\perp=0$ on $S$, 
yields
\begin{equation}
\left[\omega^2 - (\omega^2)^\ast\right]\int_V\rho_0\,|\bxi|^2\,d{\bf r}=0.
\end{equation}
Given that the integral in the previous expression is positive-definite, we deduce that 
\begin{equation}
\omega^2 = (\omega^2)^\ast.
\end{equation}
In other words,  the eigenvalue, $\omega^2$,  is a {\em real}\/ quantity. It immediately follows  from the eigenmode
equation, (\ref{e24a}), as well as the fact that all of the differential operators appearing in ${\bf F}(\bxi)$ are real,  that the eigenvector, 
$\bxi$, can be chosen to be real. However, it is often convenient to multiply $\bxi$ by a complex number of modulus unity. 

In terms of the usual definition of exponential stability, a discrete normal mode with $\omega^2>0$ corresponds to a pure oscillation, and would, therefore,
be considered stable. Conversely, a discrete mode with $\omega^2<0$ has one branch that grows exponentially in time, and
would, therefore, be considered unstable. Clearly, the transition from stability to instability occurs when $\omega^2=0$. 

\subsection{Orthogonality of Normal Modes}\label{orth}
Consider two discrete normal modes. Let the first have the eigenvalue $\omega_a^{\,2}$ and the eigenvector $\bxi_a({\bf r})$. Let the
second have the eigenvalue $\omega_b^{\,2}$ and the eigenvector $\bxi_b({\bf r})$. Let the modes satisfy the
physical boundary conditions 
\begin{equation}\label{e91}
{\bf n}\cdot\bxi_{a\,\perp}={\bf n}\cdot\bxi_{b\,\perp} = 0
\end{equation}
at the wall. It follows from Eq.~(\ref{e24a}) that
\begin{align}
\omega_a^{\,2}\,\rho_0\,\bxi_a &= -{\bf F}(\bxi_a),\\[0.5ex]
\omega_b^{\,2}\,\rho_0\,\bxi_b &=- {\bf F}(\bxi_b).
\end{align}
Forming the scalar product of the first equation with $\bxi_b^{\,\ast}$, and the scalar product of the complex conjugate of the second with $\bxi_a$, taking the difference, and integrating over the
plasma volume, we obtain
\begin{equation}
(\omega_a^{\,2}-\omega_b^{\,2})\int_V \rho_0\,\bxi_b^{\,\ast}\cdot\bxi_a\,d{\bf r} = - \int_V [\bxi_b^{\,\ast}\cdot {\bf F}(\bxi_a)-\bxi_a\cdot {\bf F}(\bxi_b^{\,\ast})]\,d{\bf r}.
\end{equation}
However, the self-adjoint property of the force operator, (\ref{e26}), yields
\begin{equation}
(\omega_a^{\,2}-\omega_b^{\,2})\int_V \rho_0\,\bxi_b^{\,\ast}\cdot\bxi_a\,d{\bf r}=0.
\end{equation}
Hence, we deduce that two discrete normal modes with different eigenvalues are {\rm orthogonal}, in the sense that 
\begin{equation}
\int_V \rho_0\,\bxi_b^{\,\ast}\cdot\bxi_a\,d{\bf r}=0.
\end{equation}

Of course, the previous proof fails if we encounter multiple  distinct normal modes that share the eigenvalue. However, any
linear combination of such degenerate modes is also a valid normal mode with the shared eigenvalue, and it is always possible to form linear
combinations that are mutually orthogonal. With this caveat, we can state that discrete normal modes are, or can be chosen to be,  mutually orthogonal. 

\subsection{Variational Principle}
Taking the scalar product of the eigenmode equation, (\ref{e24a}), with $(1/2)\,\bxi^\ast$, integrating over the
plasma volume, and rearranging,  we obtain
\begin{equation}\label{e101}
\omega^2 = \frac{\delta W(\bxi^\ast,\bxi)}{K(\bxi^\ast,\bxi)}
\end{equation}
where
\begin{align}
K(\bxi^\ast,\bxi) &= \frac{1}{2}\int_V\rho_0\,|\bxi|^2\,d{\bf r},\label{e99}\\[0.5ex]
\delta W(\bxi^\ast,\bxi) &= -\frac{1}{2}\int_V \bxi^\ast\cdot {\bf F}(\bxi)\,d{\bf r}.\label{e100}
\end{align}
Here, the quantity $\delta W(\bxi^\ast,\bxi)$ represents the change in potential energy associated with the perturbation, because it is equal to the work
done against the force density, ${\bf F}(\bxi)$, in displacing the plasma by an amount $\bxi$. The quantity $K(\bxi^\ast,\bxi)$ is
proportional to the kinetic energy. 

We wish to prove the so-called {\em variational principle}, which states that any  $\bxi({\bf r})$ function that satisfies the physical boundary condition, and for which $\omega^2$ becomes an extremum, satisfies the eigenmode equation, (\ref{e24a}). The proof follows by letting $\bxi\rightarrow \bxi + \delta\bxi$ and $\omega^2\rightarrow \omega^2+\delta\omega^2$,
and setting $\delta\omega^2=0$ (corresponding to $\omega^2$ being an extremum). 
Neglecting terms that are quadratic in perturbed quantities, we obtain
\begin{equation}
\omega^2+\delta\omega^2= \frac{\delta W(\bxi^\ast,\bxi)+\delta W(\delta\bxi^\ast,\bxi)+\delta W(\bxi^\ast,\delta\bxi)}
{K(\bxi^\ast,\bxi)+K(\delta\bxi^\ast,\bxi)+K(\bxi^\ast,\delta\bxi)}.
\end{equation}
Rearranging the previous equation, making use of Eq.~(\ref{e101}), and again neglecting terms that are quadratic in perturbed quantities,
we get
\begin{equation}
\delta \omega^2 = \frac{\delta W(\delta\bxi^\ast,\bxi)+\delta W(\bxi^\ast,\delta\bxi)-\omega^2\,[K(\delta\bxi^\ast,\bxi)+K(\bxi^\ast,\delta\bxi)]}
{K(\bxi^\ast,\bxi)}. 
\end{equation}
Setting $\delta\omega^2$ to zero yields
\begin{equation}
\omega^2\,K(\delta\bxi^\ast,\bxi)-\delta W(\delta\bxi^\ast,\bxi) + \omega^2\,K(\bxi^\ast,\delta\bxi) + \delta W(\bxi^\ast,\delta\bxi)=0.
\end{equation}
Making use of Eqs.~(\ref{e99}) and (\ref{e100}), as well as the self-adjoint property of the force operator, (\ref{e26}), we get
\begin{equation}
\int \left\{\delta\bxi^\ast\cdot[\omega^2\,\rho_0\,\bxi+{\bf F}(\bxi)]+ \delta\bxi\cdot[\omega^2\,\rho_0\,\bxi^\ast + {\bf F}(\bxi^\ast)]\right\}d{\bf r} = 0.
\end{equation}
However, in order for $\omega^2$ to be an extremum, the previous equation must hold for arbitrary $\delta\bxi$ and $\delta\bxi^\ast$. Hence, 
we obtain
\begin{equation}
-\omega^2\,\rho_0\,\bxi = {\bf F}(\bxi),
\end{equation}
which is identical to Eq.~(\ref{e24a}). 

The variation approach constitutes a relatively straightforward method of determining the ideal stability of the plasma. 
Consider a set of $N$ convenient basis functions, $\bta_n({\bf r})$, that all satisfy the physical boundary condition
\begin{equation}
{\bf n}\cdot\bta_n=0
\end{equation}
on $S$. Let the functions be chosen so as to be orthonormal with respect to the weight function $\rho_0$. In other words,
\begin{equation}
K(\bta_n^{\,\ast},\bta_m) = \delta_{nm}.
\end{equation}
[See Eq.~(\ref{e99})]. We can compute the matrix elements
\begin{equation}
W_{mn} = \delta W(\bta_m^{\,\ast},\bta_n)= -\frac{1}{2}\int_V\bta_m^{\,\ast}\cdot {\bf F}(\bta_n)\,d{\bf r}.
\end{equation}
The self-adjoint nature of the force operator ensures that $W_{mn}$ is Hermitian: that is, 
\begin{equation}
W_{nm}^{\,\ast}= W_{mn}.
\end{equation}
Let
\begin{equation}
\bta = \sum_{n=1,N} a_n\,\bta_n,
\end{equation}
where
\begin{equation}
\sum_{n=1,N} |a_n|^2 = 1,
\end{equation}
which ensures that 
\begin{equation}
K(\bta^\ast,\bta) = 1.
\end{equation}
Note that the previous equation can only be satisfied if the $\bta_n$ are well-behaved at the magnetic axis. 
Consider 
\begin{equation}
{\mit\Lambda} = \frac{\delta W(\bta^\ast,\bta)}{K(\bta^\ast,\bta)}= \sum_{n=1,N}\sum_{m=1,N}a_m^{\,\ast}\,W_{mn}\,a_n.
\end{equation}
In accordance with the variation principle, we wish to vary the $a_n$ in such a manner as to minimize ${\mit\Lambda}$. 
Now, because $W_{nm}$ is an Hermitian matrix of dimension $N$, it possess $N$ real eigenvalues, $\lambda_p$, with the
associated mutually orthogonal eigenvectors, ${\bf x}_p$. Thus,
\begin{equation}
\sum_{n=1,N}W_{mn}\,x_{np}=  \lambda_p\,x_{mp},
\end{equation}
where $x_{np}$ is the $n$th component of ${\bf x}_p$. We can normalize the eigenvectors such that
\begin{equation}
\sum_{n=1,N} x_{np}^{\,\ast}\,x_{nq} = \delta_{pq}.
\end{equation}
Let us write
\begin{equation}
a_n =\sum_{p=1,N}b_p\,x_{np},
\end{equation}
where
\begin{equation}
\sum_{p=1,N}|b_p|^2 = 1.
\end{equation}
The previous five equations can be combined to give
\begin{equation}
{\mit\Lambda} = \sum_{p=1,N}\lambda_p\,|b_p|^2= \lambda_1 +\sum_{p=2,N}(\lambda_p-\lambda_1)\,|b_p|^2.
\end{equation}
Suppose that $\lambda_1<\lambda_2<\lambda_3$, et cetera. It is clear that the second term on the right-hand side of the
previous equation is positive-definite. Hence, we can minimize ${\mit\Lambda}$ by choosing $b_p = \delta_{p1}$. In this case,
${\mit\Lambda}=\lambda_1$. Thus, our estimate for the lowest ideal-MHD eigenvalue of the plasma is
\begin{equation}
\omega_1^{\,2}= \lambda_1,
\end{equation}
with the associated properly normalized eigenvector
\begin{equation}
\bxi_1 = \sum_{n=1,N} x_{n1}\,\bta_{n}.
\end{equation}
If we set $b_1=0$ then similar reasoning reveals that our estimate for the next lowest eigenvalue is
\begin{equation}
\omega_2^{\,2}=\lambda_2,
\end{equation}
with the associated eigenvector
\begin{equation}
\bxi_2 = \sum_{n=1,N} x_{n2}\,\bta_{n}.
\end{equation}
Obviously, we can extent this argument to find estimates for the lowest $N$ eigenvalues, and the associated eigenvectors, of the plasma. If
the basis functions form a complete set, and $N\rightarrow\infty$,  then the variational approach is equivalent to solving the eigenmode equation, (\ref{e24a}), subject to
the physical boundary condition (\ref{e35}). 
Finally, the variational principle guarantees that the true value of $\omega_1^{\,2}$ is less than or equal to $\lambda_1$. Thus, 
if our calculation yields $\lambda_1<0$ then we can be sure that the plasma is ideally unstable. 

\subsection{Energy Principle}
The ideal-MHD {\em energy principle}\/ states that if
\begin{equation}
\delta W(\bxi^\ast,\bxi) \geq 0
\end{equation}
for all allowable plasma displacements (i.e., bounded in energy, and satisfying appropriate boundary conditions) then
the plasma is ideally stable. In other words, there exist no normal modes with $\omega^2<0$. Conversely, if $\delta W(\bxi^\ast,\bxi)$ is negative
for any allowable displacement then the plasma is ideally unstable. In other words, there exists at least one normal mode with $\omega^2<0$. 

The proof of the energy principle is straightforward if one assumes that the normal modes are discrete, and form a complete set of
basis functions, $\bxi_n({\bf r})$, each satisfying
\begin{equation}
-\omega_n^{\,2}\,\rho_0\,\bxi_n = {\bf F}({\bxi}_n).
\end{equation}
In this case, any arbitrary trial function, $\bxi({\bf r})$, can be represented as
\begin{equation}
\bxi({\bf r}) = \sum_n a_n\,\bxi_n({\bf r}).
\end{equation}
We demonstrated in Sect.~\ref{orth} that the normal modes are orthogonal with respect to the weight function $\rho_0({\bf r})$. Let us, now, 
normalize them such that 
\begin{equation}
\int_V \rho_0\,\bxi_n^{\,\ast}\,\cdot\bxi_m\,d{\bf r} = \delta_{nm}.
\end{equation}
It follows that
\begin{align}
\delta W(\bxi^\ast,\bxi) &= -\frac{1}{2}\int_V\bxi^\ast\cdot {\bf F}(\bxi)\,d{\bf r}\nonumber\\[0.5ex]
&= -\frac{1}{2}\sum_{n,m}a_n^{\ast}\,a_m\int_V \bxi_n^{\,\ast}\cdot {\bf F}(\bxi_m)\,d{\bf r}=\frac{1}{2}\sum_{n,m}a_n^{\ast}\,a_m\,\omega_m^{\,2}\int_V \rho_0\,\bxi_n^{\,\ast}\cdot \bxi_m\,d{\bf r}\nonumber\\[0.5ex]
&= \frac{1}{2}\sum_n|a_n|^2\,\omega_n^{\,2}.
\end{align}
The previous equation implies that if some $\bxi({\bf r})$ can be found for which $\delta W(\bxi^\ast,\bxi) <0$ then at least
one of the $\omega_n^{\,2}$ is negative, indicating instability. Conversely, if $\delta W(\bxi^\ast,\bxi)\geq 0$ for all
$\bxi({\bf r})$ then all of the $\omega_n^{\,2}$ are non-negative, indicating stability. 

Unfortunately, the previous proof of the energy principle is not completely valid because it assumes that  the normal modes of the plasma
have discrete frequencies. In reality, while purely growing or decaying normal modes, characterized by $\omega^2<0$, do indeed have discrete frequencies, oscillatory modes, characterized by 
$\omega^2>0$, can have frequencies that lie in a continuous range. Hence, we need a more general proof of the energy principle. 

Let
us work in the real time domain. The perturbed kinetic energy of the plasma is [see Eq.~(\ref{e99})] 
\begin{equation}
K\left(\frac{\partial\tilde{\bxi}}{\partial t},\frac{\partial\tilde{\bxi}}{\partial t}\right),
\end{equation}
whereas the perturbed potential energy is  [see Eq.~(\ref{e100})]
\begin{equation} \label{e131ff}
\delta W(\tilde{\bxi},\tilde{\bxi}) = - \frac{1}{2}\,\int_V \tilde{\bxi}\cdot {\bf F}(\tilde{\bxi})\,d{\bf r},
\end{equation}
Thus, the total perturbed plasma energy
is 
\begin{align}
H(t)& = K\left(\frac{\partial\tilde{\bxi}}{\partial t},\frac{\partial\tilde{\bxi}}{\partial t}\right) + \delta W(\tilde{\bxi},\tilde{\bxi})= \frac{1}{2}\int_V\left[\rho_0\left(\frac{\partial\tilde{\bxi}}{\partial t}\right)^2 - \tilde{\bxi}\cdot{\bf F}(\tilde{\bxi})\right]d{\bf r}.
\end{align}
Note that the kinetic energy is positive definite. 
It follows from the previous equation that 
\begin{equation}
\frac{dH}{dt} = \int \frac{\partial\tilde{\bxi}}{\partial t}\cdot\left[\rho_0\,\frac{\partial^2\tilde{\bxi}}{\partial t^2}-{\bf F}(\tilde{\bxi})\right]=0,
\end{equation}
where use has been made of the plasma equation of motion, (\ref{e28f}), as well as the self-adjoint property of the force operator. The
previous equation demonstrates that $H(t)= H_0$, where $H_0$ is constant in time. In other words,
\begin{equation}\label{e125f}
H_0 = K\left(\frac{\partial\tilde{\bxi}}{\partial t},\frac{\partial\tilde{\bxi}}{\partial t}\right) + \delta W(\tilde{\bxi},\tilde{\bxi}).
\end{equation}

Let us assume that $\delta W(\tilde{\bxi},\tilde{\bxi})$ is positive for all allowable $\tilde{\bxi}({\bf r},t)$. The previous equation implies that
\begin{equation}
\delta W(\tilde{\bxi},\tilde{\bxi}) = H_0 - K\left(\frac{\partial\tilde{\bxi}}{\partial t},\frac{\partial\tilde{\bxi}}{\partial t}\right) > 0.
\end{equation}
However, exponential growth of the kinetic energy, $K$,  would eventually violate the previous equation.
Hence, we deduce that the requirement that $\delta W(\tilde{\bxi},\tilde{\bxi})>0$ for all allowable  plasma displacements is a sufficient condition to prohibit exponential growth. 
However, we cannot rule out linear growth, like $t$, which leaves the kinetic energy invariant. On the other hand, we can  definitely state that the displacement cannot
grow faster than $t$. 

To show the necessity of the energy principle, let us assume that there exists a perturbation, $\bta({\bf r})$, that satisfies the physical boundary conditions,
 and is such that
$\delta W(\bta,\bta)<0$. Consider a plasma displacement, $\tilde{\bxi}({\bf r},t)$, that satisfies the initial conditions
\begin{align}
\tilde{\bxi}({\bf r},0)&= \bta({\bf r}),\\[0.5ex]
\frac{\partial\tilde{\bxi}({\bf r},0)}{\partial t}&= {\bf 0}.
\end{align}
Energy conservation implies that
\begin{equation}
H_0 = \left[K\left(\frac{\partial\tilde{\bxi}}{\partial t},\frac{\partial\tilde{\bxi}}{\partial t}\right) + \delta W(\tilde{\bxi},\tilde{\bxi})\right]_{t=0}= \delta W(\bta,\bta) < 0.
\end{equation}
Consider 
\begin{equation}
I(t) \equiv K(\tilde{\bxi},\tilde{\bxi})=\frac{1}{2}\int_V \rho_0\,\tilde{\bxi}^2\,d{\bf r}.
\end{equation}
It follows that
\begin{align}\label{e131f}
\frac{d^2 I}{dt^2} &= \int_V \rho_0\left[\left(\frac{\partial\tilde{\bxi}}{\partial t}\right)^2+ \tilde{\bxi}\cdot\frac{\partial^2\tilde{\bxi}}{\partial t^2}\right]d{\bf r}
=  \int_V \rho_0\left[\left(\frac{\partial\tilde{\bxi}}{\partial t}\right)^2+ \tilde{\bxi}\cdot{\bf F}(\tilde{\bxi})\right]d{\bf r}\nonumber\\[0.5ex]
&= 2\,K\left(\frac{\partial\tilde{\bxi}}{\partial t},\frac{\partial\tilde{\bxi}}{\partial t}\right) - 2\,\delta W(\tilde{\bxi},\tilde{\bxi}) ,
\end{align}
where use has been made of Eq.~(\ref{e28f}). 
The energy conservation equation, (\ref{e125f}), yields
\begin{align}\label{e153}
\frac{d^2 I}{dt^2}= 4\,K\left(\frac{\partial\tilde{\bxi}}{\partial t}, 
\frac{\partial\tilde{\bxi}}{\partial t}\right) -2\,H_0,
\end{align}
The previous result is known as the {\em virial theorem}. For the case in hand, given that the kinetic energy is positive definite, and $H_0<0$, we deduce that
\begin{align}
\frac{d^2 I}{dt^2} >-2\,H_0>0.
\end{align}
The previous equation implies that if $\delta W(\bta,\bta)<0$ then $I(t)$ grows without
bound as $t\rightarrow \infty$, at least as fast as $t^2$,  showing that $\tilde{\bxi}$ increases at least as fast as $t$. 

To show that $\delta W(\bta,\bta)<0$ necessitates exponential growth, let us define 
\begin{equation}
\lambda = -\frac{\delta W(\bta,\bta)}{K(\bta,\bta)} > 0.
\end{equation}
Consider a plasma displacement, $\tilde{\bxi}({\bf r},t)$, that satisfies the initial conditions
\begin{align}
\tilde{\bxi}({\bf r},0)&= \bta({\bf r}),\\[0.5ex]
\frac{\partial\tilde{\bxi}({\bf r},0)}{\partial t}&= \sqrt{\lambda}\,\bta({\bf r}).
\end{align}
Note that these are the appropriate initial conditions for a displacement that grows as $\exp(\sqrt{\lambda}\,t)$.
Energy conservation implies that
\begin{equation}
H_0 = \left[K\left(\frac{\partial\tilde{\bxi}}{\partial t},\frac{\partial\tilde{\bxi}}{\partial t}\right) + \delta W(\tilde{\bxi},\tilde{\bxi})\right]_{t=0}= \lambda\,K(\bta,\bta) 
+\delta W(\bta,\bta)=0.
\end{equation}
Thus, the virial theorem, (\ref{e153}), yields
\begin{equation}\label{e159}
\ddot{I}= 4\,K\left(\frac{\partial\tilde{\bxi}}{\partial t}, 
\frac{\partial\tilde{\bxi}}{\partial t}\right) > 0,
\end{equation}
where $\dot{\phantom{a}}\equiv d/dt$. Now,
\begin{equation}\label{e160}
\dot{I}^{\,2}(t) = 4\left[K\left(\frac{\partial\tilde{\bxi}}{\partial t}, 
\tilde{\bxi}\right)\right]^2 \leq 4\,K\left(\frac{\partial\tilde{\bxi}}{\partial t}, \frac{\partial\tilde{\bxi}}{\partial t}\right)\,K(\tilde{\bxi},\tilde{\bxi}) = \ddot{I}(t)\,I(t),
\end{equation}
where use has been made of the Schwartz inequality. Furthermore,
\begin{equation}
\dot{I}(0) = 2\,\sqrt{\lambda}\,K(\bta,\bta)=2\,\sqrt{\lambda}\,I(0) >0,
\end{equation}
which implies, from Eq.~(\ref{e159}), that $\dot{I}(t)>0$ for $t>0$. Thus, we can divide the inequality (\ref{e160}) by $\dot{I}(t)\,I(t)$ to give
\begin{equation}
\frac{\dot{I}(t)}{I(t)} \leq \frac{\ddot{I}(t)}{\dot{I}(t)}.
\end{equation}
Integration with respect to time yields
\begin{equation}
\ln\left[\frac{I(t)}{I(0)}\right] \leq \ln \left[\frac{\dot{I}(t)}{\dot{I}(0)}\right]= \ln\left[\frac{\dot{I}(t)}{2\sqrt{\lambda}\,I(0)}\right],
\end{equation}
which yields
\begin{equation}
\frac{\dot{I}(t)}{I(t)}\geq 2\sqrt{\lambda},
\end{equation}
which can be integrated to give
\begin{equation}
I(t) \geq I(0)\,\exp\left(2\sqrt{\lambda}\,t\right).
\end{equation}
Consequently, $\tilde{\bxi}$ grows at least as fast as $\exp(\sqrt{\lambda}\,t)$. 
The implication is that any allowed perturbation that makes $\delta W<0$ will grow exponentially in time. Consequently, $\delta W>0$ for all
allowed perturbations is a necessary condition to prohibit exponential growth. 

\subsection{Vacuum Region}
Let us, now,  consider the much more realistic case in which the plasma is not immediately surrounded by a perfectly
conducting wall, but is, instead, surrounded by a vacuum region, which, in turn, is surrounded by
a perfectly conducting wall. Let the plasma occupy the toroidal volume $V$, and let $S$ be its bounding surface. Let the vacuum occupy the annular
toroidal volume $V'$
that lies between $S$ and the inner surface of the wall, which corresponds to the surface $S'$. 

Let ${\bf n}$ represent an outward-directed unit normal
to either $S$ or $S'$. 
Let $\hat{\bf B}({\bf r},t)$ be the magnetic field in the vacuum region. By analogy with our previous analysis, 
\begin{equation}
\hat{\bf B}({\bf r},t)= \hat{\bf B}_0({\bf r}) + \hat{\bf Q}({\bf r})\,\exp(-{\rm i}\,\omega\,t).
\end{equation}
 Now, by definition, $S$ corresponds to an equilibrium magnetic flux-surface,
which implies that
\begin{align}\label{e166}
{\bf n}\cdot{\bf B}_0 &=0,\\[0.5ex]
{\bf n}\cdot\hat{\bf B}_0 &=0\label{e166a}
\end{align}
on $S$. The conducting wall must also lie on an equilibrium magnetic flux-surface (because the equilibrium field cannot
penetrate it), which means that
\begin{equation}\label{e167}
{\bf n}\cdot\hat{\bf B}_0 =0
\end{equation}
on $S'$. Furthermore, the perturbed magnetic field also cannot penetrate the wall, which implies that
\begin{equation}
{\bf n}\cdot\hat{\bf Q}=0
\end{equation}
on $S'$.  Finally, equilibrium force-balance at the plasma/vacuum interface requires that
\begin{equation}\label{e171v}
p_0 + \frac{B_0^{\,2}}{2\,\mu_0} = \frac{\hat{B}_0^{\,2}}{2\,\mu_0}
\end{equation}
on $S$. 

\subsection{Matching Conditions at Plasma/Vacuum Interface}
The matching conditions for perturbed quantities at the plasma/vacuum interface, $S$,  are complicated by the fact that the interface can
displace. The fundamental matching conditions are [see Eqs.~(\ref{e166}), (\ref{e166a}), and (\ref{e171v})]
\begin{align}
{\bf n}\cdot{\bf B}& = 0,\label{e171}\\[0.5ex]
{\bf n}\cdot\hat{\bf B} &= 0,\label{e171a}\\[0.5ex]
\left\llbracket p+ \frac{B^2}{2\,\mu_0}\right\rrbracket &= 0\label{e172}
\end{align}
on $S$, where $\llbracket A\rrbracket$ denotes the difference between $A$ on the outer and the inner side of $S$. In the previous
two equations, ${\bf n}$ denotes the normal to the {\em perturbed}\/ plasma/vacuum interface. As before, Eqs.~(\ref{e171}) and (\ref{e171a}) ensure that the
perturbed interface remains a magnetic flux-surface, whereas Eq.~(\ref{e172}) is an expression of force balance across the interface. 

Let $d{\bf l}$ be a line element that lies in the surface $S$. The element is convected by the plasma flow. Let $d{\bf l}_0$ be the unperturbed
line element. 
The position vectors of the two ends of the element are ${\bf r}$ and ${\bf r}+d{\bf l}_0$, where ${\bf r}$ is a position vector
of a point that lies on the unperturbed interface. Under the influence of the plasma displacement
\begin{align}
{\bf r} &\rightarrow {\bf r} +\bxi(\bf r),\\[0.5ex]
{\bf r}+ d{\bf l}_0&\rightarrow {\bf r}+ d{\bf l}_0 + \bxi({\bf r}+d{\bf l}_0) \simeq {\bf r}+ d{\bf l}_0+\bxi({\bf r})+d{\bf l}_0\cdot\nabla\bxi.
\end{align}
Taking the difference between the previous two equations, we deduce that
\begin{equation}
d{\bf l}_0\rightarrow d{\bf l}_0 + d{\bf l}_0\cdot\nabla\bxi.
\end{equation}

Let ${\bf n}_0$ be the unit outward-directed normal to the unperturbed interface at position vector ${\bf r}$, and let
${\bf n}_0+{\bf n}_1$ be the unit outward-directed normal to the perturbed interface. It follows from Eqs.~(\ref{e166}) and (\ref{e166a})
that
\begin{align}\label{e166x}
{\bf n}_0\cdot{\bf B}_0 &=0,\\[0.5ex]
{\bf n}_0\cdot\hat{\bf B}_0 &=0.\label{e166ax}
\end{align}
 The fact that $d{\bf l}_0$ lies
in the unperturbed interface implies that
\begin{equation}
{\bf n}_0\cdot d{\bf l}_0=0.
\end{equation}
The condition that the displaced
line element remain in the perturbed interface is
\begin{equation}
0=({\bf n}_0+{\bf n}_1)\cdot( d{\bf l}_0 + d{\bf l}_0\cdot\nabla\bxi),
\end{equation}
which yields
\begin{equation}
0\simeq {\bf n}_0\cdot d{\bf l}_0+d{\bf l}_0\cdot[{\bf n}_1+(\nabla\bxi)\cdot{\bf n}_0]= d{\bf l}_0\cdot[{\bf n}_1+(\nabla\bxi)\cdot{\bf n}_0].
\end{equation}
It follows that
\begin{equation}
{\bf n}_1=- (\nabla\bxi)\cdot{\bf n}_0+ \blambda,
\end{equation}
where $\blambda$ is perpendicular to $d{\bf l}_0$. However, $d{\bf l}_0$ can have any direction within the
unperturbed interface, and the only unit vector that is perpendicular to every possible orientation of  $d{\bf l}_0$ 
is ${\bf n}_0$. It follows that $\blambda= \mu\,{\bf n}_0$. Now, $|{\bf n}_0|= |{\bf n}_0+{\bf n}_1|=1$, which
implies that ${\bf n}_0\cdot{\bf n}_1\simeq 0$. It follows from the previous equation that
$\mu= {\bf n}_0\cdot(\nabla\bxi)\cdot{\bf n}_0$. Hence, we deduce that
\begin{equation}\label{e181}
{\bf n}_1 = - (\nabla\bxi)\cdot{\bf n}_0 + [{\bf n}_0\cdot(\nabla\bxi)\cdot{\bf n}_0]\,{\bf n}_0.
\end{equation}

Equation~(\ref{e171}) gives
\begin{align}
0&= ({\bf n}_0+{\bf n}_1)\,\cdot ({\bf B}_0+\bxi\cdot\nabla{\bf B}_0 + {\bf Q})\nonumber\\[0.5ex]
&\simeq {\bf n}_0\cdot{\bf B}_0+{\bf n}_0\cdot{\bf Q} + {\bf n}_0\cdot(\bxi\cdot\nabla{\bf B}_0) + {\bf n}_1\cdot{\bf B}_0\nonumber\\[0.5ex]
&= {\bf n}_0\cdot{\bf Q} +{\bf n}_0\cdot(\bxi\cdot\nabla{\bf B}_0) - {\bf B}_0\cdot(\nabla\bxi)\cdot{\bf n}_0\nonumber\\[0.5ex]
&= {\bf n}_0\cdot{\bf Q}-{\bf n}_0\cdot({\bf B}_0\cdot\nabla \bxi-\bxi\cdot\nabla{\bf B}_0)\nonumber\\[0.5ex]
&= {\bf n}_0\cdot{\bf Q} -{\bf n}_0\cdot\nabla\times (\bxi\times{\bf B}_0),
\end{align}
where use has been made of Eqs.~(\ref{e6}), (\ref{e166x}), and (\ref{e181}). 
Thus, we obtain
\begin{equation}\label{e185}
{\bf n}_0\cdot{\bf Q} = {\bf n}_0\cdot\nabla\times (\bxi_\perp\times{\bf B}_0),
\end{equation}
where all quantities are evaluated on the unperturbed interface. 
A similar calculation allows us to deduce from Eq.~(\ref{e171a}) that
\begin{equation}\label{e188v}
{\bf n}_0\cdot\hat{\bf Q} = {\bf n}_0\cdot\nabla\times (\bxi_\perp\times\hat{\bf B}_0),
\end{equation}
where all quantities are again evaluated on the unperturbed interface. Note that, according to  Eq.~(\ref{e36}),  the matching condition (\ref{e185}) is
automatically satisfied. 

Let us write
\begin{equation}
\bxi_\perp = \xi_n\,{\bf n}_0 + \xi_x\,{\bf n}_0\times \hat{\bf b},
\end{equation}
where $\hat{\bf b} = \hat{\bf B}_0/|\hat{\bf B}_0|$ and $\xi_n= {\bf n}_0\cdot\bxi_\perp$. Equation~(\ref{e166ax}) implies that ${\bf n}_0\cdot\hat{\bf b}=0$. 
  It follows that
\begin{equation}
\bxi_\perp\times \hat{\bf B}_0 = \xi_n\,{\bf n}_0\times \hat{\bf B}_0 -\xi_x\,\hat{B}_0\,{\bf n}_0.
\end{equation}
Hence,
\begin{align}
{\bf n}_0\cdot\nabla\times(\bxi_\perp\times \hat{\bf B}_0)& ={\bf n}_0\cdot\nabla\times (\xi_n\,{\bf n}_0\times\hat{\bf B}_0)
- {\bf n}_0\cdot\left[\nabla(\xi_x\,\hat{B}_0)\times {\bf n}_0+\xi_x\,\hat{B}_0\,\nabla\times {\bf n}_0\right]\nonumber\\[0.5ex]
&= {\bf n}_0\cdot\nabla\times (\xi_n\,{\bf n}_0\times\hat{\bf B}_0),
\end{align}
where use has been made of Eq.~(\ref{e98p}). Thus, we obtain
\begin{align}
{\bf n}_0\cdot\nabla\times(\bxi_\perp\times \hat{\bf B}_0)&= {\bf n}_0\cdot\left\{\nabla\xi_n\times ({\bf n}_0\times \hat{\bf B}_0)
+\xi_n\left[-(\nabla\cdot{\bf n}_0)\,\hat{\bf B}_0 + \hat{\bf B}_0\cdot\nabla{\bf n}_0 - {\bf n}_0\cdot\nabla\hat{\bf B}_0\right]\right\}\nonumber\\[0.5ex]
&= {\bf n}_0\cdot\left[(\hat{\bf B}_0\cdot\nabla\xi_n)\,{\bf n}_0 + \xi_n\,\hat{\bf B}_0\cdot\nabla{\bf n}_0 -\xi_n\,{\bf n}_0\cdot\nabla\hat{\bf B}_0\right]\nonumber\\[0.5ex]
&= \hat{\bf B}_0\cdot\nabla\xi_n -\xi_n\,{\bf n}_0\cdot\nabla \hat{\bf B}_0\cdot{\bf n}_0.
\end{align}
Here, use has been made of $\nabla\cdot\hat{\bf B}_0=0$, Eq.~(\ref{e166ax}), and  the fact that $\nabla ({\bf n}_0\cdot{\bf n}_0)=0$
(because ${\bf n}_0$ is a unit vector). It follows that the matching condition (\ref{e188v}) can be written in the form
\begin{equation}
{\bf n}_0\cdot\hat{\bf Q} =\hat{\bf B}_0\cdot\nabla({\bf n}_0\cdot\bxi_\perp) -({\bf n}_0\cdot\bxi_\perp)\,{\bf n}_0\cdot\nabla \hat{\bf B}_0\cdot{\bf n}_,
\end{equation}
where all quantities are evaluated on the unperturbed interface. 

Equation~(\ref{e172}) yields
\begin{align}
p_0 + \bxi\cdot\nabla p_0+p_1 + \frac{1}{2\,\mu_0}\left({\bf B}_0 + \bxi\cdot\nabla{\bf B}_0+{\bf Q}\right)^2=  \frac{1}{2\,\mu_0}\left(\hat{\bf B}_0 + \bxi\cdot\nabla\hat{\bf B}_0+\hat{\bf Q}\right)^2,
\end{align}
which reduces to
\begin{align}
p_0 + \bxi\cdot\nabla p_0+p_1 +\frac{B_0^{\,2}}{2\,\mu_0} + \mu_0^{-1}\,{\bf B}_0\cdot (\bxi\cdot\nabla{\bf B}_0)+\mu_0^{-1}\,{\bf B}_0\cdot {\bf Q}&\simeq 
\frac{\hat{B}_0^{\,2}}{2\,\mu_0} + \mu_0^{-1}\,\hat{\bf B}_0\cdot (\bxi\cdot\nabla\hat{\bf B}_0)\nonumber\\[0.5ex]&\phantom{=}+\mu_0^{-1}\,\hat{\bf B}_0\cdot \hat{\bf Q}.
\end{align}
Making use of Eq.~(\ref{e171v}), we get
\begin{equation}\label{e189}
p_1  + \bxi\cdot\nabla p_0+\bxi\cdot\nabla\left(\frac{B_0^{\,2}}{2\,\mu_0}\right)+ \mu_0^{-1}\,{\bf B}_0\cdot {\bf Q}=
  \bxi\cdot\nabla\left(\frac{\hat{B}_0^{\,2}}{2\,\mu_0}\right) +\mu_0^{-1}\,\hat{\bf B}_0\cdot \hat{\bf Q}.
\end{equation}
Equation~(\ref{e171v}) implies that
\begin{equation}
{\bf t}\cdot\nabla\left(p_0+\frac{B_0^{\,2}}{2\,\mu_0}-\frac{\hat{B}_0^{\,2}}{2\,\mu_0}\right)=0,
\end{equation}
where ${\bf t}$ is any vector that lies within $S$. Because ${\bf b}$ is such a vector, Eq.~(\ref{e189}) becomes 
\begin{equation}
p_1  + \bxi_\perp\cdot\nabla p_0+\bxi_\perp\cdot\nabla\left(\frac{B_0^{\,2}}{2\,\mu_0}\right)+ \mu_0^{-1}\,{\bf B}_0\cdot {\bf Q}=
  \bxi_\perp\cdot\nabla\left(\frac{\hat{B}_0^{\,2}}{2\,\mu_0}\right) +\mu_0^{-1}\,\hat{\bf B}_0\cdot \hat{\bf Q}.
\end{equation}
Finally, we obtain
\begin{equation}
-\gamma\,p_0\,\nabla\cdot \bxi + \bxi_\perp\cdot\nabla\left(\frac{B_0^{\,2}}{2\,\mu_0}\right)+ \mu_0^{-1}\,{\bf B}_0\cdot {\bf Q}=
  \bxi_\perp\cdot\nabla\left(\frac{\hat{B}_0^{\,2}}{2\,\mu_0}\right) +\mu_0^{-1}\,\hat{\bf B}_0\cdot \hat{\bf Q},
\end{equation}
where use has been made of Eq.~(\ref{e15t}) and (\ref{e37}). As before, all quantities are evaluated on the unperturbed interface. 


\end{document}

\subsection{Perturbed Potential Energy}
According to Eqs.~(\ref{e12}), (\ref{e52x}) and (\ref{e100}), the perturbed plasma potential energy can be written
\begin{equation}\label{e112}
\delta W =\frac{1}{2}\int\left\{\bxi^\ast\cdot\left[\mu_0^{-1}\,(\nabla\times {\bf Q})\times {\bf B}_0 - {\bf j}_0\times {\bf Q} -\nabla(\bxi_\perp\cdot\nabla p_0)\right]
+\gamma\,p_0\,|\nabla\cdot\bxi|^2\right\}d{\bf r}.
\end{equation}
However, Eqs.~(\ref{e12}), (\ref{e37}), and (\ref{e41}) imply that 
\begin{equation}
{\bf b}\cdot [{\bf j}_0\times {\bf Q} +\nabla(\bxi_\perp\cdot\nabla p_0)]= 0.
\end{equation}
Hence, Eq.~(\ref{e112}) becomes
\begin{equation}\label{e114}
\delta W = \frac{1}{2}\int\left[\mu_0^{-1}\,(\nabla\times {\bf Q})\cdot(\bxi_\perp^{\,\ast}\times {\bf B}_0) - \bxi_\perp^{\,\ast}\cdot{\bf j}_0\times {\bf Q}
- \bxi_\perp^{\,\ast}\cdot\nabla(\bxi_\perp\cdot\nabla p_0)+\gamma\,p_0\,|\nabla\cdot\bxi|^2\right]d{\bf r}
\end{equation}
Now,
\begin{align}\label{e115}
\int (\nabla\times {\bf Q})\cdot(\bxi_\perp^{\,\ast}\times {\bf B}_0) \,d{\bf r}&=\int\left\{\nabla\cdot[{\bf Q}\times(\bxi_\perp^{\,\ast}\times {\bf B}_0)]
+{\bf Q}\cdot\nabla\times(\bxi_\perp^{\,\ast}\times {\bf B}_0)\right\}d{\bf r}\nonumber\\[0.5ex]
&=\int |{\bf Q}|^2\,d{\bf r},
\end{align}
where use has been made of Eq.~(\ref{e36}). Here, the divergence term integrates to zero because
\begin{equation}
{\bf n}\cdot {\bf Q}\times(\bxi_\perp^{\,\ast}\times {\bf B}_0)= ({\bf Q}\cdot{\bf B}_0)\,({\bf n}\cdot\bxi_\perp^{\,\ast})- ({\bf Q}\cdot\bxi_\perp^{\,\ast})\,({\bf n}\cdot{\bf B}_0)
=0
\end{equation}
at the wall, where use has been made of Eqs.~(\ref{e31x}) and (\ref{e79}). Furthermore, 
\begin{align}\label{e117}
\int \bxi_\perp^{\,\ast}\cdot\nabla(\bxi_\perp\cdot\nabla p_0)\,d{\bf r} &= \int\left\{\nabla\cdot[(\bxi_\perp\cdot\nabla p_0)\,\bxi_\perp^{\,\ast}]
-(\bxi_\perp\cdot \nabla p_0)\,(\nabla\cdot\bxi_\perp^{\,\ast})\right\}d{\bf r}\nonumber\\[0.5ex]
&= -\int (\bxi_\perp\cdot \nabla p_0)\,(\nabla\cdot\bxi_\perp^{\,\ast})\,d{\bf r}.
\end{align}
Here, the divergence term has integrated to zero because of Eq.~(\ref{e79}). Combining Eqs.~(\ref{e114}), (\ref{e115}), and (\ref{e117}),
we obtain the following standard expression for the perturbed potential energy
\begin{equation}
\delta W = \frac{1}{2}\int\left[\mu_0^{-1}\,|{\bf Q}|^2 - \bxi_\perp^{\,\ast}\cdot{\bf j}_0\times {\bf Q} + (\bxi_\perp\cdot\nabla p_0)\,(\nabla\cdot\bxi_\perp^{\,\ast})
+ \gamma\,p_0\,|\nabla\cdot\bxi|^2\right]d{\bf r}.
\end{equation}

\end{document}