\documentclass[12pt,prb,aps,notitlepage]{revtex4-1}
\usepackage {amsmath}
\usepackage{amssymb}
\pdfoutput = 1 
\usepackage {graphicx}
\newcommand{\bomega}{\mbox{\boldmath$\omega$}}
\newcommand{\bxi}{\mbox{\boldmath$\xi$}}
\newcommand{\bta}{\mbox{\boldmath$\eta$}}
\newcommand{\bkappa}{\mbox{\boldmath$\kappa$}}
\allowdisplaybreaks

\begin{document}

\title{Ideal-MHD Energy Principle Analysis}
\maketitle

\section{Ideal Stability Analayis}
\subsection{Fundamental Equations}
The fundamental  equations of ideal magnetohydrodynamics (ideal-MHD) are
\begin{align}
\frac{\partial\rho}{\partial t} + \nabla\cdot(\rho\,{\bf v}) & =0,\label{e1}\\[0.5ex]
\rho\left[\frac{\partial{\bf v}}{\partial t} + ({\bf v}\cdot\nabla)\,{\bf v}\right]+\nabla p - {\bf j}\times {\bf B}&=0,\label{e2}\\[0.5ex]
\frac{\partial p}{\partial t}+{\bf v}\cdot\nabla p + \gamma\,p\,\nabla\cdot{\bf v} &=0,\label{e3}\\[0.5ex]
\frac{\partial {\bf B}}{\partial t} &= \nabla\times({\bf v}\times {\bf B}),\label{e4}\\[0.5ex]
\mu_0\,{\bf j} &= \nabla\times {\bf B},\label{e5}
\end{align}
where $\rho$ is the plasma mass density, ${\bf v}$ the plasma velocity, $p$ the (scalar) plasma pressure, $\gamma=5/3$ the ratio of specific heats,
${\bf B}$ the magnetic field-strength, and ${\bf j}$ the electric current density. 
Note that Eq.~(\ref{e4}) ensures that the magnetic field remains divergence-free, provided that 
this is the case initially. 

\subsection{Plasma Equilibrium}
The plasma equilibrium is such that
\begin{align}
\rho({\bf r},t) &= \rho_0({\bf r}),\\[0.5ex]
{\bf v}({\bf r},t) &= {\bf 0},\label{e7}\\[0.5ex]
p({\bf r},t) &= p_0({\bf r}),\\[0.5ex]
{\bf B}({\bf r},t) &= {\bf B}_0({\bf r}),\\[0.5ex]
{\bf j}({\bf r},t) &= {\bf j}_0({\bf r}),
\end{align}
where
\begin{align}
\nabla\cdot{\bf B}_0&= 0,\label{e6}\\[0.5ex]
\nabla p_0 &= {\bf j}_0\times {\bf B}_0,\label{e11}\\[0.5ex]
\mu_0\,{\bf j}_0 &= \nabla\times {\bf B}_0.\label{e12}
\end{align}
Note that we are neglecting equilibrium plasma flows, as these are generally unimportant provided that they remain sub-sonic and sub-Alfv\'{e}nic. 

\subsection{Perturbed Quantities}
Let us formulate the linear stability problem as a normal mode problem. This goal is
achieved by letting all perturbed quantities vary in time as $\exp(-{\rm i}\,\omega\,t)$. 
Thus, the perturbation to the plasma equilibrium  is written
\begin{align}
\rho_1({\bf r},t) &= \rho_1({\bf r})\,{\rm e}^{-{\rm i}\,\omega\,t},\\[0.5ex]
{\bf v}_1({\bf r},t) &={\bf v}_1({\bf r})\,{\rm e}^{-{\rm i}\,\omega\,t}=  {-{\rm i}\,\omega}\,\bxi({\bf r})\,{\rm e}^{-{\rm i}\,\omega\,t},\label{e15}\\[0.5ex]
p({\bf r},t) &= p_1({\bf r})\,{\rm e}^{-{\rm i}\,\omega\,t},\\[0.5ex]
{\bf B}({\bf r},t) &= {\bf B}_1({\bf r})\,{\rm e}^{-{\rm i}\,\omega\,t},\\[0.5ex]
{\bf j}({\bf r},t) &= {\bf j}_1({\bf r})\,{\rm e}^{-{\rm i}\,\omega\,t}.
\end{align}
Note that $\bxi({\bf r})$ represents the displacement of the plasma from its equilibrium position.

The linearized perturbed versions of  Eqs.~(\ref{e1})--(\ref{e5}) are
\begin{align}\label{e19}
\rho_1 &=-\nabla\cdot(\rho_0\,\bxi),\\[0.5ex]
-\omega^2\,\rho_0\,\bxi &=  {\bf j}_1\times {\bf B}_0 + {\bf j}_0\times {\bf B}_1-\nabla p_1,\\[0.5ex]
p_1&= - \bxi\cdot\nabla p_0-\gamma\,p_0\,\nabla\cdot\bxi,\\[0.5ex]
{\bf B}_1 &=\nabla\times (\bxi\times {\bf B}_0),\label{e22}\\[0.5ex]
\mu_0\,{\bf j}_1&= \nabla\times {\bf B}_1.
\end{align}
Note that Eq.~(\ref{e19}) is not coupled to the other linearized equations. 
Combining the previous four equations, we obtain the perturbed plasma equation of motion, 
\begin{equation}\label{e24a}
-\omega^2\,\rho_0\,\bxi= {\bf F}(\bxi),
\end{equation}
where
\begin{align}\label{e24}
{\bf F}(\bxi) = \mu_0^{-1}\,(\nabla\times {\bf Q})\times{\bf B}_0+ \mu_0^{-1}\, (\nabla\times {\bf B}_0)\times {\bf Q} + 
\nabla(\bxi\cdot \nabla p_0 + \gamma\,p_0\,\nabla\cdot\bxi)
\end{align}
is known as the {\em force operator}, 
and
\begin{equation}\label{e25}
{\bf Q} = \nabla\times (\bxi\times {\bf B}_0)
\end{equation}
is the perturbed magnetic field. The force operator clearly specifies the perturbed force density that develops in the plasma in
response to the displacement $\bxi$. 

\subsection{Perfectly Conducting Wall}
Suppose that the plasma is surrounded by a perfectly conducting wall whose inner surface has the outward-directed unit normal  ${\bf n}$. 
Both the electric field and the magnetic field are zero within the wall. Standard electromagnetic boundary conditions 
require that
\begin{align}\label{e27a}
{\bf n}\times {\bf E}&= 0,\\[0.5ex]
{\bf n} \cdot {\bf B}&=0,\label{e27b}
\end{align}
at the wall. Here, ${\bf E}$ is the electric field. 

The plasma velocity is assumed to be dominated by the ${\bf E}\times {\bf B}$ drift velocity. In other words,
\begin{equation}
{\bf v} = \frac{{\bf E}\times {\bf B}}{B^2}.
\end{equation}
If we write ${\bf E}({\bf r},t)= {\bf E}_0({\bf r})+ {\bf E}_1({\bf r})\,\exp(-{\rm i}\,\omega\,t)$
then it is clear from Eq.~(\ref{e7}) that ${\bf E}_0={\bf 0}$. It follows from the previous equation that
\begin{equation}\label{e30a}
{\bf v}_1 = \frac{{\bf E}_1\times {\bf B}_0}{B_0^{\,2}}.
\end{equation}
Now, Eqs.~(\ref{e27a}) and (\ref{e27b}) imply that
\begin{equation}\label{e31x}
{\bf n} \cdot {\bf B}_0=0,
\end{equation}
and 
\begin{align}
{\bf n}\times {\bf E}_1&= 0,\label{e31a}\\[0.5ex]
{\bf n} \cdot {\bf B}_1&=0\label{e33}
\end{align}
at the wall. 
Equations~(\ref{e30a}) and (\ref{e31a}) yield
\begin{equation}
{\bf n}\cdot{\bf v}_1=0
\end{equation}
at the wall. Hence, it follows from Eq.~(\ref{e15}) that
\begin{equation}\label{e35}
{\bf n}\cdot\bxi = 0
\end{equation}
at the wall. 

\subsection{Self-Adjoint Property of Force Operator}
We wish to demonstrate that the force operator is {\em self-adjoint}. In other words, we wish to prove that 
\begin{equation}\label{e26}
\int \bta\cdot{\bf F}(\bxi)\,d{\bf r} = \int \bxi\cdot {\bf F}(\bta)\,d{\bf r},
\end{equation}
where the integrals are taken over the whole plasma volume, and  $\bxi({\bf r})$ and $\bta({\bf r})$ are two arbitrary vector fields that satisfy the 
physical boundary conditions
\begin{equation}\label{e27}
{\bf n}\cdot\bxi={\bf n}\cdot\bta=0
\end{equation}
at the wall.

According to Eq.~(\ref{e24}), the integrand of the left-hand side of Eq.~(\ref{e26}) takes the form 
\begin{equation}
\bta\cdot {\bf F}(\bxi)= \bta\cdot\left[ \mu_0^{-1}\,(\nabla\times {\bf Q})\times{\bf B}_0+ \mu_0^{-1}\, (\nabla\times {\bf B}_0)\times {\bf Q}
+\nabla(\bxi\cdot\nabla p_0) + \nabla(\gamma\,p_0\,\nabla\cdot\bxi)\right].
\end{equation}
The final term can be written
\begin{equation}
\bta\cdot \nabla(\gamma\,p_0\,\nabla\cdot\bxi)= \nabla\cdot (\bta\,\gamma\,p_0\,\nabla\cdot\bxi)-\gamma\,p_0\,(\nabla\cdot\bta)\,(\nabla\cdot\bxi).
\end{equation}
However, according to Eq.~(\ref{e27}), the divergence term integrates to zero in Eq.~(\ref{e26}). Hence, we can safely neglect this term, 
to give 
\begin{align}
\bta\cdot {\bf F}(\xi)&= \bta\cdot\left[ \mu_0^{-1}\,(\nabla\times {\bf Q})\times{\bf B}_0+ \mu_0^{-1}\, (\nabla\times {\bf B}_0)\times {\bf Q}
+\nabla(\bxi\cdot\nabla p_0)\right]\nonumber\\[0.5ex]
&\phantom{=} -\gamma\,p_0\,(\nabla\cdot\bta)\,(\nabla\cdot\bxi).\label{e31}
\end{align}

Let us write
\begin{align}\label{e32}
\bxi &=\bxi_\perp + \xi_\parallel\,{\bf b},\\[0.5ex]
\bta&=\bta_\perp+\eta_\parallel\,{\bf b},\label{e33x}
\end{align}
where 
\begin{align}
{\bf b} &=\frac{{\bf B}_0}{B_0},\\[0.5ex]
{\bf b} \cdot\bxi_\perp &= {\bf b}\cdot\bta_\perp = 0.
\end{align}
Thus, $\xi_\parallel\,{\bf b}$ and $\bxi_\perp$ are the component of $\bxi$ that are parallel to and perpendicular to the equilibrium magnetic field, et cetera.
According to Eqs.~(\ref{e31x}) and (\ref{e27}),
\begin{equation}\label{ecc}
{\bf n}\cdot\bxi_\perp = {\bf n}\cdot\bta_\perp
\end{equation}
at the wall. 
It follows from Eq.~(\ref{e25}) that
\begin{equation}\label{e36}
{\bf Q} = \nabla\times (\bxi_\perp\times {\bf B}_0).
\end{equation}
Moreover, Eq.~(\ref{e11}) implies that
\begin{equation}\label{e37}
\bxi\cdot\nabla p_0 = \bxi\cdot{\bf j}_0\times {\bf B}_0= \bxi_\perp\cdot{\bf j}_0\times {\bf B}_0= \bxi_\perp\cdot\nabla p_0.
\end{equation}

Now,
\begin{equation}\label{e38}
{\bf B}_0 \cdot \left[\mu_0^{-1}\,(\nabla\times {\bf Q})\times{\bf B}_0\right] =0,
\end{equation}
and
\begin{align}
{\bf B}_0\cdot\left[\mu_0^{-1}\, (\nabla\times {\bf B}_0)\times {\bf Q}\right]= {\bf B}_0\cdot{\bf j}_0\times {\bf Q} =-{\bf j}_0\times {\bf B}_0\cdot {\bf Q}
=-\nabla p_0\cdot{\bf Q},
\end{align}
where use has been made of Eqs.~(\ref{e11}) and (\ref{e12}). However, according to Eq.~(\ref{e36}), 
\begin{align}
-\nabla p_0\cdot{\bf Q} &= -\nabla p_0\cdot\nabla\times(\bxi_\perp\times {\bf B}_0)=\nabla\cdot\left[\nabla p_0\times(\bxi_\perp\times {\bf B}_0)\right]
=-\nabla\cdot[(\bxi_\perp\cdot\nabla p_0)\, {\bf B}_0]\nonumber\\[0.5ex]
&= - {\bf B}_0\cdot\nabla (\bxi_\perp\cdot\nabla p_0),
\end{align}
where use has been made of Eqs.~(\ref{e6}) and (\ref{e11}). The previous two equations, combined with Eq.~(\ref{e37}), imply that
\begin{equation}\label{e41}
{\bf B}_0\cdot\left[\mu_0^{-1}\, (\nabla\times {\bf B}_0)\times {\bf Q}+\nabla(\bxi\cdot\nabla p_0)\right] =0.
\end{equation}
Thus, Eqs.~(\ref{e31}), (\ref{e33x}), (\ref{e37}), (\ref{e38}), and (\ref{e41}) yield 
\begin{align}\label{e52x}
\bta\cdot {\bf F}(\bxi)&= \bta_\perp\cdot\left[ \mu_0^{-1}\,(\nabla\times {\bf Q})\times{\bf B}_0+ \mu_0^{-1}\, (\nabla\times {\bf B}_0)\times {\bf Q}
+\nabla(\bxi_\perp\cdot\nabla p_0)\right] \nonumber\\[0.5ex]
&\phantom{=}-\gamma\,p_0\,(\nabla\cdot\bta)\,(\nabla\cdot\bxi).
\end{align}
Note that $\xi_\parallel$ and $\eta_\parallel$ only occur on the right-hand side of the previous equation in the final term. 

Let 
\begin{equation}\label{e43}
I= \bta_\perp\cdot\left[ \mu_0^{-1}\,(\nabla\times {\bf Q})\times{\bf B}_0+ \mu_0^{-1}\, (\nabla\times {\bf B}_0)\times {\bf Q}
+\nabla(\bxi_\perp\cdot\nabla p_0)\right].
\end{equation}
Now,
\begin{equation}
\bta_\perp\cdot\nabla (\bxi_\perp\cdot\nabla p_0)= \nabla\cdot[\bta_\perp\,(\bxi_\perp\cdot \nabla p_0)] - (\bxi_\perp\cdot\nabla p_0)\,(\nabla\cdot
\bta_\perp) =-(\bxi_\perp\cdot\nabla p_0)\,(\nabla\cdot
\bta_\perp),
\end{equation}
where the divergence term has integrated to zero because of Eq.~(\ref{ecc}). Furthermore, 
\begin{equation}
(\nabla\times {\bf Q})\times{\bf B}_0+ \mu_0^{-1}\, (\nabla\times {\bf B}_0)\times {\bf Q}= {\bf Q}\cdot\nabla{\bf B}_0
+ {\bf B}_0\cdot\nabla{\bf Q} - \nabla({\bf B}_0\cdot{\bf Q}).
\end{equation}
Hence,
\begin{equation}\label{e46}
I = \mu_0^{-1}\,\bta_\perp\cdot\left[ {\bf Q}\cdot\nabla{\bf B}_0
+ {\bf B}_0\cdot\nabla{\bf Q} - \nabla({\bf B}_0\cdot{\bf Q})\right]  - (\bxi_\perp\cdot\nabla p_0)\,(\nabla\cdot
\bta_\perp).
\end{equation}

Note that
\begin{equation}\label{e47}
Q_i = \frac{\partial}{\partial x_j}\,(\xi_{\perp\,i}\,B_{0\,j} - \xi_{\perp\,j}\,B_{0\,i}).
\end{equation}
It follows that
\begin{align}
({\bf Q}\cdot\nabla {\bf B}_0)_i&= Q_j\,\frac{\partial B_{0\,i}}{\partial x_j} 
= \frac{\partial}{\partial x_k}\,(\xi_{\perp\,j}\,B_{0\,k} - \xi_{\perp\,k}\,B_{0\,j})\,\frac{\partial B_{0\,i}}{\partial x_j}\nonumber\\[0.5ex]
&= B_{0\,k}\,\frac{\partial\xi_{\perp\,j}}{\partial x_k}\,\frac{\partial B_{0\,i}}{\partial x_j}-\xi_{\perp\,k}\,\frac{\partial B_{0\,j}}{\partial x_k}\,\frac{\partial B_{0\,i}}{\partial x_j}
-B_{0\,j}\,\frac{\partial B_{0\,i}}{\partial x_j}\,\frac{\partial \xi_{\perp\,k}}{\partial x_k} \nonumber\\[0.5ex]
&= \left[ ({\bf B}_0\cdot
\nabla\bxi_\perp)\cdot\nabla {\bf B}_0 - (\bxi_\perp\cdot \nabla{\bf B}_0)\cdot\nabla {\bf B}_0- ({\bf B}_0\cdot\nabla {\bf B}_0)\,(\nabla\cdot\bxi_\perp)\right]_i,
\end{align}
where use has been made of Eq.~(\ref{e6}). However,
\begin{equation}\label{e49}
({\bf B}_0\cdot\nabla {\bf B}_0) = B_0^{\,2}\,\bkappa+ ({\bf B}_0\cdot\nabla B_0)\,{\bf b},
\end{equation}
where
\begin{equation}
\bkappa = {\bf b}\cdot\nabla {\bf b} = \frac{{\bf R}_c}{R_c^{\,2}}
\end{equation}
is the curvature vector of the equilibrium magnetic field (i.e., ${\bf R}_c$ is the local radius of curvature of equilibrium magnetic field-lines).
Hence, we deduce that
\begin{equation}\label{e51}
\bta_\perp \cdot({\bf Q}\cdot\nabla {\bf B}_0)= \bta_\perp\cdot\left[({\bf B}_0\cdot
\nabla\bxi_\perp)\cdot\nabla {\bf B}_0 - (\bxi_\perp\cdot \nabla{\bf B}_0)\cdot\nabla {\bf B}_0- B_0^{\,2}\,(\nabla\cdot\bxi_\perp)\,\bkappa\right].
\end{equation}

Now,
\begin{align}
\bta_\perp\cdot({\bf B}_0\cdot\nabla{\bf Q}) 
= \eta_{\perp\,i}\,B_{0\,j}\,\frac{\partial Q_i}{\partial x_j}=
\frac{\partial}{\partial x_j}\,(\eta_{\perp i}\,Q_i\,B_{0\,j})
- Q_i\,B_{0\,j}\,\frac{\partial\eta_{\perp\,i}}{\partial x_j}\nonumber\\[0.5ex]
= \nabla\cdot[(\bta_\perp\cdot {\bf Q})\,{\bf B}_0] - {\bf Q}\cdot ({\bf B}_0\cdot\nabla \bta_\perp)=- {\bf Q}\cdot ({\bf B}_0\cdot\nabla \bta_\perp),
\end{align}
where we have  used Eq.~(\ref{e6}), and the divergence term has integrated to zero because of Eq.~(\ref{e31x}). 
However, from Eqs.~(\ref{e6}) and (\ref{e47}),
\begin{equation}\label{e53}
{\bf Q}  ={\bf B}_0\cdot\nabla \bxi_\perp - \bxi_\perp\cdot\nabla {\bf B}_0 - (\nabla\cdot\bxi_\perp)\,{\bf B}_0.
\end{equation}
Thus,
\begin{align}
\bta_\perp \cdot({\bf B}_0\cdot\nabla {\bf Q})&=- ({\bf B}_0\cdot\nabla\bxi_\perp)\cdot({\bf B}_0\cdot\nabla \bta_\perp)+ (\bxi_\perp\cdot\nabla {\bf B}_0)\cdot({\bf B}_0\cdot\nabla \bta_\perp) \nonumber\\[0.5ex]
&\phantom{=}+ (\nabla\cdot\bxi_\perp)\,{\bf B}_0\cdot({\bf B}_0\cdot\nabla \bta_\perp). 
\end{align}
But,
\begin{align}
{\bf B}_0\cdot({\bf B}_0\cdot\nabla \bta_\perp)&= B_{0\,i}\,B_{0\,j}\,\frac{\partial \eta_{\perp\,i}}{\partial x_j}= \frac{\partial}{\partial x_j}\,( B_{0\,i}\,B_{0\,j}\, \eta_{\perp\,i}) - B_{0\,i}\,\frac{\partial B_{0\,j}}{\partial x_j}\,\eta_{\perp\,i}\nonumber\\[0.5ex]
&= \nabla\cdot [({\bf B}_0\cdot\bta_\perp)\,{\bf B}_0] -B_0^{\,2}\,\bkappa\cdot \bta_\perp=-B_0^{\,2}\,\bkappa\cdot \bta_\perp,
\end{align}
where we have made use of Eq.~(\ref{e6}), as well as Eq.~(\ref{e49}), and the divergence term has integrated to zero because of Eq.~(\ref{e31x}). Thus, we arrive at
\begin{equation}\label{e56}
\bta_\perp \cdot({\bf B}_0\cdot\nabla {\bf Q})=- ({\bf B}_0\cdot\nabla\bxi_\perp)\cdot({\bf B}_0\cdot\nabla \bta_\perp)+ (\bxi_\perp\cdot\nabla {\bf B}_0)\cdot({\bf B}_0\cdot\nabla \bta_\perp) - B_0^{\,2}\,(\bta_\perp\cdot\bkappa)\,(\nabla\cdot\bxi_\perp).
\end{equation}

Now,
\begin{align}
-\bta_\perp\cdot\nabla ({\bf B}_0\cdot{\bf Q}) &= -\eta_{\perp i}\,\frac{\partial}{\partial x_i}\,(B_{0\,j}\,Q_j) = - \frac{\partial}{\partial x_i}\,(\eta_{\perp\,i}\,B_{0\,j}\,Q_j)+ B_{0\,j}\,Q_j\,\frac{\partial \eta_{\perp i}}{\partial x_i}\nonumber\\[0.5ex]
&= -\nabla\cdot[({\bf B}_0\cdot {\bf Q})\,\bta_\perp]  +({\bf B}_0\cdot{\bf Q})\,(\nabla\cdot\bta_\perp)=   ({\bf B}_0\cdot{\bf Q})\,(\nabla\cdot\bta_\perp),
\end{align}
where the divergence term has  integrated to zero because of Eq.~(\ref{ecc}). 
But, from Eq.~(\ref{e53}), 
\begin{align}
{\bf B}_0\cdot{\bf Q} &= - B_0^{\,2}\,\nabla\cdot\bxi_\perp + {\bf B}_0\cdot ({\bf B}_0\cdot\nabla\bxi_\perp) - {\bf B}_0\cdot
(\bxi_\perp\cdot\nabla {\bf B}_0)\nonumber\\[0.5ex]
&=- B_0^{\,2}\,\nabla\cdot\bxi_\perp - \bxi_\perp\cdot\nabla(B_0^{\,2}/2) + {\bf B}_0\cdot ({\bf B}_0\cdot\nabla\bxi_\perp).
\end{align}
However,
\begin{align}
 {\bf B}_0\cdot ({\bf B}_0\cdot\nabla\bxi_\perp) &= 
 B_{0\,i}\,B_{0\,j}\,\frac{\partial \xi_{\perp\,i}}{\partial x_j}= \frac{\partial}{\partial x_j}\,(B_{0\,i}\,B_{0\,j}\,\xi_{\perp\,i})- B_{0\,j} \,\frac{\partial B_{0\,i}}{\partial x_j}\,\xi_{\perp\,i}\nonumber\\[0.5ex]
 &= \nabla\cdot[({\bf B}_0\cdot\bxi_\perp)\,{\bf B}_0] - \bxi_\perp\cdot({\bf B}_0\cdot\nabla{\bf B}_0) = -B_0^{\,2}\,(\bxi_\perp\cdot\bkappa),
 \end{align}
 where we have  used Eqs.~(\ref{e6}) and (\ref{e49}), and the divergence term has  integrated to zero because of Eq.~(\ref{e31x}). 
 Thus, we deduce that
 \begin{equation}\label{e60}
 -\bta_\perp\cdot\nabla ({\bf B}_0\cdot{\bf Q}) = -B_0^{\,2}\,(\nabla\cdot\bxi_\perp)\,(\nabla\cdot\bta_\perp)
 -[\bxi_\perp\cdot\nabla(B_0^{\,2}/2) + B_0^{\,2}\,(\bxi_\perp\cdot\bkappa)]\,(\nabla\cdot\bta_\perp).
 \end{equation}
 
 Hence, it follows from Eqs.~(\ref{e31}), (\ref{e43}), (\ref{e46}),  (\ref{e51}), (\ref{e56}), and (\ref{e60}) that
 \begin{align}\label{e61}
\bta\cdot {\bf F}(\bxi) &=\frac{B_0^{\,2}}{\mu_0}\,(\nabla\cdot\bxi_\perp)\,(\nabla\cdot\bta_\perp) 
 - \frac{1}{\mu_0}\,({\bf B}_0\cdot\nabla\bxi_\perp)\cdot
 ({\bf B}_0\cdot\nabla\bta_\perp)-\gamma\,p_0\,(\nabla\cdot\bxi)\,(\nabla\cdot\bta)\nonumber\\[0.5ex]
 &-\left[\bxi_\perp\cdot\nabla\left(p_0+\frac{B_0^{\,2}}{2\,\mu_0}\right)+ \frac{B_0^{\,2}}{\mu_0}\,\bxi_\perp\cdot\bkappa\right]\nabla\cdot\bta_\perp\nonumber
\\[0.5ex]&
-\frac{2\,B_0^{\,2}}{\mu_0}\,(\bta_\perp\cdot\bkappa)\,(\nabla\cdot\bxi_\perp) +R,
 \end{align}
 where
 \begin{equation}
 \mu_0\,R= \bta_\perp\cdot\left[({\bf B}_0\cdot\nabla\bxi_\perp)\cdot\nabla {\bf B}_0 - (\bxi_\perp\cdot\nabla {\bf B}_0)\cdot\nabla {\bf B}_0\right]
 + (\bxi_\perp\cdot\nabla {\bf B}_0)\cdot({\bf B}_0\cdot\nabla\bta_\perp).
 \end{equation}
 
 However, from Eqs.~(\ref{e11}) and (\ref{e12}), 
 \begin{equation}\label{e64}
 \nabla p_0 = \mu_0^{-1}\,(\nabla\times{\bf B}_0)\times {\bf B}_0= \mu_0^{-1}\left[({\bf B}_0\cdot\nabla)\,{\bf B}_0-\nabla(B_0^{\,2}/2)\right].
 \end{equation}
 Thus, Eq.~(\ref{e49}) yields
 \begin{equation}
 \bxi_\perp\cdot\nabla\left(p_0 +\frac{B_0^{\,2}}{2}\right) = \frac{B_0^{\,2}}{\mu_0}\,\bxi_\perp\cdot\bkappa.
 \end{equation}
 Hence, Eq.~(\ref{e61}) simplifies to give 
 \begin{align}\label{e65}
\bta\cdot {\bf F}(\bxi) &=\frac{B_0^{\,2}}{\mu_0}\,(\nabla\cdot\bxi_\perp)\,(\nabla\cdot\bta_\perp) 
 - \frac{1}{\mu_0}\,({\bf B}_0\cdot\nabla\bxi)\cdot
 ({\bf B}_0\cdot\nabla\bta_\perp)-\gamma\,p_0\,(\nabla\cdot\bxi)\,(\nabla\cdot\bta)\nonumber\\[0.5ex]
 &-\frac{2\,B_0^{\,2}}{\mu_0}\,(\bxi_\perp\cdot\bkappa)\, (\nabla\cdot\bta_\perp)-\frac{2\,B_0^{\,2}}{\mu_0}\,(\bta_\perp\cdot\bkappa)\,(\nabla\cdot\bxi_\perp) +R.
 \end{align}

Now, 
\begin{align}
\mu_0\,R &= \eta_{\perp\,i}\,B_{0\,k}\,\frac{\partial\xi_{\perp\,j}}{\partial x_k}\,\frac{\partial B_{0\,i}}{\partial x_j}
- \eta_{\perp\,i}\,\xi_{\perp\,k}\,\frac{\partial B_{0\,j}}{\partial x_k}\,\frac{\partial B_{0\,i}}{\partial x_j}
+\xi_{\perp\,j}\,\frac{\partial B_{0\,i}}{\partial x_j}\,B_{0\,k}\,\frac{\partial\eta_{\perp\,i}}{\partial x_k}\nonumber\\[0.5ex]
&= \frac{\partial}{\partial x_k}\!\left(\eta_{\perp\,i}\,B_{0\,k}\,\xi_{\perp\,j}\,\frac{\partial B_{0\,i}}{\partial x_j}\right)
-\xi_{\perp\,j}\,\frac{\partial}{\partial x_k}\!\left(\eta_{\perp\,i}\,B_{0\,k}\,\frac{\partial B_{0\,i}}{\partial x_j}\right)\nonumber\\[0.5ex]
&\phantom{=}- \eta_{\perp\,i}\,\xi_{\perp\,k}\,\frac{\partial B_{0\,j}}{\partial x_k}\,\frac{\partial B_{0\,i}}{\partial x_j}
+\xi_{\perp\,j}\,\frac{\partial B_{0\,i}}{\partial x_j}\,B_{0\,k}\,\frac{\partial\eta_{\perp\,i}}{\partial x_k}\nonumber\\[0.5ex]
&=-\eta_{\perp\,i}\,\xi_{\perp\,j}\,\frac{\partial}{\partial x_k}\!\left(B_{0\,k}\,\frac{\partial B_{0\,i}}{\partial x_j}\right)- \eta_{\perp\,i}\,\xi_{\perp\,j}\,\frac{\partial B_{0\,k}}{\partial x_j}\,\frac{\partial B_{0\,i}}{\partial x_k}\nonumber\\[0.5ex]
&=-\eta_{\perp\,i}\,\xi_{\perp\,j}\,B_{0\,k}\,\frac{\partial^2 B_{0\,i}}{\partial x_j\,\partial x_k}- \eta_{\perp\,i}\,\xi_{\perp\,j}\,\frac{\partial B_{0\,k}}{\partial x_j}\,\frac{\partial B_{0\,i}}{\partial x_k} \nonumber\\[0.5ex]
&=- \eta_{\perp\,i}\,\xi_{\perp\,j}\,\frac{\partial}{\partial x_j}\!\left(B_{0\,k}\,\frac{\partial B_{0\,i}}{\partial x_k}\right),
\end{align}
where the divergence term has integrated to zero because of Eq.~(\ref{e31x}), and use has been made of Eq.~(\ref{e6}). 
However, from Eq.~(\ref{e64}), 
\begin{equation}
\mu_0^{-1}\,B_{0\,k}\,\frac{\partial B_{0\,i}}{\partial x_k} = \frac{\partial}{\partial x_i}\!\left(p_0 + \frac{B_0^{\,2}}{2\,\mu_0}\right).
\end{equation}
Thus,
\begin{equation}
R=  -\eta_{\perp\,i}\,\xi_{\perp\,j}\,\frac{\partial^2}{\partial x_i\,\partial x_j}\!\left(p_0 + \frac{B_0^{\,2}}{2\,\mu_0}\right)=-
(\bta_\perp\,\bxi_\perp : \nabla\nabla)\left(p_0 + \frac{B_0^{\,2}}{2\,\mu_0}\right).
\end{equation}

Thus, it follows from Eq.~(\ref{e65}) that 
 \begin{align}
 \bta\cdot {\bf F}(\bxi) &= - \frac{1}{\mu_0}\,({\bf B}_0\cdot\nabla\bxi_\perp)\cdot
 ({\bf B}_0\cdot\nabla\bta_\perp)-\gamma\,p_0\,(\nabla\cdot\bxi)\,(\nabla\cdot\bta)+\frac{B_0^{\,2}}{\mu_0}\,(\nabla\cdot\bxi_\perp)\,(\nabla\cdot\bta_\perp) 
\nonumber\\[0.5ex]
 &\phantom{=}-\frac{2\,B_0^{\,2}}{\mu_0}\,(\bxi_\perp\cdot\bkappa)\, (\nabla\cdot\bta_\perp)-\frac{2\,B_0^{\,2}}{\mu_0}\,(\bta_\perp\cdot\bkappa)\,(\nabla\cdot\bxi_\perp)\nonumber\\[0.5ex]
 &\phantom{=}
 -(\bta_\perp\,\bxi_\perp : \nabla\nabla)\left(p_0 + \frac{B_0^{\,2}}{2\,\mu_0}\right).
\end{align}
The self-adjointness property (\ref{e26}) is now obviously satisfied. 

\subsection{Boundary Conditions at Perfectly Conducting Wall}
Equations~(\ref{e31x}), (\ref{e35}), and (\ref{e32}) can be combined to give 
\begin{equation}\label{e79}
{\bf n} \cdot\bxi_\perp = 0
\end{equation}
at the wall. Making use of Eqs.~(\ref{e22}), (\ref{e33}) and (\ref{e32}),
we also require
\begin{equation}\label{e81}
{\bf n}\cdot\nabla\times(\bxi_\perp\times {\bf B}_0) = 0
\end{equation}
at the wall. Now, Eqs.~(\ref{e31x}) and (\ref{e79})  imply that
\begin{equation}
\bxi_\perp\times {\bf B}_0= f\,{\bf n}
\end{equation}
at the wall, where $f$ is some scalar.
Thus, the boundary condition (\ref{e81}) becomes
\begin{equation}
{\bf n}\cdot \nabla\times (f\,{\bf n}) = {\bf n}\cdot[\nabla f\times {\bf n} + f\,\nabla\times {\bf n})= f\,{\bf n}\cdot\nabla\times {\bf n} = 0.
\end{equation}
Now, according to Eq.~(\ref{e31x}), the inner surface of the perfectly conducting wall must correspond to a contour of the equilibrium poloidal
magnetic flux,  $\psi({\bf r})$. 
It follows that
\begin{equation}
{\bf n} = \frac{\nabla\psi}{|\nabla\psi|}.
\end{equation}
Thus,
\begin{equation}
{\bf n}\cdot\nabla\times {\bf n} = \frac{\nabla\psi}{|\nabla\psi|}\cdot \left[\nabla\left(\frac{1}{|\nabla\psi|}\right)\times \nabla\psi\right]=0.
\end{equation}
Hence, we deduce that the boundary condition (\ref{e81}) is satisfied provided that the boundary condition (\ref{e79}) is satisfied. 

\subsection{Reality of $\omega^2$}
Consider a discrete normal mode with frequency $\omega$ and displacment $\bxi({\bf r})$. 
Taking the scalar product of Eq.~(\ref{e24}) with $\bxi^\ast({\bf r})$, and integrating over the plasma volume, we obtain
\begin{equation}
\omega^2\int\rho_0\,|\bxi|^2\,d{\bf r} = - \int\bxi^\ast\cdot {\bf F}(\bxi)\,d{\bf r}.
\end{equation}
Likewise, taking the scalar product of the complex conjugate of Eq.~(\ref{e24}) with $\bxi({\bf r})$, and integrating over the whole plasma
volume, we get
\begin{equation}
(\omega^2)^\ast\int\rho_0\,|\bxi|^2\,d{\bf r} = - \int\bxi\cdot {\bf F}(\bxi^\ast)\,d{\bf r}.
\end{equation}
Here, we have made use of the fact that $[{\bf F}(\bxi)]^\ast={\bf F}(\bxi^\ast)$. 
Taking the difference between the previous two equations, we obtain
\begin{equation}
\left[\omega^2 - (\omega^2)^\ast\right]\int\rho_0\,|\bxi|^2\,d{\bf r}=  - \int\bxi^\ast\cdot {\bf F}(\bxi)\,d{\bf r}+ \int\bxi\cdot {\bf F}(\bxi^\ast)\,d{\bf r}.
\end{equation}
However, the self-adjoint property of the force operator, (\ref{e26}), which is validated by the physical boundary condition (\ref{e79}), 
yields
\begin{equation}
\left[\omega^2 - (\omega^2)^\ast\right]\int\rho_0\,|\bxi|^2\,d{\bf r}=0.
\end{equation}
Given that the integral in the previous expression is positive-definite, we deduce that 
\begin{equation}
\omega^2 = (\omega^2)^\ast.
\end{equation}
In other words, $\omega^2$ is a real quantity. Furthermore, the fact that all of the differential operators appearing on the right-hand side of
Eq.~(\ref{e24a}) are real implies that $\bxi({\bf r})$ is real. More generally, $\bxi({\bf r})$ is a real function multiplied by a spatially uniform complex
number. 

In terms of the usual definition of exponential stability, a discrete normal mode with $\omega^2>0$ corresponds to a pure oscillation, and would, therefore,
be considered stable. Conversely, a discrete mode with $\omega^2<0$ has one branch that grows exponentially in time, and
would, therefore, be considered unstable. Clearly, the transition from stability to instability occurs when $\omega^2=0$. 

\subsection{Orthogonality of Normal Modes}\label{orth}
Consider two discrete normal modes. Let the first have the frequency $\omega_a$ and the displacement $\bxi_a({\bf r})$. Let the
second have the frequency $\omega_b$ and the displacement $\bxi_b({\bf r})$. Let the modes satisfy the
physical boundary conditions 
\begin{equation}\label{e91}
{\bf n}\cdot\bxi_{a\,\perp}={\bf n}\cdot\bxi_{b\,\perp} = 0
\end{equation}
at the wall. It follows from Eq.~(\ref{e24a}) that
\begin{align}
\omega_a^{\,2}\,\rho_0\,\bxi_a &= -{\bf F}(\bxi_a),\\[0.5ex]
\omega_b^{\,2}\,\rho_0\,\bxi_b &=- {\bf F}(\bxi_b).
\end{align}
Forming the scalar product of the first equation with $\bxi_b$, and the second with $\bxi_a$, taking the difference, and integrating over the
plasma volume, we obtain
\begin{equation}
(\omega_a^{\,2}-\omega_b^{\,2})\int \rho_0\,\bxi_a\cdot\bxi_b\,d{\bf r} = - \int [\bxi_b\cdot {\bf F}(\bxi_a)-\bxi_a\cdot {\bf F}(\bxi_b)]\,d{\bf r}.
\end{equation}
However, the self-adjoint property of the force operator, (\ref{e26}), which is validated by the physical boundary conditions (\ref{e91}), 
yields
\begin{equation}
(\omega_a^{\,2}-\omega_b^{\,2})\int \rho_0\,\bxi_a\cdot\bxi_b\,d{\bf r}=0.
\end{equation}
Hence, we deduce that two discrete normal modes with different frequencies are orthogonal, in the sense that 
\begin{equation}
\int \rho_0\,\bxi_a\cdot\bxi_b\,d{\bf r}=0.
\end{equation}

Of course, the previous proof fails if we encounter multiple  distinct normal modes that share the same frequency, $\omega$. However, any
linear combination of such degenerate modes is also a valid normal mode with frequency $\omega$, and it is always possible to form linear
combinations that are mutually orthogonal. With this caveat, we can state that discrete normal modes are mutually orthogonal. 

\subsection{Energy Conservation}
Taking the scalar product of the perturbed plasma equation of motion, (\ref{e24a}), with $(1/2)\,\bxi^\ast$, and integrating over the
plasma volume, we obtain
\begin{equation}\label{e97}
\delta K +\delta W = 0,
\end{equation}
where
\begin{align}
\delta K(\bxi^\ast,\bxi) &= -\omega^2\,K(\bxi^\ast,\bxi),\\[0.5ex]
K(\bxi^\ast,\bxi) &= \frac{1}{2}\int\rho_0\,|\bxi|^2\,d{\bf r},\label{e99}\\[0.5ex]
\delta W(\bxi^\ast,\bxi) &= -\frac{1}{2}\int \bxi^\ast\cdot {\bf F}(\bxi)\,d{\bf r}.\label{e100}
\end{align}
Equation~(\ref{e97}) is clearly an energy conservation equation. We recognize $\delta K$ as the perturbed kinetic
energy associated with the normal mode. (Actually, for purely growing modes it is the perturbed {\em kinetic energy}\/ at $t=0$, and
for purely oscillatory modes it is the peak perturbed kinetic energy.) It follows that $\delta W$ is the perturbed {\em potential
energy}\/ associated with the mode. (With the same caveats as for the kinetic energy.)  Of course, energy is conserved because
the ideal-MHD equations, (\ref{e1})--(\ref{e5}), contain no dissipative terms. 

\subsection{Variational Formulation}
Equations~(\ref{e97})--(\ref{e99}) can be rearranged to give
\begin{equation}\label{e101}
\omega^2 = \frac{\delta W(\bxi^\ast,\bxi)}{K(\bxi^\ast,\bxi)}
\end{equation}
We wish to demonstrate that any allowable $\bxi({\bf r})$ function for which $\omega^2$ becomes an extremum satisfies the perturbed plasma equation of motion, (\ref{e24a}). The proof follows by letting $\bxi\rightarrow \bxi + \delta\bxi$ and $\omega^2\rightarrow \omega^2+\delta\omega^2$,
and setting $\delta\omega^2=0$ (corresponding to $\omega^2$ being an extremum). 
Neglecting terms that are quadratic in small quantities, we obtain
\begin{equation}
\omega^2+\delta\omega^2= \frac{\delta W(\bxi^\ast,\bxi)+\delta W(\delta\bxi^\ast,\bxi)+\delta W(\bxi^\ast,\delta\bxi)}
{K(\bxi^\ast,\bxi)+K(\delta\bxi^\ast,\bxi)+K(\bxi^\ast,\delta\bxi)}.
\end{equation}
Rearranging the previous equation, making use of Eq.~(\ref{e101}), and again neglecting terms that are quadratic in small quantities,
we get
\begin{equation}
\delta \omega^2 = \frac{\delta W(\delta\bxi^\ast,\bxi)+\delta W(\bxi^\ast,\delta\bxi)-\omega^2\,[K(\delta\bxi^\ast,\bxi)+K(\bxi^\ast,\delta\bxi)]}
{K(\bxi^\ast,\bxi)}. 
\end{equation}
Setting $\delta\omega^2$ to zero yields
\begin{equation}
\omega^2\,K(\delta\bxi^\ast,\bxi)-\delta W(\delta\bxi^\ast,\bxi) + \omega^2\,K(\bxi^\ast,\delta\bxi) + \delta W(\bxi^\ast,\delta\bxi)=0.
\end{equation}
Making use of Eqs.~(\ref{e99}) and (\ref{e100}), as well as the self-adjoint property of the force operator, (\ref{e26}), we get
\begin{equation}
\int \left\{\delta\bxi^\ast\cdot[\omega^2\,\rho_0\,\bxi-{\bf F}(\bxi)]+ \delta\bxi\cdot[\omega^2\,\rho_0\,\bxi^\ast + {\bf F}(\bxi^\ast)]\right\}d{\bf r} = 0.
\end{equation}
However, in order for $\omega^2$ to be an extremum, the previous equation must hold for arbitrary $\delta\bxi$. Hence, 
we obtain
\begin{equation}
-\omega^2\,\rho_0\,\bxi = {\bf F}(\bxi),
\end{equation}
which is identical to Eq.~(\ref{e24a}). 

\subsection{Energy Principle}
The ideal-MHD energy principle states that if
\begin{equation}
\delta W(\bxi^\ast,\bxi) \geq 0
\end{equation}
for all allowable plasma displacements (i.e., bounded in energy and satisfying appropriate boundary conditions) then
the plasma is ideally stable. In other words, there exist no normal modes with $\omega^2<0$. Conversely, if $\delta W$ is negative
for any allowable displacement then the plasma is ideally unstable. In other words, there exists at least one normal mode with $\omega^2<0$. 

The proof of the energy principle is straightforward if one assumes that the normal modes are discrete, and form a complete set of
basis functions, $\bxi_n({\bf r})$, each satisfying
\begin{equation}
-\omega_n^{\,2}\,\rho_0\,\bxi_n = {\bf F}({\bxi}_n).
\end{equation}
In this case, any arbitrary trial function, $\bxi({\bf r})$, can be represented as
\begin{equation}
\bxi({\bf r}) = \sum_n a_n\,\bxi_n({\bf r}).
\end{equation}
Now, we demonstrated in Sect.~\ref{orth} that the normal modes are orthogonal with respect to the weight function $\rho_0$. Let us
normalize them such that they are orthonormal with respect to this weight function:
\begin{equation}
\int \rho_0\,\bxi_n^{\,\ast}\,\cdot\bxi_m\,d{\bf r} = \delta_{nm}.
\end{equation}
It follows that
\begin{align}
\delta W(\bxi^\ast,\bxi) &= -\frac{1}{2}\int\bxi^\ast\cdot {\bf F}(\bxi)\,d{\bf r}\nonumber\\[0.5ex]
&= -\frac{1}{2}\sum_{n,m}a_n^{\ast}\,a_m\int \bxi_n^{\,\ast}\cdot {\bf F}(\bxi_m)\,d{\bf r}=\frac{1}{2}\sum_{n,m}a_n^{\ast}\,a_m\,\omega_m^{\,2}\int \rho_0\,\bxi_n^{\,\ast}\cdot \bxi_m\,d{\bf r}\nonumber\\[0.5ex]
&= \frac{1}{2}\sum_n|a_n|^2\,\omega_n^{\,2}.
\end{align}
The previous equation implies that if a $\bxi({\bf r})$ can be found for which $\delta W <0$ then at least
one of the $\omega_n^{\,2}$ is negative, indicating instability. Conversely, if $\delta W\geq 0$ for all
$\bxi({\bf r})$ then all of the $\omega_n^{\,2}$ are non-negative, indicating stability. 

\subsection{Perturbed Potential Energy}
According to Eqs.~(\ref{e12}), (\ref{e52x}) and (\ref{e100}), the perturbed plasma potential energy can be written
\begin{equation}\label{e112}
\delta W =\frac{1}{2}\int\left\{\bxi^\ast\cdot\left[\mu_0^{-1}\,(\nabla\times {\bf Q})\times {\bf B}_0 - {\bf j}_0\times {\bf Q} -\nabla(\bxi_\perp\cdot\nabla p_0)\right]
+\gamma\,p_0\,|\nabla\cdot\bxi|^2\right\}d{\bf r}.
\end{equation}
However, Eqs.~(\ref{e12}), (\ref{e37}), and (\ref{e41}) imply that 
\begin{equation}
{\bf b}\cdot [{\bf j}_0\times {\bf Q} +\nabla(\bxi_\perp\cdot\nabla p_0)]= 0.
\end{equation}
Hence, Eq.~(\ref{e112}) becomes
\begin{equation}\label{e114}
\delta W = \frac{1}{2}\int\left[\mu_0^{-1}\,(\nabla\times {\bf Q})\cdot(\bxi_\perp^{\,\ast}\times {\bf B}_0) - \bxi_\perp^{\,\ast}\cdot{\bf j}_0\times {\bf Q}
- \bxi_\perp^{\,\ast}\cdot\nabla(\bxi_\perp\cdot\nabla p_0)+\gamma\,p_0\,|\nabla\cdot\bxi|^2\right]d{\bf r}
\end{equation}
Now,
\begin{align}\label{e115}
\int (\nabla\times {\bf Q})\cdot(\bxi_\perp^{\,\ast}\times {\bf B}_0) \,d{\bf r}&=\int\left\{\nabla\cdot[{\bf Q}\times(\bxi_\perp^{\,\ast}\times {\bf B}_0)]
+{\bf Q}\cdot\nabla\times(\bxi_\perp^{\,\ast}\times {\bf B}_0)\right\}d{\bf r}\nonumber\\[0.5ex]
&=\int |{\bf Q}|^2\,d{\bf r},
\end{align}
where use has been made of Eq.~(\ref{e36}). Here, the divergence term integrates to zero because
\begin{equation}
{\bf n}\cdot {\bf Q}\times(\bxi_\perp^{\,\ast}\times {\bf B}_0)= ({\bf Q}\cdot{\bf B}_0)\,({\bf n}\cdot\bxi_\perp^{\,\ast})- ({\bf Q}\cdot\bxi_\perp^{\,\ast})\,({\bf n}\cdot{\bf B}_0)
=0
\end{equation}
at the wall, where use has been made of Eqs.~(\ref{e31x}) and (\ref{e79}). Furthermore, 
\begin{align}\label{e117}
\int \bxi_\perp^{\,\ast}\cdot\nabla(\bxi_\perp\cdot\nabla p_0)\,d{\bf r} &= \int\left\{\nabla\cdot[(\bxi_\perp\cdot\nabla p_0)\,\bxi_\perp^{\,\ast}]
-(\bxi_\perp\cdot \nabla p_0)\,(\nabla\cdot\bxi_\perp^{\,\ast})\right\}d{\bf r}\nonumber\\[0.5ex]
&= -\int (\bxi_\perp\cdot \nabla p_0)\,(\nabla\cdot\bxi_\perp^{\,\ast})\,d{\bf r}.
\end{align}
Here, the divergence term has integrated to zero because of Eq.~(\ref{e79}). Combining Eqs.~(\ref{e114}), (\ref{e115}), and (\ref{e117}),
we obtain the following standard expression for the perturbed potential energy
\begin{equation}
\delta W = \frac{1}{2}\int\left[\mu_0^{-1}\,|{\bf Q}|^2 - \bxi_\perp^{\,\ast}\cdot{\bf j}_0\times {\bf Q} + (\bxi_\perp\cdot\nabla p_0)\,(\nabla\cdot\bxi_\perp^{\,\ast})
+ \gamma\,p_0\,|\nabla\cdot\bxi|^2\right]d{\bf r}.
\end{equation}

\end{document}