\documentclass[12pt,prb,aps]{revtex4-1}
\usepackage{amsmath}           		          	
\usepackage{graphicx,epstopdf}					
\usepackage{amssymb}
\usepackage{fullpage}
\usepackage{color}
\usepackage{esint}
\pdfoutput = 1 
\newcommand {\bxi}{\mbox{\boldmath$\xi$}}
\allowdisplaybreaks

\begin{document}
\title{Further Calculation of Tearing Mode Stability in an Inverse Aspect-Ratio Expanded Tokamak Plasma Equilibrium}
\author{Richard Fitzpatrick\,\footnote{rfitzp@utexas.edu}}
\affiliation{Institute for Fusion Studies, Department of Physics, University of Texas at Austin, Austin, TX 78712}

\begin{abstract}

\end{abstract}
\maketitle

\section{Introduction} 
The calculation of the stability of a tokamak plasma equilibrium to tearing perturbations is most efficiently formulated as  an asymptotic
matching problem in which the  plasma is  divided into two distinct regions.\cite{fkr}  In the ``outer region'', which comprises most
of the plasma, the tearing perturbation is described by the equations of linearized, marginally-stable, ideal magnetohydrodynamics (a.k.a.\ the ``ideal-MHD'' equations). 
However, these equations become singular on so-called ``rational'' magnetic flux-surfaces at which the perturbed magnetic field resonates with the equilibrium field. In the so-called ``inner region'', which
consists of a set of narrow layers centered on the various rational surfaces, non-ideal-MHD effects such as plasma inertia, resistivity, 
viscosity,  diamagnetic flows, and the ion sound radius,  become important.  The resistive layer
solutions in the various segments of the inner region must be simultaneously asymptotically matched to the ideal-MHD solution in the outer region to produce a matrix 
dispersion relation that determines the growth-rates and angular rotation frequencies of the various tearing modes to which the plasma is subject.\cite{con0,cht}

In general, the  determination of the ideal-MHD contributions to the elements of the  tearing mode dispersion relation is an exceptionally challenging computational task.\cite{connor,nish,gal,pletz,pletz1,
tokuda,brennan,ham,ham1,ham2,am1,am2,am3,aglas,aglas1,aglas2}
One way of greatly reducing the complexity of this task is to employ an inverse aspect-ratio expanded plasma equilibrium.\cite{greene,gim,inverse} In such an equilibrium,
the metric elements of the flux-coordinate system can be expressed analytically in terms of a relatively small number of  flux-surface functions,
which represents a major simplification.\cite{con0} Another significant advantage of an inverse aspect-ratio expanded equilibrium is that the magnetic perturbation in the plasma can be efficiently 
matched to an exterior vacuum solution  that is expressed as an expansion in toroidal functions.\cite{am1} The aspect-ratio expansion approach to the tearing mode stability problem has been realized in the recently developed TJ code,
which is described in Ref.~\onlinecite{tj}.

\appendix

\section{General Definitions}
\subsection{Normalization}\label{coords}
All lengths are normalized to  the major radius of the plasma magnetic axis, $R_0$. All magnetic field-strengths
are normalized to the  toroidal field-strength at the magnetic axis, $B_0$. All currents are normalized to $B_0\,R_0/\mu_0$. All current densities are normalized to $B_0/(\mu_0\,R_0)$.  All plasma pressures are normalized to $B_0^{\,2}/\mu_0$.
All toroidal electromagnetic torques are normalized to $B_0^{\,2}\,R_0^{\,3}/\mu_0$. All energies are normalized to $B_0^{\,2}\,R_0^{\,3}/\mu_0$. 

\subsection{Axisymmetric Tokamak Plasma Equilibrium}\label{s3}
Let $R$, $\phi$, $Z$ be right-handed cylindrical coordinates whose Jacobian 
is $ (\nabla R\times \nabla\phi\cdot\nabla Z)^{-1} = R$, 
and whose symmetry axis corresponds to the symmetry axis of the axisymmetric toroidal plasma equilibrium. 
Note that $|\nabla\phi|=1/R$. 

Let $r$, $\theta$, $\phi$ be right-handed flux-coordinates whose
Jacobian is
\begin{equation}\label{jac}
{\cal J}(r,\theta)\equiv (\nabla r\times \nabla\theta\cdot\nabla\phi)^{-1}= r\,R^{\,2}.
\end{equation}
Note that $r=r(R,Z)$ and $\theta=\theta(R,Z)$. 
The magnetic axis corresponds to $r=0$, and the plasma-vacuum interface to $r=a$. 

Consider an axisymmetric tokamak plasma equilibrium whose magnetic field takes the form
\begin{equation}
{\bf B}(r,\theta) = f(r)\,\nabla\phi\times \nabla r + g(r)\,\nabla\phi = f\,\nabla(\phi-q\,\theta)\times \nabla r,
\end{equation}
where
$q(r) = r\,g/f$ is the safety-factor. Note that ${\bf B}\cdot\nabla r=0$, which implies that $r$ is a magnetic flux-surface label.
Furthermore,   $B^r=0$, $B^\theta= f/{\cal J}$ and  $B^\phi =f\,q/{\cal J}$. Here, superscript/subscript denote contravariaint/covariant
vector components. 

Equilibrium force balance requires that
$ \nabla P={\bf J}\times {\bf B}$, 
where $P(r)$ is the equilibrium scalar plasma pressure, and ${\bf J}=\nabla\times {\bf B}$ the equilibrium plasma current density. 

\section{Calculation of Ideal Stability}

\subsection{Plasma Perturbation}
The perturbed magnetic field associated with an ideal perturbation is written\,\cite{freidberg,ideal}
\begin{equation}
{\bf b} = \nabla\times (\bxi\times{\bf B}),
\end{equation}
where $\bxi$ is the plasma displacement. All perturbed quantities are assumed to vary toroidally as $\exp(-{\rm i}\,n\,\phi)$,
where the integer  $n>0$ is the toroidal mode number of the perturbation.  According to Eqs.~(2), (25), (26), and
(101) of Ref.~\onlinecite{tj}, 
\begin{align}\label{e9}
{\cal J}\,b^r&=\left(\frac{\partial}{\partial\theta}-{\rm i}\,n\,q\right)y = {\rm i}\,\psi(r,\theta),
\end{align}
where
\begin{align}\label{e10}
y(r,\theta)&= f\,\xi^r.
\end{align}
Furthermore, if
\begin{align}
b_\phi = x(r,\theta)= n\,z(r,\theta)
\end{align}
then Eqs.~(38), (39), (47), (48), and (78) of Ref.~\onlinecite{tj} yield
\begin{align}
b_\theta = -\frac{\alpha_g}{{\rm i}\,n}\left(\frac{\partial}{\partial\theta} - {\rm i}\,n\,q\right)y +\alpha_p\,R^{\,2}\,y +{\rm i}\,\frac{\partial z}{\partial\theta},
\end{align}
where
\begin{align}
\alpha_p(r) &= \frac{r\,P'}{f^2},\label{ap}\\[0.5ex]
\alpha_g (r)&= \frac{g'}{f}.\label{ag}
\end{align}
Here, $'$ denotes $d/dr$. 
Thus,
\begin{equation}\label{e15}
{\bf B}\cdot{\bf b} -\xi^r\,P' = B^\theta\,b_\theta+B^\phi\,b_\phi - \xi^r\,P'=
\frac{{\rm i}\,f}{\cal J}\left(\frac{\partial}{\partial\theta}- {\rm i}\,n\,q\right)\left(\frac{\alpha_g}{n}\,y+z\right).
\end{equation}

\subsection{Plasma Potential Energy}
The perturbed ideal-MHD  force operator in the plasma takes the well-known form\,\cite{freidberg,ideal}
\begin{equation}
{\bf F}(\bxi)= \nabla({\mit\Gamma}\,P\,\nabla\cdot\bxi) - {\bf B}\times(\nabla\times {\bf b})+\nabla(\bxi\cdot\nabla P)+{\bf J}\times  {\bf b},
\end{equation}
where ${\mit\Gamma}=5/3$ is the plasma ratio of specific heats. 
The perturbed  plasma potential energy in the region lying between the magnetic  flux-surfaces whose labels are $r_1$ and $r_2$ can be written\,\cite{ideal}
\begin{align}
\delta W_{12} &= \frac{1}{2}\int_{r_1}^{r_2}\oint\oint\left[{\mit\Gamma}\,P\,(\nabla\cdot\bxi^\ast)\,(\nabla\cdot\bxi)+ \nabla\times (\bxi^\ast\times {\bf B})\cdot {\bf b}
+(\nabla\cdot\bxi^\ast)\,(\bxi\cdot\nabla P)\right.\nonumber\\[0.5ex]&\phantom{=}
\left.+{\bf J}\times \bxi^\ast\,\cdot{\bf b}\right]{\cal J}\,dr\,d\theta\,d\phi.
\end{align}

Now,
\begin{align}
{\mit\Gamma}\,P\,(\nabla\cdot\bxi^\ast)\,(\nabla\cdot\bxi)&=\nabla\cdot({\mit\Gamma}\,P\,\bxi^\ast\,\nabla\cdot\bxi)-\bxi^\ast\cdot\nabla(
{\mit\Gamma}\,P\,\nabla\cdot\bxi),\\[0.5ex]
 \nabla\times (\bxi^\ast\times {\bf B})\cdot {\bf b}&= \nabla\cdot[(\bxi^\ast\times {\bf B})\times {\bf b}] + \bxi^\ast\times {\bf B}\cdot\nabla\times {\bf b},\\[0.5ex]
 (\nabla\cdot\bxi^\ast)\,(\bxi\cdot\nabla P)&=\nabla\cdot(\bxi^\ast\,\bxi\cdot\nabla P) -\bxi^\ast\cdot\nabla(\bxi\cdot\nabla P),
\end{align}
so
\begin{align}
\delta W_{12} &= \frac{1}{2}\int_{r_1}^{r_2}\oint\oint\left\{\nabla\cdot[{\mit\Gamma}\,P\,\bxi^\ast\,\nabla\cdot\bxi+ (\bxi^\ast\times {\bf B})\times {\bf b}+\bxi^\ast\,\bxi\cdot\nabla P]\right.\nonumber\\[0.5ex]&\left.-\bxi^\ast\cdot[ \nabla({\mit\Gamma}\,P\,\nabla\cdot\bxi) - {\bf B}\times(\nabla\times {\bf b})+\nabla(\bxi\cdot\nabla P)+{\bf J}\times  {\bf b}]\right\}{\cal J}\,dr\,d\theta\,d\phi,
\end{align}
which yields
\begin{align}
\delta W_{12} &= \frac{1}{2}\left(\oint\oint{\cal J}\,\nabla r\cdot[{\mit\Gamma}\,P\,\bxi^\ast\,\nabla\cdot\bxi+ (\bxi^\ast\times {\bf B})\times {\bf b}+\bxi^\ast\,\bxi\cdot\nabla P]\,d\theta\,d\phi\right)_{r_1}^{r_2}\nonumber\\[0.5ex]
&-\frac{1}{2}\int_{r_1}^{r_2}\oint\oint \bxi^\ast\cdot{\bf F}(\bxi)\,{\cal J}\,dr\,d\theta\,d\phi,
\end{align}
or
\begin{align}
\delta W_{12} &= \frac{1}{2}\left[\oint\oint{\cal J}\,\xi^{r\,\ast}\,({\mit\Gamma}\,P\,\nabla\cdot\bxi -{\bf B}\cdot{\bf b} + \xi^r\,P')\,d\theta\,d\phi\right]_{r_1}^{r_2}\nonumber\\[0.5ex]&\phantom{=}-\frac{1}{2}\int_{r_1}^{r_2}\oint\oint\bxi^\ast\cdot{\bf F}(\bxi)\,{\cal J}\,dr\,d\theta\,d\phi,
\end{align}
where use has been made of the fact that ${\bf B}\cdot\nabla r = 0$.  However, in the TJ code, the marginally-stable ideal-MHD  plasma perturbation in the outer region is calculated assuming that ${\bf F}(\bxi)={\bf 0}$ and
$\nabla\cdot\bxi=0$.\cite{tj}   Thus, making use of Eqs.~(\ref{e10}) and (\ref{e15}), we obtain 
\begin{equation}
\delta W_{12} = \frac{1}{2}\int_{r_1}^{r_2}\left[-{\rm i}\,y^\ast\left(\frac{\partial}{\partial\theta}-{\rm i}\,n\,q\right)\left(\frac{\alpha_g}{n}\,y+z\right)d\theta\,d\phi\right]_{r_1}^{r_2},
\end{equation}
which reduces to
\begin{equation}
\delta W_{12}= \pi^2\left[\sum_m\,y_m^\ast\,(m-n\,q)\left(\frac{\alpha_g}{n}\,y_m+z_m\right)\right]_{r_1}^{r_2},
\end{equation}
where
\begin{align}\label{e26}
y(r,\theta)&=\sum_m y_m(r)\,\exp(\,{\rm i}\,m\,\theta),\\[0.5ex]
z(r,\theta)&=\sum_m z_m(r)\,\exp(\,{\rm i}\,m\,\theta),\\[0.5ex]
\psi(r,\theta)&=\sum_m \psi_m(r)\,\exp(\,{\rm i}\,m\,\theta).\label{e28}
\end{align}
Here, $m$ is a poloidal mode number, and the sum is over all mode numbers included in the calculation. 

Now, according to Eqs.~(98) and (99) of Ref.~\onlinecite{tj}, 
\begin{align}\label{epsi}
y_m(r) &= \left(\frac{\psi_m}{m-n\,q}\right)_r,\\[0.5ex]
z_m(r)&= \left(\frac{k_m\,\psi_m+Z_m}{m-n\,q}\right)_r,
\end{align}
where
$k_m(r)$ is real, and is specified in Eq.~(100) Ref.~\onlinecite{tj}. 
Thus, we get 
\begin{equation}
\delta W_{12}=\pi^2\left(\sum_m\psi_m^\ast\,\chi_m\right)_{r_1}^{r_2}
\end{equation}
where
\begin{align}\label{e29}
\chi_m(r)&=
 \left(\frac{\tilde{k}_m\,\psi_m+Z_m}{m-n\,q}\right)_r,\\[0.5ex]
\tilde{k}_m(r) &=\left( k_m + \frac{\alpha_g}{n} \right)_r= \left[\frac{\alpha_g\,(m\,q\,c_m^{\,m}+n\,r^2)+\alpha_p\,m\,d_m^{\,m}}{m^2\,c_m^{\,m}+n^2\,r^2}\right]_r,\label{e30}
\end{align}
and use has been made of Eq.~(100) Ref.~\onlinecite{tj}. Here, $\tilde{k}_m(r)$ is real, and
\begin{align}
c_m^{\,m}(r) &=\oint|\nabla r|^{-2}\,\frac{d\theta}{2\pi},\\[0.5ex]
d_m^{\,m}(r) &=\oint|\nabla r|^{-2}\,R^{\,2}\,\frac{d\theta}{2\pi}.
\end{align}

\subsection{Magnetic Axis}
Let the $\psi_m(r)$ and the $Z_m(r)$ be solutions of the ideal-MHD equations in the outer region that are well behaved at $r=0$. It follows that\,\cite{tj}
\begin{equation}
\left(\sum_m\psi_m^\ast\,\chi_m\right)_{0}=0.
\end{equation}
Hence, 
\begin{equation}\label{e1}
\delta W_p(r) =\pi^2\left(\sum_m\psi_m^\ast\,\chi_m\right)_r
\end{equation}
is the perturbed plasma potential energy in the region lying between the magnetic axis and the magnetic flux-surface whose label is $r$. 

\subsection{Rational Surfaces}
At a rational surface, $r=r_m$, at which $m-n\,q(r_m)=0$ for some $m$, the non-resonant components of $\psi_m$ and $\chi_m$ are continuous, the 
``large'' solution is absent (i.e., there is zero reconnected magnetic flux), and the ``small'' solution is discontinuous (i.e., there is a current sheet). It is easily shown that\,\cite{tj}
\begin{equation}
\left(\sum_m\psi_m^\ast\,\chi_m\right)_{r_{m-}}^{r_{m+}}=0.
\end{equation}
In other words, there is no contribution to the perturbed potential energy from the surface. Thus, Eq.~(\ref{e1}) holds even when the region between the
magnetic axis and the flux-surface whose label is $r$ contains rational surfaces. 
Thus, the total plasma potential energy is 
\begin{equation}\label{a32}
\delta W_p =\pi^2\left(\sum_m\psi_m^\ast\,\chi_m\right)_{a_-},
\end{equation}
where $r=a_-$ corresponds to an equilibrium magnetic flux that lies just inside the plasma-vacuum interface. 

\subsection{Toroidal Electromagnetic Torque}
The flux-surface averaged toroidal electromagnetic torque acting on the plasma   is\,\cite{tj,ideal}
\begin{align}
T_\phi& = {\rm i}\,n\,\pi^2\left(\sum_m\frac{Z_m^\ast\,\psi_m-\psi_m^\ast\,Z_m}{m-n\,q}\right)_{a_-}= {\rm i}\,n\,\pi^2\left(\sum_m
\chi_m^{\,\ast}\,\psi_m - \psi_m^{\,\ast}\,\chi_m\right)_{a_-}\nonumber\\[0.5ex]
&= 2\,n\,{\rm Im}(\delta W_p),
\end{align}
where use has been made of Eqs.~(\ref{e29}) and (\ref{e1}). 
However, this torque must be zero for ideal solutions, which are characterized by zero reconnected magnetic flux at the various rational surfaces in the plasma.\cite{tj}
It follows that
 the total plasma potential energy, $\delta W_p$, is necessarily a real quantity. 
 
 \subsection{Marginally-Stable Ideal Eigenfunctions}\label{sideal}
We can construct a complete set of  marginally-stable ideal eigenfuctions in the outer region from the tearing eigenfunctions calculated by the TJ code as follows:
\begin{align}
\psi_{m\,m'}^i(r)&= \psi_{m\,m'}^a(r)-\sum_{k}\psi_{m\,k}^u(r)\,{\mit\Pi}_{k\,m'}^a,\\[0.5ex]
Z_{m\,m'}^i(r)&= Z_{m\,m'}^a(r)-\sum_{k}Z_{m\,k}^u(r)\,{\mit\Pi}_{k\,m'}^a.
\end{align}
 Here, $m$ labels the poloidal harmonic of the solution,  $m'$ labels the solution itself, and $k$ labels the various rational surfaces in the plasma. There are as many solutions as there are poloidal harmonics included in the calculation. All other quantities are defined in Sects.~VIII.B and VIII.D of Ref.~\onlinecite{tj}. 
 It follows
from the definitions of the various quantities that the reconnected magnetic  fluxes at the rational surfaces in the plasma
associated with the ideal eigenfunctions are all zero. We can also write
\begin{equation}\label{e39}
\chi^i_{m\,m'}(r) = \left(\frac{\tilde{k}_{m}\,\psi^i_{m\,m'}+Z_{m\,m'}^i}{m-n\,q}\right)_r,
\end{equation}
where use has been made of Eq.~(\ref{e29}).

\subsection{Plasma Energy Matrix}
We can write a general ideal solution at the plasma boundary as
\begin{align}
\psi_m(a) &= \sum_{m'}\psi_{m\,m'}^i(a)\,\alpha_{m'},\\[0.5ex]
\chi_m(a_-) &= \sum_{m'}\chi_{m\,m'}^i(a_-)\,\alpha_{m'},
\end{align}
where the $\alpha_m$ are arbitrary complex coefficients. 
The previous two equations can be written more succinctly as
\begin{align}\label{e41}
\underline{\psi}&= \underline{\underline{\psi}}_{\,i}\,\underline{\alpha},\\[0.5ex]
\underline{\chi}&= \underline{\underline{\chi}}_{\,i}\,\underline{\alpha},\label{e42}
\end{align}
where $\underline{\psi}$ is the column vector of the $\psi_m(a)$ values, $\underline{\chi}$ is the column vector of the $\chi_m(a_-)$ values,
$\underline{\underline{\psi}}_{\,i}$ is the matrix of the $\psi_{mm'}^i(a)$ values, $\underline{\underline{\chi}}_{\,i}$ is the matrix of the $\chi_{mm'}^i(a_-)$ values,
and $\underline{\alpha}$ the column vector of the $\alpha_m$ values. 
Equation~(\ref{a32}) then becomes
\begin{equation}
\delta W_p=\pi^2\,\underline{\psi}^\dag\,\underline{\chi},
\end{equation} 
or
\begin{equation}\label{e44}
\delta W_p =\pi^2\, \underline{\alpha}^\dag\,\underline{\underline{\psi}}^{\dag}_{\,i}\,\underline{\underline{\chi}}_{\,i}\,\underline{\alpha}.
\end{equation}
The fact that $\delta W_p$ is real implies that
 \begin{equation}\label{e47}
 \underline{\underline{\psi}}^{\dag}_{\,i}\,\underline{\underline{\chi}}_{\,i}= \underline{\underline{\chi}}^{\dag}_{\,i}\,\underline{\underline{\psi}}_{\,i}.
 \end{equation}

Let us define the plasma energy matrix, $\underline{\underline{W}}_{\,p}$, such that 
\begin{equation}\label{e48}
\underline{\underline{\chi}}_{\,i} = \underline{\underline{W}}_{\,p}\,\underline{\underline{\psi}}_{\,i}.
\end{equation}
 It is easily seen that
 \begin{align}
 \underline{\underline{\psi}}^{\dag}_{\,i}\,\underline{\underline{\chi}}_{\,i}= \underline{\underline{\psi}}^{\dag}_{\,i}\,\underline{\underline{W}}_{\,p}\,
 \underline{\underline{\psi}}_{\,i},\\[0.5ex]
 \underline{\underline{\chi}}^{\dag}_{\,i}\,\underline{\underline{\psi}}_{\,i}= \underline{\underline{\psi}}^{\dag}_{\,i}\,\underline{\underline{W}}_{\,p}^{\dag}\,
 \underline{\underline{\psi}}_{\,i}.
 \end{align}
 Making use of Eq.~(\ref{e47}), we obtain
 \begin{equation}
 \underline{\underline{\psi}}^{\dag}_{\,i}\,\underline{\underline{W}}_{\,p}\,
 \underline{\underline{\psi}}_{\,i}=
 \underline{\underline{\psi}}^{\dag}_{\,i}\,\underline{\underline{W}}_{\,p}^{\dag}\,
 \underline{\underline{\psi}}_{\,i},
 \end{equation}
 which implies that $\underline{\underline{W}}_{\,p}$ is an Hermitian matrix. Equations~(\ref{e44}) and (\ref{e48}) yield
 \begin{equation}\label{e52}
 \delta W_p = \pi^2\,\underline{\alpha}^\dag\,\underline{\underline{\psi}}^{\dag}_{\,i}\,\underline{\underline{W}}_{\,p}\,\underline{\underline{\psi}}_{\,i}\,\underline{\alpha}.
 \end{equation}
 
 \subsection{Total Potential Energy}
The total perturbed potential energy of the plasma-vacuum system can be written\,\cite{freidberg,ideal}
\begin{equation}\label{e79}
\delta W= \delta W_p + \delta W_v,
\end{equation}
where
\begin{equation}
\delta W_v = \frac{1}{2}\int_{a+}^\infty\oint\oint {\bf b}^\ast\cdot{\bf b} \,{\cal J}\,dr\,d\theta\,d\phi
\end{equation}
is the perturbed potential energy of the surrounding vacuum. Here, $r=a_+$ corresponds to an equilibrium magnetic flux-surface that lies just outside the plasma-vacuum interface. 

\subsection{Vacuum Potential Energy}
In the vacuum region, we can write the curl- and divergence-free perturbed magnetic field in the form 
${\bf b} = {\rm i}\,\nabla V$,
where
$\nabla^2 V =0$.
Hence, the vacuum potential energy is
\begin{align}
\delta W_v &= \frac{1}{2}\int_{a+}^\infty\oint\oint\nabla V\,\cdot\nabla V^\ast\,{\cal J}\,dr\,d\theta\,d\phi\nonumber\\[0.5ex]
&= \frac{1}{2}\int_{a+}^\infty\oint\oint  \nabla\cdot(V\,\nabla V^\ast)\,{\cal J}\,dr\,d\theta\,d\phi=-\frac{1}{2}\left(\oint\oint {\cal J}\,\nabla r\cdot\nabla V^\ast\,V\,d\theta\,d\phi\right)_{a_+},
\end{align}
assuming that $V(r,\theta)\rightarrow 0$ as $r\rightarrow \infty$ if there is no wall surrounding the plasma, or  $\nabla V\cdot d{\bf S}=0$ if a perfectly-conducting wall
surrounds the plasma. 
But, Eq.~(210) of Ref.~\onlinecite{tj} implies that 
\begin{equation}
{\cal J}\,\nabla V\cdot\nabla r = \psi,
\end{equation}
so we deduce that
\begin{equation}\label{e87}
\delta W_v = -\frac{1}{2}\left(\oint\oint \psi^\ast\,V\,d\theta\,d\phi\right)_{a_{+}} =- \pi^2\left(\sum_m \psi_m^\ast\,V_m\right)_{a_{+}},
\end{equation}
where
$V(r,\theta)= \sum_m V_m(r)\,\exp(\,{\rm i}\,m\,\theta)$. 
However, making use of Eq.~(214) of Ref.~\onlinecite{tj}, we get
\begin{equation}\label{e88}
\delta W_v =-\pi^2\left(\sum_m\psi_m^\ast\,\chi_m\right)_{a_+},
\end{equation}
where
\begin{equation}\label{e89}
\chi_m=\frac{Z_m}{m-n\,q}.
 \end{equation}
 Note that the previous equation is consistent with Eq.~(\ref{e29}) because, according to Eq.~(\ref{e30}),  $\tilde{k}_m=0$ in the vacuum region, given that $\alpha_g=\alpha_p=0$ in the vacuum. 

Combining Eqs.~(\ref{a32}), (\ref{e79}),  and (\ref{e88}), we deduce that
\begin{equation}\label{e89a}
\delta W = \pi^2\left(\sum_m\psi_m^\ast\,J_m\right)_\epsilon,
\end{equation}
where
\begin{equation}\label{e90}
J_m= -\left[\chi_m\right]_{a_-}^{a_+}.
\end{equation}
Note that the $\psi_m(r)$ are continuous across the plasma-vacuum interface, whereas the $\chi_m(r)$ are generally discontinuous. 

\subsection{Boundary Current Sheet}
Now, $\alpha_p=\alpha_g=0$ at the plasma-vacuum interface, assuming that the  equilibrium current is zero there,
which implies that $k_m=\tilde{k}_m=0$ at the interface, where use has been made of Eq.~(\ref{e30}),
as well as Eq.~(100) of Ref.~\onlinecite{tj}.  It follows from Eqs.~(78) and (99) of Ref.~\onlinecite{tj}, combined with Eq.~(\ref{e89}),  that 
\begin{equation}\label{e91}
x_m= n\,\chi_m
\end{equation}
at the plasma-vacuum interface, where $x(r,\theta)=\sum_m x_m(r)\,\exp(\,{\rm i}\,m\,\theta)$. 
Suppose that there is a perturbed current sheet on the interface. 
Thus, if ${\bf K}$ is the current sheet density then Eqs.~(66) and (67) of Ref.~\onlinecite{tj} suggest that
\begin{align}
{\cal J}\,K_m^{\,\theta} &=  n\,J_m,\\[0.5ex]
{\cal J}\,K_m^{\,\phi} &=m\,J_m,
\end{align}
where use has been made of Eqs.~(\ref{e90}) and (\ref{e91}). 
Let us write
\begin{equation}\label{e94}
{\bf K} = {\rm i}\,\nabla J\times \nabla r,
\end{equation}
where $J(\theta,\phi)=J(\theta)\,\exp(-{\rm i}\,n\,\phi)$, which ensures that $\nabla\cdot{\bf K}=0$. It follows from (A8) and (A9) of Ref.~\onlinecite{tj} that
\begin{align}
{\cal J}\,K^\theta&= {\rm i}\,\frac{\partial J}{\partial \phi},\\[0.5ex]
{\cal J}\,K^\phi &=-{\rm i}\,\frac{\partial J}{\partial\theta}.
\end{align}
Hence, we deduce that 
\begin{align}
{\cal J}\,K^\theta_m&= n\,J_m,\\[0.5ex]
{\cal J}\,K^\phi_m &=m\,J_m.
\end{align}
Thus, it is clear that the $J_m$ are the Fourier components of the $J(\theta)$ function introduced in Eq.~(\ref{e94}). 

As described in Ref.~\onlinecite{ideal}, the marginally-stable ideal eigenfunctions   feature a perturbed current sheet on the vacuum-plasma interface because they do not satisfy the
perturbed  pressure-balance boundary condition.
However, these current sheets are entirely fictitious. The true ideal eigenfunctions (which, unlike the marginally-stable ideal eigenfunctions, take plasma inertia into account)
satisfy the pressure-balance boundary condition (and, therefore, have no associated current sheets). 

\subsection{Determination of Ideal Stability}
According to Eq.~(215) of Ref.~\onlinecite{tj}, combined with Eq.~(\ref{e89}), 
\begin{equation}
\chi_m(a_+)=\sum_{m'}\,H_{m\,m'}\,\psi_{m'}(a),
\end{equation}
where the vacuum matrix $H_{mm'}$ is Hermitian. 
Hence, it follows from Eq.~(\ref{e90}) that
\begin{equation}
J_m = \chi_m(a_-)-\sum_{m'} H_{m\,m'}\,\psi_m(a).
\end{equation}
Making use of Eq.~(\ref{e89a}), we can write
\begin{equation}
\delta W =\pi^2\, \underline{\psi}^\dag\,\underline{J},
\end{equation}
where 
\begin{equation}
\underline{J} = \underline{\chi}+ \underline{\underline{W}}_{\,v}\,\underline{\psi},
\end{equation}
and  $\underline{J}$ the column vector of the $J_m$ values, 
and $\underline{\underline{W}}_{\,v}$  the Hermitian matrix of the $-H_{m\,m'}$ values. 
Making use of Eqs.~(\ref{e41}), (\ref{e42}), and (\ref{e48}), we get 
\begin{equation}
\underline{J} = \underline{\underline{W}}\,\underline{\psi},
\end{equation}
and
\begin{equation}
\delta W = \underline{\alpha}^\dag\,\underline{\underline{\psi}}^{\dag}_{\,i}\,(\underline{\underline{\chi}}_{\,i} + \underline{\underline{W}}_{\,v}\,\underline{\underline{\psi}}_{\,i})\,\underline{\alpha}=  \underline{\alpha}^\dag\,\underline{\underline{\psi}}^{\dag}_{\,i}\,\underline{\underline{W}}\,\underline{\underline{\psi}}_{\,i}\,\underline{\alpha},
\end{equation}
where
\begin{equation}
\underline{\underline{W}}=\underline{\underline{W}}_{\,p}+\underline{\underline{W}}_{\,v}.
\end{equation}

Note that $\underline{\underline{W}}_{\,p}$ and $\underline{\underline{W}}_{\,v}$ are both Hermitian, so $\underline{\underline{W}}$ is also
Hermitian. Thus the energy matrix, $\underline{\underline{W}}$, possesses real eigenvalues and orthonormal eigenvectors, $\underline{\beta}_m$. 
We conclude that there are as many linearly independent ideal eigenmodes of the plasma as there are poloidal harmonics included in the calculation. The $m$th
eigenmode has the associated energy
\begin{equation}
\delta W_m = \delta W_{p\,m} + \delta W_{v\,m},
\end{equation}
where 
\begin{align}
\delta W_m &= \underline{\beta}_m^{\,\dag}\,\underline{\underline{W}}\, \underline{\beta}_m,\\[0.5ex]
\delta W_{p\,m} &= \underline{\beta}_m^{\,\dag}\,\underline{\underline{W}}_{\,p}\, \underline{\beta}_m,\\[0.5ex]
\delta W_{v\,m} &= \underline{\beta}_m^{\,\dag}\,\underline{\underline{W}}_{\,v}\, \underline{\beta}_m.
\end{align}
Note that $\delta W_m$, $\delta W_{p\,m}$ and $\delta W_{v\,m}$ are all real quantities. Moreover, $\delta W_{p\,m}$ and $\delta W_{v\,m}$ can be interpreted as the
plasma and vacuum contributions to $\delta W_m$, respectively. Finally, if any of the $\delta W_m$ are negative then the plasma is ideally unstable.\cite{freidberg,ideal}


\begin{thebibliography}{99}\baselineskip 5ex

\bibitem{fkr} H.P.~Furth,  J.~Killeen and M.N.~Rosenbluth,  Phys.\ Fluids {\bf 6}, 459 (1963).

\bibitem{con0} J.W.~Connor and R.J.~Hastie, {\em The Effect of Shaped Plasma Cross Sections on the Ideal Kink Mode in a Tokamak}\/ (Rep. CLM-M106, Culham Laboratory, Abingdon UK, 1985).

\bibitem{cht} J.W.~Connor, R.J.~Hastie and J.B.~Taylor, Phys.\ Fluids B {\bf 3}, 1539 (1991).

\bibitem{connor} J.W.~Connor,  S.C.~Cowley, R.J.~Hastie,  T.C.~Hender,  A.~Hood  and T.J.~Martin,  Phys.\ Fluids {\bf 31}, 577 (1988).

\bibitem{pletz} A.~Pletzer and R.L.~Dewar, J.\ Plasma Physics {\bf 45}, 427 (1991).

\bibitem{am1} R.~Fitzpatrick, R.J.~Hastie, T.J.~Martin and C.M.~Roach, Nucl.\ Fusion {\bf 33}, 1533 (1993).

\bibitem{pletz1}  A.~Pletzer, A.~Bondeson and R.L.~Dewar, J.\ Comput.\ Phys.\ {\bf 115}, 530 (1994).

\bibitem{nish} Y.~Nishimura, J.D.~Callen and C.C.~Hegna, Phys.\ Plasmas {\bf 5}, 4292 (1998).

\bibitem{gal} S.A.~Galkin, A.D.~Turnbull, J.M.~Greene and M.S.~Chu, Phys.\ Plasmas {\bf 7}, 4070 (2000). 

\bibitem{tokuda} S.~Tokuda, Nucl.\ Fusion {\bf 41}, 1037 (2001).

\bibitem{brennan} D.P.~Brennan, R.J.~La Haye, A.D.~Turnbull, M.S.~Chu, T.H.~Jensen, L.L.~Lao, T.C.~Luce, P.A.~Politzer and E.J.~Strait, Phys.\ Plasmas {\bf 10}, 1643 (2003).

\bibitem{ham1}  C.J.~Ham, Y.Q.~Liu, J.W.~Connor, S.C.~Cowley, R.J.~Hastie. T.C.~Hender and T.J.~Martin, Plasma Phys.\ Controlled Fusion {\bf 54}, 105014 (2012). 

\bibitem{ham} C.J.~Ham, J.W.~Connor, S.C.~Cowley, C.G.~Gimblett, R.J.~Hastie, T.C.~Hender and T.J.~Martin, Plasma Phys.\ Controlled Fusion {\bf 54}, 025009 (2012). 

\bibitem{ham2} C.J.~Ham, J.W.~Connor, S.C.~Cowley, R.J.~Hastie, T.C.~Hender and Y.Q.~Liu, Plasma Phys.\ Control.\ Fusion {\bf 55} 125015 (2013).

\bibitem{am2} A.H.~Glasser, Z.R.~Wang and J.-K.~Park, Phys.\ Plasmas {\bf 23}, 112506 (2016).

\bibitem{am3} R.~Fitzpatrick, Phys.\ Plasmas {\bf 24}, 072506 (2017). 

\bibitem{aglas} A.S.~Glasser, E.~Kolemen and A.H.~Glasser, Phys.\ Plasmas {\bf 25}, 032507 (2018).

\bibitem{aglas1} A.S.~Glasser and E.~Koleman, Phys.\ Plasmas {\bf 25}, 082502 (2018). 

\bibitem{aglas2} Z.~Wang, A.H.~Glasser, D.~Brennan, Y.~Liu and J.-K.~Park, Phys.\ Plasmas {\bf 27}, 122503 (2020).

\bibitem{greene} J.M.~Greene, J.L.~Johnson and K.E.~Weimer,  Phys.\  Fluids  {\bf 14}, 671 (1971).

\bibitem{gim} R.~Fitzpatrick, C.G.~Gimblett and R.J.~Hastie, Plasma Phys.\ Control.\ Fusion {\bf 34}, 161 (1992). 

\bibitem{inverse} R.~Fitzpatrick, Phys.\ Plasmas {\bf 31}, 082505 (2024).

\bibitem{tj} R.~Fitzpatrick, Phys.\ Plasmas {\bf 31}, 102507 (2024).

\bibitem{freidberg} J.P.~Freidberg, Rev.\ Mod.\ Phys.\ 54, {\bf 801} (1982).

\bibitem{ideal} R.~Fitzpatrick, Phys.\ Plasmas {\bf 31}, 112502 (2024).

%\bibitem{pletzer} A.~Pletzer, A.~Bondeson, and R.L.~Dewar, J.\ Comp.\ Phys.\ {\bf 115}, 530 (1994).

%\bibitem{rf} R.~Fitzpatrick, Phys.\ Plasmas {\bf 1}, 3308 (1994). 

%\bibitem{wesson} J.A.~Wesson, Nucl.\ Fusion {\bf 18},  87 (1978). 





\end{thebibliography}
\end{document}


