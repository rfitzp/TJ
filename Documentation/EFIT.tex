\documentclass[12pt,prb,aps,notitlepage]{revtex4-1}
\usepackage {amsmath}
\usepackage{amssymb}
\pdfoutput = 1 
\usepackage {graphicx}
\newcommand{\bomega}{\mbox{\boldmath$\omega$}}
\allowdisplaybreaks

\begin{document}

\title{Construction of EFIT File}
\author{R.~Fitzpatrick\,\footnote{rfitzp@utexas.edu}}
\affiliation{Institute for Fusion Studies,  Department of Physics,  University of Texas at Austin,  Austin TX 78712, USA}\begin{abstract}
\end{abstract}
\maketitle

The EFIT equilibrium magnetic field is written
\begin{equation}
{\bf B} = T({\mit\Psi})\,\nabla\phi + \nabla\phi\times \nabla{\mit\Psi},
\end{equation}
whereas the TJ equilibrium field is written
\begin{equation}
{\bf B} = B_0\,R_0\left[g(r)\,\nabla\phi+R_0\,f(r)\,\nabla\phi\times\nabla r\right].
\end{equation}
So,
\begin{align}
T &= B_0\,R_0\,g,\\[0.5ex]
\frac{d{\mit\Psi}}{dr} &= B_0\,R_0^{\,2}\,f.
\end{align}
Here, $B_0$ and $R_0$ are the toroidal magnetic field strength at the magnetic axis, and the major radius of the 
magnetic axis, respectively. Let ${\mit\Psi}=0$ at the plasma boundary, $r=\epsilon$. It follows that
\begin{equation}
{\mit\Psi}(r) = B_0\,R_0^{\,2}\int_r^\epsilon f(r')\,dr'.
\end{equation}
Also,
\begin{equation}
q =\frac{r\,g}{f}.
\end{equation}

Let $\hat{r}=r/\epsilon$. It follows that
\begin{equation}
\frac{d{\mit\Psi}}{d\hat{r}} = \epsilon\,B_0\,R_0^{\,2}\,f,
\end{equation}
and
\begin{equation}
{\mit\Psi}(\hat{r}) = \epsilon\,B_0\,R_0^{\,2}\int_{\hat{r}}^1 f(\hat{r}')\,d\hat{r}'.
\end{equation}
Furthermore,
\begin{equation}
T\,\frac{dT}{d{\mit\Psi}}= B_0\,\frac{g}{\epsilon\,f}\,\frac{dg}{d\hat{r}},
\end{equation}
and
\begin{equation}
\frac{dP}{d{\mit\Psi}}= \frac{B_0}{\mu_0\,R_0^{\,2}}\,\frac{1}{\epsilon\,f}\,\frac{dP}{d\hat{r}}.
\end{equation}

Let $\hat{r}\,f = \epsilon\,f_1+\epsilon^3\,f_3$. 
It follows that
\begin{align}
{\mit\Psi}(\hat{r}) &= \epsilon^2\,B_0\,R_0^{\,2}\int_{\hat{r}}^1\frac{f_1(\hat{r}') +\epsilon^2\,f_3(\hat{r}')}{\hat{r}'}\,d\hat{r}',\\[0.5ex]
q({\mit\Psi})&= \frac{\hat{r}^2\,(1+\epsilon^2\,g_2)}{f_1+\epsilon^2\,f_3},\\[0.5ex]
T({\mit\Psi}) &= B_0\,R_0\left(1+\epsilon^2\,g_2\right),\\[0.5ex]
p({\mit\Psi})&= \frac{B_0^{\,2}}{\mu_0}\,\epsilon^2\,p_2(\hat{r}),\\[0.5ex]
T\,\frac{dT}{d{\mit\Psi}}&= B_0\,\frac{1}{f_1+\epsilon^2\,f_3}\,\hat{r}\,\frac{dg_2}{d\hat{r}}\left(1+\epsilon^2\,g_2\right),\\[0.5ex]
\frac{dP}{d{\mit\Psi}} &=\frac{B_0}{\mu_0\,R_0^{\,2}}\,\frac{1}{f_1+\epsilon^2\,f_3}\,\hat{r}\,\frac{dp_2}{d\hat{r}}.
\end{align}
At small $\hat{r}$, $g_2=0$, $f_1=f_{1\,c}\,\hat{r}^{\,2}$,  $f_3=f_{3\,c}\,\hat{r}^{\,2}$, $\hat{r}\,p_2'=p_{2\,c}''\,\hat{r}^{\,2}$, $\hat{r}\,g_2' = g_{2\,c}'\,\hat{r}^{\,2}$, 
${\mit\Psi}= {\mit\Psi}(0) + \epsilon^2\,B_0\,R_0^{\,2}\,(1/2)\,(f_{1\,c}+\epsilon^2\,f_{3\,c})\,\hat{r}^{\,2}$, 
where $g_{2\,c}'= -2\,(f_{1\,c}^{\,2}+p_{2\,c}''/2)$, and $f_{3\,c}=-f_{1\,c}\,(\hat{H}_{2\,c}^{\,2}+\hat{V}_{2\,c}^{\,2})$. 
For $\hat{r}>1$, 
\begin{equation}
{\mit\Psi}(\hat{r}) = \epsilon^2\,B_0\,R_0^{\,2}\,\ln\hat{r}\,[f_1(1)+\epsilon^2\,f_3(1)].
\end{equation}

TJ flux surfaces are parameterized via
\begin{align}
R(\hat{r},\omega) &= 1 -\epsilon\,\hat{r}\,\cos\omega +f_R(\hat{r},\omega),\\[0.5ex]
Z(\hat{r},\omega)&= \epsilon\,\hat{r}\,\sin\omega +f_Z(\hat{r},\omega),
\end{align}
where 
\begin{align}
f_R(\hat{r},\omega) &=  \epsilon^{\,2}\sum_{j>0}H_j(\hat{r})\,\cos[(j-1)\,\omega] + \epsilon^{\,2}\sum_{j>1}V_j(\hat{r})\,\sin[(j-1)\,\omega]\nonumber\\[0.5ex]
&\phantom{=}+\epsilon^{\,3}\,[L_3(\hat{r})+\epsilon\,L_4(\hat{r})]\,\cos\omega,\\[0.5ex]
f_Z(\hat{r},\omega)&= \epsilon^{\,2}\,\sum_{j>1}H_j(\hat{r})\,\sin[(j-1)\,\omega]
-\epsilon^{\,2}\sum_{j>1}V_j(\hat{r})\,\cos[(j-1)\,\omega]\nonumber\\[0.5ex]&\phantom{=}-
\epsilon^{\,3}\,[L_3(\hat{r})+\epsilon\,L_4(\hat{r})]\,\sin\omega.
\end{align}
For $\hat{r}>1$, let $L_3(\hat{r}) = L_3(1)$, $L_4(\hat{r})=L_4(1)$, $H_j(\hat{r})= H_j(1)$, and $V_j(\hat{r})= V_j(1)$. 
To obtain the $\hat{r}$, $\omega$ coordinates that correspond to the point $R$, $Z$, our initial guess is
\begin{align}
\hat{r} &= \frac{[(R-1)^2+Z^2]^{1/2}}{\epsilon},\\[0.5ex]
\frac{\sin\omega}{\cos\omega} &= \frac{Z}{1-R}.
\end{align}
We can then iterate the following equations:
\begin{align}
\hat{r} &= \frac{[(\hat{R}-1)^2+\hat{Z}^2]^{1/2}}{\epsilon},\\[0.5ex]
\frac{\sin\omega}{\cos\omega} &= \frac{\hat{Z}}{1-\hat{R}},
\end{align}
where
\begin{align}
\hat{R} &=R-f_R(\hat{r},\omega),\\[0.5ex]
\hat{Z} &= Z-f_Z(\hat{r},\omega).
\end{align}

EFIT integer parameters are NRBOX, NZBOX, NPBOUND, NLIMITER. EFIT float parameters are RBOXLEN, ZBOXLEN, RBOXLFT, ZOFF. 
Also, R0EXP $= R_0$, B0EXP = $B_0$, RAXIS = $R_0$, ZAXIS $=0$, PSIAXIS = ${\mit\Psi}(0)$, PSIBOUND $=0$, CURRENT$= \epsilon^2\,(B_0\,R_0/\mu_0)\,\hat{I}_t(1)$. The $R$, $Z$ grid-points are 
\begin{align}
R\,R_0&= {\rm RBOXLFT} + {\rm RBOXLEN} *  i /({\rm NRBOX}-1),\\[0.5ex]
Z\,Z_0 &= {\rm ZOFF} - {\rm ZBOXLEN}/2 + {\rm ZBOXLEN} * j /({\rm NZBOX}-1),
\end{align}
for $i=0,{\rm NBBOX}-1$, $j=0,{\rm NZBOX}-1$. The profiles are evaluated on the grid ${\rm PSI} = {\rm PSIAXIS} * ({\rm NRBOX}- 1 - i)/({\rm NRBOX}-1)$
for $i=0, {\rm NRBOX}-1$. 

\end{document}