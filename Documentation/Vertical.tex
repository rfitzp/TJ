\documentclass[12pt,prb,aps,notitlepage]{revtex4-1}
\usepackage{amsmath}           		          	
\usepackage{graphicx,epstopdf}					
\usepackage{amssymb}
\usepackage{fullpage}
\usepackage{color}
\usepackage{esint}
\pdfoutput = 1 
\newcommand {\bxi}{\mbox{\boldmath$\xi$}}
\allowdisplaybreaks

\begin{document}
\title{Calculation of Vertical Stability in an Inverse Aspect-Ratio Expanded Tokamak Plasma Equilibrium}
\author{Richard Fitzpatrick\,\footnote{rfitzp@utexas.edu}}
\affiliation{Institute for Fusion Studies, Department of Physics, University of Texas at Austin, Austin, TX 78712}
\maketitle

\section{Plasma Equilibrium}\label{geq}
All lengths  are normalized to  the major radius of the plasma magnetic axis, $R_0$. All magnetic field-strengths
are normalized to the  toroidal field-strength at the magnetic axis, $B_0$. All current densities are normalized to $B_0/(\mu_0\,R_0)$.  All plasma pressures are normalized to $B_0^{\,2}/\mu_0$.

Let $R$, $\phi$, $Z$ be right-handed cylindrical coordinates whose Jacobian 
is
\begin{equation}
(\nabla R\times \nabla\phi\cdot\nabla Z)^{-1} = R.
\end{equation}
Note that $|\nabla\phi|=1/R$. 

Let $r$, $\theta$, $\phi$ be right-handed flux-coordinates whose
Jacobian is
\begin{equation}\label{jac}
{\cal J}(r,\theta)\equiv (\nabla r\times \nabla\theta\cdot\nabla\phi)^{-1} \equiv R\left(\frac{\partial R}{\partial\theta}\,\frac{\partial Z}{\partial r} -\frac{\partial R}{\partial r}\,\frac{\partial Z}{\partial \theta}\right)= r\,R^{\,2}.
\end{equation}
Note that $r=r(R,Z)$ and $\theta=\theta(R,Z)$. 
The magnetic axis corresponds to $r=0$. The inboard mid-plane corresponds to $\theta=0$. 

Consider an axisymmetric tokamak equilibrium whose magnetic field takes the form
\begin{equation}
{\bf B}(r,\theta) = f(r)\,\nabla\phi\times \nabla r + g(r)\,\nabla\phi = f\,\nabla(\phi-q\,\theta)\times \nabla r,
\end{equation}
where
\begin{equation}\label{q}
q(r) = \frac{r\,g}{f}
\end{equation}
is the safety-factor (i.e., the inverse of the rotational transform). Note that ${\bf B}\cdot\nabla r=0$, which implies that $r$ is a magnetic flux-surface label.
We require $g=1$ on the magnetic axis in order to ensure that the normalized toroidal magnetic field-strength at the  axis is unity.  

It is easily demonstrated that
\begin{align}
B^{\,r}&={\bf B}\cdot\nabla r= 0,\label{bup1}\\[0.5ex]
B^{\,\theta} &={\bf B}\cdot\nabla \theta= \frac{f}{r\,R^{\,2}},\label{bup2}\\[0.5ex]
B^{\,\phi} &={\bf B}\cdot\nabla \phi= \frac{g}{R^{\,2}},\label{bup3}\\[0.5ex]
B_r &={\cal J}\,\nabla\theta\times\nabla\phi\cdot{\bf B}= -r\,f\,\nabla r\cdot\nabla\theta,\label{bdown1}\\[0.5ex]
B_\theta &={\cal J}\,\nabla\phi\times\nabla r\cdot{\bf B}= r\,f\,|\nabla r|^2,\\[0.5ex]
B_\phi &={\cal J}\,\nabla r\times\nabla \theta \cdot{\bf B}= g.\label{bdown3}
\end{align}

The Maxwell equation (neglecting the displacement current, because the plasma velocity perturbations due to axisymmetric modes are far smaller than the velocity of light in vacuum)
${\bf J}= \nabla\times{\bf B}$
yields
\begin{align}
{\cal J}\,J^{\,r} &= \frac{\partial B_\phi}{\partial \theta} =0,\label{jup1}\\[0.5ex]
{\cal J}\,J^{\,\theta} &= -\frac{\partial B_\phi}{\partial r} = - g',\label{jup2}\\[0.5ex]
{\cal J}\,J^{\,\phi}&= \frac{\partial B_\theta}{\partial r} -\frac{\partial B_r}{\partial\theta}=\frac{\partial}{\partial r}\!\left(r\,f\,|\nabla r|^2\right)+ \frac{\partial}{\partial\theta}\!\left(r\,f\,\nabla r\cdot\nabla\theta\right),\label{jup3}
\end{align}
where ${\bf J}$ is the equilibrium current density, $'\equiv d/dr$, and use has been made of  Eqs.~(\ref{bdown1})--(\ref{bdown3}).

Equilibrium force balance requires that
\begin{equation}\label{e15c}
 \nabla P={\bf J}\times {\bf B},
\end{equation}
where $P(r)$ is the equilibrium scalar plasma pressure. Here, for the sake of simplicity, we have neglected the small centrifugal modifications to force balance due to subsonic plasma
rotation.
It follows that 
\begin{align}\label{eg1}
P'&= {\cal J}(J^{\,\theta}\,B^{\,\phi}-J^{\,\phi}\,B^{\,\theta})= -g'\,\frac{g}{R^{\,2}} - \frac{f}{r\,R^{\,2}}\left[\frac{\partial}{\partial r}\!\left(r\,f\,|\nabla r|^2\right)+ \frac{\partial}{\partial\theta}\!\left(r\,f\,\nabla r\cdot\nabla\theta\right)\right],
\end{align}
where use has been made of Eqs.~(\ref{bup1})--(\ref{bup3}), and  (\ref{jup1})--(\ref{jup3}). The
other two components of Eq.~(\ref{e15c}) are identically zero. 

Equation~(\ref{eg1}) yields the {\em inverse Grad-Shafranov equation},
\begin{equation}\label{gs}
\frac{f}{r}\,\frac{\partial}{\partial r}\!\left(r\,f\,|\nabla r|^2\right) +\frac{f}{r}\,\frac{\partial}{\partial\theta}\!\left(r\,f\,\nabla r\cdot\nabla\theta\right)+g\,g' + R^{\,2}\,P'=0.
\end{equation}
It follows from Eqs.~(\ref{q}), (\ref{jup3}), and (\ref{gs}) that
\begin{equation}\label{jup3a}
{\cal J}\,J^{\,\phi} = -q\,g' - \frac{r\,R^{\,2}\,P'}{f}.
\end{equation}
It is clear from Eqs.~(\ref{jup2}) and (\ref{jup3a}) that $g'=P'=0$ in the  current-free ``vacuum'' region surrounding the plasma.
We shall also assume that $g'=P'=0$ at the plasma-vacuum interface, so as to ensure that the equilibrium plasma
current density is zero at the interface. 

\section{Axisymmetric Plasma Perturbation}\label{opde}

\subsection{Derivation of Axisymmetric Ideal-MHD P.D.E.s}
Let us assume that all perturbed quantities are independent of the toroidal angle, $\phi$. 
The perturbed plasma equilibrium satisfies the  marginally-stable ideal-MHD equations
\begin{align}
{\bf b} &= \nabla\times (\bxi\times {\bf B}),\label{e21}\\[0.5ex]
\nabla p &={\bf j}\times {\bf B}  +{\bf J}\times {\bf b},\label{e22}\\[0.5ex]
{\bf j} &= \nabla\times {\bf b},\label{e23}\\[0.5ex]
p&= -\bxi\cdot\nabla P,\label{e24}
\end{align}
where $\bxi(r,\theta)$ is the plasma displacement, ${\bf b}(r,\theta)$ the perturbed magnetic field,
${\bf j}(r,\theta)$ the perturbed current density, and $p(r,\theta)$ the perturbed scalar pressure. 

Now, 
\begin{align}
(\bxi\times {\bf B})_\theta&= {\cal J}\,(\xi^{\,\phi}\,B^{\,r} - \xi^{\,r}\,B^{\,\phi}) = -{\cal J}\,B^{\,\phi}\,\xi^{\,r},\\[0.5ex]
(\bxi\times {\bf B})_\phi &= {\cal J}\,(\xi^{\,r}\,B^{\,\theta} - \xi^{\,\theta}\,B^{\,r})= {\cal J}\,B^{\,\theta}\,\xi^{\,r},\label{e23x}
\end{align}
where use has been made of  the fact that $B^{\,r}=J^{\,r}=0$. [See Eqs.~(\ref{bup1}) and (\ref{jup1}).]
 Combining Eqs.~(\ref{e21}) and (\ref{e23x}), we obtain
\begin{align}
{\cal J}\,b^{\,r} &= \frac{\partial}{\partial\theta}\left({\cal J}\,B^{\,\theta}\,\xi^{\,r}\right).
\end{align}
Thus, Eqs.~(\ref{jac}), (\ref{q}), (\ref{bup2}), and (\ref{bup3}) give
\begin{align}\label{e41}
r\,R^{\,2}\,b^{\,r}& = \frac{\partial y}{\partial\theta},
\end{align}
where 
\begin{align}\label{e42}
y(r,\theta) &=f\,\xi^{\,r}.
\end{align}
The constraint $\nabla\cdot{\bf b} =0$, which follows from Eq.~(\ref{e21}),  immediately yields
\begin{equation}\label{e43y}
r\,R^{\,2}\,b^{\,\theta} = - \frac{\partial y}{\partial r}.
\end{equation}

According to Eq.~(\ref{e24}), 
\begin{equation}
p =-P'\,\nabla r\cdot\bxi=- P'\,\xi^{\,r}.
\end{equation}
So, the perturbed force balance equation, (\ref{e22}), yields
\begin{align}
-\frac{\partial\, (P'\,\xi^{\,r})}{\partial r} &= ({\bf j}\times {\bf B})_r+({\bf J}\times {\bf b})_r,\\[0.5ex]
-\frac{\partial\,(P'\,\xi^{\,r})}{\partial \theta}&= ({\bf j}\times {\bf B})_\theta+({\bf J}\times {\bf b})_\theta,\\[0.5ex]
0&= ({\bf j}\times {\bf B})_\phi+({\bf J}\times {\bf b})_\phi,
\end{align}
giving
\begin{align}
-\frac{\partial\, (P'\,\xi^{\,r})}{\partial r} &=r\,R^{\,2}\,(j^{\,\theta}\,B^{\,\phi}-j^{\,\phi}\,B^{\,\theta}) + r\,R^{\,2}\,(J^{\,\theta}\,b^{\,\phi}-J^{\,\phi}\,b^{\,\theta}),\\[0.5ex]
-\frac{\partial\,(P'\,\xi^{\,r})}{\partial \theta}&=r\,R^{\,2}\,(j^{\,\phi}\,B^{\,r}-j^{\,r}\,B^{\,\phi}) + r\,R^{\,2}\,(J^{\,\phi}\,b^{\,r}-J^{\,r}\,b^{\,\phi}),\\[0.5ex]
0&=r\,R^{\,2}\,(j^{\,r}\,B^{\,\theta}-j^{\,\theta}\,B^{\,r}) + r\,R^{\,2}\,(J^{\,r}\,b^{\,\theta}-J^{\,\theta}\,b^{\,r}),
\end{align}
where use has been made of Eq.~(\ref{jac}). 
Thus, according to Eqs.~(\ref{bup1})--(\ref{bup3}), (\ref{jup1}), (\ref{jup2}), and (\ref{jup3a}), 
\begin{align}
-\frac{\partial\, (P'\,\xi^{\,r})}{\partial r} &= f\,(q\,j^{\,\theta} -j^{\,\phi}) - g'\,b^{\,\phi} + \left(q\,g'+\frac{r\,R^{\,2}\,P'}{f}\right)b^{\,\theta},\label{e51}\\[0.5ex]
-\frac{\partial\,(P'\,\xi^{\,r})}{\partial \theta}&=-r\,g\,j^{\,r} - \left(q\,g'+\frac{r\,R^{\,2}\,P'}{f}\right)b^{\,r},\label{e44}\\[0.5ex]
0&= f\,j^{\,r}+g'\,b^{\,r}.\label{e53}
\end{align}
It follows from Eqs.~(\ref{e41}) and (\ref{e53}) that 
\begin{equation}\label{e54}
r\,R^{\,2}\,j^{\,r} = -\alpha_g\,\frac{\partial y}{\partial\theta},
\end{equation}
where
\begin{align}
\alpha_g (r)&= \frac{g'}{f}.\label{ag}
\end{align}
Note that Eq.~(\ref{e44}) is trivially satisfied. Hence, of the three components of the perturbed force balance equation, only Eq.~(\ref{e51}) remains to be solved. 

Equation~(\ref{e23}) yields 
\begin{align}
r\,R^{\,2}\,j^{\,r} &= \frac{\partial b_\phi}{\partial\theta},\label{e57}\\[0.5ex]
r\,R^{\,2}\,j^{\,\theta} &= -\frac{\partial b_\phi}{\partial r},\label{e58}\\[0.5ex]
r\,R^{\,2}\,j^{\,\phi}&= \frac{\partial b_\theta}{\partial r} -\frac{\partial b_r}{\partial \theta},\label{e59}
\end{align}
where use has been made of Eq.~(\ref{jac}). It follows from Eqs.~(\ref{e54}), (\ref{e57}), and (\ref{e58}) that
\begin{align}\label{e43yy}
b_\phi &=-\alpha_g\,y,\\[0.5ex]
r\,R^{\,2}\,j^{\,\theta}&=  \frac{\partial (\alpha_g\,y)}{\partial r}.\label{e44yy}
\end{align}
Note that $\nabla\cdot{\bf j} = 0$, in accordance with Eq.~(\ref{e23}).

Now, 
\begin{equation}
{\bf b} = b_r\,\nabla r + b_\theta\,\nabla\theta+b_\phi\,\nabla\phi,
\end{equation}
so
\begin{align}
b^{\,r} &= {\bf b}\cdot\nabla r = |\nabla r|^2\,b_r + (\nabla r\cdot\nabla\theta)\,b_\theta,\label{e61}\\[0.5ex]
b^{\,\theta} &= {\bf b}\cdot\nabla \theta = (\nabla r\cdot\nabla\theta)\,b_r + |\nabla\theta|^2\,b_\theta,\label{e62}\\[0.5ex]
b^{\,\phi}&={\bf b}\cdot\nabla\phi =\frac{b_\phi}{R^{\,2}}.\label{e63}
\end{align}

Equations~(\ref{jac}), (\ref{e61}), and (\ref{e62}) can be rearranged to give
\begin{align}
b_r &= \left(\frac{1}{|\nabla r|^2}\right)b^{\,r}- \left(\frac{\nabla r\cdot\nabla\theta}{|\nabla r|^2}\right)b_\theta,\label{e69}\\[0.5ex]
\label{e58x}
b^{\,\theta}& = \left(\frac{\nabla r\cdot\nabla\theta}{|\nabla r|^2}\right)b^{\,r} + \left(\frac{1}{r^2\,R^{\,2}\,|\nabla r|^2}\right) b_\theta.
\end{align}
Let 
\begin{equation}\label{zdef}
z= |\nabla r|^2\,r\,\frac{\partial y}{\partial r} + r\,\nabla r\cdot\nabla\theta\,\frac{\partial y}{\partial\theta}.
\end{equation}
Equations~(\ref{e41}), (\ref{e43y}), (\ref{e43yy}), (\ref{e69}) and (\ref{e58x}) yield 
\begin{align}\label{e53y}
b_r &=\frac{1}{r\,|\nabla r|^2\,R^{\,2}}\,\frac{\partial y}{\partial \theta} + \frac{\nabla r\cdot\nabla\theta}{|\nabla r|^2}\,z,\\[0.5ex]
b_\theta &= -z,\label{e54y}\\[0.5ex]
b^{\,\phi} &= -\frac{\alpha_g}{R^{\,2}}\,y.\label{e55yy}
\end{align}
Equations~(\ref{e59}), (\ref{e53y}), and (\ref{e54y}) give
\begin{align}\label{e56yy}
r\,R^{\,2}\,j^{\,\phi} &=-\frac{\partial z}{\partial r}-\frac{\partial}{\partial\theta}\!\left[\frac{1}{r\,|\nabla r|^2\,R^{\,2}}\,\frac{\partial y}{\partial \theta} + \frac{\nabla r\cdot\nabla\theta}{|\nabla r|^2}\,z\right].
\end{align}

It follows from Eqs.~(\ref{e42}), (\ref{e43y}), (\ref{e51}), (\ref{e44yy}), (\ref{e55yy}), and (\ref{e56yy}) that
\begin{align}
-\frac{\partial}{\partial r}\!\left(\frac{P'}{f}\,y\right)&= \frac{f\,q}{r\,R^{\,2}}\,\frac{\partial (\alpha_g\,y)}{\partial r} 
+ \frac{f}{r\,R^{\,2}}\,\frac{\partial z}{\partial r}\nonumber\\[0.5ex]
&\phantom{=} +\frac{f}{r\,R^{\,2}}\frac{\partial}{\partial\theta}\!\left[\frac{1}{r\,|\nabla r|^2\,R^{\,2}}\,\frac{\partial y}{\partial \theta} + \frac{\nabla r\cdot\nabla\theta}{|\nabla r|^2}\,z\right]
\nonumber\\[0.5ex]
&\phantom{=} + \frac{g'\,\alpha_g}{R^{\,2}}\,y - \left(q\,g'+\frac{r\,R^{\,2}\,P'}{f}\right)\frac{1}{r\,R^{\,2}}\,\frac{\partial y}{\partial r}.
\end{align}
Hence,
\begin{align}\label{e58u}
-\left[(\alpha_p\,\alpha_f+ r\,\alpha_p')R^{\,2} + q\,r\,\alpha_g'+r^2\,\alpha_g^{\,2}\right]y&= 
 r\,\frac{\partial z}{\partial r}+\frac{\partial}{\partial\theta}\!\left[\frac{1}{|\nabla r|^2\,R^{\,2}}\,\frac{\partial y}{\partial \theta} + \frac{r\,\nabla r\cdot\nabla\theta}{|\nabla r|^2}\,z\right],
\end{align}
where
\begin{align}\label{alpp}
\alpha_p(r)&= \frac{r\,P'}{f^2},\\[0.5ex]
\alpha_f(r) &= \frac{r^2}{f}\,\frac{d}{dr}\!\left(\frac{f}{r}\right).\label{alpf}
\end{align}

Finally, Eqs.~(\ref{zdef}) and (\ref{e58u})  yield the {\em axisymmetric ideal-MHD p.d.e.s}:
\begin{align}\label{e63y}
r\,\frac{\partial y}{\partial r} &= \frac{z}{|\nabla r|^2} - \frac{r\,\nabla r\cdot\nabla\theta}{|\nabla r|^2}\,\frac{\partial y}{\partial\theta},\\[0.5ex]
r\,\frac{\partial z}{\partial r}&= -\left[(\alpha_p\,\alpha_f+ r\,\alpha_p')R^{\,2} + q\,r\,\alpha_g'+r^2\,\alpha_g^{\,2}\right]y-\frac{\partial}{\partial\theta}\!\left(\frac{1}{|\nabla r|^2\,R^{\,2}}\,\frac{\partial y}{\partial\theta}\right)\nonumber\\[0.5ex]
&\phantom{=} -\frac{\partial}{\partial\theta}\!\left(\frac{r\,\nabla r\cdot\nabla\theta}{|\nabla r|^2}\,z\right).\label{e64y}
\end{align}

\subsection{Properties of Axisymmetric Ideal-MHD P.D.E.s}
The previous two equations can be combined to give
\begin{align}
r\,\frac{\partial (z\,y^\ast)}{\partial r} &= \frac{|z|^2}{|\nabla r|^2} - \frac{\partial y^\ast}{\partial \theta}\left(\frac{r\,\nabla r\cdot\nabla\theta}{|\nabla r|^2}\,z\right)
 -\left[(\alpha_p\,\alpha_f+ r\,\alpha_p')R^{\,2} + q\,r\,\alpha_g'+r^2\,\alpha_g^{\,2}\right]|y|^2\nonumber\\[0.5ex]
 &\phantom{=} - y^\ast\,\frac{\partial}{\partial\theta}\!\left(\frac{1}{|\nabla r|^2\,R^{\,2}}\,\frac{\partial y}{\partial\theta}\right)-y^\ast\,\frac{\partial}{\partial\theta}\!\left(\frac{r\,\nabla r\cdot\nabla\theta}{|\nabla r|^2}\,z\right).
\end{align}
Hence,
\begin{align}
r\,\frac{d}{dr}\!\left(\oint z\,y^\ast\,\frac{d\theta}{2\pi}\right)&= \oint\left[\frac{|z|^2}{|\nabla r|^2}- \left[(\alpha_p\,\alpha_f+ r\,\alpha_p')R^{\,2} + q\,r\,\alpha_g'+r^2\,\alpha_g^{\,2}\right]|y|^2\right.\nonumber\\[0.5ex]
&\phantom{=}\left.
+ \frac{1}{|\nabla r|^2\,R^{\,2}}\,\left|\frac{\partial y}{\partial \theta}\right|^2\right]\frac{d\theta}{2\pi},
\end{align}
which implies that $\oint z\,y^\ast\,d\theta/2\pi$ is a real quantity, and also that
\begin{equation}
r\,\frac{d}{dr}\!\left[\oint(z\,y^\ast-y^\ast\,z)\,\frac{d\theta}{2\pi}\right] =0. 
\end{equation}

\subsection{Derivation of the Axisymmetric Ideal-MHD O.D.E.s}
Let
\begin{align}
y(r,\theta)&= \sum_m y_m(r)\,{\rm e}^{\,{\rm i}\,m\,\theta},\\[0.5ex]
z(r,\theta) &= \sum_m z_m(r)\,{\rm e}^{\,{\rm i}\,m\,\theta}.
\end{align}
Equations~(\ref{e63y}) and (\ref{e64y}) yield the {\em axisymmetric ideal-MHD o.d.e.s}:
\begin{align}\label{e69u}
r\,\frac{dy_m}{dr}&= \sum_{m'}\left(A_{m}^{\,m'}\,z_{m'} + B_{m}^{\,m'}\,y_{m'}\right),\\[0.5ex]
r\,\frac{dz_m}{dr}&= \sum_{m'}\left(C_{m}^{\,m'}\,z_{m'} + D_{m}^{\,m'}\,y_{m'}\right),\label{e70uu}
\end{align}
where
\begin{align}
A_m^{\,m'} &= c_{m}^{\,m'},\\[0.5ex]
B_m^{\,m'} &= - m'\,f_m^{\,m'},\\[0.5ex]
C_{m}^{\,m'} &= -m\,f_m^{\,m'},\\[0.5ex]
D_{m}^{\,m'}&= -(\alpha_f\,\alpha_p+ r\,\alpha_p')\,a_m^{\,m'} - (q\,r\,\alpha_g' +r^{\,2}\,\alpha_g^{\,2})\,\delta_m^{\,m'}+m\,m'\,b_m^{\,m'},\label{Ddef}
\end{align}
where
\begin{align}\label{e73er}
a_m^{\,m'}(r)  &=\oint R^{\,2}\,\exp[-{\rm i}\,(m-m')\,\theta]\,\frac{d\theta}{2\pi},\\[0.5ex]
b_m^{\,m'}(r)  &=\oint|\nabla r|^{-2}\, R^{-2}\,\exp[-{\rm i}\,(m-m')\,\theta]\,\frac{d\theta}{2\pi},\\[0.5ex]
c_m^{\,m'}(r)  &=\oint|\nabla r|^{-2}\,\exp[-{\rm i}\,(m-m')\,\theta]\,\frac{d\theta}{2\pi},\\[0.5ex]
f_{m}^{\,m'}(r)&= \oint \frac{{\rm i}\,r\,\nabla r\cdot\nabla\theta}{|\nabla r|^2}\,\exp[-{\rm i}\,(m-m')\,\theta]\,\frac{d\theta}{2\pi}.\label{e76er}
\end{align}

\subsection{Properties of Axisymmetric Ideal-MHD O.D.E.s}
Note that
$a_{m'}^{\,m} =a_{m}^{\,m'\ast}$, 
$b_{m'}^{\,m} =b_{m}^{\,m'\ast}$, 
$c_{m'}^{\,m} =c_{m}^{\,m'\ast}$, and 
$f_{m'}^{\,m} =-f_{m}^{\,m'\ast}$,
which implies that
\begin{align}\label{e70u}
A_{m'}^{\,m} &= A_{m'}^{\,m\,\ast},\\[0.5ex]
B_{m'}^{\,m} &= -C_{m'}^{\,m\,\ast},\\[0.5ex]
C_{m'}^{\,m} &= -B_{m'}^{\,m\,\ast},\\[0.5ex]
D_{m'}^{\,m} &= D_{m'}^{\,m\,\ast}.\label{e82u}
\end{align}
It follows from Eqs.~(\ref{e69u}), (\ref{e70uu}), and (\ref{e70u})--(\ref{e82u}) that 
\begin{align}
r\,\frac{d}{dr}\!\left(\sum_m z_m\,y_m^{\,\ast} \right)&= \sum_{m,m'}\left(z_m^{\,\ast}\,A_{m}^{\,m'}\,z_{m'} + y_m^{\,\ast}\,D_{m}^{\,m'}\,y_{m'}\right),\\[0.5ex]
r\,\frac{d}{dr}\!\left[\sum_m (z_m\,y_m^{\,\ast}- y_m\,z_m^{\,\ast}) \right]&= 0.
\end{align}

\subsection{Toroidal Electromagnetic Torque}
The net toroidal electromagnetic torque exerted on the plasma lying within the magnetic flux-surface whose label is $r$ is
\begin{equation}
T_\phi(r)= \oint\oint r\,R^{\,2}\,b_\phi\,b^{\,r}\,d\theta\,d\phi.
\end{equation}
It follows from Eqs.~(\ref{e41}) and (\ref{e43yy}) that 
\begin{equation}
T_\phi(r) = -\pi\,\alpha_g\oint\left(y^\ast\,\frac{\partial y}{\partial\theta}+y\,\frac{\partial y^\ast}{\partial\theta}\right)d\theta
= -\pi\,\alpha_g\oint\frac{\partial|y|^2}{\partial\theta}\,d\theta = 0.
\end{equation}
We conclude that an axisymmetric perturbation is incapable of exerting a net toroidal electromagnetic torque on the plasma. 

\subsection{Perturbed Plasma Potential Energy}
The perturbed plasma potential energy in the region of the plasma lying within the magnetic flux-surface whose label is $r$ is 
\begin{equation}
\delta W_p = \frac{1}{2}\oint\oint r\,R^{\,2}\,\xi^{\,r\ast}(-{\bf B}\cdot{\bf b} + \xi^{\,r}\,P')\,d\theta\,d\phi.
\end{equation}
However,
\begin{equation}
{\bf B}\cdot{\bf b} -\xi^{\,r}\,P' = B^{\,\theta}\,b_\theta+B^{\,\phi}\,b_\phi - \xi^{\,r}\,P'
= - \frac{f}{r\,R^{\,2}}\left(z + q\,\alpha_g\,y + \alpha_p\,R^{\,2}\right),
\end{equation}
where use has been made of Eqs.~(\ref{q})--(\ref{bup3}), (\ref{e42}), (\ref{e43yy}), (\ref{e54y}),  and (\ref{alpp}). Hence,
we obtain
\begin{align}
\delta W_p(r) &=\frac{1}{2}\oint\oint y^{\ast}\left[z + (q\,\alpha_g + \alpha_p\,R^{\,2})\,y\right]d\theta\,d\phi= \pi^2\sum_{m}y_m^{\,\ast}\,\chi_m,
\end{align}
where 
\begin{equation}
\chi_m(r)=z_m + q\,\alpha_g\,y_m + \alpha_p\sum_{m'} a_m^{\,m'}\,y_{m'}.
\end{equation}

\section{Inverse Aspect-Ratio Expanded Tokamak Equilibrium}\label{large}

\subsection{Equilibrium Magnetic Flux-Surfaces}\label{flux}
Let us assume that the inverse aspect-ratio of the plasma, $\epsilon$, is such that $0<\epsilon\ll 1$.  
Let $r=\epsilon\,\hat{r}$, $\nabla =\epsilon^{\,-1}\,\hat{\nabla}$, and $'\rightarrow \epsilon^{\,-1}\,'$. 
Suppose that the loci of the equilibrium magnetic flux-surfaces can be written in the parametric form:
\begin{align}
R(\hat{r},\omega) &= 1 -\epsilon\,\hat{r}\,\cos\omega + \epsilon^{\,2}\,\sum_{j>0}H_j(\hat{r})\,\cos[(j-1)\,\omega] + \epsilon^{\,2}\,\sum_{j>1}V_j(\hat{r})\,\sin[(j-1)\,\omega] \nonumber\\[0.5ex]
&\phantom{=}+\epsilon^{\,3}\,L(\hat{r})\,\cos\omega,\label{e19x}\\[0.5ex]
Z(\hat{r},\omega)&= \epsilon\,\hat{r}\,\sin\omega +\epsilon^{\,2}\,\sum_{j>1}H_j(\hat{r})\,\sin[(j-1)\,\omega]
-\epsilon^{\,2}\,\sum_{j>1}V_j(\hat{r})\,\cos[(j-1)\,\omega]\nonumber\\[0.5ex]&\phantom{=}-\epsilon^{\,3}\,L(\hat{r})\,\sin\omega,\label{e20x}
\end{align}
where $j$ is a positive integer. 
Here, $H_1(\hat{r})$  controls the relative horizontal locations of the flux-surface centroids, $H_2(\hat{r})$ and $V_2(\hat{r})$ control the 
magnitudes and vertical tilts of the flux-surface ellipticities, $H_3(\hat{r})$ and
$V_3(\hat{r})$ control the magnitudes and vertical tilts of the flux-surface triangularities, et cetera, whereas $L(\hat{r})$ is a
flux-surface re-labelling parameter. Moreover, $\omega(R,Z)$ is a  poloidal angle that is distinct from $\theta$. Note that $V_1$ does not appear in Eq.~(\ref{e20x})
because such a factor merely gives rise to a rigid vertical shift of the plasma that can be eliminated by a suitable choice of the
origin of the flux-coordinate system.

Let
\begin{equation}
J(\hat{r},\omega) = \frac{1}{\epsilon^{\,2}}\left(\frac{\partial R}{\partial\omega}\,\frac{\partial Z}{\partial \hat{r}} -\frac{\partial R}{\partial \hat{r}}\,\frac{\partial Z}{\partial \omega}\right)
\end{equation}
be the Jacobian of the $\hat{r}$, $\omega$ coordinate system. We can transform to the $\hat{r}$, $\theta$ coordinate system 
by writing
\begin{align}\label{e11}
\theta(\hat{r},\omega) &= \left.2\pi\int_0^\omega \frac{J(\hat{r},\tilde{\omega})}{R(\hat{r},\tilde{\omega})}\,d\tilde{\omega}\right/\oint\frac{J(\hat{r},\omega)}{R(\hat{r},\omega)}\,d\omega,\\[0.5ex]
\hat{r}&=\frac{1}{2\pi}\oint\frac{J(\hat{r},\omega)}{R(\hat{r},\omega)}\,d\omega.\label{e12}
\end{align}
This transformation ensures that 
\begin{equation}
\frac{\partial\theta}{\partial\omega} = \frac{J}{\hat{r}\,R},
\end{equation}
and, hence, that 
\begin{equation}
{\cal J} \equiv \frac{R}{\epsilon} \left(\frac{\partial R}{\partial\theta}\,\frac{\partial Z}{\partial \hat{r}} -\frac{\partial R}{\partial \hat{r}}\,\frac{\partial Z}{\partial \hat{r}}\right)
=\epsilon\, R\,J\,\frac{\partial\omega}{\partial\theta} =r\,R^{\,2},
\end{equation}
in accordance with Eq.~(\ref{jac}). 

\subsection{Metric Elements}\label{metric}
We can determine the metric elements of the flux-coordinate system by combining Eqs.~(\ref{e19x})--(\ref{e12}).
Evaluating the elements up to ${\cal O}(\epsilon)$, but retaining ${\cal O}(\epsilon^{\,2})$ contributions to terms that are independent of
$\omega$, we obtain,
\begin{align}\label{epdef}
L(\hat{r})&= \frac{\hat{r}^{\,3}}{8} -\frac{\hat{r}\,H_1}{2}-\frac{1}{2}\sum_{j>1}(j-1)\,\frac{H_j^{\,2}}{\hat{r}}
-\frac{1}{2}\sum_{j>1}(j-1)\,\frac{V_j^{\,2}}{\hat{r}},\\[0.5ex]
\theta &= \omega+\epsilon\,\hat{r}\,\sin\omega - \epsilon\sum_{j>0}\frac{1}{j}\left[H_j'-(j-1)\,\frac{H_j}{\hat{r}}\right]\sin(j\,\omega)
\nonumber\\[0.5ex]&\phantom{=}+ \epsilon\sum_{j>1}\frac{1}{j}\left[V_j'-(j-1)\,\frac{V_j}{\hat{r}}\right]\cos(j\,\omega),\label{e22y}\\[0.5ex]
|\hat{\nabla} \hat{r}|^2 &= 1 +2\,\epsilon\sum_{j>0}H_j'\,\cos(j\,\theta) +2\,\epsilon\sum_{j>1}V_j'\,\sin(j\,\theta) \nonumber\\[0.5ex]
&\phantom{=}+\epsilon^{\,2}\left(\frac{3\,\hat{r}^{\,2}}{4}-H_1+
\frac{1}{2}\sum_{j>0}\left[H_j'^{\,2}+(j^2-1)\,\frac{H_j^{\,2}}{\hat{r}^{\,2}}\right]\right.\nonumber\\[0.5ex]&\phantom{=}\left.+
\frac{1}{2}\sum_{j>1}\left[V_j'^{\,2}+(j^2-1)\,\frac{V_j^{\,2}}{\hat{r}^{\,2}}\right]
\right),\label{e19}\\[0.5ex]
\hat{\nabla}\hat{r}\cdot\hat{\nabla}\theta&=\epsilon\,\sin\theta
-\epsilon\sum_{j>0}\frac{1}{j}\left[H_j''+\frac{H_j'}{\hat{r}}+(j^2-1)\,\frac{H_j}{\hat{r}^{\,2}}\right]\sin(j\,\theta)\nonumber\\[0.5ex]&
\phantom{=}+\epsilon\sum_{j>1}\frac{1}{j}\left[V_j''+\frac{V_j'}{\hat{r}}+(j^2-1)\,\frac{V_j}{\hat{r}^{\,2}}\right]\cos(j\,\theta),\label{e20uu}
\\[0.5ex]
R^{\,2}&= 1-2\,\epsilon\,\hat{r}\,\cos\theta -\epsilon^{\,2}\left(\frac{\hat{r}^{\,2}}{2}-\hat{r}\,H_1'-2\,H_1\right).\label{e25a}
\end{align}
Here, $'\equiv d/d\hat{r}$. Moreover, we have made use of the fact that $V_j\propto H_j$, for $j>1$, because
$V_j$ and $H_j$ satisfy the identical differential equations, (\ref{e33x}) and (\ref{e28}). 

\subsection{Expansion of Grad-Shafranov Equation}\label{exp}
Let us write
\begin{align}\label{e26v}
f(\hat{r})&= \epsilon\,\frac{\hat{r}\,g}{q},\\[0.5ex]
g(\hat{r}) &= 1+ \epsilon^{\,2}\,g_2(\hat{r}) + \epsilon^{\,4}\,g_4(\hat{r}),\label{e27v}\\[0.5ex]
P'(\hat{r}) &= \epsilon^{\,2}\,p_2'(\hat{r}),\label{eq1}
\end{align}
where $q$,  $g_2$, $g_4$, and $p_2$ are all ${\cal O}(1)$. Here, the safety-factor, $q(\hat{r})$, and the second-order plasma
pressure gradient, $p_2'(\hat{r})$, are the two free flux-surface functions that characterize the plasma equilibrium.

Expanding the Grad-Shafranov equation, (\ref{gs}), order by order in the
small parameter $\epsilon$, making use of Eqs.~(\ref{e19})--(\ref{eq1}), we obtain
\begin{align}
g_2'&=- p_2' - \frac{\hat{r}}{q^2}\,(2-s),\label{e26}\\[0.5ex]
H_1''&= -(3-2\,s)\,\frac{H_1' }{\hat{r}}
-1+\frac{2\,p_2'\,q^2}{\hat{r}},\label{e27}\\[0.5ex]
H_j''&= -(3-2\,s)\,\frac{H_j'}{\hat{r}}+(j^2-1)\,\frac{H_j}{\hat{r}^{\,2}}~~~~~\mbox{for $j>1$},\label{e33x}\\[0.5ex]
V_j''&= -(3-2\,s)\,\frac{V_j'}{\hat{r}}+(j^2-1)\,\frac{V_j}{\hat{r}^{\,2}}~~~~~~\mbox{for $j>1$},\label{e28}\\[0.5ex]
g_4'&= g_2\left[p_2' - \frac{\hat{r}}{q^2}\,(2-s)\right] - \frac{\hat{r}}{q}\,{\mit\Sigma}
+p_2'\left(\frac{\hat{r}^{\,2}}{2}+\frac{\hat{r}^{\,2}}{q^2}-2\,H_1-3\,\hat{r}\,H_1'\right),\label{e31}
\end{align}
where $s=\hat{r}\,q'/q$, 
\begin{align}\label{e109}
{\mit\Sigma} &=\frac{1}{q}\left(\frac{3\,\hat{r}^{\,2}}{2} -2\,\hat{r}\,H_1'+S_2\right) - \frac{2-s}{q}\left(-\frac{3\,\hat{r}^{\,2}}{4}+\frac{\hat{r}^{\,2}}{q^2} +H_1+S_1\right),\\[0.5ex]
S_1(\hat{r})&=\frac{1}{2}\sum_{j>0} \left[3\,H_j'^{\,2} -(j^2-1)\,\frac{H_j^{\,2}}{\hat{r}^2}\right]
 +\frac{1}{2}\sum_{j>1} \left[3\,V_j'^{\,2} -(j^2-1)\,\frac{V_j^{\,2}}{\hat{r}^2}\right],\\[0.5ex]
S_2(\hat{r})&=\sum_{j>0}\left[H_j'^{\,2}+2\,(j^2-1)\,\frac{H_j'\,H_j}{\hat{r}}-(j^2-1)\,\frac{H_j^{\,2}}{\hat{r}^{\,2}}\right]\nonumber\\[0.5ex]
&\phantom{===}+\sum_{j>1}\left[V_j'^{\,2}+2\,(j^2-1)\,\frac{V_j'\,V_j}{\hat{r}}-(j^2-1)\,\frac{V_j^{\,2}}{\hat{r}^{\,2}}\right].
\end{align}
Note that the relative horizontal shift of magnetic flux-surfaces, $H_1$, otherwise known as the {\em Shafranov shift}, is driven by toroidicity [the second term on
the right-hand side of Eq.~(\ref{e27})], and plasma pressure gradients (the third term). All of the other shaping terms (i.e., the $H_j$, for $j>1$, and
the $V_j$) are driven by axisymmetric currents flowing in external  magnetic field-coils.

Finally, it follows from Eqs.~(\ref{ag}), (\ref{alpp}), (\ref{alpf}), and (\ref{e26v})--(\ref{eq1}) that
\begin{align}
\alpha_p(\hat{r}) &= \frac{p_2'\,q^2}{\hat{r}}\left(1-2\,\epsilon^{\,2}\,g_2\right),\\[0.5ex]
\alpha_g(\hat{r}) &= \frac{q}{\hat{r}}\left(g_2' -\epsilon^{\,2}\,g_2\,g_2'+\epsilon^{\,2}\,g_4'\right),\\[0.5ex]
\alpha_f(\hat{r}) &= -s + \epsilon^{\,2}\,\hat{r}\,g_2'.
\end{align}
Finally, it follows from Eqs.~(\ref{e20uu}) and (\ref{e27})--(\ref{e28}) that
\begin{align}\label{e114}
\hat{\nabla}\hat{r}\cdot\hat{\nabla}\theta &= 2\,\epsilon\left[1-\frac{p_2'\,q^2}{\hat{r}} +(1-s)\,\frac{H_1'}{\hat{r}}\right]\sin\theta\nonumber\\[0.5ex]
&\phantom{=}-2\,\epsilon\sum_{j>1}\frac{1}{j}\left[-(1-s)\,\frac{H_j'}{\hat{r}} + (j^2-1)\,\frac{H_j}{\hat{r}^{\,2}}\right]\sin(j\,\theta)\nonumber\\[0.5ex]
&\phantom{=}+2\,\epsilon\sum_{j>1}\frac{1}{j}\left[-(1-s)\,\frac{V_j'}{\hat{r}} + (j^2-1)\,\frac{V_j}{\hat{r}^{\,2}}\right]\cos(j\,\theta).
\end{align}

\subsection{Calculation of Coupling Coefficients}
Equations~(\ref{e19}) and (\ref{e109}) yield 
\begin{align}\label{e115}
|\hat{\nabla}\hat{r}|^{-2} &= 1 - 2\,\epsilon\sum_{j>0}H_j'\,\cos(j\,\theta) - 2\,\epsilon\sum_{j>0}V_j'\sin(j\,\theta)
+ \epsilon^2\left(-\frac{3\,\hat{r}^{\,2}}{4} + H_1 + S_1\right),
\end{align}
Equation~(\ref{e25a}) gives
\begin{align}
R^{-2}= 1 + 2\,\epsilon\,\hat{r}\,\cos\theta +\epsilon^2\left(\frac{5\,\hat{r}^{\,2}}{2} - \hat{r}\,H_1'-2\,H_1\right).
\end{align}
The previous two equations imply that
\begin{align}\label{e117}
|\hat{\nabla}\hat{r}|^{-2} \,R^{-2}&= 1 +2\,\epsilon\,\hat{r}\,\cos\theta -2\,\epsilon \sum_{j>0}H_j'\,\cos(j\,\theta) -2\,\epsilon\sum_{>1}V_j'\,\sin(j\,\theta)\nonumber\\[0.5ex]
&\phantom{=} + \epsilon^2\left(\frac{7\,\hat{r}^{\,2}}{4} - H_1 -3\,\hat{r}\,H_1' + S_1\right).
\end{align}
Finally, Eqs.~(\ref{e114}) and (\ref{e115}) give
\begin{align}\label{e118}
\hat{\nabla}\hat{r}\cdot\hat{\nabla}\theta \,|\hat{\nabla}\hat{r}|^{-2} &= 2\,\epsilon\left[1-\frac{p_2'\,q^2}{\hat{r}} +(1-s)\,\frac{H_1'}{\hat{r}}\right]\sin\theta\nonumber\\[0.5ex]
&\phantom{=}-2\,\epsilon\sum_{j>1}\frac{1}{j}\left[-(1-s)\,\frac{H_j'}{\hat{r}} + (j^2-1)\,\frac{H_j}{\hat{r}^{\,2}}\right]\sin(j\,\theta)\nonumber\\[0.5ex]
&\phantom{=}+2\,\epsilon\sum_{j>1}\frac{1}{j}\left[-(1-s)\,\frac{V_j'}{\hat{r}} + (j^2-1)\,\frac{V_j}{\hat{r}^{\,2}}\right]\cos(j\,\theta),
\end{align}
where use has been made of the fact that $V_j'\propto H_j'$ for $j>1$. 

It follows from Eqs.~(\ref{e73er})--(\ref{e76er}), (\ref{e25a}), (\ref{e115}), (\ref{e117}), and (\ref{e118}) that 
\begin{align}
a_m^{\,m'} &= \delta_{m}^{\,m'} -\epsilon\,\hat{r}\left(\delta_{m'-m-1}+\delta_{m'-m+1}\right)-\epsilon^2\left(\frac{\hat{r}^{\,2}}{2}-\hat{r}\,H_1'-2\,H_1\right)\delta_m^{\,m'},\\[0.5ex]
b_m^{\,m'}&= \delta_m^{\,m'}+\epsilon\,\hat{r}\left(\delta_{m'-m-1}+\delta_{m'-m+1}\right)-\epsilon\sum_{j>0}H_j'\left(\delta_{m'-m-j}+\delta_{m'-m+j}\right)\nonumber\\[0.5ex]
&- \epsilon\sum_{j>1}{\rm i}\,V_j'\left(\delta_{m'-m-j} - \delta_{m'-m+j}\right)+ \epsilon^2\left(\frac{7\,\hat{r}^{\,2}}{4} - H_1 -3\,\hat{r}\,H_1' + S_1\right)\delta_{m}^{\,m'},\\[0.5ex]
c_m^{\,m'} &= \delta_m^{\,m'}-\epsilon\sum_{j>0}H_j'\left(\delta_{m'-m-j}+\delta_{m'-m+j}\right)- \epsilon\sum_{j>1}{\rm i}\,V_j'\left(\delta_{m'-m-j} - \delta_{m'-m+j}\right)\nonumber\\[0.5ex]
&\phantom{=}+ \epsilon^2\left(-\frac{3\,\hat{r}^{\,2}}{4} + H_1 + S_1\right)\delta_{m}^{\,m'},\\[0.5ex]
f_m^{\,m'} &= -\epsilon\left[\hat{r} - p_2'\,q^2+ (1-s)\,H_1'\right]\left(\delta_{m'-m-1}-\delta_{m'-m+1}\right) \nonumber\\[0.5ex]
&\phantom{=}+ \epsilon\sum_{j>1}\frac{1}{j}\left[-(1-s)\,H_j'+(j^2-1)\,\frac{H_j}{\hat{r}}\right]\left(\delta_{m'-m-j}-\delta_{m'-m+j}\right)\nonumber\\[0.5ex]
&\phantom{=}+ \epsilon\sum_{j>1}\frac{{\rm i}}{j}\left[-(1-s)\,V_j'+(j^2-1)\,\frac{V_j}{\hat{r}}\right]
\left(\delta_{m'-m-j}+\delta_{m'-m+j}\right).
\end{align}

If we write
\begin{align}
\alpha_g&= \alpha_g^{\,(0)} + \epsilon^2\,\alpha_g^{\,(2)},\\[0.5ex]
\alpha_p&= \alpha_p^{\,(0)} + \epsilon^2\,\alpha_p^{\,(2)},\\[0.5ex]
\alpha_f&= \alpha_f^{\,(0)} + \epsilon^2\,\alpha_f^{\,(2)},\\[0.5ex]
a_{m}^{m'} &= 1 + \epsilon\,a_m^{\,m'\,(1)} + \epsilon^2\,a_{m}^{\,m'\,(2)},\\[0.5ex]
b_{m}^{\,m'} &= 1 + \epsilon\,b_m^{\,m'\,(1)} + \epsilon^2\,b_{m}^{\,m'\,(2)},\\[0.5ex]
D_{m}^{\,m'} &= D_m^{\,m'\,(0)} + \epsilon\,D_m^{\,m'\,(1)} + \epsilon^2\,D_{m}^{\,m'\,(2)}
\end{align}
then it follows from Eq.~(\ref{Ddef}) that
\begin{align}
D_m^{\,m\,(0)} &= -\alpha_f^{\,(0)}\,\alpha_p^{\,(0)} - \hat{r}\,\alpha_p^{\prime\,(0)} - q\,\hat{r}\,\alpha_g^{\prime\,(0)} + m^2,\\[0.5ex]
D_{m}^{\,m'\,(1)}&=   - \left[\alpha_f^{\,(0)}\,\alpha_p^{\,(0)} + \hat{r}\,\alpha_p^{\prime\,(0)}\right]a_m^{\,m'\,(1)} + m\,m'\,b_m^{\,m'\,(1)},\\[0.5ex]
D_{m}^{\,m\,(2)}&=  -\left[\alpha_f^{\,(0)}\,\alpha_p^{\,(0)} + \hat{r}\,\alpha_p^{\prime\,(0)}\right]a_m^{\,m'\,(2)} 
-\alpha_f^{\,(0)}\,\alpha_p^{\,(2)}-\alpha_f^{\,(2)}\,\alpha_p^{\,(0)}-\hat{r}\,\alpha_p^{\prime\,(2)} - q\,\hat{r}\,\alpha_g^{\prime\,(2)}\nonumber\\[0.5ex]&\phantom{=}
-\hat{r}^{\,2}\,[\alpha_g^{\,(0)}]^2+m^2\,b_m^{\,m\,(2)}.
\end{align}

It follows that
\begin{align}
A_m^{\,m}(\hat{r}) &= 1 + \epsilon^2\left(-\frac{3\,\hat{r}^{\,2}}{4} + H_1+S_1\right),\\[0.5ex]
A_m^{\,m\pm 1}(\hat{r}) &= - \epsilon\,H_1',\\[0.5ex]
A_m^{\,m\pm j}(\hat{r}) &= -\epsilon\,(H_j'\pm {\rm i}\, V_j'),\\[0.5ex]
B_m^{\,m}(\hat{r}) &= 0,\\[0.5ex]
B_m^{\,m\pm 1}(\hat{r}) &= \pm \epsilon\,(m\pm 1)\left[\hat{r}-p_2'\,q^2+(1-s)\,H_1'\right],\\[0.5ex]
B_m^{\,m\pm j}(\hat{r}) &= \pm\epsilon\,\frac{m\pm j}{j}\,\left[(1-s)\,(H_j'\pm {\rm i}\,V_j')- (j^2-1)\,\frac{H_j\pm {\rm i}\,V_j}{\hat{r}}\right],\\[0.5ex]
C_m^{\,m}(\hat{r}) &= 0,\\[0.5ex]
C_m^{\,m\pm 1}(\hat{r}) &= \pm \epsilon\,m\left[\hat{r}-p_2'\,q^2+(1-s)\,H_1'\right],\\[0.5ex]
C_m^{\,m\pm j}(\hat{r}) &= \pm\epsilon\,\frac{m}{j}\,\left[(1-s)\,(H_j'\pm {\rm i}\,V_j')- (j^2-1)\,\frac{H_j\pm {\rm i}\,V_j}{\hat{r}}\right],\\[0.5ex]
D_m^{\,m}(\hat{r}) &= m^2 + q\,\hat{r}\,\frac{d}{d\hat{r}}\!\left(\frac{2-s}{q}\right) +\epsilon^2\,m^2\left(\frac{7\,\hat{r}^{\,2}}{4} - H_1 -3\,\hat{r}\,H_1' + S_1\right)\nonumber\\[0.5ex]
&\phantom{=}+\epsilon^2\left\{-\hat{r}^{\,2}\,\frac{(2-s)^2}{q^2}- \hat{r}\,\frac{d}{d\hat{r}}(\hat{r}\,p_2') -2\,\hat{r}\,p'\,(2-s)+q\,\hat{r}\,\frac{d{\mit\Sigma}}{d\hat{r}}
+2\,\hat{r}\,p_2'\,(1-q^2)\right.\nonumber\\[0.5ex]
&\phantom{=}\left.+2\,\hat{r}\,p_2'\left(-1 + \frac{3\,p_2'\,q^2}{\hat{r}}\right) + 2\,H_1'\,q^2\left[\frac{d}{d\hat{r}}(\hat{r}\,p_2')+4\,(s-1)\,p_2'\right]\right\},\\[0.5ex]
D_m^{\,m\pm 1}(\hat{r}) &=\epsilon\left[(\hat{r}\,p_2')' - (2-s)\,p_2'\right]q^2 + \epsilon\,m\,(m\pm 1)\,(\hat{r}-H_1'),\\[0.5ex]
D_m^{\,m\pm j}(\hat{r}) &= -\epsilon\,m\,(m\pm j)\,(H_j'\pm {\rm i}\,V_j').
\end{align}

\end{document}