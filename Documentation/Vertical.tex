\documentclass[12pt,prb,aps,notitlepage]{revtex4-1}
\usepackage{amsmath}           		          	
\usepackage{graphicx,epstopdf}					
\usepackage{amssymb}
\usepackage{fullpage}
\usepackage{color}
\usepackage{esint}
\pdfoutput = 1 
\newcommand {\bxi}{\mbox{\boldmath$\xi$}}
\allowdisplaybreaks

\begin{document}
\title{Calculation of Vertical Stability in an Inverse Aspect-Ratio Expanded Tokamak Plasma Equilibrium}
\author{Richard Fitzpatrick\,\footnote{rfitzp@utexas.edu}}
\affiliation{Institute for Fusion Studies, Department of Physics, University of Texas at Austin, Austin, TX 78712}
\maketitle

\section{Plasma Equilibrium}\label{geq}
All lengths  are normalized to  the major radius of the plasma magnetic axis, $R_0$. All magnetic field-strengths
are normalized to the  toroidal field-strength at the magnetic axis, $B_0$. All current densities are normalized to $B_0/(\mu_0\,R_0)$.  All plasma pressures are normalized to $B_0^{\,2}/\mu_0$.

Let $R$, $\phi$, $Z$ be right-handed cylindrical coordinates whose Jacobian 
is
\begin{equation}
(\nabla R\times \nabla\phi\cdot\nabla Z)^{-1} = R.
\end{equation}
Note that $|\nabla\phi|=1/R$. 

Let $r$, $\theta$, $\phi$ be right-handed flux-coordinates whose
Jacobian is
\begin{equation}\label{jac}
{\cal J}(r,\theta)\equiv (\nabla r\times \nabla\theta\cdot\nabla\phi)^{-1} \equiv R\left(\frac{\partial R}{\partial\theta}\,\frac{\partial Z}{\partial r} -\frac{\partial R}{\partial r}\,\frac{\partial Z}{\partial \theta}\right)= r\,R^{\,2}.
\end{equation}
Note that $r=r(R,Z)$ and $\theta=\theta(R,Z)$. 
The magnetic axis corresponds to $r=0$. The inboard mid-plane corresponds to $\theta=0$. 

Consider an axisymmetric tokamak equilibrium whose magnetic field takes the form
\begin{equation}
{\bf B}(r,\theta) = f(r)\,\nabla\phi\times \nabla r + g(r)\,\nabla\phi = f\,\nabla(\phi-q\,\theta)\times \nabla r,
\end{equation}
where
\begin{equation}\label{q}
q(r) = \frac{r\,g}{f}
\end{equation}
is the safety-factor (i.e., the inverse of the rotational transform). Note that ${\bf B}\cdot\nabla r=0$, which implies that $r$ is a magnetic flux-surface label.
We require $g=1$ on the magnetic axis in order to ensure that the normalized toroidal magnetic field-strength at the  axis is unity.  

It is easily demonstrated that
\begin{align}
B^{\,r}&={\bf B}\cdot\nabla r= 0,\label{bup1}\\[0.5ex]
B^{\,\theta} &={\bf B}\cdot\nabla \theta= \frac{f}{r\,R^{\,2}},\label{bup2}\\[0.5ex]
B^{\,\phi} &={\bf B}\cdot\nabla \phi= \frac{g}{R^{\,2}},\label{bup3}\\[0.5ex]
B_r &={\cal J}\,\nabla\theta\times\nabla\phi\cdot{\bf B}= -r\,f\,\nabla r\cdot\nabla\theta,\label{bdown1}\\[0.5ex]
B_\theta &={\cal J}\,\nabla\phi\times\nabla r\cdot{\bf B}= r\,f\,|\nabla r|^2,\\[0.5ex]
B_\phi &={\cal J}\,\nabla r\times\nabla \theta \cdot{\bf B}= g.\label{bdown3}
\end{align}

The Maxwell equation (neglecting the displacement current, because axisymmetric modes are comparatively low-frequency phenomena)
${\bf J}= \nabla\times{\bf B}$
yields
\begin{align}
{\cal J}\,J^{\,r} &= \frac{\partial B_\phi}{\partial \theta} =0,\label{jup1}\\[0.5ex]
{\cal J}\,J^{\,\theta} &= -\frac{\partial B_\phi}{\partial r} = - g',\label{jup2}\\[0.5ex]
{\cal J}\,J^{\,\phi}&= \frac{\partial B_\theta}{\partial r} -\frac{\partial B_r}{\partial\theta}=\frac{\partial}{\partial r}\!\left(r\,f\,|\nabla r|^2\right)+ \frac{\partial}{\partial\theta}\!\left(r\,f\,\nabla r\cdot\nabla\theta\right),\label{jup3}
\end{align}
where ${\bf J}$ is the equilibrium current density, $'\equiv d/dr$, and use has been made of  Eqs.~(\ref{bdown1})--(\ref{bdown3}).

Equilibrium force balance requires that
\begin{equation}\label{e15c}
 \nabla P={\bf J}\times {\bf B},
\end{equation}
where $P(r)$ is the equilibrium scalar plasma pressure. Here, for the sake of simplicity, we have neglected the small centrifugal modifications to force balance due to subsonic plasma
rotation.
It follows that 
\begin{align}\label{eg1}
P'&= {\cal J}(J^{\,\theta}\,B^{\,\phi}-J^{\,\phi}\,B^{\,\theta})= -g'\,\frac{g}{R^{\,2}} - \frac{f}{r\,R^{\,2}}\left[\frac{\partial}{\partial r}\!\left(r\,f\,|\nabla r|^2\right)+ \frac{\partial}{\partial\theta}\!\left(r\,f\,\nabla r\cdot\nabla\theta\right)\right],
\end{align}
where use has been made of Eqs.~(\ref{bup1})--(\ref{bup3}), and  (\ref{jup1})--(\ref{jup3}). The
other two components of Eq.~(\ref{e15c}) are identically zero. 

Equation~(\ref{eg1}) yields the {\em Grad-Shafranov equation},
\begin{equation}\label{gs}
\frac{f}{r}\,\frac{\partial}{\partial r}\!\left(r\,f\,|\nabla r|^2\right) +\frac{f}{r}\,\frac{\partial}{\partial\theta}\!\left(r\,f\,\nabla r\cdot\nabla\theta\right)+g\,g' + R^{\,2}\,P'=0.
\end{equation}
It follows from Eqs.~(\ref{q}), (\ref{jup3}), and (\ref{gs}) that
\begin{equation}\label{jup3a}
{\cal J}\,J^{\,\phi} = -q\,g' - \frac{r\,R^{\,2}\,P'}{f}.
\end{equation}
It is clear from Eqs.~(\ref{jup2}) and (\ref{jup3a}) that $g'=P'=0$ in the  current-free ``vacuum'' region surrounding the plasma.
We shall also assume that $g'=P'=0$ at the plasma-vacuum interface, so as to ensure that the equilibrium plasma
current density is zero at the interface. 

\section{Axisymmetric Plasma Perturbation}\label{opde}
Let us assume that all perturbed quantities are independent of the toroidal angle, $\phi$. 
The perturbed plasma equilibrium satisfies the  marginally-stable ideal-MHD equations
\begin{align}
{\bf b} &= \nabla\times (\bxi\times {\bf B}),\label{e21}\\[0.5ex]
\nabla p &={\bf j}\times {\bf B}  +{\bf J}\times {\bf b},\label{e22}\\[0.5ex]
{\bf j} &= \nabla\times {\bf b},\label{e23}\\[0.5ex]
p&= -\bxi\cdot\nabla P,\label{e24}
\end{align}
where $\bxi(r,\theta)$ is the plasma displacement, ${\bf b}(r,\theta)$ the perturbed magnetic field,
${\bf j}(r,\theta)$ the perturbed current density, and $p(r,\theta)$ the perturbed scalar pressure. 

Now, 
\begin{align}
(\bxi\times {\bf B})_\theta&= {\cal J}\,(\xi^{\,\phi}\,B^{\,r} - \xi^{\,r}\,B^{\,\phi}) = -{\cal J}\,B^{\,\phi}\,\xi^{\,r},\\[0.5ex]
(\bxi\times {\bf B})_\phi &= {\cal J}\,(\xi^{\,r}\,B^{\,\theta} - \xi^{\,\theta}\,B^{\,r})= {\cal J}\,B^{\,\theta}\,\xi^{\,r},\label{e23x}
\end{align}
where use has been made of  the fact that $B^{\,r}=J^{\,r}=0$. [See Eqs.~(\ref{bup1}) and (\ref{jup1}).]
 Combining Eqs.~(\ref{e21}) and (\ref{e23x}), we obtain
\begin{align}
{\cal J}\,b^{\,r} &= \frac{\partial}{\partial\theta}\left({\cal J}\,B^{\,\theta}\,\xi^{\,r}\right).
\end{align}
Thus, Eqs.~(\ref{jac}), (\ref{q}), (\ref{bup2}), and (\ref{bup3}) give
\begin{align}\label{e41}
r\,R^{\,2}\,b^{\,r}& = \frac{\partial y}{\partial\theta},
\end{align}
where 
\begin{align}\label{e42}
y(r,\theta) &=f\,\xi^{\,r}.
\end{align}
The constraint $\nabla\cdot{\bf b} =0$, which follows from Eq.~(\ref{e21}),  immediately yields
\begin{equation}\label{e43y}
r\,R^{\,2}\,b^{\,\theta} = - \frac{\partial y}{\partial r}.
\end{equation}

According to Eq.~(\ref{e24}), 
\begin{equation}
p =-P'\,\nabla r\cdot\bxi=- P'\,\xi^{\,r}.
\end{equation}
So, the perturbed force balance equation, (\ref{e22}), yields
\begin{align}
-\frac{\partial\, (P'\,\xi^{\,r})}{\partial r} &= ({\bf j}\times {\bf B})_r+({\bf J}\times {\bf b})_r,\\[0.5ex]
-\frac{\partial\,(P'\,\xi^{\,r})}{\partial \theta}&= ({\bf j}\times {\bf B})_\theta+({\bf J}\times {\bf b})_\theta,\\[0.5ex]
0&= ({\bf j}\times {\bf B})_\phi+({\bf J}\times {\bf b})_\phi,
\end{align}
giving
\begin{align}
-\frac{\partial\, (P'\,\xi^{\,r})}{\partial r} &=r\,R^{\,2}\,(j^{\,\theta}\,B^{\,\phi}-j^{\,\phi}\,B^{\,\theta}) + r\,R^{\,2}\,(J^{\,\theta}\,b^{\,\phi}-J^{\,\phi}\,b^{\,\theta}),\\[0.5ex]
-\frac{\partial\,(P'\,\xi^{\,r})}{\partial \theta}&=r\,R^{\,2}\,(j^{\,\phi}\,B^{\,r}-j^{\,r}\,B^{\,\phi}) + r\,R^{\,2}\,(J^{\,\phi}\,b^{\,r}-J^{\,r}\,b^{\,\phi}),\\[0.5ex]
0&=r\,R^{\,2}\,(j^{\,r}\,B^{\,\theta}-j^{\,\theta}\,B^{\,r}) + r\,R^{\,2}\,(J^{\,r}\,b^{\,\theta}-J^{\,\theta}\,b^{\,r}),
\end{align}
where use has been made of Eq.~(\ref{jac}). 
Thus, according to Eqs.~(\ref{bup1})--(\ref{bup3}), (\ref{jup1}), (\ref{jup2}), and (\ref{jup3a}), 
\begin{align}
-\frac{\partial\, (P'\,\xi^{\,r})}{\partial r} &= f\,(q\,j^{\,\theta} -j^{\,\phi}) - g'\,b^{\,\phi} + \left(q\,g'+\frac{r\,R^{\,2}\,P'}{f}\right)b^{\,\theta},\label{e51}\\[0.5ex]
-\frac{\partial\,(P'\,\xi^{\,r})}{\partial \theta}&=-r\,g\,j^{\,r} - \left(q\,g'+\frac{r\,R^{\,2}\,P'}{f}\right)b^{\,r},\label{e44}\\[0.5ex]
0&= f\,j^{\,r}+g'\,b^{\,r}.\label{e53}
\end{align}
It follows from Eqs.~(\ref{e41}) and (\ref{e53}) that 
\begin{equation}\label{e54}
r\,R^{\,2}\,j^{\,r} = -\alpha_g\,\frac{\partial y}{\partial\theta},
\end{equation}
where
\begin{align}
\alpha_g (r)&= \frac{g'}{f}.\label{ag}
\end{align}
Note that Eq.~(\ref{e44}) is trivially satisfied. Hence, of the three components of the perturbed force balance equation, only Eq.~(\ref{e51}) remains to be solved. 

Equation~(\ref{e23}) yields 
\begin{align}
r\,R^{\,2}\,j^{\,r} &= \frac{\partial b_\phi}{\partial\theta},\label{e57}\\[0.5ex]
r\,R^{\,2}\,j^{\,\theta} &= -\frac{\partial b_\phi}{\partial r},\label{e58}\\[0.5ex]
r\,R^{\,2}\,j^{\,\phi}&= \frac{\partial b_\theta}{\partial r} -\frac{\partial b_r}{\partial \theta},\label{e59}
\end{align}
where use has been made of Eq.~(\ref{jac}). It follows from Eqs.~(\ref{e54}), (\ref{e57}), and (\ref{e58}) that
\begin{align}\label{e43yy}
b_\phi &=-\alpha_g\,y,\\[0.5ex]
r\,R^{\,2}\,j^{\,\theta}&=  \frac{\partial (\alpha_g\,y)}{\partial r}.\label{e44yy}
\end{align}

Now, 
\begin{equation}
{\bf b} = b_r\,\nabla r + b_\theta\,\nabla\theta+b_\phi\,\nabla\phi,
\end{equation}
so
\begin{align}
b^{\,r} &= {\bf b}\cdot\nabla r = |\nabla r|^2\,b_r + (\nabla r\cdot\nabla\theta)\,b_\theta,\label{e61}\\[0.5ex]
b^{\,\theta} &= {\bf b}\cdot\nabla \theta = (\nabla r\cdot\nabla\theta)\,b_r + |\nabla\theta|^2\,b_\theta,\label{e62}\\[0.5ex]
b^{\,\phi}&={\bf b}\cdot\nabla\phi =\frac{b_\phi}{R^{\,2}}.\label{e63}
\end{align}

Equations~(\ref{e61}) and (\ref{e62}) can be rearranged to give
\begin{align}
b_r &= \left(\frac{1}{|\nabla r|^2}\right)b^{\,r}- \left(\frac{\nabla r\cdot\nabla\theta}{|\nabla r|^2}\right)b_\theta,\label{e69}\\[0.5ex]
b^{\,\theta} &= \left(\frac{\nabla r\cdot\nabla\theta}{|\nabla r|^2}\right)b^{\,r} +\left[|\nabla\theta|^2 -\frac{(\nabla r\cdot\nabla\theta)^2}{|\nabla r|^2}\right]b_\theta.\label{e70}
\end{align}
But, from Eq.~(\ref{jac}), 
\begin{equation}
|\nabla r|^2\,|\nabla\theta|^2-(\nabla r \cdot\nabla\theta)^2 = \frac{1}{r^2\,R^{\,2}}.
\end{equation}
Thus, Eq.~(\ref{e70}) reduces to 
\begin{equation}\label{e58x}
b^{\,\theta} = \left(\frac{\nabla r\cdot\nabla\theta}{|\nabla r|^2}\right)b^{\,r} + \left(\frac{1}{r^2\,R^{\,2}\,|\nabla r|^2}\right) b_\theta.
\end{equation}
Let 
\begin{equation}\label{zdef}
z= |\nabla r|^2\,r\,\frac{\partial y}{\partial r} + r\,\nabla r\cdot\nabla\theta\,\frac{\partial y}{\partial\theta}.
\end{equation}
Equations~(\ref{e41}), (\ref{e43y}), (\ref{e43yy}), (\ref{e69}) and (\ref{e58x}) yield 
\begin{align}\label{e53y}
b_r &=\frac{1}{r\,|\nabla r|^2\,R^{\,2}}\,\frac{\partial y}{\partial \theta} + \frac{\nabla r\cdot\nabla\theta}{|\nabla r|^2}\,z,\\[0.5ex]
b_\theta &= -z,\label{e54y}\\[0.5ex]
b^{\,\phi} &= -\frac{\alpha_g}{R^{\,2}}\,y.\label{e55yy}
\end{align}
Equations~(\ref{e59}), (\ref{e53y}), and (\ref{e54y}) give
\begin{align}\label{e56yy}
r\,R^{\,2}\,j^{\,\phi} &=-\frac{\partial z}{\partial r}-\frac{\partial}{\partial\theta}\!\left[\frac{1}{r\,|\nabla r|^2\,R^{\,2}}\,\frac{\partial y}{\partial \theta} + \frac{\nabla r\cdot\nabla\theta}{|\nabla r|^2}\,z\right].
\end{align}

It follows from Eqs.~(\ref{e42}), (\ref{e43y}), (\ref{e51}), (\ref{e44yy}), (\ref{e55yy}), and (\ref{e56yy}) that
\begin{align}
-\frac{\partial}{\partial r}\!\left(\frac{P'}{f}\,y\right)&= \frac{f\,q}{r\,R^{\,2}}\,\frac{\partial (\alpha_g\,y)}{\partial r} 
+ \frac{f}{r\,R^{\,2}}\,\frac{\partial z}{\partial r}\nonumber\\[0.5ex]
&\phantom{=} +\frac{f}{r\,R^{\,2}}\frac{\partial}{\partial\theta}\!\left[\frac{1}{r\,|\nabla r|^2\,R^{\,2}}\,\frac{\partial y}{\partial \theta} + \frac{\nabla r\cdot\nabla\theta}{|\nabla r|^2}\,z\right]
\nonumber\\[0.5ex]
&\phantom{=} + \frac{g'\,\alpha_g}{R^{\,2}}\,y - \left(q\,g'+\frac{r\,R^{\,2}\,P'}{f}\right)\frac{1}{r\,R^{\,2}}\,\frac{\partial y}{\partial r}.
\end{align}
Hence,
\begin{align}\label{e58u}
-\left[(\alpha_p\,\alpha_f+ r\,\alpha_p')R^{\,2} + q\,r\,\alpha_g'+r^2\,\alpha_g^{\,2}\right]y&= 
 r\,\frac{\partial z}{\partial r}+\frac{\partial}{\partial\theta}\!\left[\frac{1}{|\nabla r|^2\,R^{\,2}}\,\frac{\partial y}{\partial \theta} + \frac{r\,\nabla r\cdot\nabla\theta}{|\nabla r|^2}\,z\right],
\end{align}
where
\begin{align}\label{alpp}
\alpha_p(r)&= \frac{r\,P'}{f^2},\\[0.5ex]
\alpha_f(r) &= \frac{r^2}{f}\,\frac{d}{dr}\!\left(\frac{f}{r}\right).
\end{align}

Finally, it follows from Eqs.~(\ref{zdef}) and (\ref{e58u})  that
\begin{align}\label{e63y}
r\,\frac{\partial y}{\partial r} &= \frac{z}{|\nabla r|^2} - \frac{r\,\nabla r\cdot\nabla\theta}{|\nabla r|^2}\,\frac{\partial y}{\partial\theta},\\[0.5ex]
r\,\frac{\partial z}{\partial r}&= -\left[(\alpha_p\,\alpha_f+ r\,\alpha_p')R^{\,2} + q\,r\,\alpha_g'+r^2\,\alpha_g^{\,2}\right]y-\frac{\partial}{\partial\theta}\!\left(\frac{1}{|\nabla r|^2\,R^{\,2}}\,\frac{\partial y}{\partial\theta}\right)\nonumber\\[0.5ex]
&\phantom{=} -\frac{\partial}{\partial\theta}\!\left(\frac{r\,\nabla r\cdot\nabla\theta}{|\nabla r|^2}\,z\right).\label{e64y}
\end{align}

The previous two equations can be combined to give
\begin{align}
r\,\frac{\partial (z\,y^\ast)}{\partial r} &= \frac{|z|^2}{|\nabla r|^2} - \frac{\partial y^\ast}{\partial \theta}\left(\frac{r\,\nabla r\cdot\nabla\theta}{|\nabla r|^2}\,z\right)
 -\left[(\alpha_p\,\alpha_f+ r\,\alpha_p')R^{\,2} + q\,r\,\alpha_g'+r^2\,\alpha_g^{\,2}\right]|y|^2\nonumber\\[0.5ex]
 &\phantom{=} - y^\ast\,\frac{\partial}{\partial\theta}\!\left(\frac{1}{|\nabla r|^2\,R^{\,2}}\,\frac{\partial y}{\partial\theta}\right)-y^\ast\,\frac{\partial}{\partial\theta}\!\left(\frac{r\,\nabla r\cdot\nabla\theta}{|\nabla r|^2}\,z\right).
\end{align}
Hence,
\begin{align}
r\,\frac{d}{dr}\!\left(\oint z\,y^\ast\,\frac{d\theta}{2\pi}\right)&= \oint\left[\frac{|z|^2}{|\nabla r|^2}- \left[(\alpha_p\,\alpha_f+ r\,\alpha_p')R^{\,2} + q\,r\,\alpha_g'+r^2\,\alpha_g^{\,2}\right]|y|^2\right.\nonumber\\[0.5ex]
&\phantom{=}\left.
+ \frac{1}{|\nabla r|^2\,R^{\,2}}\,\left|\frac{\partial y}{\partial \theta}\right|^2\right]\frac{d\theta}{2\pi},
\end{align}
which implies that $\oint z\,y^\ast\,d\theta/2\pi$ is a real quantity, and also that
\begin{equation}
r\,\frac{d}{dr}\!\left[\oint(z\,y^\ast-y^\ast\,z)\,\frac{d\theta}{2\pi}\right] =0. 
\end{equation}

Let
\begin{align}
y(r,\theta)&= \sum_m y_m(r)\,{\rm e}^{\,{\rm i}\,m\,\theta},\\[0.5ex]
z(r,\theta) &= \sum_m z_m(r)\,{\rm e}^{\,{\rm i}\,m\,\theta}.
\end{align}
Equations~(\ref{e63y}) and (\ref{e64y}) yield
\begin{align}\label{e69u}
r\,\frac{dy_m}{dr}&= \sum_{m'}\left(A_{m}^{\,m'}\,z_{m'} + B_{m}^{\,m'}\,y_{m'}\right),\\[0.5ex]
r\,\frac{dz_m}{dr}&= \sum_{m'}\left(C_{m}^{\,m'}\,z_{m'} + D_{m}^{\,m'}\,y_{m'}\right),\label{e70u}
\end{align}
where
\begin{align}
A_m^{\,m'} &= c_{m}^{\,m'},\\[0.5ex]
B_m^{\,m'} &= - m'\,f_m^{\,m'},\\[0.5ex]
C_{m}^{\,m'} &= -m\,f_m^{\,m'},\\[0.5ex]
D_{m}^{\,m'}&= -(\alpha_f\,\alpha_p+ r\,\alpha_p')\,a_m^{\,m'} - (q\,r\,\alpha_g' +r^{\,2}\,\alpha_g^{\,2})\,\delta_m^{\,m'}+m\,m'\,b_m^{\,m'},
\end{align}
where
\begin{align}
a_m^{\,m'}(r)  &=\oint R^{\,2}\,\exp[-{\rm i}\,(m-m')\,\theta]\,\frac{d\theta}{2\pi},\\[0.5ex]
b_m^{\,m'}(r)  &=\oint|\nabla r|^{-2}\, R^{-2}\,\exp[-{\rm i}\,(m-m')\,\theta]\,\frac{d\theta}{2\pi},\\[0.5ex]
c_m^{\,m'}(r)  &=\oint|\nabla r|^{-2}\,\exp[-{\rm i}\,(m-m')\,\theta]\,\frac{d\theta}{2\pi},\\[0.5ex]
f_{m}^{\,m'}(r)&= \oint \frac{{\rm i}\,r\,\nabla r\cdot\nabla\theta}{|\nabla r|^2}\,\exp[-{\rm i}\,(m-m')\,\theta]\,\frac{d\theta}{2\pi}.
\end{align}
Note that
$a_{m'}^{\,m} =a_{m}^{\,m'\ast}$, 
$b_{m'}^{\,m} =b_{m}^{\,m'\ast}$, 
$c_{m'}^{\,m} =c_{m}^{\,m'\ast}$, and 
$f_{m'}^{\,m} =-f_{m}^{\,m'\ast}$,
which implies that
\begin{align}\label{e70u}
A_{m'}^{\,m} &= A_{m'}^{\,m\,\ast},\\[0.5ex]
B_{m'}^{\,m} &= -C_{m'}^{\,m\,\ast},\\[0.5ex]
C_{m'}^{\,m} &= -B_{m'}^{\,m\,\ast},\\[0.5ex]
D_{m'}^{\,m} &= A_{m'}^{\,m\,\ast}.\label{e82u}
\end{align}
It follows from Eqs.~(\ref{e69u}), (\ref{e70u}), and (\ref{e70u})--(\ref{e82u}) that 
\begin{align}
r\,\frac{d}{dr}\!\left(\sum_m z_m\,y_m^{\,\ast} \right)&= \sum_{m,m'}\left(z_m^{\,\ast}\,A_{m}^{\,m'}\,z_{m'} + y_m^{\,\ast}\,D_{m}^{\,m'}\,y_{m'}\right),\\[0.5ex]
r\,\frac{d}{dr}\!\left[\sum_m (z_m\,y_m^{\,\ast}- y_m\,z_m^{\,\ast}) \right]&= 0.
\end{align}

\section{Toroidal Electromagnetic Torque}
The net toroidal electromagnetic torque exerted on the plasma lying within the magnetic flux-surface whose label is $r$ is
\begin{equation}
T_\phi(r)= \oint\oint r\,R^{\,2}\,b_\phi\,b^r\,d\theta\,d\phi.
\end{equation}
It follows from Eqs.~(\ref{e41}) and (\ref{e43yy}) that 
\begin{equation}
T_\phi(r) = -\pi\,\alpha_g\oint\left(y^\ast\,\frac{\partial y}{\partial\theta}+y\,\frac{\partial y^\ast}{\partial\theta}\right)d\theta
= -\pi\,\alpha_g\oint\frac{\partial|y|^2}{\partial\theta}\,d\theta = 0.
\end{equation}
We conclude that an axisymmetric perturbation is incapable of exerting a net toroidal electromagnetic torque on the plasma. 

\section{Perturbed Plasma Potential Energy}
The perturbed plasma potential energy in the region of the plasma lying within the magnetic flux-surface whose label is $r$ is 
\begin{equation}
\delta W_p = \frac{1}{2}\oint\oint r\,R^{\,2}\,\xi_r^{\,\ast}(-{\bf B}\cdot{\bf b} + \xi^{\,r}\,P')\,d\theta\,d\phi.
\end{equation}
However,
\begin{equation}
{\bf B}\cdot{\bf b} -\xi^{\,r}\,P' = B^{\,\theta}\,b_\theta+B^{\,\phi}\,b_\phi - \xi^{\,r}\,P'
= - \frac{f}{r\,R^{\,2}}\left(z + q\,\alpha_g\,y + \alpha_p\,R^{\,2}\right),
\end{equation}
where use has been made of Eqs.~(\ref{q}), (\ref{bup2}), (\ref{bup3}), (\ref{e42}), (\ref{e43yy}), (\ref{e54y}),  and (\ref{alpp}). Hence,
we obtain
\begin{align}
\delta W_p(r) &=\frac{1}{2}\oint\oint y^{\ast}\left[z + (q\,\alpha_g + \alpha_p\,R^{\,2})\,y\right]d\theta\,d\phi= \pi^2\sum_{m}y_m^{\,\ast}\,\chi_m,
\end{align}
where 
\begin{equation}
\chi_m(r)=z_m + q\,\alpha_g\,y_m + \alpha_p\sum_{m'} a_m^{\,m'}\,y_{m'}.
\end{equation}
\end{document}