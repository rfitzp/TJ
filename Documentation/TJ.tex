\documentclass[12pt,prb,aps]{revtex4-1}
\usepackage{amsmath}           		          	
\usepackage{graphicx,epstopdf}					
\usepackage{amssymb}
\usepackage{fullpage}
\usepackage{color}
\usepackage{esint}
\pdfoutput = 1 
\newcommand {\bxi}{\mbox{\boldmath$\xi$}}
\allowdisplaybreaks

\begin{document}
\title{Calculation of Tearing-Mode Stability in an Inverse Aspect-Ratio Expanded Tokamak Plasma Equilibrium}
\author{Richard Fitzpatrick\,\footnote{rfitzp@utexas.edu}}
\affiliation{Institute for Fusion Studies, Department of Physics, University of Texas at Austin, Austin, TX 78712}

\maketitle

\section{Introduction} 
The calculation of tearing-mode stability in a high temperature tokamak plasma is most efficiently formulated as  an asymptotic
matching problem.\cite{fkr} In such a problem, the  plasma is  divided into two regions. In the ``outer region'', which comprises most
of the plasma, the tearing perturbation is described by the equations of linearized, marginally-stable, ideal magnetohydrodynamics (ideal-MHD). (See Sect.~\ref{mhd}.)
However, these equations become singular on so-called ``rational'' magnetic flux-surfaces at which the perturbed magnetic field resonates with the equilibrium field. In the so-called ``inner region'', which
consists of a set of narrow layers centered on the various rational surfaces, non-ideal-MHD effects such as plasma inertia, resistivity, and
viscosity become important. The growth-rate and angular rotation frequency of the reconnected magnetic flux at a given rational
surface (numbered $k$) are fixed by asymptotically matching the resistive layer
solution in the associated segment of the inner region, which is characterized by a dimensionless complex quantity ${\mit\Delta}_k$,  to the ideal-MHD solution in the outer region. In a realistic axisymmetric tokamak  plasma equilibrium, tearing
perturbations with different toroidal mode numbers are independent of one another, whereas perturbations with different poloidal
mode numbers are coupled together via toroidicity and the non-circular shaping of magnetic flux-surfaces.\cite{con0}
Consequently, for a tearing perturbation with a given toroidal mode number, 
 the ${\mit\Delta}_k$ values associated with  the various rational surfaces in the plasma are interrelated via a matrix equation.\cite{connor,cht,tokuda,brennan, ham,am1,am3}

In general, the  numerical determination of the elements of the matrix equation that links the various ${\mit\Delta}_k$ values from the ideal-MHD equations 
in the outer region is a exceptionally challenging task.\cite{nish,pletz,am2,aglas,aglas1}
One way of greatly reducing the complexity of this task is to employ an inverse aspect-ratio expanded plasma equilibrium.\cite{greene} In such an equilibrium,
the metric elements of the flux-coordinate system can be expressed analytically in terms of a relatively small number of  flux-surface functions,
which represents a major simplification.\cite{con0} Another significant advantage of an inverse aspect-ratio expanded equilibrium is that the magnetic perturbation in the plasma can be
matched to a vacuum solution  expressed as an expansion in toroidal functions.\cite{am1} The alternative approach of using a Green's
function solution in the vacuum region is much more computationally intensive.\cite{chance,xu} The inverse aspect-ratio expansion approach to determining tearing-mode stability in tokamak plasmas 
was first introduced in Ref.~\onlinecite{connor} in a calculation that allows for three  poloidal harmonics coupled via toroidicity. The
inverse aspect-ratio expansion approach was extended in Ref.~\onlinecite{am1} in a calculation that allows for seven poloidal harmonics coupled via toroidicity, flux-surface elongation, and
flux-surface triangularity. In this paper, we intend to generalize the inverse aspect-ratio expansion  approach to allow for an arbitrary number of poloidal harmonics coupled
by flux-surfaces of general shape. Furthermore, unlike Refs.~\onlinecite{connor} and \onlinecite{am1}, we shall not assume that the plasma
equilibrium 
is up-down symmetric. 

\section{Preliminary Analysis}
\subsection{Normalization}\label{coords}
All lengths in this paper are normalized to  the major radius of the plasma magnetic axis, $R_0$. All magnetic field-strengths
are normalized to the  toroidal field-strength at the magnetic axis, $B_0$. All current densities are normalized to $B_0/(\mu_0\,R_0)$. All plasma pressures are normalized to $B_0^{\,2}/\mu_0$.
All toroidal electromagnetic torques are normalized to $B_0^{\,2}\,R_0^{\,3}/\mu_0$. 

\subsection{Axisymmetric Tokamak Plasma Equilibrium}\label{s3}
Let $R$, $\phi$, $Z$ be right-handed cylindrical coordinates whose Jacobian 
is
\begin{equation}
(\nabla R\times \nabla\phi\cdot\nabla Z)^{-1} = R.
\end{equation}
Note that $|\nabla\phi|=1/R$. 

Let $r$, $\theta$, $\phi$ be right-handed flux-coordinates whose
Jacobian is\,\cite{connor,bussac}
\begin{equation}\label{jac}
{\cal J}(r,\theta)\equiv (\nabla r\times \nabla\theta\cdot\nabla\phi)^{-1} \equiv R\left(\frac{\partial R}{\partial\theta}\,\frac{\partial Z}{\partial r} -\frac{\partial R}{\partial r}\,\frac{\partial Z}{\partial \theta}\right)= r\,R^{\,2}.
\end{equation}
Note that $r=r(R,Z)$ and $\theta=\theta(R,Z)$. 
The magnetic axis corresponds to $r=0$. The inboard mid-plane corresponds to $\theta=0$. 

Consider an axisymmetric tokamak equilibrium\,\cite{gs1} whose magnetic field takes the form\,\cite{connor,am1}
\begin{equation}
{\bf B}(r,\theta) = f(r)\,\nabla\phi\times \nabla r + g(r)\,\nabla\phi = f\,\nabla(\phi-q\,\theta)\times \nabla r,
\end{equation}
where
\begin{equation}\label{q}
q(r) = \frac{r\,g}{f}
\end{equation}
is the safety-factor (i.e., the inverse of the rotational transform). Note that ${\bf B}\cdot\nabla r=0$, which implies that $r$ is a magnetic flux-surface label.
We require $g=1$ on the magnetic axis in order to ensure that the normalized toroidal magnetic field-strength at the  axis is unity.  

It is easily demonstrated that
\begin{align}
B^{\,r}&={\bf B}\cdot\nabla r= 0,\label{bup1}\\[0.5ex]
B^{\,\theta} &={\bf B}\cdot\nabla \theta= \frac{f}{r\,R^{\,2}},\label{bup2}\\[0.5ex]
B^{\,\phi} &={\bf B}\cdot\nabla \phi= \frac{g}{R^{\,2}},\label{bup3}\\[0.5ex]
B_r &={\cal J}\,\nabla\theta\times\nabla\phi\cdot{\bf B}= -r\,f\,\nabla r\cdot\nabla\theta,\label{bdown1}\\[0.5ex]
B_\theta &={\cal J}\,\nabla\phi\times\nabla r\cdot{\bf B}= r\,f\,|\nabla r|^2,\\[0.5ex]
B_\phi &={\cal J}\,\nabla r\times\nabla \theta \cdot{\bf B}= g,\label{bdown3}
\end{align}
where use has been made of Sect.~\ref{s2}. 

The Maxwell equation (neglecting the displacement current, because tearing modes are comparatively low-frequency phenomena)
${\bf J}= \nabla\times{\bf B}$
yields
\begin{align}
{\cal J}\,J^{\,r} &= \frac{\partial B_\phi}{\partial \theta} =0,\label{jup1}\\[0.5ex]
{\cal J}\,J^{\,\theta} &= -\frac{\partial B_\phi}{\partial r} = - g',\label{jup2}\\[0.5ex]
{\cal J}\,J^{\,\phi}&= \frac{\partial B_\theta}{\partial r} -\frac{\partial B_r}{\partial\theta}=\frac{\partial}{\partial r}\!\left(r\,f\,|\nabla r|^2\right)+ \frac{\partial}{\partial\theta}\!\left(r\,f\,\nabla r\cdot\nabla\theta\right),\label{jup3}
\end{align}
where ${\bf J}$ is the equilibrium current density, $'\equiv d/dr$, and use has been made of  Eqs.~(\ref{bdown1})--(\ref{bdown3}) and (\ref{curl1})--(\ref{curl3}).

Equilibrium force balance requires that
\begin{equation}\label{e15c}
 \nabla P={\bf J}\times {\bf B},
\end{equation}
where $P(r)$ is the equilibrium plasma pressure. Here, for the sake of simplicity, we have neglected the small centrifugal modifications to force balance due to plasma
rotation.\cite{flow,flow1}
It follows that 
\begin{align}\label{eg1}
P'&= {\cal J}(J^{\,\theta}\,B^{\,\phi}-J^{\,\phi}\,B^{\,\theta})= -g'\,\frac{g}{R^{\,2}} - \frac{f}{r\,R^{\,2}}\left[\frac{\partial}{\partial r}\!\left(r\,f\,|\nabla r|^2\right)+ \frac{\partial}{\partial\theta}\!\left(r\,f\,\nabla r\cdot\nabla\theta\right)\right],
\end{align}
where use has been made of Eqs.~(\ref{bup1})--(\ref{bup3}),  (\ref{jup1})--(\ref{jup3}), and (\ref{crossdown1})--(\ref{crossdown3}). The
other two components of Eq.~(\ref{e15c}) are identically zero. 

Equation~(\ref{eg1}) yields the {\em Grad-Shafranov equation},\cite{gs1}
\begin{equation}\label{gs}
\frac{f}{r}\,\frac{\partial}{\partial r}\!\left(r\,f\,|\nabla r|^2\right) +\frac{f}{r}\,\frac{\partial}{\partial\theta}\!\left(r\,f\,\nabla r\cdot\nabla\theta\right)+g\,g' + R^{\,2}\,P'=0.
\end{equation}
It follows from Eqs.~(\ref{q}), (\ref{jup3}), and (\ref{gs}) that
\begin{equation}\label{jup3a}
{\cal J}\,J^{\,\phi} = -q\,g' - \frac{r\,R^{\,2}\,P'}{f}.
\end{equation}
It is clear from Eqs.~(\ref{jup2}) and (\ref{jup3a}) that $g'=P'=0$ in the  current-free vacuum region surrounding the plasma.
We shall also assume that $g'=P'=0$ at the plasma/vacuum interface, so as to ensure that the equilibrium plasma
current density is zero at the interface. 

\section{Derivation of Outer-Region PDEs}
\subsection{Introduction}
The outer-region partial differential equations (PDEs) were first presented in Ref.~\onlinecite{connor} without an explicit
derivation. However, the derivation is sufficiently non-obvious that it is worth outlining in this section. 

\subsection{Governing Equations}\label{mhd}
In the outer region, the perturbed plasma equilibrium satisfies the  ideal-MHD equations\,\cite{connor,am1,am3,gs1}
\begin{align}
{\bf b} &= \nabla\times (\bxi\times {\bf B}),\label{e21}\\[0.5ex]
\nabla p &={\bf j}\times {\bf B}  +{\bf J}\times {\bf b},\label{e22}\\[0.5ex]
{\bf j} &= \nabla\times {\bf b},\label{e23}\\[0.5ex]
p&= -\bxi\cdot\nabla P,\label{e24}
\end{align}
where $\bxi(r,\theta,\phi)$ is the plasma displacement, ${\bf b}(r,\theta,\phi)$ the perturbed magnetic field,
${\bf j}(r,\theta,\phi)$ the perturbed current density, and $p(r,\theta,\phi)$ the perturbed pressure. 
Let us assume that all perturbed quantities vary with $\phi$ as $\exp(-{\rm i}\,n\,\phi)$, where the real positive integer $n$ is the
toroidal mode number of the tearing mode. For example, $p(r,\theta,\phi) = p(r,\theta)\,\exp(-{\rm i}\,n\,\phi)$. 

\subsection{Radial Plasma Displacement}
Equations~(\ref{crossdown2}) and (\ref{crossdown3}) yield
\begin{align}
(\bxi\times {\bf B})_\theta&= {\cal J}\,(\xi^{\,\phi}\,B^{\,r} - \xi^{\,r}\,B^{\,\phi}) = -{\cal J}\,B^{\,\phi}\,\xi^{\,r},\\[0.5ex]
(\bxi\times {\bf B})_\phi &= {\cal J}\,(\xi^{\,r}\,B^{\,\theta} - \xi^{\,\theta}\,B^{\,r})= {\cal J}\,B^{\,\theta}\,\xi^{\,r},
\end{align}
where use has been made of  the fact that $B^{\,r}=J^{\,r}=0$. [See Eqs.~(\ref{bup1}) and (\ref{jup1}).]
Combining the previous two equations with  Eqs.~(\ref{e21}) and (\ref{curl1}), we obtain
\begin{align}
{\cal J}\,b^{\,r} &= \frac{\partial}{\partial\theta}\left({\cal J}\,B^{\,\theta}\,\xi^{\,r}\right)-{\rm i}\,n\,{\cal J}\,B^{\,\phi}\,\xi^{\,r}.
\end{align}
Thus, Eqs.~(\ref{jac}), (\ref{q}), (\ref{bup2}), and (\ref{bup3}) give
\begin{align}\label{e41}
r\,R^{\,2}\,b^{\,r}& = \left(\frac{\partial}{\partial\theta}-{\rm i}\,n\,q\right)y,
\end{align}
where 
\begin{align}\label{e42}
y(r,\theta) &=f\,\xi^{\,r}.
\end{align}

\subsection{Perturbed Force Balance}
According to Eq.~(\ref{e24}), 
\begin{equation}
p =-P'\,\nabla r\cdot\bxi=- P'\,\xi^{\,r}.
\end{equation}
So, the perturbed force balance equation, (\ref{e22}), yields
\begin{align}
-\frac{\partial\, (P'\,\xi^{\,r})}{\partial r} &= ({\bf j}\times {\bf B})_r+({\bf J}\times {\bf b})_r,\\[0.5ex]
-\frac{\partial\,(P'\,\xi^{\,r})}{\partial \theta}&= ({\bf j}\times {\bf B})_\theta+({\bf J}\times {\bf b})_\theta,\\[0.5ex]
{\rm i}\,n\,P'\,\xi^{\,r} &= ({\bf j}\times {\bf B})_\phi+({\bf J}\times {\bf b})_\phi,
\end{align}
giving
\begin{align}
-\frac{\partial\, (P'\,\xi^{\,r})}{\partial r} &=r\,R^{\,2}\,(j^{\,\theta}\,B^{\,\phi}-j^{\,\phi}\,B^{\,\theta}) + r\,R^{\,2}\,(J^{\,\theta}\,b^{\,\phi}-J^{\,\phi}\,b^{\,\theta}),\\[0.5ex]
-\frac{\partial\,(P'\,\xi^{\,r})}{\partial \theta}&=r\,R^{\,2}\,(j^{\,\phi}\,B^{\,r}-j^{\,r}\,B^{\,\phi}) + r\,R^{\,2}\,(J^{\,\phi}\,b^{\,r}-J^{\,r}\,b^{\,\phi}),\\[0.5ex]
{\rm i}\,n\,P'\,\xi^{\,r} &=r\,R^{\,2}\,(j^{\,r}\,B^{\,\theta}-j^{\,\theta}\,B^{\,r}) + r\,R^{\,2}\,(J^{\,r}\,b^{\,\theta}-J^{\,\theta}\,b^{\,r}),
\end{align}
where use has been made of Eqs.~(\ref{jac}) and (\ref{crossdown1})--(\ref{crossdown3}). 
Thus, according to Eqs.~(\ref{bup1})--(\ref{bup3}), (\ref{jup1}), (\ref{jup2}), and (\ref{jup3a}), 
\begin{align}
-\frac{\partial\, (P'\,\xi^{\,r})}{\partial r} &= f\,(q\,j^{\,\theta} -j^{\,\phi}) - g'\,b^{\,\phi} + \left(q\,g'+\frac{r\,R^{\,2}\,P'}{f}\right)b^{\,\theta},\label{e51}\\[0.5ex]
-\frac{\partial\,(P'\,\xi^{\,r})}{\partial \theta}&=-r\,g\,j^{\,r} - \left(q\,g'+\frac{r\,R^{\,2}\,P'}{f}\right)b^{\,r},\label{e44}\\[0.5ex]
{\rm i}\,n\,P'\,\xi^{\,r} &= f\,j^{\,r}+g'\,b^{\,r}.\label{e53}
\end{align}
It follows from Eqs.~(\ref{e42}), (\ref{e41}), and (\ref{e53}) that 
\begin{equation}\label{e54}
r\,j^{\,r} = {\rm i}\,n\,\alpha_p\,y - \frac{\alpha_g}{R^{\,2}}\left(\frac{\partial}{\partial\theta}-{\rm i}\,n\,q\right)y,
\end{equation}
where
\begin{align}
\alpha_p(r) &= \frac{r\,P'}{f^2},\label{ap}\\[0.5ex]
\alpha_g (r)&= \frac{g'}{f}.\label{ag}
\end{align}
Note that Eq.~(\ref{e44}) is trivially satisfied. Hence, only Eq.~(\ref{e51}) remains to be solved. 

\subsection{Perturbed Plasma Current Density}
Equation~(\ref{e23}) yields 
\begin{align}
r\,R^{\,2}\,j^{\,r} &= \frac{\partial b_\phi}{\partial\theta}+{\rm i}\,n\,b_\theta,\label{e57}\\[0.5ex]
r\,R^{\,2}\,j^{\,\theta} &=-{\rm i}\,n\,b_r -\frac{\partial b_\phi}{\partial r},\label{e58}\\[0.5ex]
r\,R^{\,2}\,j^{\,\phi}&= \frac{\partial b_\theta}{\partial r} -\frac{\partial b_r}{\partial \theta},\label{e59}
\end{align}
where use has been made of Eqs.~(\ref{jac}) and (\ref{curl1})--(\ref{curl3}). 
Now, according to Sect.~\ref{s2}, 
\begin{equation}
{\bf b} = b_r\,\nabla r + b_\theta\,\nabla\theta+b_\phi\,\nabla\phi,
\end{equation}
so
\begin{align}
b^{\,r} &= {\bf b}\cdot\nabla r = |\nabla r|^2\,b_r + (\nabla r\cdot\nabla\theta)\,b_\theta,\label{e61}\\[0.5ex]
b^{\,\theta} &= {\bf b}\cdot\nabla \theta = (\nabla r\cdot\nabla\theta)\,b_r + |\nabla\theta|^2\,b_\theta,\label{e62}\\[0.5ex]
b^{\,\phi}&={\bf b}\cdot\nabla\phi =\frac{b_\phi}{R^{\,2}}.\label{e63}
\end{align}

Let us define
\begin{equation}\label{z}
x(r,\theta) = b_\phi.
\end{equation}
It follows from Eqs.~(\ref{e54}), (\ref{e57}), (\ref{e63}), and (\ref{z}) that
\begin{align}
b_\theta &= -\frac{\alpha_g}{{\rm i}\,n}\left(\frac{\partial}{\partial \theta}-{\rm i}\,n\,q\right)y+\alpha_p\,R^{\,2}\,y- \frac{1}{{\rm i}\,n}\,\frac{\partial x}{\partial \theta},\label{e65}\\[0.5ex]
b^{\,\phi}&=\frac{x}{R^{\,2}}.\label{e66}
\end{align}

Equations~(\ref{e61}) and (\ref{e62}) can be rearranged to give
\begin{align}
b_r &= \left(\frac{1}{|\nabla r|^2}\right)b^{\,r}- \left(\frac{\nabla r\cdot\nabla\theta}{|\nabla r|^2}\right)b_\theta,\label{e69}\\[0.5ex]
b^{\,\theta} &= \left(\frac{\nabla r\cdot\nabla\theta}{|\nabla r|^2}\right)b^{\,r} +\left[|\nabla\theta|^2 -\frac{(\nabla r\cdot\nabla\theta)^2}{|\nabla r|^2}\right]b_\theta.\label{e70}
\end{align}
But, from Eq.~(\ref{jac}), 
\begin{equation}
|\nabla r|^2\,|\nabla\theta|^2-(\nabla r \cdot\nabla\theta)^2 = \frac{1}{r^2\,R^{\,2}}.
\end{equation}
Thus, Eq.~(\ref{e70}) reduces to 
\begin{equation}\label{e58x}
b^{\,\theta} = \left(\frac{\nabla r\cdot\nabla\theta}{|\nabla r|^2}\right)b^{\,r} + \left(\frac{1}{r^2\,R^{\,2}\,|\nabla r|^2}\right) b_\theta.
\end{equation}
Making use of Eqs.~(\ref{e41}) and (\ref{e65}), we obtain
\begin{equation}\label{e74}
r^2\,R^{\,2}\,b^{\,\theta} = T\left(\frac{\partial}{\partial\theta}-{\rm i}\,n\,q\right)y + U\,y -Q\,\frac{\partial x}{\partial\theta},
\end{equation}
where
\begin{align}
Q(r,\theta)&= \frac{1}{{\rm i}\,n\,|\nabla r|^2},\label{eq}\\[0.5ex]
U(r,\theta)&= \frac{\alpha_p\,R^{\,2}}{|\nabla r|^2},\label{eu}\\[0.5ex]
T(r,\theta)&= \frac{r\,\nabla r\cdot\nabla\theta}{|\nabla r|^2}-\frac{\alpha_g}{{\rm i}\,n\,|\nabla r|^2}.\label{et}
\end{align}

Equation~(\ref{e69}) gives
\begin{equation}\label{e76}
b_r = A\left(\frac{\partial}{\partial\theta}-{\rm i}\,n\,q\right)y - B\,y + C\,\frac{\partial x}{\partial\theta},
\end{equation}
where
\begin{align}
A(r,\theta)&= \frac{1}{r\,R^{\,2}\,|\nabla r|^2}+\frac{\alpha_g}{{\rm i}\,n}\,\frac{\nabla r\cdot\nabla\theta}{|\nabla r|^2},\\[0.5ex]
B(r,\theta)&= \alpha_p\,\frac{R^{\,2}\,\nabla r\cdot\nabla\theta}{|\nabla r|^2},\\[0.5ex]
C(r,\theta)&= \frac{1}{{\rm i}\,n}\,\frac{\nabla r\cdot\nabla\theta}{|\nabla r|^2},
\end{align}
and use has been made of Eqs.~(\ref{e41}) and (\ref{e65}).

\subsection{First Outer-Region PDE}
According to Eq.~(\ref{e21}), 
\begin{equation}\label{divb}
\nabla \cdot\,{\bf b} =0,
\end{equation}
which implies that
\begin{equation}
r\,\frac{\partial}{\partial r}\!\left[\left(\frac{\partial}{\partial\theta}-{\rm i}\,n\,q\right)y\right]+\frac{\partial (r^2\,R^{\,2}\,b^{\,\theta})}{\partial \theta} - S\,x=0,
\end{equation}
where
\begin{equation}\label{s}
S(r) = {\rm i}\,n\,r^2,
\end{equation}
and use has been made of Eqs.~(\ref{jac}),  (\ref{e41}),  (\ref{e66}), and (\ref{div}).
Thus, employing Eq.~(\ref{e74}), we obtain the {\em first outer-region partial differential equation}\/ (PDE), \cite{connor}
\begin{equation}\label{efirst}
r\,\frac{\partial}{\partial r}\!\left[\left(\frac{\partial}{\partial\theta}-{\rm i}\,n\,q\right)y\right]=\frac{\partial}{\partial\theta}\!\left(Q\,\frac{\partial x}{\partial\theta}\right)
+S\,x -\frac{\partial}{\partial\theta}\!\left[T\left(\frac{\partial}{\partial\theta}-{\rm i}\,n\,q\right)y+U\,y\right].
\end{equation}

\subsection{Second Outer-Region PDE}
According to Eqs.~(\ref{e58}), (\ref{e59}), (\ref{z}), (\ref{e65}), and (\ref{e76}), 
\begin{align}
r\,R^{\,2}\,j^{\,\theta} &= -{\rm i}\,n\left[A\left(\frac{\partial}{\partial \theta}-{\rm i}\,n\,q\right)y-B\,y+C\,\frac{\partial x}{\partial\theta}\right]-\frac{\partial x}{\partial r},\\[0.5ex]
r\,R^{\,2}\,j^{\,\phi} &= \frac{\partial}{\partial r}\!\left[-\frac{\alpha_g}{{\rm i}\,n}\left(\frac{\partial}{\partial \theta}-{\rm i}\,n\,q\right)y+\alpha_p\,R^{\,2}\,y\right]-\frac{1}{{\rm i}\,n}\,\frac{\partial^2 x}{\partial r\,\partial\theta}\nonumber\\[0.5ex]&\phantom{=}-\frac{\partial}{\partial\theta}\!\left[A\left(\frac{\partial}{\partial \theta}-{\rm i}\,n\,q\right)y-B\,y+C\,\frac{\partial x}{\partial\theta}\right].
\end{align}
So,
\begin{align}
r\,R^{\,2}\,(q\,j^{\,\theta}-j^{\,\phi}) &=\left(\frac{\partial}{\partial\theta}-{\rm i}\,n\,q\right)\left[A\left(\frac{\partial}{\partial \theta}-{\rm i}\,n\,q\right)y-B\,y+C\,\frac{\partial x}{\partial\theta}\right]\nonumber\\[0.5ex]&\phantom= +\frac{1}{{\rm i}\,n}\left(\frac{\partial}{\partial\theta}-{\rm i}\,n\,q\right)\frac{\partial x}{\partial r} \nonumber\\[0.5ex]
&\phantom{=} -\frac{\partial}{\partial r}\!\left[-\frac{\alpha_g}{{\rm i}\,n}\left(\frac{\partial}{\partial \theta}-{\rm i}\,n\,q\right)y+\alpha_p\,R^{\,2}\,y\right].
\end{align}
Thus, Eq.~(\ref{e51}) gives
\begin{align}
-\frac{r\,R^{\,2}}{f}\frac{\partial}{\partial r}\!\left(\frac{f}{r}\,\alpha_p\,y\right)&=\left(\frac{\partial}{\partial\theta}-{\rm i}\,n\,q\right)\left[A\left(\frac{\partial}{\partial \theta}-{\rm i}\,n\,q\right)y-B\,y+C\,\frac{\partial x}{\partial\theta}\right]\nonumber\\[0.5ex]&\phantom= +\frac{1}{{\rm i}\,n}\left(\frac{\partial}{\partial\theta}-{\rm i}\,n\,q\right)\frac{\partial x}{\partial r} \nonumber\\[0.5ex]
&\phantom{=} -\frac{\partial}{\partial r}\!\left[-\frac{\alpha_g}{{\rm i}\,n}\left(\frac{\partial}{\partial \theta}-{\rm i}\,n\,q\right)y+\alpha_p\,R^{\,2}\,y\right]-r\,\alpha_g\,x\nonumber\\[0.5ex]
&\phantom{=} +\frac{1}{r}\left(q\,\alpha_g + R^{\,2}\,\alpha_p\right) \left[T\left(\frac{\partial}{\partial\theta}-{\rm i}\,n\,q\right)y + U\,y -Q\,\frac{\partial x}{\partial\theta}\right],
\end{align}
where use has been made of Eqs.~(\ref{e42}), (\ref{ap}), (\ref{ag}), (\ref{e66}),  (\ref{e74}). The
previous equation reduces to
\begin{align}
-{\rm i}\,n\,\alpha_p\,\alpha_f\,R^{\,2}\,y&={\rm i}\,n\left(\frac{\partial}{\partial\theta}-{\rm i}\,n\,q\right)\left[r\,A\left(\frac{\partial}{\partial \theta}-{\rm i}\,n\,q\right)y-r\,B\,y+r\,C\,\frac{\partial x}{\partial\theta}\right]\nonumber\\[0.5ex]&\phantom= +\left(\frac{\partial}{\partial\theta}-{\rm i}\,n\,q\right)r\,\frac{\partial x}{\partial r}  +r\,\alpha_g'\left(\frac{\partial}{\partial\theta}-{\rm i}\,n\,q\right)y \nonumber\\[0.5ex]
&\phantom{=}+ \alpha_g\,r\,\frac{\partial}{\partial r}\!\left[\left(\frac{\partial}{\partial\theta}-{\rm i}\,n\,q\right)y\right]
-{\rm i}\,n\,r\,\frac{\partial R^{\,2}}{\partial r}\,\alpha_p\,y-\alpha_g\,S\,x\nonumber\\[0.5ex]
&\phantom{=} +{\rm i}\,n\left(q\,\alpha_g + R^{\,2}\,\alpha_p\right) \left[T\left(\frac{\partial}{\partial\theta}-{\rm i}\,n\,q\right)y + U\,y -Q\,\frac{\partial x}{\partial\theta}\right],
\end{align}
where
\begin{equation}\label{af}
\alpha_f(r) = \frac{r^2}{f}\,\frac{d}{dr}\!\left(\frac{f}{r}\right),
\end{equation}
and use has been made of Eq.~(\ref{s}). 
Employing Eq.~(\ref{efirst}), we obtain
\begin{align}
-{\rm i}\,n\,\alpha_p\,\alpha_f\,R^{\,2}\,y&={\rm i}\,n\left(\frac{\partial}{\partial\theta}-{\rm i}\,n\,q\right)\left[r\,A\left(\frac{\partial}{\partial \theta}-{\rm i}\,n\,q\right)y-r\,B\,y+r\,C\,\frac{\partial x}{\partial\theta}\right]\nonumber\\[0.5ex]&\phantom= +\left(\frac{\partial}{\partial\theta}-{\rm i}\,n\,q\right)r\,\frac{\partial x}{\partial r}  +r\,\alpha_g'\left(\frac{\partial}{\partial\theta}-{\rm i}\,n\,q\right)y \nonumber\\[0.5ex]
&\phantom{=}+ \alpha_g\,\frac{\partial}{\partial\theta}\!\left(Q\,\frac{\partial x}{\partial\theta}\right)
+\alpha_g\,S\,x -\alpha_g\,\frac{\partial}{\partial\theta}\!\left[T\left(\frac{\partial}{\partial\theta}-{\rm i}\,n\,q\right)y+U\,y\right]\nonumber\\[0.5ex]&\phantom{=}
-{\rm i}\,n\,r\,\frac{\partial R^{\,2}}{\partial r}\,\alpha_p\,y-\alpha_g\,S\,x\nonumber\\[0.5ex]
&\phantom{=} +{\rm i}\,n\left(q\,\alpha_g + R^{\,2}\,\alpha_p\right) \left[T\left(\frac{\partial}{\partial\theta}-{\rm i}\,n\,q\right)y + U\,y -Q\,\frac{\partial x}{\partial\theta}\right],
\end{align}
which yields
\begin{align}
-{\rm i}\,n\,\alpha_p\,\alpha_f\,R^{\,2}\,y&={\rm i}\,n\left(\frac{\partial}{\partial\theta}-{\rm i}\,n\,q\right)\left[r\,A\left(\frac{\partial}{\partial \theta}-{\rm i}\,n\,q\right)y-r\,B\,y+r\,C\,\frac{\partial x}{\partial\theta}\right]\nonumber\\[0.5ex]&\phantom= +\left(\frac{\partial}{\partial\theta}-{\rm i}\,n\,q\right)r\,\frac{\partial x}{\partial r}  +r\,\alpha_g'\left(\frac{\partial}{\partial\theta}-{\rm i}\,n\,q\right)y \nonumber\\[0.5ex]
&\phantom{=}+ \alpha_g\left(\frac{\partial}{\partial\theta}-{\rm i}\,n\,q\right)\left[Q\,\frac{\partial x}{\partial\theta}
-T\left(\frac{\partial}{\partial\theta}-{\rm i}\,n\,q\right)y-U\,y\right]\nonumber\\[0.5ex]&\phantom{=}
-{\rm i}\,n\,r\,\frac{\partial R^{\,2}}{\partial r}\,\alpha_p\,y+{\rm i}\,n\,R^{\,2}\,\alpha_p \left[T\left(\frac{\partial}{\partial\theta}-{\rm i}\,n\,q\right)y + U\,y -Q\,\frac{\partial x}{\partial\theta}\right],
\end{align}
which reduces to the {\em second outer-region PDE},\cite{connor}
\begin{align}\label{esecond}
\left(\frac{\partial}{\partial\theta}-{\rm i}\,n\,q\right)r\,\frac{\partial x}{\partial r} &= -\left(\frac{\partial}{\partial\theta}-{\rm i}\,n\,q\right)T^\ast\,\frac{\partial x}{\partial \theta} + U\,\frac{\partial x}{\partial\theta}+X\,y\nonumber\\[0.5ex]&\phantom{=} -\left(\frac{\partial}{\partial\theta}-{\rm i}\,n\,q\right)V\left(\frac{\partial}{\partial\theta}-{\rm i}\,n\,q\right)y + W\left(\frac{\partial}{\partial\theta}-{\rm i}\,n\,q\right)y,
\end{align}
where
\begin{align}
V(r,\theta)&= \frac{1}{|\nabla r|^2}\left(\frac{{\rm i}\,n}{R^{\,2}}+\frac{\alpha_g^{\,2}}{{\rm i}\,n}\right),\label{ev}\\[0.5ex]
W(r,\theta)&= \frac{2\,\alpha_g\,\alpha_p\,R^{\,2}}{|\nabla r|^2} -r\,\alpha_g',\label{ew}\\[0.5ex]
X(r,\theta)&= {\rm i}\,n\,\alpha_p\left[\frac{\partial}{\partial \theta}(T^\ast\,R^{\,2})+r\,\frac{\partial R^{\,2}}{\partial r}-\alpha_f\,R^{\,2}-U\,R^{\,2}\right],\label{ex}
\end{align}
and $\phantom{!}^\ast$ denotes a complex conjugate. 

\section{Outer-Region ODEs}\label{sode}
\subsection{Primitive Outer-Region ODEs}\label{ode1}
Let
\begin{equation}\label{e79o}
x(r,\theta) = n\,z(r,\theta),
\end{equation}
and
let us express $y(r,\theta)$ and $z(r,\theta)$ as  Fourier series in the poloidal angle, $\theta$:
\begin{align}
y(r,\theta)&= \sum_{m}y_m(r)\,\exp(\,{\rm i}\,m\,\theta),\\[0.5ex]
z(r,\theta) &= \sum_{m}z_m(r)\,\exp(\,{\rm i}\,m\,\theta),\label{e97}
\end{align}
Here, the  (not necessarily positive) integers $m$ are the  poloidal mode numbers of the coupled Fourier harmonics included in the calculation. 
The outer-region PDEs, (\ref{efirst}) and (\ref{esecond}),  reduce to the {\em primitive outer-region ordinary differential equations}\/ 
(ODEs),\cite{connor,am1,am3}
\begin{align}\label{e41x}
r\,\frac{d}{dr}\left[(m-n\,q)\,y_m\right]&= \sum_{m'}\left(A_m^{\,m'}\,z_{m'}+B_m^{\,m'}\,y_{m'}\right),\\[0.5ex]
(m-n\,q)\,r\,\frac{dz_m}{dr} &= \sum_{m'}\left(C_{m}^{\,m'}\,z_{m'}+D_m^{\,m'}\,y_{m'}\right),\label{e42x}
\end{align}
where
\begin{align}
n^{-1}\,A_m^{\,m'}(r) &= \frac{1}{2\pi\,{\rm i}}\oint{\rm e}^{-{\rm i}\,m\,\theta}\left(\frac{\partial}{\partial\theta}\,Q\,\frac{\partial}{\partial\theta}+S\right){\rm e}^{\,{\rm i}\,m'\,\theta}\,d\theta,\\[0.5ex]
B_m^{\,m'}(r) &= \frac{1}{2\pi\,{\rm i}}\oint{\rm e}^{-{\rm i}\,m\,\theta}\left[-\frac{\partial}{\partial\theta}\,T\left(\frac{\partial}{\partial\theta}-{\rm i}\,n\,q\right)-\frac{\partial U}{\partial\theta}\right]{\rm e}^{\,{\rm i}\,m'\,\theta}\,d\theta,\\[0.5ex]
C_m^{\,m'}(r) &= \frac{1}{2\pi\,{\rm i}}\oint{\rm e}^{-{\rm i}\,m\,\theta}\left[-\left(\frac{\partial}{\partial\theta}-{\rm i}\,n\,q\right)T^\ast\,\frac{\partial}{\partial\theta}+U\,\frac{\partial}{\partial\theta}\right]{\rm e}^{\,{\rm i}\,m'\,\theta}\,d\theta,\\[0.5ex]
n\,D_m^{\,m'}(r)&= \frac{1}{2\pi\,{\rm i}}\oint{\rm e}^{-{\rm i}\,m\,\theta}\left[-\left(\frac{\partial}{\partial\theta}-{\rm i}\,n\,q\right)V\left(\frac{\partial}{\partial\theta}-{\rm i}\,n\,q\right)
\right. \nonumber\\[0.5ex]&\phantom{=}\left.+W\left(\frac{\partial}{\partial\theta}-{\rm i}\,n\,q\right)+ X\right]{\rm e}^{\,{\rm i}\,m'\,\theta}\,d\theta.
\end{align}
Hence, it follows from Eqs.~(\ref{eq})--(\ref{et}), (\ref{s}), and (\ref{ev})--(\ref{ex}) that\,\cite{am3}
\begin{align}\label{e104}
A_m^{\,m'}&= m\,m'\,c_m^{\,m'} + n^2\,r^2\,\delta_m^{\,m'}\\[0.5ex]
B_m^{\,m'}&= m\,(m'-n\,q)\left(-f_m^{\,m'}+n^{-1}\,\alpha_g\,c_m^{\,m'}\right) -m\,\alpha_p\,d_m^{\,m'},\\[0.5ex]
C_m^{\,m'}&= -(m-n\,q)\,m'\left(f_m^{\,m'}+n^{-1}\,\alpha_g\,c_m^{\,m'}\right)+m'\,\alpha_p\,d_m^{\,m'},\\[0.5ex]
D_m^{\,m'}&= (m-n\,q)\,(m'-n\,q)\left(b_m^{\,m'}-n^{-2}\,\alpha_g^{\,2}\,c_m^{\,m'}\right) - (m-n\,q)\,n^{-1}\,r\,\alpha_g'\,\delta_m^{\,m'}\label{e107}\\[0.5ex]
&\phantom{=} + 
\alpha_p\left[(m-m')\,g_m^{\,m'}+n^{-1}\,\alpha_g\,(m+m'-2\,n\,q)\,d_m^{\,m'} + r\,\frac{d a_m^{\,m'}}{dr}-\alpha_f\,a_m^{\,m'}
-\alpha_p\,e_m^{\,m'}\right],\nonumber
\end{align}
where
\begin{align}\label{e108}
a_m^{\,m'}(r) &= \oint R^{\,2}\,\exp[-{\rm i}\,(m-m')\,\theta]\,\frac{d\theta}{2\pi},\\[0.5ex]
b_m^{\,m'}(r) &= \oint |\nabla r|^{-2}\,R^{\,-2}\,\exp[-{\rm i}\,(m-m')\,\theta]\,\frac{d\theta}{2\pi},\\[0.5ex]
c_m^{\,m'}(r) &= \oint |\nabla r|^{-2}\,\exp[-{\rm i}\,(m-m')\,\theta]\,\frac{d\theta}{2\pi},\\[0.5ex]
d_m^{\,m'}(r) &= \oint |\nabla r|^{-2}\,R^{\,2}\,\exp[-{\rm i}\,(m-m')\,\theta]\,\frac{d\theta}{2\pi},\\[0.5ex]
e_m^{\,m'}(r) &= \oint |\nabla r|^{-2}\,R^{\,4}\,\exp[-{\rm i}\,(m-m')\,\theta]\,\frac{d\theta}{2\pi},\\[0.5ex]
f_m^{\,m'}(r) &= \oint \frac{{\rm i}\,r\,\nabla r\cdot\nabla\theta}{|\nabla r|^2}\,\exp[-{\rm i}\,(m-m')\,\theta]\,\frac{d\theta}{2\pi},\label{fdef}\\[0.5ex]
g_m^{\,m'}(r) &= \oint \frac{{\rm i}\,r\,\nabla r\cdot\nabla\theta}{|\nabla r|^2}\,R^{\,2}\,\exp[-{\rm i}\,(m-m')\,\theta]\,\frac{d\theta}{2\pi}.\label{e114}
\end{align}
Here, we have extended the analysis of Ref.~\onlinecite{am3} to take into account the fact that the $A_m^{\,m'}$, $B_m^{\,m'}$, $a_m^{\,m'}$, $b_m^{\,m'}$, et cetera,
are complex quantities in a realistic non-up-down-symmetric plasma equilibrium. 

\subsection{Outer-Region ODEs}\label{ode2}
Let 
\begin{align}
y_m(r) &= \frac{\psi_m(r)}{m-n\,q},\label{e115}\\[0.5ex]
z_m(r) &= \frac{Z_m(r)+k_j\,\psi_m(r)}{m-n\,q},\label{Zdef}
\end{align}
where
\begin{equation}\label{e117}
k_m(r) = -{\rm Re}\left(\frac{B_m^{\,m}}{A_m^{\,m}}\right) = -\left[\frac{m\,(m-n\,q)\,
n^{-1}\,\alpha_g\,c_m^{\,m} - m\,\alpha_p\,d_m^{\,m}}{m^2\,c_m^{\,m}+n^2\,r^2}\right].
\end{equation}
Here, we have made use of the fact that $f_m^{\,m}$ is imaginary. [See Eq.~(\ref{fdef}).] 
It follows from Eq.~(\ref{e41}) that
\begin{equation}\label{psidef}
b^{\,r}(r,\theta)= {\rm i}\sum_{m}\frac{\psi_m(r)}{r\,R^{\,2}}\,\exp(\,{\rm i}\,m\,\theta). 
\end{equation}
Furthermore, Eqs.~(\ref{e41x}) and (\ref{e42x}) transform to give the  {\em outer-region ODEs},\cite{am1,am3}
\begin{align}\label{e61x}
r\,\frac{d\psi_m}{dr} &=\sum_{m'}\frac{L_m^{\,m'}\,Z_{m'}+M_m^{\,m'}\,\psi_{m'}}{m'-n\,q},\\[0.5ex]
(m-n\,q)\,r\,\frac{d}{dr}\!\left(\frac{Z_m}{m-n\,q}\right)&=\sum_{m'}\frac{N_m^{\,m'}\,Z_{m'}+P_m^{\,m'}\,\psi_{m'}}{m'-n\,q},\label{e62x}
\end{align}
where
\begin{align}\label{e121}
L_m^{\,m'}(r) &=A_m^{\,m'},\\[0.5ex]
M_m^{\,m'}(r)& = B_m^{\,m'}+k_{m'}\,L_m^{\,m'},\\[0.5ex]
N_m^{\,m'}(r)&= C_m^{\,m'}-k_m\,L_m^{\,m'},\\[0.5ex]
P_m^{\,m'}(r) &=D_m^{\,m'}+k_{m'}\,C_m^{\,m'}  -k_m\,M_m^{\,m'}-k_m\,n\,q\,s\,\delta_m^{\,m'} - (m-n\,q)\,r\,\frac{dk_{m}}{dr}\,\delta_m^{\,m'},\label{e124}
\end{align}
with 
\begin{equation}
s(r)=\frac{r\,q'}{q}.
\end{equation}
Note that
\begin{equation}
M_m^{\,m}=N_m^{\,m} = - m\,(m-n\,q)\,f_m^{\,m}.
\end{equation}

\subsection{Symmetry Properties}
Equations~(\ref{e108})--(\ref{e114}) imply that
$a_{m'}^{\,m}= a_m^{\,m'\ast}$,
$b_{m'}^{\,m}= b_m^{\,m'\ast}$, 
$c_{m'}^{\,m}= c_m^{\,m'\ast}$,
$d_{m'}^{\,m}= d_m^{\,m'\ast}$,
$e_{m'}^{\,m}= e_m^{\,m'\ast}$,
$f_{m'}^{\,m}= -f_m^{\,m'\ast}$,
$g_{m'}^{\,m}= -g_m^{\,m'\ast}$,
for all $m$, $m'$.
 Hence, Eqs.~(\ref{e104})--(\ref{e107}), Eqs.~(\ref{e117}), and (\ref{e121})--(\ref{e124}) give
\begin{align}
L_{m'}^{\,m}&= L_m^{\,m'\ast},\label{e137}\\[0.5ex]
M_{m'}^{\,m}&=-N_m^{\,m'\ast},\\[0.5ex]
N_{m'}^{\,m}&=-M_m^{\,m'\ast},\label{e139}\\[0.5ex]
P_{m'}^{\,m}&= P_m^{\,m'\ast}.\label{e140}
\end{align}

\subsection{Toroidal Electromagnetic Torque}
The volume integrated toroidal electromagnetic torque acting between the magnetic axis and a magnetic flux-surface whose label is $r$ is given by
\begin{align}
T_\phi(r) &=\int_0^r\oint\oint R^{\,2}\,\nabla\phi\cdot({\bf J}+{\bf j})\times ({\bf B}\times {\bf b})\,{\cal J}\,d\tilde{r}\,d\theta\,d\phi\nonumber\\[0.5ex]
&=
\int_0^r\oint\oint\,  ({\bf j}\times {\bf b})_\phi\,{\cal J}\,d\tilde{r}\,d\theta\,d\phi
\end{align}
Here, use has been made of Eq.~(\ref{e15c}), as well as the fact that $P=P(r)$. We have also taken into account that ${\bf b}$ and ${\bf j}$ vary with $\phi$ as
$\exp(-{\rm i}\,n\,\phi)$, whereas ${\bf B}$, ${\bf J}$, ${\cal J}$, and $|\nabla\phi|$ are independent of $\phi$. It is clear that the zeroth-order (in perturbed quantities) contribution
to $T_\phi$ is identically zero, whereas the first-order contributions average to zero, leaving only second-order (i.e., nonlinear in perturbed quantities) contributions. 
Making use of Sect.~\ref{s2}, as well as Eqs.~(\ref{jac}), (\ref{e23}), (\ref{e69}), (\ref{e58x}), and (\ref{divb}), we deduce that
\begin{align}
{\cal J}\,({\bf j}\times {\bf b})_\phi&=\frac{\partial}{\partial r}\!\left({\cal J}\,b_\phi\,b^{\,r}\right)+\frac{\partial}{\partial\theta}\!\left({\cal J}\,b_\phi\,b^{\,\theta}\right)+\frac{\partial}{\partial \phi}\!\left({\cal J}\,b_\phi\,b^{\,\phi}\right)\nonumber\\[0.5ex]
&\phantom{=}-\frac{1}{2}\,\frac{\partial}{\partial\phi}\!\left[{\cal J}\left(b_r\,b^{\,r} + b_\theta\,b^{\,\theta}+b_\phi\,b^{\,\phi}\right)\right].
\end{align}
Hence, we obtain 
\begin{equation}\label{torque}
T_\phi(r)=\oint\oint {\cal J}\,b_\phi\,b^{\,r}\,d\theta\,d\phi=  r\oint\oint R^{\,2}\,b_\phi\,b^{\,r}\,d\theta\,d\phi,
\end{equation}
where the integral on the right-hand side is evaluated on the magnetic flux-surface whose label is $r$. We can reinterpret the
previous expression as specifying the net outward flux of toroidal electromagnetic angular momentum across the magnetic flux-surface whose label is $r$. 
Finally, making use of Eqs.~(\ref{z}), (\ref{e79o}), (\ref{e97}), and (\ref{Zdef})--(\ref{psidef}), the previous expression reduces to\,\cite{am1}
\begin{align}\label{etorque}
T_\phi(r) &= {\rm i}\,\pi^2\,n\sum_{m}\frac{Z_m^{\,\ast}\,\psi_m-\psi_m^{\,\ast}\,Z_m}{m-n\,q}.
\end{align}
 
 It follows from Eqs.~(\ref{e61x}), (\ref{e62x}) and (\ref{e137})--(\ref{e140}) that
\begin{equation}\label{e141c}
r\,\frac{d}{dr}\!\left(\sum_{m} \frac{Z_m^{\,\ast}\,\psi_m-\psi_m^{\,\ast}\,Z_m}{m-n\,q}\right)= 0.
\end{equation}
Hence, we deduce that\,\cite{am1}
\begin{equation}\label{etcons}
\frac{dT_\phi}{dr}=0
\end{equation}
in any region of the plasma that satisfies the outer-region ODEs. Thus, the volume integrated toroidal electromagnetic torque is
constant between rational magnetic flux-surfaces. As will become apparent in Sect.~\ref{sa7}, the integrated torque can have discontinuous jumps across rational flux-surfaces. It follows that
net electromagnetic torques can only develop in the plasma in the immediate vicinity of rational magnetic flux-surfaces, where the ideal-MHD equations  become singular. \cite{rfa}

\section{Behavior in Vicinity of Rational Surface}\label{snus}
\subsection{Introduction}
The analysis of this section is a generalization of the analysis of Ref.~\onlinecite{am3} that takes into
account the fact that the $L_m^{\,m'}$, $M_m^{\,m'}$, et cetera, are complex quantities in a realistic non-up-down-symmetric tokamak
plasma equilibrium.

Let there be $K$ rational magnetic flux-surfaces in the plasma. Suppose that the $k$th surface lies at $r=r_k$, and possesses the resonant
poloidal mode number $m_k$, where $q(r_k)=m_k/n$.   

\subsection{General Case}\label{sgen}
Consider the solution of the outer-region ODEs, (\ref{e61x}) and (\ref{e62x}), in the
vicinity of the $k$th rational surface. Let  $x=r-r_k$.  The most general small-$|x|$ solution of the ODEs
can be shown to take the form\,\cite{am1,am3}
\begin{align}
\psi_{m_k}(r_k+x)&=A_{L\,k}^\pm \,|x|^{\nu_{L\,k}}\,(1+\lambda_{L}\,x+\cdots) + A_{S\,k}^{\pm}\,{\rm sgn}(x)\,|x|^{\nu_{S\,k}}\,(1+\cdots) \nonumber\\[0.5ex]
& \phantom{=}+ A_{C}\,x\,(1+\cdots),\label{ex1}\\[0.5ex]
Z_{m_k}(r_k+x)&= A_{L\,k}^\pm\,|x|^{\nu_{L\,k}}(b_{L}  + \gamma_{L}\,x+\cdots) + A_{S\,k}^{\pm}\,{\rm sgn}(x)\,|x|^{\nu_{S\,k}}\,(b_{S}+\cdots)\nonumber\\[0.5ex]
& \phantom{=}+ B_{C}\,x\,(1+\cdots),\label{ex2}
\end{align}
and 
\begin{align}
\psi_{m_k+j}(r_k+x)&=A_{L\,k}^\pm\,|x|^{\nu_{L\,k}}\,(a_j+c_j\,x+\cdots)  
+A_{S\,k}^\pm\,{\rm sgn}(x)\,|x|^{\nu_{S\,k}}\,(\tilde{a}_{j}+\cdots)\nonumber\\[0.5ex]
&\phantom{=}+ (\bar{\psi}_{m_k+j}+\bar{\psi}_{m_j+k}'\,x+\cdots),\\[0.5ex]
Z_{m_k+j}(r_k+x)&= A_{L\,k}^\pm\,|x|^{\nu_{L\,k}}\,(b_j+d_j\,x+\cdots) +A_{S\,k}^\pm\,{\rm sgn}(x)\,|x|^{\nu_{S\,k}}\,(\tilde{b}_{j}+\cdots)\nonumber\\[0.5ex]
&\phantom{=}+(\bar{Z}_{m_k+j}\,+\bar{Z}_{m_k+j}'\,x+\cdots).\label{ex4}
\end{align}
The superscripts $\phantom{!}^+$ and $\phantom{!}^-$ correspond  to $x>0$ and $x<0$, respectively. Here, $A_{L\,k}$ is known as the ``coefficient of
the large solution,'' whereas $A_{S\,k}$ is termed the ``coefficient of the small solution.''\cite{am1,am3,ggj}
Moreover, 
\begin{align}
\nu_{L\,k}&= \frac{1}{2}-\sqrt{-D_{I\,k}},\label{nul}\\[0.5ex]
\nu_{S\,k} &=  \frac{1}{2}+\sqrt{-D_{I\,k}},\label{nus}\\[0.5ex]
D_{I\,k} &= -L_0\,P_0-\frac{1}{4},\\[0.5ex]
L_0 &=-\left(\frac{L_{m_k}^{\,m_k}}{m_k\,s}\right)_{r_k},\label{lkk}\\[0.5ex]
P_0 &= -\left(\frac{P_{m_k}^{\,m_k}}{m_k\,s}\right)_{r_k}.\label{pkk}
\end{align}
Note that, ordinarily, $\nu_{L\,k}$, $\nu_{S\,k}$, $D_{I\,k}$, $L_0$ and $P_0$ are all real quantities. 
Furthermore, 
\begin{align}
b_{L} &= \frac{\nu_{L\,k}}{L_0},\label{ebl}\\[0.5ex]
b_{S} &= \frac{\nu_{S\,k}}{L_0},\label{ebs}\\[0.5ex]
A_{C} &= - \frac{1}{r_k\,P_0}\sum_{j\neq 0}\frac{1}{j}\left(N_{m_k}^{\,m_k+j}\,\bar{Z}_{m_k+j}+ P_{m_k}^{\,m_k+j}\,\bar{\psi}_{m_k+j}\right)_{r_k},\\[0.5ex]
B_{C} &= - \frac{1}{r_k\,L_0}\sum_{j\neq 0}\frac{1}{j}\left(L_{m_k}^{\,m_k+j}\,\bar{Z}_{m_k+j}+ M_{m_k}^{\,m_k+j}\,\bar{\psi}_{m_k+j}\right)_{r_k}+\frac{A_{C}}{L_0},\\[0.5ex]
\lambda_{L} &= \frac{1}{2\,r_k}\left[\frac{P_1\,L_0}{\nu_{L\,k}} + T_1 + \nu_{L\,k}\left(\frac{L_1}{L_0}-2\right)+2\,M_1\right]_{r_k}\\[0.5ex]&
\phantom{=}-\frac{1}{2\,(m_k\,s)_{r_k}}\,\frac{1}{r_k\,\nu_{L\,k}}\sum_{j\neq0}\frac{1}{j}\left[L_{m_k}^{\,m_k+j}\,P_{m_k+j}^{\,m_k}+P_{m_k}^{\,m_k+j}\,L_{m_k+j}^{\,m_k} +
M_{m_k}^{\,m_k+j}\,M_{m_k+j}^{\,m_k}+N_{m_k}^{\,m_k+j}\,N_{m_k+j}^{\,m_k}\phantom{\frac{a}{b_L}}\right.\nonumber\\[0.5ex]&
\phantom{=}\left. + b_{L}\,(L_{m_k}^{\,{m_k}+j}\,N_{m_k+j}^{\,m_k}+M_{m_k}^{\,m_k+j}\,L_{m_k+j}^{\,m_k})+\frac{1}{b_{L}}\,(N_{m_k}^{\,m_k+j}\,P_{m_k+j}^{\,m_k}+ P_{m_k}^{\,m_k+j}\,M_{m_k+j}^{\,m_k})\right]_{r_k},\nonumber\\[0.5ex]
\gamma_{L} &=\frac{1}{2\,r_k}\left[(1+\nu_{L\,k})\left(\frac{P_1}{\nu_{L\,k}}+\frac{T_1}{L_0}-\frac{\nu_{L\,k}}{L_0}\right)+
P_0\left(\frac{L_1}{L_0}-1\right)+2\,b_{L}\,M_1\right]_{r_k}\nonumber\\[0.5ex]
&\phantom{=}-\frac{1}{2\,(m_k\,s)_{r_k}}\,\frac{1}{r_k\,\nu_{L\,k}\,L_0}\sum_{j\neq 0}\frac{1}{j}\left[(\nu_{L\,k}+1)\,(P_{m_k}^{\,m_k+j}\,L_{m_k+j}^{\,m_k}+N_{m_k}^{\,m_k+j}\,N_{m_k+j}^{\,m_k})\phantom{\frac{a}{b_L}}\right.\nonumber\\[0.5ex]
&\phantom{=} +(\nu_{L\,k}-1)\,(L_{m_k}^{\,m_k+j}\,P_{m_k+j}^{\,m_k}+M_{m_k}^{\,m_k+j}\,M_{m_k+j}^{\,m_k})+b_{L}\,(\nu_{L\,k}-1)\,(L_{m_k}^{\,m_k+j}\,N_{m_k+j}^{\,m_k}+M_{m_k}^{\,m_k+j}\,L_{m_k+j}^{\,m_k}) \nonumber\\[0.5ex]
&\phantom{=} \left. + \frac{1}{b_{L}}\left(\nu_{L\,k}+1)\,(N_{m_k}^{\,m_k+j}\,P_{m_k+j}^{\,m_k}+P_{m_k}^{\,m_k+j}\,M_{m_k+j}^{\,m_k}\right)\right]_{r_k},\\[0.5ex]
a_j&=-\frac{1}{(m_k\,s)_{r_k}}\left(\frac{L^{\,m_k}_{m_k+j}}{L_0}+\frac{M^{\,m_k}_{m_k+j}}{\nu_{L\,k}}\right)_{r_k},\label{e135}\\[0.5ex]
b_j&=- \frac{1}{(m_k\,s)_{r_k}}\left(\frac{P^{\,m_k}_{m_k+j}}{\nu_{L\,k}}+\frac{N^{\,m_k}_{m_k+j}}{L_0}\right)_{r_k},\\[0.5ex]
\tilde{a}_{j}&= -\frac{1}{(m_k\,s)_{r_k}}\left(\frac{L^{\,m_k}_{m_k+j}}{L_0}+\frac{M^{\,m_k}_{m_k+j}}{\nu_{S\,k}}\right)_{r_k},\\[0.5ex]
\tilde{b}_{j}&=- \frac{1}{(m_k\,s)_{r_k}}\left(\frac{P^{\,m_k}_{m_k+j}}{\nu_{S\,k}}+\frac{N^{\,m_k}_{m_k+j}}{L_0}\right)_{r_k},\label{e138}\\[0.5ex]
c_j&= \frac{1}{(1+\nu_{L\,k})\,r_k}\left[-\nu_{L\,k}\,a_j+L_{j\,1}\,b_{L}+M_{j\,1}\phantom{\frac{1}{j}}\right.\nonumber\\[0.5ex]
&\phantom{=}-
\frac{r_k}{m_k\,s}\left(L_{m_k+j}^{\,m_k}\,\gamma_L+M_{m_k+j}^{\,m_k}\,\lambda_L\right) +\left.\sum_{j'\neq 0} \frac{1}{j'}\left(L_{m_k+j}^{\,m_k+j'}\,b_{j'}+ M_{m_k+j}^{\,m_k+j'}\,a_{j'}\right)\right]_{r_k},\\[0.5ex]
d_j&= \frac{1}{(1+\nu_{L\,k})\,r_k}\left[-\left(\nu_{L\,k}+\frac{m_k\,s}{j}\right)b_j+N_{j\,1}\,b_{L}+P_{j\,1}\right.\nonumber\\[0.5ex]
&\phantom{=}-
\frac{r_k}{m_k\,s}\left(N_{m_k+j}^{\,m_k}\,\gamma_L+P_{m_k+j}^{\,m_k}\,\lambda_L\right) +\left.\sum_{j'\neq 0} \frac{1}{j'}\left(N_{m_k+j}^{\,m_k+j'}\,b_{j'}+ P_{m_k+j}^{\,m_k+j'}\,a_{j'}\right)\right]_{r_k},\\[0.5ex]
\bar{\psi}_{m_k+j}' &= \frac{1}{r_k}\left[-\frac{r_k}{m_k\,s}\left(L_{m_k+j}^{\,m_k}\,B_C+ M_{m_k+j}^{m_k}\,A_C\right)\right.\nonumber\\[0.5ex]&\phantom{=}\left.
+ \sum_{j'\neq 0} \frac{1}{j'}\left(L_{m_k+j}^{\,m_k+j'}\,\bar{Z}_{m_k+j'} +M_{m_k+j}^{\,m_k+j'}\,\bar{\psi}_{m_k+j'}\right)\right]_{r_k},\\[0.5ex]
\bar{Z}_{m_k+j}' &= \frac{1}{r_k}\left[-\frac{m_k\,s}{j}\,\bar{Z}_{m_k+j}-\frac{r_k}{m_k\,s}\left(N_{m_k+j}^{\,m_k}\,B_C+ P_{m_k+j}^{m_k}\,A_C\right)\right.\nonumber\\[0.5ex]&\phantom{=}\left.+ 
\sum_{j'\neq 0} \frac{1}{j'}\left(N_{m_k+j}^{\,m_k+j'}\,\bar{Z}_{m_k+j'} +P_{m_k+j}^{\,m_k+j'}\,\bar{\psi}_{m_k+j'}\right)\right]_{r_k},
\end{align}
and
\begin{align}
L_1&= \lim_{x\rightarrow 0}\left(\frac{L_{m_k}^{\,m_k}}{m_k-n\,q}-\frac{r_k\,L_0}{x}\right),\\[0.5ex]
P_1&=  \lim_{x\rightarrow 0}\left(\frac{P_{m_k}^{\,m_k}}{m_k-n\,q}-\frac{r_k\,P_0}{x}\right),\\[0.5ex]
T_1 &= \lim_{x\rightarrow 0}\left(\frac{-n\,q\,s}{m_k-n\,q}-\frac{r_k}{x}\right),\\[0.5ex]
M_1 &= \lim_{x\rightarrow 0}\left(\frac{M_{m_k}^{\,m_k}}{m_k-n\,q}\right),\\[0.5ex]
L_{j\,1}&=\lim_{x\rightarrow 0}\left(\frac{L_{m_k+j}^{\,m_k}}{m_k-n\,q}+\frac{r_k}{m_k\,s}\,\frac{L_{m_k+j}^{\,m_k}}{x}\right),\\[0.5ex]
M_{j\,1}&=\lim_{x\rightarrow 0}\left(\frac{M_{m_k+j}^{\,m_k}}{m_k-n\,q}+\frac{r_k}{m_k\,s}\,\frac{M_{m_k+j}^{\,m_k}}{x}\right),\\[0.5ex]
N_{j\,1}&=\lim_{x\rightarrow 0}\left(\frac{N_{m_k+j}^{\,m_k}}{m_k-n\,q}+\frac{r_k}{m_k\,s}\,\frac{N_{m_k+j}^{\,m_k}}{x}\right),\\[0.5ex]
P_{j\,1}&=\lim_{x\rightarrow 0}\left(\frac{P_{m_k+j}^{\,m_k}}{m_k-n\,q}+\frac{r_k}{m_k\,s}\,\frac{P_{m_k+j}^{\,m_k}}{x}\right),
\end{align}
where $j\neq 0$. 

The coefficients of the large and the small solutions at the $k$th rational surface are evaluated as follows:
\begin{align}
\bar{\psi}_{m_k+j} &= \psi_{m_k+j}(r_k+\delta)- (a_j+\delta\,c_j)\,A_{L\,k}\,|\delta|^{\,\nu_{L\,k}}-\tilde{a}_j\,{\rm sgn}(\delta)\,|\delta|^{\nu_{S\,k}}\,A_{S\,k}\nonumber\\[0.5ex]
&\phantom{=}-\bar{\psi}'_{m_k+j}\,\delta+{\cal O}(\delta^2),\\[0.5ex]
\bar{Z}_{m_k+j}&= Z_{m_k+j}(r_k+\delta)- (b_j+\delta\,d_j)\,A_{L\,k}\,|\delta|^{\,\nu_{L\,k}}-\tilde{b}_j\,{\rm sgn}(\delta)\,|\delta|^{\nu_{S\,k}}\,A_{S\,k}\nonumber\\[0.5ex]
&\phantom{=}-\bar{Z}'_{m_k+j}\,\delta+{\cal O}(\delta^2),\\[0.5ex]
A_{S\,k} &= \frac{Z_{m_k}(r_k+\delta)-b_L\,\psi_{m_k}(r_k+\delta)-\delta\,(B_C-b_L\,A_C)-\delta\,(\gamma_L-b_L\,\lambda_L)\,A_{L\,k}\,|\delta|^{\,\nu_{L\,k}}}{
(b_S-b_L)\,{\rm sgn}(\delta)\,|\delta|^{\nu_{S\,k}}}\nonumber\\[0.5ex]
&\phantom{=}+{\cal O}(\delta),\\[0.5ex]
A_{L\,k} &= \frac{\psi_{m_k}(r_k+\delta)-A_{S\,k}\,{\rm sgn}(\delta)\,|\delta|^{\nu_{S\,k}}-A_C\,\delta}{(1+\delta\,\lambda_L)\,|\delta|^{\nu_{L\,k}}}+{\cal O}(\delta^2)
\end{align}
for $j\neq 0$. The previous set of equations can be solved via iteration.

The  analysis in this section is based on the assumption that $ D_{I\,k}< 0$.
 If $D_{I\,k}>0$ then the indices $\nu_{L\,k}$ and $\nu_{S\,k}$ become
complex, indicating that the plasma in the vicinity of the $k$th rational surface is unstable to localized ideal interchange modes.\cite{mercier}

\subsection{Special Case}\label{sspec}
In the limit $\nu_{L\,k}\rightarrow 0$, some of the previous expressions become singular, and a special treatment is required. The most general small-$|x|$ solution of the outer-region ODEs 
takes the form
\begin{align}
\psi_{m_k}(r_k+x)&=A_{L\,k}^\pm \,[1+\nu_{L\,k}\,\ln|x|+\hat{\lambda}_{L}\,x\,(\ln |x|-1)+\mu_{L}\,x\,(\ln^2\!|x|-2\,\ln|x|-2) + \xi_{L}\,x+\cdots]  \nonumber\\[0.5ex]&\phantom{=}+ A_{S\,k}^{\pm}\,x\,(1+\cdots) + \hat{A}_{C}\,x\,(1+\cdots)+ A_{D}\,x\,(\ln |x|-1 + \cdots),\\[0.5ex]
Z_{m_k}(r_k+x)&= A_{L\,k}^\pm\left(b_L+\hat{\gamma}_{L}\,x\,\ln |x|+\delta_{L}\,x\,\ln^2\!|x|+\cdots\right) +A_{S\,k}^\pm\,x\,(b_{S}+\cdots)\nonumber\\[0.5ex]&\phantom{=} + B_{D}\,x\,(\ln |x|+\cdots),
\end{align}
and 
\begin{align}
\psi_{m_k+j}(r_k+x)&=A_{L\,k}^\pm\,(\hat{a}_j\,\ln |x|+\cdots)+A_{S\,k}^\pm\,x\,(\tilde{a}_j+\cdots)  + \bar{\psi}_{m_k+j}\,(1+\cdots),\\[0.5ex]
Z_{m_k+j}(r_k+x)&= A_{L\,k}^\pm\,(\hat{b}_j \,\ln |x|+ \cdots) +  A_{S\,k}^\pm\,x\,(\tilde{b}_j + \cdots) +\bar{Z}_{m_k+j}\,(1+\cdots).
\end{align}
Here, 
\begin{align}
\hat{A}_{C} &= \frac{1}{r_k}\sum_{j\neq 0}\frac{1}{j}\left(L_{m_k}^{\,m_k+j}\,\bar{Z}_{m_k+j}+ M_{m_k}^{\,m_k+j}\,\bar{\psi}_{m_k+j}\right)_{r_k},\\[0.5ex]
A_{D} &=  \frac{L_0}{r_k}\sum_{j\neq 0}\frac{1}{j}\left(N_{m_k}^{\,m_k+j}\,\bar{Z}_{m_k+j}+ P_{m_k}^{\,m_k+j}\,\bar{\psi}_{m_k+j}\right)_{r_k}
\nonumber\\[0.5ex]&\phantom{=}-
\frac{\nu_{L\,k}}{r_k}\,\sum_{j\neq 0}\frac{1}{j}\left(L_{m_k}^{\,m_k+j}\,\bar{Z}_{m_k+j}+ M_{m_k}^{\,m_k+j}\,\bar{\psi}_{m_k+j}\right)_{r_k},
\\[0.5ex]
B_{D} &= \frac{A_{D}}{L_0},\\[0.5ex]
\hat{\lambda}_{L} &= \frac{P_1\,L_0\,(1+\nu_{L\,k}) }{r_k} + \frac{\nu_{L\,k}\,T_1}{r_k}
-\frac{1}{(m_k\,s)_{r_k}}\,\frac{1}{r_k}\sum_{j\neq 0}\frac{1}{j}\left(L_{m_k}^{\,m_k+j}\,P_{m_k+j}^{\,m_k}+M_{m_k}^{\,m_k+j}\,M_{m_k+j}^{m_k}\right)_{r_k}\nonumber\\[0.5ex]&\phantom{=}
- \frac{1}{(m_k\,s)_{r_k}}\,\frac{\nu_{L\,k}}{r_k}
\left(L_{m_k}^{\,m_k+j}\,N_{m_k+j}^{\,m_k}+M_{m_k}^{\,m_k+j}\,L_{m_k+j}^{m_k}\right)_{r_k},\\[0.5ex]
\mu_{L}&= -\frac{1}{2\,(m_k\,s)_{r_k}}\frac{L_0}{r_k}\sum_{j\neq 0}\frac{1}{j}\,(N_{m_k}^{\,m_k+j}\,P_{m_k+j}^{\,m_k}+P_{m_k}^{\,m_k+j}\,M_{m_k+j}^{\,m_k})_{r_k},\\[0.5ex]
\xi_{L} &= M_1+ \frac{\nu_{L\,k}}{r_k} \left(\frac{L_1}{L_0}-1\right),\\[0.5ex]
\hat{\gamma}_{L} &=\frac{P_1\,(1+\nu_{L\,k})}{L_0\,r_k} + \frac{\nu_{L\,k}\,T_1}{L_0\,r_k},\\[0.5ex]
\delta_{L} &= \frac{\mu_{L}}{L_0}.
\end{align}
Moreover, $\hat{a}_j$, $\hat{b}_j$, $\tilde{a}_j$ and $\tilde{b}_j$ are again specified by Eqs.~(\ref{e135})--(\ref{e138}). 

The coefficients of the large and the small solutions at the $k$th rational surface are evaluated as follows:
\begin{align}
\bar{\psi}_{m_k+j} &= \psi_{m_k+j}(r_k+\delta)- \hat{a}_j\,A_{L\,k}\,\ln|\delta| +{\cal O}(\delta),\\[0.5ex]
\bar{Z}_{m_k+j}&= Z_{m_k+j}(r_k+\delta)-\hat{b}_j\,A_{L\,k}\,\ln|\delta|+{\cal O}(\delta),\\[0.5ex]
A_{S\,k} &= \frac{Z_{m_k}(r_k+\delta)-b_{L}\,A_{L\,k}-\delta\,\ln|\delta|\,[B_D+(\hat{\gamma}_L+\delta_L\,\ln|\delta|)\,A_{L\,k}]}
{b_S\,\delta}+{\cal O}(\delta),\\[0.5ex]
A_{L\,k} &= \frac{\psi_{m_k}(r_k+\delta)-\delta\,[A_{S\,k}+A_C+A_D\,(\ln|\delta|-1)]}
{1+\nu_{L\,k}\,\ln|\delta|+\delta\,[\hat{\lambda}_L\,(\ln|\delta|-1)+ \mu_L\,(\ln^2|\delta|-2\,\ln|\delta|-2)+\xi_L]}+{\cal O}(\delta^{\,2})
\end{align}
for $j\neq 0$. As before, the previous equations can be solved via iteration.

\subsection{Asymptotic Matching Across  Rational Surfaces}\label{sa7}
Consider the resonant layer solution in the vicinity of the $k$th  rational surface, whose resonant poloidal mode number is $m_k$. 
This solution can be separated into independent tearing and twisting parity components.\cite{ggj}
The tearing parity   component is such that $\psi_{m_k}(r_k-x)=\psi_{m_k}(r_k+x)$ throughout the layer, whereas the  twisting parity component is such that
$\psi_{m_k}(r_k-x)=-\psi_{m_k}(r_k+x)$. It turns out, however, that the twisting parity response of a resonant layer to the solution in the outer region is generally negligible compared to the tearing parity response.\cite{connor,twist,am3}
Hence,  we shall neglect the twisting parity responses of the various resonant layers  in the plasma all together. 

The neglect of the twisting parity responses of the various resonant layers in the plasma implies that the coefficients of the large solution to the left and to the right of
each rational surface in the plasma are equal to one another.\cite{am1} In other words,
\begin{equation}\label{tear}
A_{L\,k}^- = A_{L\,k}^+= A_{L\,k}
\end{equation}
for all $k$. 
 Note, however, that the coefficients of the small solution to the left and to the right of a given rational surface are not, in general, equal to one another. 

Consider a solution that is completely continuous across the $k$th rational surface, so that $A_{S\,k}^-=A_{S\,k}^+$. According to the
preceding analysis, the continuity condition for the resonant harmonic can be written as
\begin{align}
\psi_{m_k}(r_k+|\delta|) &= \psi_{m_k}(r_k-|\delta|) + 2\,|\delta|\,[A_{L\,k}\,|\delta|^{\nu_{L\,k}}\,\lambda_L + A_C] + 2\,A_{S\,k}^-\,|\delta|^{\nu_{S\,k}}+ {\cal O}(\delta^{\,2}),\\[0.5ex]
Z_{m_k} (r_k+|\delta|) &=  Z_{m_k}(r_k-|\delta|) + 2\,|\delta|\,[A_{L\,k}\,|\delta|^{\nu_{L\,k}}\,\gamma_L + B_C] + 2\,A_{S\,k}^-\,b_S\,|\delta|^{\nu_{S\,k}} 
\nonumber\\[0.5ex]&\phantom{=}+ {\cal O}(\delta^{\,2}),\\[0.5ex]
\psi_{m_k+j}(r_k+|\delta|) &= \psi_{m_k+j}(r_k-|\delta|) +2\,|\delta|\,[A_{L\,k}\,|\delta|^{\nu_{L\,k}}\,c_j + \bar{\psi}_{m_k+j}'] + 2\,A_{S\,k}^-\,a_j\,|\delta|^{\nu_{S\,k}}\nonumber\\[0.5ex]&\phantom{=} +{\cal O}(\delta^{\,2}),\\[0.5ex]
Z_{m_k+j}(r_k+|\delta|) &= Z_{m_k+j}(r_k-|\delta|) +2\,|\delta|\,[A_{L\,k}\,|\delta|^{\nu_{L\,k}}\,d_j + \bar{Z}_{m_k+j}'] + 2\,A_{S\,k}^-\,b_j\,|\delta|^{\nu_{S\,k}}
\nonumber\\[0.5ex]&\phantom{=}+ {\cal O}(\delta^{\,2})
\end{align}
in the general case, and 
\begin{align}
\psi_{m_k}(r_k+|\delta|) &= \psi_{m_k}(r_k-|\delta|) \nonumber\\[0.5ex]
&\phantom{=}+ 2\,|\delta|\,\{A_{L\,k}\,[\hat{\lambda}_L\,(\ln|\delta|-1)+\hat{\mu}_L\,(\ln^2|\delta|-2\,\ln|\delta|-2)+ \xi_L]\nonumber\\[0.5ex]
&\phantom{=}+ A_C
+A_D\,(\ln|\delta|-1) + A_{S\,k}^-\}+ {\cal O}(\delta^{\,2}),\\[0.5ex]
Z_{m_k} (r_k+|\delta|) &=  Z_{m_k}(r_k-|\delta|) + 2\,|\delta|\,[A_{L\,k}\,\ln|\delta| \,(\hat{\gamma}_L+\delta_L\,\ln|\delta|)+ B_D\,\ln|\delta|
+A_{S\,k}^-\,b_{S}]\nonumber\\[0.5ex]&\phantom{=}+ {\cal O}(\delta^{\,2}),\\[0.5ex]
\psi_{m_k+j}(r_k+|\delta|) &= \psi_{m_k+j}(r_k-|\delta|)+{\cal O}(\delta),\\[0.5ex]
Z_{m_k+j}(r_k+|\delta|) &= Z_{m_k+j}(r_k-|\delta|)+{\cal O}(\delta)
\end{align}
in the special case. In both cases, $j\neq 0$.

Consider a solution that is launched from the $k$th rational surface, so that $A_{L\,k} = A_{S\,k}^-=0$. It follows from the preceding analysis that
\begin{align}
\psi_{m_k}(r_k+|\delta|) &= A_{S\,k}^+\,|\delta|^{\nu_{S\,k}}+ {\cal O}(\delta^{\,2}),\\[0.5ex]
Z_{m_k}(r_k+|\delta|) &= A_{S\,k}^+\,b_S\,|\delta|^{\nu_{S\,k}}+ {\cal O}(\delta^{\,2}),\\[0.5ex]
\psi_{m_k+j}(r_k+|\delta|)& = A_{S\,k}^+\,\tilde{a}_j\,|\delta|^{\nu_{S\,k}}+ {\cal O}(\delta^{\,2}),\\[0.5ex]
Z_{m_k+j}(r_k+|\delta|) &= A_{S\,k}^+\,\tilde{b}_j\,|\delta|^{\nu_{S\,k}}+ {\cal O}(\delta^{\,2})
\end{align}
for $j\neq 0$. 

It is helpful to define the quantities\,\cite{am1}
\begin{align}\label{Psidef}
{\mit\Psi}_k&= r_k^{\,\nu_{L\,k}}\left(\frac{\nu_{S\,k}-\nu_{L\,k}}{L_{m_k}^{\,{m_k}}}\right)^{1/2}_{r_k} A_{L\,k},\\[0.5ex]
{\mit\Delta\Psi}_k &= r_k^{\,\nu_{S\,k}}\left(\frac{\nu_{S\,k}-\nu_{L\,k}}{L_{m_k}^{\,m_k}}\right)^{1/2}_{r_k} (A_{S\,k}^+ - A_{S\,k}^-)\label{edpp}
\end{align}
at each rational surface in the plasma. Here, the complex parameter ${\mit\Psi}_k$ is a measure of the reconnected helical magnetic flux at the $k$th rational surface, whereas
the complex parameter ${\mit\Delta\Psi}_k$ is a measure of the strength of a localized current sheet that flows parallel to the equilibrium magnetic field at the surface. 
It is evident from Eqs.~(\ref{etorque}), (\ref{etcons}), (\ref{ex1}), (\ref{ex2}), (\ref{nul}), (\ref{nus}), (\ref{lkk}), (\ref{ebl}), (\ref{ebs}), and (\ref{tear})--(\ref{edpp}) that\,\cite{am1,am3}
\begin{equation}\label{e204z}
T_\phi(r) =\int_0^r \sum_{k=1,K}\delta T_k\,\delta(\tilde{r}-r_k)\,d\tilde{r},
\end{equation}
where
\begin{equation}\label{e194t}
\delta T_k = 2\pi^2\,n\,{\rm Im}({\mit\Psi}_k^\ast\,{\mit\Delta\Psi}_k).
\end{equation}
Here, $\delta T_k$ is the net toroidal electromagnetic torque exerted on the plasma in the immediate vicinity of the $k$th rational
surface. 

\section{Inverse Aspect-Ratio Expanded Tokamak Equilibrium}
\subsection{Equilibrium Magnetic Flux-Surfaces}
Let the plasma/vacuum interface  correspond to $r=\epsilon$, where $\epsilon\ll 1$ is the inverse aspect-ratio of the
plasma. In other words, $\epsilon=a/R_0$, where $a\ll R_0$ is the effective minor radius of the plasma. 
Let $r=\epsilon\,\hat{r}$, $\nabla =\epsilon^{\,-1}\,\hat{\nabla}$, and $'\rightarrow \epsilon^{\,-1}\,'$. 
Suppose that the loci of the equilibrium magnetic flux-surfaces can be written in the parametric form:\,\cite{con0,gim,am1,fitz2024}
\begin{align}
R(\hat{r},\omega) &= 1 -\epsilon\,\hat{r}\,\cos\omega + \epsilon^{\,2}\,\sum_{j>0}H_j(\hat{r})\,\cos[(j-1)\,\omega] + \epsilon^{\,2}\,\sum_{j>1}V_j(\hat{r})\,\sin[(j-1)\,\omega] \nonumber\\[0.5ex]
&\phantom{=}+\epsilon^{\,3}\,L(\hat{r})\,\cos\omega,\label{e19x}\\[0.5ex]
Z(\hat{r},\omega)&= \epsilon\,\hat{r}\,\sin\omega +\epsilon^{\,2}\,\sum_{j>1}H_j(\hat{r})\,\sin[(j-1)\,\omega]
-\epsilon^{\,2}\,\sum_{j>1}V_j(\hat{r})\,\cos[(j-1)\,\omega]\nonumber\\[0.5ex]&\phantom{=}-\epsilon^{\,3}\,L(\hat{r})\,\sin\omega,\label{e20x}
\end{align}
where $j$ is a positive integer. 
Here, $H_1(\hat{r})$  controls the relative horizontal locations of the flux-surface centroids, $H_2(\hat{r})$ and $V_2(\hat{r})$ control the 
magnitudes and vertical tilts of the flux-surface ellipticities, $H_3(\hat{r})$ and
$V_3(\hat{r})$ control the magnitudes and vertical tilts of the flux-surface triangularities, et cetera, whereas $L(\hat{r})$ is a
re-labelling parameter. Moreover, $\omega(R,Z)$ is a  poloidal angle that is distinct from $\theta$. Note that $V_1$ does not appear in Eq.~(\ref{e20x})
because such a factor merely gives rise to a rigid vertical shift of the plasma that can be eliminated by a suitable choice of the
origin of the flux-coordinate system.\cite{fitz2024}

Let
\begin{equation}
J(\hat{r},\omega) = \frac{1}{\epsilon^{\,2}}\left(\frac{\partial R}{\partial\omega}\,\frac{\partial Z}{\partial \hat{r}} -\frac{\partial R}{\partial \hat{r}}\,\frac{\partial Z}{\partial \omega}\right)
\end{equation}
be the Jacobian of the $\hat{r}$, $\omega$ coordinate system. We can transform to the $\hat{r}$, $\theta$ coordinate system 
by writing
\begin{align}\label{e11}
\theta(\hat{r},\omega) &= \left.2\pi\int_0^\omega \frac{J(\hat{r},\tilde{\omega})}{R(\hat{r},\tilde{\omega})}\,d\tilde{\omega}\right/\oint\frac{J(\hat{r},\omega)}{R(\hat{r},\omega)}\,d\omega,\\[0.5ex]
\hat{r}&=\frac{1}{2\pi}\oint\frac{J(\hat{r},\omega)}{R(\hat{r},\omega)}\,d\omega.\label{e12}
\end{align}
This transformation ensures that 
\begin{equation}
\frac{\partial\theta}{\partial\omega} = \frac{J}{\hat{r}\,R},
\end{equation}
and, hence, that 
\begin{equation}
{\cal J} \equiv \frac{R}{\epsilon} \left(\frac{\partial R}{\partial\theta}\,\frac{\partial Z}{\partial \hat{r}} -\frac{\partial R}{\partial \hat{r}}\,\frac{\partial Z}{\partial \hat{r}}\right)
=\epsilon\, R\,J\,\frac{\partial\omega}{\partial\theta} =r\,R^{\,2},
\end{equation}
in accordance with Eq.~(\ref{jac}). 

\subsection{Metric Elements}\label{metric}
We can determine the metric elements of the flux-coordinate system by combining Eqs.~(\ref{e19x})--(\ref{e12}).
Evaluating the elements up to ${\cal O}(\epsilon)$, but retaining ${\cal O}(\epsilon^{\,2})$ contributions to terms that are independent of
$\omega$, we obtain,\cite{am1,gim,fitz2024}
\begin{align}\label{epdef}
L(\hat{r})&= \frac{\hat{r}^{\,3}}{8} -\frac{\hat{r}\,H_1}{2}-\frac{1}{2}\sum_{j>1}(j-1)\,\frac{H_j^{\,2}}{\hat{r}}
-\frac{1}{2}\sum_{j>1}(j-1)\,\frac{V_j^{\,2}}{\hat{r}},\\[0.5ex]
\theta &= \omega+\epsilon\,\hat{r}\,\sin\omega - \epsilon\sum_{j>0}\frac{1}{j}\left[H_j'-(j-1)\,\frac{H_j}{\hat{r}}\right]\sin(j\,\omega)
\nonumber\\[0.5ex]&\phantom{=}+ \epsilon\sum_{j>1}\frac{1}{j}\left[V_j'-(j-1)\,\frac{V_j}{\hat{r}}\right]\cos(j\,\omega),\label{e22y}\\[0.5ex]
|\hat{\nabla} \hat{r}|^2 &= 1 +2\,\epsilon\sum_{j>0}H_j'\,\cos(j\,\theta) +2\,\epsilon\sum_{j>1}V_j'\,\sin(j\,\theta) \nonumber\\[0.5ex]
&\phantom{=}+\epsilon^{\,2}\left(\frac{3\,\hat{r}^2}{4}-H_1+
\frac{1}{2}\sum_{j>0}\left[H_j'^{\,2}+(j^2-1)\,\frac{H_j^{\,2}}{\hat{r}^{\,2}}\right]\right.\nonumber\\[0.5ex]&\phantom{=}\left.+
\frac{1}{2}\sum_{j>1}\left[V_j'^{\,2}+(j^2-1)\,\frac{V_j^{\,2}}{\hat{r}^{\,2}}\right]
\right),\label{e19}\\[0.5ex]
\hat{\nabla}\hat{r}\cdot\hat{\nabla}\theta&=\epsilon\,\sin\theta
-\epsilon\sum_{j>0}\frac{1}{j}\left[H_j''+\frac{H_j'}{\hat{r}}+(j^2-1)\,\frac{H_j}{\hat{r}^{\,2}}\right]\sin(j\,\theta)\nonumber\\[0.5ex]&
\phantom{=}+\epsilon\sum_{j>1}\frac{1}{j}\left[V_j''+\frac{V_j'}{\hat{r}}+(j^2-1)\,\frac{V_j}{\hat{r}^{\,2}}\right]\cos(j\,\theta),
\\[0.5ex]
R^{\,2}&= 1-2\,\epsilon\,\hat{r}\,\cos\theta -\epsilon^{\,2}\left(\frac{\hat{r}^{\,2}}{2}-\hat{r}\,H_1'-2\,H_1\right).\label{e25a}
\end{align}
Here, $'\equiv d/d\hat{r}$. Moreover, we have made use of the fact that $V_j\propto H_j$, for $j>1$, because
$V_j$ and $H_j$ satisfy the identical differential equations, (\ref{e33x}) and (\ref{e28}). 

\subsection{Expansion of Grad-Shafranov Equation}\label{exp}
Let us write
\begin{align}\label{e26v}
f(\hat{r})&= \epsilon\,\frac{\hat{r}\,g}{q},\\[0.5ex]
g(\hat{r}) &= 1+ \epsilon^{\,2}\,g_2(\hat{r}) + \epsilon^{\,4}\,g_4(\hat{r}),\label{e27v}\\[0.5ex]
P'(\hat{r}) &= \epsilon^{\,2}\,p_2'(\hat{r}),\label{eq1}
\end{align}
where $q$,  $g_2$, $g_4$, and $p_2$ are all ${\cal O}(1)$. Here, the safety-factor, $q(\hat{r})$, and the second-order plasma
pressure gradient, $p_2'(\hat{r})$, are the two free flux-surface functions that characterize the plasma equilibrium.\cite{gs1} 

Expanding the Grad-Shafranov equation, (\ref{gs}), order by order in the
small parameter $\epsilon$, making use of Eqs.~(\ref{e19})--(\ref{eq1}), we obtain\,\cite{con0,connor,gim,fitz2024}
\begin{align}
g_2'&=- p_2' - \frac{\hat{r}}{q^2}\,(2-s),\label{e26}\\[0.5ex]
H_1''&= -(3-2\,s)\,\frac{H_1' }{\hat{r}}
-1+\frac{2\,p_2'\,q^2}{\hat{r}},\label{e27}\\[0.5ex]
H_j''&= -(3-2\,s)\,\frac{H_j'}{\hat{r}}+(j^2-1)\,\frac{H_j}{\hat{r}^{\,2}}~~~~~\mbox{for $j>1$},\label{e33x}\\[0.5ex]
V_j''&= -(3-2\,s)\,\frac{V_j'}{\hat{r}}+(j^2-1)\,\frac{V_j}{\hat{r}^{\,2}}~~~~~~\mbox{for $j>1$},\label{e28}\\[0.5ex]
g_4'&= -\frac{\hat{r}}{q^2}\left(
\frac{3\,\hat{r}^{\,2}}{2}-2\,\hat{r}\,H_1'\right.\nonumber\\[0.5ex]
&\phantom{===}
+\sum_{j>0}\left[H_j'^{\,2}+2\,(j^2-1)\,\frac{H_j'\,H_j}{\hat{r}}-(j^2-1)\,\frac{H_j^{\,2}}{\hat{r}^{\,2}}\right]\nonumber\\[0.5ex]
&\phantom{===}\left.+\sum_{j>1}\left[V_j'^{\,2}+2\,(j^2-1)\,\frac{V_j'\,V_j}{\hat{r}}-(j^2-1)\,\frac{V_j^{\,2}}{\hat{r}^{\,2}}\right]\right)\nonumber\\[0.5ex]
&\phantom{=}
+\frac{\hat{r}}{q^2}\,(2-s)\left(-g_2-\frac{3\,\hat{r}^{\,2}}{4} +\frac{\hat{r}^{\,2}}{q^2}+H_1 +\frac{1}{2}\sum_{j>0}\left[3\,H_j'^{\,2}- (j^2-1)\,\frac{H_j^{\,2}}{\hat{r}^{\,2}}\right]
\right.\nonumber\\[0.5ex]
&\phantom{===}\left.+\frac{1}{2}\sum_{j>1}\left[3\,V_j'^{\,2}- (j^2-1)\,\frac{V_j^{\,2}}{\hat{r}^{\,2}}\right]\right)\nonumber\\[0.5ex]
&\phantom{=}
+p_2'\left(g_2+\frac{\hat{r}^{\,2}}{2}+\frac{\hat{r}^{\,2}}{q^2}-2\,H_1-3\,\hat{r}\,H_1'\right).\label{e31}
\end{align}
Note that the relative horizontal shift of magnetic flux-surfaces, $H_1$, otherwise known as the {\em Shafranov shift},\cite{shaf} is driven by toroidicity [the second term on
the right-hand side of Eq.~(\ref{e27})], and plasma pressure gradients (the third term). All of the other shaping terms (i.e., the $H_j$, for $j>1$, and
the $V_j$) are driven by axisymmetric currents flowing in external  magnetic field-coils.\cite{fitz2024} 

Finally, it follows from Eqs.~(\ref{ap}), (\ref{ag}), (\ref{af}), and (\ref{e26v})--(\ref{eq1}) that
\begin{align}
\alpha_p(\hat{r}) &= \frac{p_2'\,q^2}{\hat{r}}\left(1-2\,\epsilon^{\,2}\,g_2\right),\\[0.5ex]
\alpha_g(\hat{r}) &= \frac{q}{\hat{r}}\left(g_2' -\epsilon^{\,2}\,g_2\,g_2'+\epsilon^{\,2}\,g_4'\right),\\[0.5ex]
\alpha_f(\hat{r}) &= -s + \epsilon^{\,2}\,\hat{r}\,g_2'.
\end{align}

\subsection{Self-Inductance and $\beta$ values}
The conventionally defined normalized self-inductance, toroidal beta, poloidal beta, and normalized beta values of the plasma equilibrium 
can be written\,\cite{gs1}
\begin{align}
l_i&= \frac{2\int_0^1 \hat{r}\,f^2\,\langle |\nabla r|^2\rangle\,d\hat{r}}{(f^2\,\langle|\nabla r|^2\rangle^2)_{\hat{r}=1}},\\[0.5ex]
\beta_t &=  \frac{2\,\epsilon^{\,2}\int_0^1 \hat{r}\,\langle R^{\,2}\rangle\,p_2\,d\hat{r}}{\int_0^1 \hat{r}\,\langle R^{\,2}\rangle\,d\hat{r}},\\[0.5ex]
\beta_p &=  \frac{2\,\epsilon^{\,2}\int_0^1 \hat{r}\,\langle R^{\,2}\rangle\,p_2\,d\hat{r}}
{(f^2\,\langle|\nabla r|^2/R^{\,2}\rangle)_{\hat{r}=1}\int_0^1 \hat{r}\,\langle R^{\,2}\rangle\,d\hat{r}},\\[0.5ex]
\beta_N &=  \frac{20\,\epsilon^{\,2}\int_0^1 \hat{r}\,\langle R^{\,2}\rangle\,p_2\,d\hat{r}}
{(f\,\langle|\nabla r|^2\rangle)_{\hat{r}=1}\int_0^1 \hat{r}\,\langle R^{\,2}\rangle\,d\hat{r}},
\end{align}
respectively. Here, $\langle\cdots\rangle \equiv \oint (\cdots)\,d\theta/2\pi$.

\subsection{Coupling Coefficients}
Let
\begin{align}
S_1(\hat{r})&= \frac{1}{2}\sum_{j>0}\left[3\,(H_j'^{\,2}+V_j'^{\,2}) - (j^2-1)\,\frac{H_j^{\,2}+V_j^{\,2}}{\hat{r}^{\,2}}\right],\\[0.5ex]
S_2(\hat{r})&= \sum_{j>1}2\,(j^2-1)\left(H_j'^{\,2}+V_j'^{\,2}-\frac{11}{3}\,\frac{H_j'\,H_j+V_j'\,V_j}{\hat{r}}+j^2\,\frac{H_j^{\,2}+V_j^{\,2}}{\hat{r}^{\,2}}\right)\nonumber\\[0.5ex]&\phantom{=} -\sum_{j>0}(1-s)\left(\frac{H_j'\,H_j+V_j'\,V_j}{\hat{r}}+\frac{1}{3}\,\frac{H_j^{\,2}+V_j^{\,2}}{\hat{r}^{\,2}}\right).
\end{align}
The analysis of Sects.~\ref{ode1}, \ref{ode2}, \ref{metric}, and \ref{exp} can be combined to give
the following expressions for the coupling coefficients appearing in the outer-region ODES, (\ref{e61x}) and (\ref{e62x}):\,\cite{am1}
\begin{align}
L_m^{\,m}(\hat{r})&= m^2 + \epsilon^{\,2}\,m^2\left(-\frac{3\,\hat{r}^{\,2}}{4} + H_1+S_1\right)+ \epsilon^{\,2}\,n^2\,\hat{r}^{\,2},\\[0.5ex]
M_m^{\,m}(\hat{r})&= 0,\\[0.5ex]
N_m^{\,m}(\hat{r})&= 0,\\[0.5ex]
P_m^{\,m}(\hat{r}) &= (m-n\,q)^2+ \frac{m-n\,q}{m}\,q\,\hat{r}\,\frac{d}{d\hat{r}}\!\left(\frac{2-s}{q}\right)\nonumber\\[0.5ex]
&\phantom{=}+\epsilon^{\,2}\,(m-n\,q)^2\left\{\frac{7\,\hat{r}^{\,2}}{4} -H_1-3\,\hat{r}\,H_1'+S_1\right.\nonumber\\[0.5ex]
&\phantom{=} \left.+\frac{1}{m^2}\left[\frac{n}{m}\,\hat{r}\,\frac{d}{d\hat{r}}\!\left(\hat{r}^{\,2}\,\frac{2-s}{q}\right)
-\hat{r}^{\,2}\,\frac{(2-s)^2}{q^2}-\hat{r}\,\frac{d}{d\hat{r}}(\hat{r}\,p_2')\right]\right\}\nonumber\\[0.5ex]
&\phantom{=}-\epsilon^{\,2}\,\frac{m-n\,q}{m}\left\{2\,\hat{r}\,p_2'\,(2-s) +q\,\hat{r}\,\frac{d}{d\hat{r}}\!\left[
\hat{r}^{\,2}\,\frac{2-s}{q^3}+\frac{s}{q}\left(\frac{3\,\hat{r}^{\,2}}{4}-H_1-S_1\right)\right.\right.\nonumber\\[0.5ex]
&\left.\left.-\frac{2}{q}\left(\frac{3\,\hat{r}^{\,2}}{2}-H_1-\hat{r}\,H_1'-\frac{2}{3}\,S_1\right)\right]-S_2\right\}+\epsilon^{\,2}\,2\,\hat{r}\,p_2'\,(1-q^2),\\[0.5ex]
L_m^{\,m\pm 1}(\hat{r}) &= -\epsilon\,m\,(m\pm 1)\,H_1',\\[0.5ex]
L_m^{\,m\pm j}(\hat{r})&= - \epsilon\,m\,(m\pm j)\,(H_j'\pm {\rm i}\,V_j')\mbox{~~~~~for $j>1$},\\[0.5ex]
M_m^{\,m\pm 1}(\hat{r}) &= \mp \epsilon\,m\,(m-n\,q)\,p_2'\,q^2\pm \epsilon\,m\,(m\pm 1-n\,q)\,[\hat{r}+ (1-s)\,H_1'],\\[0.5ex]
M_m^{\,m\pm j}(\hat{r}) &= \pm \epsilon\,\frac{m}{j}\,(m\pm j-n\,q)\left[(1-s)\,(H_j'\pm {\rm i}\,V_j')-(j^2-1)\,\frac{H_j\pm{\rm i} \,V_j}{\hat{r}}\right]\nonumber\\[0.5ex]&\phantom{=}\mbox{~~~~~for $j>1$},\\[0.5ex]
N_m^{\,m\pm 1}(\hat{r}) &= \mp \epsilon\,(m\pm 1)\,(m\pm 1-n\,q)\,p_2'\,q^2\pm \epsilon\,(m\pm 1)\,(m-n\,q)\,[\hat{r}+ (1-s)\,H_1'],\\[0.5ex]
N_m^{\,m\pm j}(\hat{r}) &= \pm \epsilon\,\frac{(m\pm j)}{j}\,(m-n\,q)\left[(1-s)\,(H_j'\pm {\rm i}\,V_j')-(j^2-1)\,\frac{H_j\pm{\rm i}\, V_j}{\hat{r}}\right]\nonumber\\[0.5ex]&\phantom{=}\mbox{~~~~~for $j>1$},\\[0.5ex]
P_m^{\,m\pm 1}(\hat{r})&= -\epsilon\,(1+s)\,p_2'\,q^2+ \epsilon\,(m-n\,q)\,(m\pm 1-n\,q)\,(\hat{r}-H_1'),\\[0.5ex]
P_m^{\,m\pm j}(\hat{r})&= -\epsilon\,(m-n\,q)\,(m\pm j-n\,q)\,(H_j' \pm {\rm i} \,V_j')]\mbox{~~~~~for $j>1$}.
\end{align}
When $m=0$, some of the coupling coefficient take on special values: 
\begin{align}
P_0^{\,0}(\hat{r})&=n^2\,q^2-\frac{q^2}{\hat{r}}\,\frac{d}{d\hat{r}}\!\left(\hat{r}^{\,2}\,\frac{2-s}{q^2}\right)-q^2\,\hat{r}\,\frac{d}{d\hat{r}}\!\left(
\frac{p_2'}{\hat{r}}\right),\\[0.5ex]
M_{\pm 1}^{\,0}(\hat{r}) &= \lim_{m\rightarrow 0}\,M_{m\pm 1}^{\,m} \mp\epsilon\,(2-s)\,H_1',\\[0.5ex]
M_{\pm j}^{\,0}(\hat{r}) &= \lim_{m\rightarrow 0}\,M_{m\pm j}^{\,m} \mp\epsilon\,(2-s)\,j\,(H_j'\mp{\rm i}\,V_j')\mbox{~~~~~for $j>1$},\\[0.5ex]
N_0^{\,\pm 1}(\hat{r}) &= \lim_{m\rightarrow 0}\,N_m^{\,m\pm 1} \pm\epsilon\,(2-s)\,H_1',\\[0.5ex]
N_0^{\,\pm j}(\hat{r}) &= \lim_{m\rightarrow 0}\,N_m^{\,m\pm j} \pm\epsilon\,(2-s)\,j\,(H_j'\pm{\rm i}\,V_j')\mbox{~~~~~for $j>1$},\\[0.5ex]
P_{\pm 1}^{\,0}(\hat{r})&= \lim_{m\rightarrow 0}\,P_{m\pm 1}^{\,m}-\epsilon\,(2-s)\left\{\pm n\,q^3\,p_2'+(1\mp n\,q)\left[\hat{r}+(1-s)\,H_1'\right]\right\}\\[0.5ex]
P_{\pm j}^{\,0}(\hat{r}) & =\lim_{m\rightarrow 0}\,P_{m\pm j}^{\,m}-\epsilon\,(2-s)\,\frac{(j\mp n\,q)}{j}
\left[(1-s)\,(H_j'\mp{\rm i}\,V_j')-(j^2-1)\left(\frac{H_j\mp {\rm i}\,V_j}{\hat{r}}\right)\right]\nonumber\\[0.5ex]&\mbox{~~~~~for $j>1$},\\[0.5ex]
P_0^{\,\pm 1}(\hat{r})&= \lim_{m\rightarrow 0}\,P_{m\pm 1}^{\,m}-\epsilon\,(2-s)\left\{\pm n\,q^3\,p_2'+(1\mp n\,q)\left[\hat{r}+(1-s)\,H_1'\right]\right\},\\[0.5ex]
P_0^{\,\pm j}(\hat{r})&=\lim_{m\rightarrow 0}\,P_m^{\,m\pm j}-\epsilon\,(2-s)\,\frac{(j\mp n\,q)}{j}
\left[(1-s)\,(H_j'\pm {\rm i}\,V_j')-(j^2-1)\left(\frac{H_j\pm {\rm i}\,V_j}{\hat{r}}\right)\right]\nonumber\\[0.5ex]&\mbox{~~~~~for $j>1$}.
\end{align}
Note that the coupling coefficients satisfy the symmetry requirements (\ref{e137})--(\ref{e140}).

\subsection{Behavior Close to Magnetic Axis}\label{axis}
In the limit $\hat{r}\ll 1$,  a well-behaved solution of the outer-region ODEs, (\ref{e61x}) and (\ref{e62x}), with a dominant poloidal
mode number $m>0$ is such that\,\cite{am1}
\begin{align}\label{e213}
Z_m(\hat{r})&\simeq \frac{m-n\,q}{m}\,\psi_m(\hat{r}),\\[0.5ex]
\psi_{m+1}(\hat{r})&\simeq- \epsilon\,\frac{\hat{r}\left[(m-n\,q)-2\,p_2''\,q^2\right]}{2\,(m-n\,q)}\,\psi_m(\hat{r}),\\[0.5ex]
\psi_{m+j} (\hat{r})&\simeq 
\epsilon\,\frac{2\,\hat{r}^{\,2}\,q''\,(H_j'-{\rm i}\,V_j')}{(m-n\,q)\,(m+1)\,q}\,\psi_m(\hat{r})\mbox{~~~~for $j>1$}\label{e215}
\end{align}
to lowest order, with all of the other $Z_m$ and $\psi_m$ approximately zero. 
A well-behaved solution with a dominant poloidal mode number $m<0$ is such that 
\begin{align}
Z_m(\hat{r})&\simeq \frac{m-n\,q}{|m|}\,\psi_m(\hat{r}),\label{e216}\\[0.5ex]
\psi_{m-1}(\hat{r})&\simeq -\epsilon\,\frac{\hat{r}\left[(m-n\,q)+2\,p_2''\,q^2\right]}{2\,(m-n\,q)}\,\psi_m(\hat{r}),\\[0.5ex]
\psi_{m-j} (\hat{r})&\simeq 
-\epsilon\,\frac{2\,\hat{r}^{\,2}\,q''\,(H_j'+{\rm i}\,V_j')}{(m-n\,q)\,(|m|+1)\,q}\,\psi_m(\hat{r})\mbox{~~~~for $j>1$}\label{e218}
\end{align}
to lowest order, with all of the other $Z_m$ and $\psi_m$ approximately zero. For the special case in which the dominant poloidal
mode number is zero, the well-behaved solution is
\begin{equation}
Z_0(\hat{r})\simeq {\rm constant}
\end{equation}
to lowest order, with all of the other $Z_m$ and $\psi_m$ approximately zero. 

Note that the solutions (\ref{e213})--(\ref{e218}) only exhibit ``outward'' coupling of different poloidal harmonics
(i.e., coupling in the direction away from the $m=0$ harmonic). However, the solutions whose central poloidal
mode numbers are $m=\pm 1$ are special cases, and also exhibit inward coupling. Thus, in addition, to the
couplings described in Eqs.~(\ref{e213})--(\ref{e215}), an $m=1$ solution drives the harmonics
\begin{equation}
\psi_{1-j}(\hat{r})\simeq 2\,\epsilon\,(H_j'-{\rm i}\,V_j')\,\psi_1(\hat{r})\mbox{~~~~for $j>1$}.
\end{equation}
Likewise, in addition to the couplings described in Eqs.~(\ref{e216})--(\ref{e218}), an $m=-1$ solution drives the
harmonics
\begin{equation}
\psi_{-1+j}(\hat{r})\simeq 2\,\epsilon\,(H_j'+{\rm i}\,V_j')\,\psi_{-1}(\hat{r})\mbox{~~~~for $j>1$}.
\end{equation}

\subsection{Plasma/Vacuum Interface}
We require the equilibrium plasma current to be zero at the plasma/vacuum interface, $\hat{r}=1$, which
implies that $g'(1)=P'(1)= 0$. (See Sect.~\ref{s3}.) It follows from Eqs.~(\ref{e27v})--(\ref{e26}) and (\ref{e31}) that we need\,\cite{am1}
\begin{align}
p_2'(1) &= 0,\label{e222}\\[0.5ex]
s(1) &= 2+ \epsilon^{\,2}\left(\frac{3\,\hat{r}^{\,2}}{2}-2\,\hat{r}\,H_1'\right.\nonumber\\[0.5ex]
&\phantom{===}
+\sum_{j>0}\left[H_j'^{\,2}+2\,(j^2-1)\,\frac{H_j'\,H_j}{\hat{r}}-(j^2-1)\,\frac{H_j^{\,2}}{\hat{r}^{\,2}}\right]\nonumber\\[0.5ex]
&\phantom{===}\left.+\sum_{j>1}\left[V_j'^{\,2}+2\,(j^2-1)\,\frac{V_j'\,V_j}{\hat{r}}-(j^2-1)\,\frac{V_j^{\,2}}{\hat{r}^{\,2}}\right]\right)_{\hat{r}=1}
+{\cal O}(\epsilon^4).\label{e223}
\end{align}

\section{Inverse Aspect-Ratio Expanded Vacuum Solution}\label{vacx}
\subsection{Perturbed Vacuum Magnetic Field}\label{pertb}
In the vacuum region surrounding the plasma, the curl-free perturbed magnetic field can be written in the form\,\cite{am1}
\begin{equation}
{\bf b} = {\rm i}\,\nabla V,
\end{equation}
where the scalar magnetic potential, $V(\hat{r},\theta,\phi)= V(\hat{r},\theta)\,\exp(-{\rm i}\,n\,\phi)$,  is expanded as 
\begin{equation}\label{e225}
V(\hat{r},\theta) = \sum_{m}V_m(\hat{r})\,\exp({\rm i}\,m\,\theta).
\end{equation}
It follows from Eqs.~(\ref{e66}), (\ref{e79o}), (\ref{e97}), and (\ref{Zdef})--(\ref{psidef}) that
\begin{align}
Z_m(\hat{r})&= (m-n\,q)\,V_m(\hat{r}),\label{evdef}\\[0.5ex]
\psi_m(\hat{r}) &= \oint \hat{r}\,R^{\,2}\,\hat{\nabla}\hat{r}\cdot\hat{\nabla} V\,\exp(-{\rm i}\,m\,\theta)\,\frac{d\theta}{2\pi}= \sum_{m'}\left[h_m^{\,m'}\,\hat{r}\,\frac{dV_{m'}}{d\hat{r}}+ i_{m}^{\,m'}\,m'\,V_{m'}(\hat{r})\right],
\end{align}
where
\begin{align}
h_m^{\,m'}(\hat{r})&= \oint |\hat{\nabla}\hat{r}|^2\,R^{\,2}\,\exp[-{\rm i}\,(m-m')\,\theta]\,\frac{d\theta}{2\pi},\\[0.5ex]
i_m^{\,m'}(\hat{r})&= \oint {\rm i}\,\hat{r}\,\hat{\nabla} \hat{r}\cdot\hat{\nabla}\theta\,R^{\,2} \,\exp[-{\rm i}\,(m-m')\,\theta]\,\frac{d\theta}{2\pi}.\label{e229}
\end{align}

In the vacuum region, the physical constraint $\nabla\cdot {\bf b} =0$ requires the scalar magnetic potential to satisfy Laplace's equation,
\begin{equation}
\nabla^2 V= 0.
\end{equation}
It is necessary to obtain a solution of the previous equation that extends beyond the region of validity of the
$\hat{r}$, $\theta$, $\phi$ coordinate system. This goal can be achieved using {\em orthogonal toroidal coordinates}, $\mu$, $\eta$, $\phi$, 
where\,\cite{morse}
\begin{align}\label{e54cx}
R &= \frac{\sinh\mu}{\cosh\mu-\cos\eta},\\[0.5ex]
Z&= \frac{\sin\eta}{\cosh\mu-\cos\eta}.\label{e55cc}
\end{align}
Here, $\mu(R,Z)\rightarrow 0$ corresponds to either $R\rightarrow 0$ or $(R^{\,2}+Z^{\,2})^{1/2}\rightarrow\infty$ (i.e.,
an approach to the toroidal symmetry axis or to infinity), whereas $\mu(R,Z)\rightarrow \infty$
corresponds to $(R, Z) \rightarrow (1, 0)$ (i.e., an approach to the magnetic axis). Furthermore, $\eta(R,Z)$ is an angular variable in the poloidal
plane.  The most general solution of Laplace's equation in toroidal coordinates for a potential that has $n$ periods around the
toroidal symmetry axis is\,\cite{morse1}
\begin{equation}
V(\mu,\eta) = \sum_m (z-\cos\eta)^{1/2}\,[A_m\,P_{m-1/2}^{\,n}(z) + B_m\,Q_{m-1/2}^{\,n}(z)]\,\exp(-{\rm i}\,m\,\eta),\label{e233}
\end{equation}
where $z=\cosh\mu$, $P_{m-1/2}^{\,n}(z)$ and $Q_{m-1/2}^{\,n}(z)$ are {\em toroidal functions},\cite{abrama} and the $A_m$ and
$B_m$ are arbitrary complex coefficients.  Note that $P_{m-1/2}^{\,n}(z)$ is well behaved
in the limit $z\rightarrow 1$, whereas $Q_{m-1/2}^{\,n}(z)$ is badly behaved.  Conversely, $P_{m-1/2}^{\,n}(z)$ is badly behaved in the limit
$z\rightarrow\infty$, whereas $Q_{m-1/2}^{\,n}(z)$ is well behaved. 

In the immediate vicinity of the plasma/vacuum interface, Eqs.~(\ref{e19x}), (\ref{e20x}), (\ref{epdef}), (\ref{e54cx}), and (\ref{e55cc}) can
be combined to give\,\cite{fitz2024}
\begin{equation}\label{e73h}
\eta=\pi-\theta-\epsilon\,F_\theta(\hat{r},\theta)+{\cal O}(\epsilon^{\,2}),
\end{equation}
where
\begin{equation}\label{e234}
F_\theta(\hat{r},\theta) = -\frac{\hat{r}}{2}\,\sin\theta+\sum_{j>0}\frac{1}{j}\left(H_j' + \frac{H_j}{\hat{r}}\right)\sin(j\,\theta)
-\sum_{j>1}\frac{1}{j}\left(V_j' + \frac{V_j}{\hat{r}}\right)\cos(j\,\theta).
\end{equation}
Furthermore, 
\begin{equation}\label{e74cc}
z = \frac{\xi(\hat{r},\theta)}{\epsilon\,\hat{r}},
\end{equation}
where 
\begin{align}\label{e75cc}
\xi(\hat{r},\theta)&= 1+ \epsilon\left[-\frac{\hat{r}}{2}\,\cos\theta
 +\sum_{j>0}\frac{H_j}{\hat{r}}\,\cos(j\,\theta)+\sum_{j>1}\frac{V_j}{\hat{r}}\,\sin(j\,\theta)\right] \\[0.5ex]
 &\phantom{=}+ \epsilon^{\,2}\left[\frac{3\,\hat{r}^{\,2}}{16}+\frac{H_1}{4}-\frac{\hat{r}\,H_1'}{4}
-\frac{1}{2}\sum_{j>0}\left(\frac{H_j'\,H_j}{\hat{r}}-\frac{H_j^{\,2}}{2\,\hat{r}^{\,2}}\right)-\frac{1}{2}\sum_{j>1}\left(\frac{V_j'\,V_j}{\hat{r}}-\frac{V_j^{\,2}}{2\,\hat{r}^{\,2}}\right)\right].\nonumber
\end{align}
Note that  $\hat{r}\sim{\cal O}(1)$ in the immediate vicinity of the plasma/vacuum interface, which implies that $z\gg 1$. 

It is easily demonstrated that $P_{m-1/2}^{\,n}(z)= P_{|m|-1/2}^{\,n}(z)$
and $Q_{m-1/2}^{\,n}(z) = Q_{|m|-1/2}^{\,n}(z)$.\cite{abram2}
In the limit $z\gg 1$, the toroidal functions $P_{|m|-1/2}^{\,n}(z)$ and $Q_{|m|-1/2}^{\,n}(z)$
have the following asymptotic behaviors:\,\cite{morse2,bate}
\begin{align}
\frac{z^{1/2}\,P_{-1/2}^{\,n}(z)}{\tanh^{n}z}&= \frac{2}{\sqrt{2\pi}\,\Gamma(1/2-n)}
\left\{\ln\left(\frac{8\,z}{\zeta_n}\right)\right.\nonumber\\[0.5ex]
&\phantom{=}\left.
+ \frac{1}{4}\left[(n^2-5/4)+ (n+1/2)\,(n+3/2)\,\ln\left(\frac{8\,z}{\zeta_n}\right)\right]\frac{1}{z^2}+{\cal O}\left(\frac{1}{z^4}\right)\right\},\\[0.5ex]
\frac{z^{1/2}\,P_{1/2}^{\,n}(z)}{\tanh^n z}&= \frac{2\,z}{\sqrt{2\pi}\,\Gamma(3/2-n)}
\Biggl\{1\nonumber\\[0.5ex]&\phantom{=}-\left[\frac{n^2-1/4}{2}\,\ln\left(\frac{8\,z}{\zeta_n}\right)+\frac{(n-1/2)\,(n-3/2)}{4} \right]\frac{1}{z^2}+{\cal O}\left(\frac{1}{z^4}\right)\Biggr\},\\[0.5ex]
\frac{z^{1/2}\,P_{|m|-1/2}^{\,n}(z)}{\tanh^n z}&= \frac{(|m|-1)!\,2^{\,|m|}\,z^{\,|m|}}{\sqrt{2\pi}\,\Gamma(|m|+1/2-n)}
\left[1- \frac{(n-|m|+1/2)\,(n-|m|+3/2)}{4\,(|m|-1)}\,\frac{1}{z^2}\right.\nonumber\\[0.5ex]&\phantom{=}\left.+{\cal O}\left(\frac{1}{z^4}\right)\right],\label{e8nm}\\[0.5ex]
\frac{z^{1/2}\,Q_{|m|-1/2}^{\,n}(z)}{\tanh^{-n}z}&=  \frac{\sqrt{\pi}\,\Gamma(|m|+1/2+n)\,z^{-|m|}}{\sqrt{2}\,2^{\,|m|}\,|m|!}
\Biggl[1\nonumber\\[0.5ex]&\phantom{=}+ \frac{(|m|+1/2-n)\,(|m|+3/2-n)}{4\,(|m|+1)}\,\frac{1}{z^2}+{\cal O}\left(\frac{1}{z^4}\right)\Biggr],\label{e241}
\end{align}
where
\begin{equation}
\zeta_n = \exp\left(\sum_{i=1,n}\frac{2}{2\,i-1}\right).
\end{equation}
Here, Eq.~(\ref{e8nm}) only applies to $|m|>1$. 
Note that there is a factor ${\rm i}^{\,n}$ difference between the definition of the $P_{m-1/2}^{\,n}(z)$ used in this paper and that employed in Ref.~\onlinecite{morse1}.
Moreover, $\Gamma(z)$ is a Gamma function.\cite{abram1} 

Let 
\begin{align}
a_m &= \frac{\cos(|m|\,\pi)\,2^{\,|m|-1/2}\,|m|!\,\epsilon^{\,-|m|}}{\sqrt{\pi}\,\Gamma(|m|+1/2-n)}\,A_m,\label{e11cc}\\[0.5ex]
b_m &= \frac{\cos(|m|\,\pi)\,\sqrt{\pi}\,\Gamma(|m|+1/2+n)\,\epsilon^{\,|m|}}{2^{\,|m|+1/2}\,(|m|-1)!}\,B_m\label{e13cc}
\end{align}
for general $m$, and
\begin{align}
a_0 &= \frac{\sqrt{2}}{\sqrt{\pi}\,\Gamma(1/2-n)}\,A_0,\\[0.5ex]
b_0 &= \frac{\sqrt{\pi}\,\Gamma(1/2+n)}{\sqrt{2}}\,B_0\label{e15cc}
\end{align}
for the special case $m=0$. 
Equations~(\ref{e19})--(\ref{e25a}), (\ref{e27})--(\ref{e28}), (\ref{e222})--(\ref{e223}), (\ref{e225})--(\ref{e229}), (\ref{e233}),  (\ref{e234})--(\ref{e241}),
and (\ref{e11cc})--(\ref{e15cc}) 
can be combined to give
\begin{align}
\frac{Z_m(1)}{m-n\,q(1)} &= \sum_{m'}\left({\cal P}_m^{\,m'} \,a_{m'}+ {\cal Q}_{m}^{\,m'}\,b_{m'}\right),\label{e27cc}\\[0.5ex]
\psi_m(1)& = \sum_{m'}\left({\cal R}_m^{\,m'}\,a_{m'} + {\cal S}_{m}^{\,m'}\,b_{m'}\right),\label{e54cc}
\end{align}
where\,\cite{am1}
\begin{align}
{\cal P}_{m+\sigma\,j}^{\,m}  &=\frac{\epsilon}{2\,j}\left[(H_j'-{\rm i}\,\sigma\,V_j')+(j+1)\left(H_j-{\rm i}\,\sigma\,V_j\right)\right]\mbox{~~~for $j>1$},\\[0.5ex]
{\cal P}_{m+\sigma}^{\,m} &=\epsilon\left[\left(\frac{1}{4\,|m|}-\frac{1}{2}\right)+\left(H_1+\frac{H_1'}{2}\right)\right],\\[0.5ex]
{\cal P}_m^{\,m}  &= \frac{1}{|m|}\left(1+\epsilon^{\,2}\,[G_0 (|m|)- |m|\,G_1- |m|^2\,G_2]\right),\\[0.5ex]
{\cal P}_{m-\sigma}^{\,m} &= \epsilon\left(\frac{1}{4\,|m|}- \frac{H_1'}{2}\right),\\[0.5ex]
{\cal P}_{m-\sigma\,j}^{\,m} &= \frac{\epsilon}{2\,j}\,\left[-\left(H_j'+{\rm i}\,\sigma\,V_j'\right)+(j-1)\left(
H_j+{\rm i}\,\sigma\,V_j\right)\right]\mbox{~~~for $j>1$},
\end{align}
and 
\begin{align}
{\cal Q}_{m+\sigma\,j}^{\,m} &=\frac{\epsilon}{2\,j}\,\left[\left(H_j'-{\rm i}\,\sigma\,V_j'\right)-(j-1)\left(
H_j-{\rm i}\,\sigma\,V_j\right)\right]\mbox{~~~for $j>1$}, \\[0.5ex]
{\cal Q}_{m+\sigma}^{\,m} &=\epsilon\left(\frac{1}{4\,|m|}+ \frac{H_1'}{2}\right)\\[0.5ex]
{\cal Q}_m^{\,m} &= \frac{1}{|m|}\left(1+\epsilon^{\,2}\,[G_0 (-|m|)+ |m|\,G_1- |m|^2\,G_2]\right),\\[0.5ex]
{\cal Q}_{m-\sigma}^{\,m} &=\epsilon\left[\left(\frac{1}{4\,|m|}+\frac{1}{2}\right)-\left(H_1+\frac{H_1'}{2}\right)\right], \\[0.5ex]
{\cal Q}_{m-\sigma\,j}^{\,m} &=  \frac{\epsilon}{2\,j}\left[-(H_j'+{\rm i}\,\sigma\,V_j')-(j+1)\left(H_j+{\rm i}\,\sigma\,V_j\right)\right]\mbox{~~~for $j>1$},
\end{align}
and
\begin{align}
{\cal R}_{m+\sigma\,j}^{\,m}  &=\frac{\epsilon}{2}\left(1+\frac{|m|}{j}\right)\left[-(H_j'-{\rm i}\,\sigma\,V_j')-(j+1)\left(H_j-{\rm i}\,\sigma\,V_j\right)\right]\mbox{~~~for $j>1$},\\[0.5ex]
{\cal R}_{m+\sigma}^{\,m} &=\epsilon\,(1+|m|)\left[\left(\frac{1}{4\,|m|}+\frac{1}{2}\right)-\left(H_1+\frac{H_1'}{2}\right)\right],\\[0.5ex]
{\cal R}_m^{\,m} &= -1-\epsilon^{\,2}\left[G_3 (|m|)- |m|\left(G_1+\frac{H_1}{2}\right)- |m|^2\,G_2\right]\,\\[0.5ex]
{\cal R}_{m-\sigma}^{\,m} &=\epsilon\left[\left(\frac{1}{4\,|m|}+\frac{1}{4}\right)
-(1-|m|)\,\frac{H_1'}{2}\right],\\[0.5ex]
{\cal R}_{m-\sigma\,j}^{\,m} &= \frac{\epsilon}{2}\left(1-\frac{|m|}{j}\right)\left[-\left(H_j'+{\rm i}\,\sigma\,V_j'\right)+(j-1)\left(
H_j+{\rm i}\,\sigma\,V_j\right)\right]\mbox{~~~for $j>1$},
\end{align}
and 
\begin{align}
{\cal S}_{m+\sigma\,j}^{\,m}  &=\frac{\epsilon}{2}\left(1+\frac{|m|}{j}\right)\left[\left(H_j'-{\rm i}\,\sigma\,V_j'\right)-(j-1)\left(
H_j-{\rm i}\,\sigma\,V_j\right)\right]\mbox{~~~for $j>1$}, \\[0.5ex]
{\cal S}_{m+\sigma}^{\,m} &= \epsilon\left[\left(\frac{1}{4\,|m|}-\frac{1}{4}\right)
+(1+|m|)\,\frac{H_1'}{2}\right],\\[0.5ex]
{\cal S}_m^{\,m} &=1+\epsilon^{\,2}\left[G_3 (-|m|)+ |m|\left(G_1+\frac{H_1}{2}\right)- |m|^2\,G_2\right],\\[0.5ex]
{\cal S}_{m-\sigma}^{\,m} &=\epsilon\,(1-|m|)\left[\left(\frac{1}{4\,|m|}-\frac{1}{2}\right)+\left(H_1+\frac{H_1'}{2}\right)\right], \\[0.5ex]
{\cal S}_{m-\sigma\,j}^{\,m} &=\frac{\epsilon}{2}\left(1-\frac{|m|}{j}\right)\left[(H_j'+{\rm i}\,\sigma\,V_j')+(j+1)\left(H_j+{\rm i}\,\sigma\,V_j\right)\right]\mbox{~~~for $j>1$},
\end{align}
and 
\begin{align}
G_0(|m|) &= -\left[\frac{n^2+(|m|-2)\,(|m|-3/4)}{4\,(|m|-1)}\right]-\frac{H_1}{2}-\frac{H_1'}{4}\mbox{~~~~for $|m|\neq 1$},\\[0.5ex]
G_0(1) &= -\frac{n^2}{4}-\left(\frac{n^2}{2}-\frac{1}{8}\right)\ln\left(\frac{8}{\zeta_n\,\epsilon}\right)-\frac{H_1}{2}-\frac{H_1'}{4},\\[0.5ex]
G_1  &= -\frac{3\,H_1}{4} - \frac{H_1'}{4} + \frac{1}{2}\sum_{j>0}H_j'\,H_j+ \frac{1}{2}\sum_{j>1}V_j'\,V_j,\\[0.5ex]
G_2 &= -\frac{H_1'}{4}+\frac{1}{4}\sum_{j>0}\frac{1}{j^2}\left[H_j'^{\,2}+ 2\,H_j'\,H_j - (j^2-1)\,H_j^{\,2}\right]\nonumber\\[0.5ex]&\phantom{=}+\frac{1}{4}\sum_{j>1}\frac{1}{j^2}\left[V_j'^{\,2}+ 2\,V_j'\,V_j - (j^2-1)\,V_j^{\,2}\right],\\[0.5ex]
G_3(|m|)& =- \left[\frac{(|m|-2)\,n^2+ |m|^2/4}{4\,(|m|-1)\,|m|}\right]\mbox{~~~~for $|m|\neq 1$},\\[0.5ex]
G_3(1)&= -\frac{n^2}{4} +\frac{1}{8}+\left(\frac{n^2}{2}-\frac{1}{8}\right)\ln\left(\frac{8}{\zeta_n\,\epsilon}\right).
\end{align}
Here $\sigma={\rm sgn}(m)$. Moreover, the shaping functions and their derivatives are all evaluated at $\hat{r}=1$. 
There are a number of special cases: 
\begin{align}
{\cal P}_j^{\,0} &= \epsilon\left(\frac{H_j-{\rm i}\,V_j}{2}\right)\mbox{~~~for $j>1$},\\[0.5ex]
{\cal P}_1^{\,0} &=\epsilon\left[\frac{1}{4}\ln\left(\frac{8}{\zeta_n\,\epsilon}\right)-\frac{1}{4}+\frac{H_1}{2}\right],\\[0.5ex]
{\cal P}_0^{\,0}  &= \ln\left(\frac{8}{\zeta_n\,\epsilon}\right)+\epsilon^{\,2}\left[\frac{n^2}{4} -\frac{5}{16} + \left(\frac{n^2}{4}+\frac{3}{8}\right)\ln\left(\frac{8}{\zeta_n\,\epsilon}\right)\right]\nonumber\\[0.5ex]
&\phantom{=}-\left(\frac{H_1}{2}+\frac{H_1'}{4}\right)\ln\left(\frac{8}{\zeta_n\,\epsilon}\right)-G_1,\\[0.5ex]
{\cal P}_{-1}^{\,0} &=\epsilon\left[\frac{1}{4}\ln\left(\frac{8}{\zeta_n\,\epsilon}\right)-\frac{1}{4}+\frac{H_1}{2}\right],\\[0.5ex]
{\cal P}_{-j}^{\,0}   &= \epsilon\left(\frac{H_j+{\rm i}\,V_j}{2}\right)\mbox{~~~for $j>1$},
\end{align}
and
\begin{align}
{\cal Q}_j^{\,0} &= 0\mbox{~~~for $j>1$},\\[0.5ex]
{\cal Q}_1^{\,0} &=\frac{\epsilon}{4},\\[0.5ex]
{\cal Q}_0^{\,0}  &= 1+ \epsilon^{\,2}\left(\frac{n^2}{4} + \frac{3}{8} - \frac{H_1}{2} - \frac{H_1'}{4}\right),\\[0.5ex]
{\cal Q}_{-1}^{\,0} &=\frac{\epsilon}{4},\\[0.5ex]
{\cal Q}_{-j}^{\,0}   &= 0\mbox{~~~for $j>1$},
\end{align}
and 
\begin{align}
{\cal R}_j^{\,0}&= \epsilon\left(-\frac{H_j'-{\rm i}\,V_j'}{2}-\frac{H_j-{\rm i}\,V_j}{2}\right)\mbox{~~~for $j>1$},\\[0.5ex]
{\cal R}_1^{\,0} &=\epsilon\left[\frac{1}{4}\ln\left(\frac{8}{\zeta_n\,\epsilon}\right)+\frac{1}{2}-\frac{H_1}{2}-\frac{H_1'}{2}\right],\\[0.5ex]
{\cal R}_0^{\,0}  &= -1+\epsilon^{\,2}\,n^2\,\frac{1}{2}\left[\ln\left(\frac{8}{\zeta_n\,\epsilon}\right)+\frac{1}{2}\right],\\[0.5ex]
{\cal R}_{-1}^{\,0} &=\epsilon\left[\frac{1}{4}\ln\left(\frac{8}{\zeta_n\,\epsilon}\right)+\frac{1}{2}-\frac{H_1}{2}-\frac{H_1'}{2}\right],\\[0.5ex]
{\cal R}_{-j}^{\,0}   &= \epsilon\left(-\frac{H_j'+{\rm i}\,V_j'}{2}-\frac{H_j+{\rm i}\,V_j}{2}\right)\mbox{~~~for $j>1$},
\end{align}
and
\begin{align}
{\cal S}_j^{\,0} &= 0\mbox{~~~for $j>1$},\\[0.5ex]
{\cal S}_1^{\,0} &=\frac{\epsilon}{4},\\[0.5ex]
{\cal S}_0^{\,0} &= \epsilon^{\,2}\,\frac{n^2}{2},\\[0.5ex]
{\cal S}_{-1}^{\,0} &=\frac{\epsilon}{4},\\[0.5ex]
{\cal S}_{-j}^{\,0}   &= 0\mbox{~~~for $j>1$}.
\end{align}

It can be demonstrated that\,\cite{am1}
\begin{align}\label{e286}
{\cal A}^{mm'}\equiv\sum_{m''}\left({\cal P}_{m''}^{\,m\,\ast}\,{\cal R}_{m''}^{\,m'}- {\cal R}_{m''}^{\,m\,\ast}\,{\cal P}_{m''}^{\,m'}\right)&= \delta^{\,mm'}\,{\cal O}(\epsilon^{\,4})+ (1-\delta^{\,mm'})\,{\cal O}(\epsilon^{\,2}),\\[0.5ex]
{\cal B}^{mm'}\equiv\sum_{m''}\left({\cal Q}_{m''}^{\,m\,\ast}\,{\cal S}_{m''}^{\,m'}- {\cal S}_{m''}^{\,m\,\ast}\,{\cal Q}_{m''}^{\,m'}\right)&= \delta^{\,mm'}\,{\cal O}(\epsilon^{\,4})+ (1-\delta^{\,mm'})\,{\cal O}(\epsilon^{\,2}),\\[0.5ex]
{\cal C}^{mm'}\equiv\sum_{m''}\left({\cal P}_{m''}^{\,m\,\ast}\,{\cal S}_{m''}^{\,m'}- {\cal R}_{m''}^{\,m\,\ast}\,{\cal Q}_{m''}^{\,m'}\right)&= \delta^{\,mm'}\,[h_m+{\cal O}(\epsilon^{\,4})]+ (1-\delta^{\,mm'})\,{\cal O}(\epsilon^{\,2}),\label{e288}
\end{align}
where 
\begin{equation}
h_m = \left\{\begin{array}{ccc}2/|m|&~~~~~~&\mbox{$|m|>0$}\\[0.5ex]
1&&\mbox{$|m|=0$}\end{array}\right..
\end{equation}
Thus, we can determine the parameters in the expansions (\ref{e27cc}) and (\ref{e54cc}) as follows:
\begin{align}
a_m &= \frac{1}{h_m}\sum_{m'}\left[-{\cal Q}_{m'}^{\,m\,\ast}\,\psi_{m'}(1) + \frac{{\cal S}_{m'}^{\,m\,\ast}\,Z_{m'}(1)}{m'-n\,q(1)}\right]+{\cal O}(\epsilon^{\,2}),\\[0.5ex]
b_m &= \frac{1}{h_m}\sum_{m'}\left[{\cal P}_{m'}^{\,m\,\ast}\,\psi_{m'}(1) -\frac{{\cal R}_{m'}^{\,m\,\ast}\,Z_{m'}(1)}{m'-n\,q(1)}\right]+{\cal O}(\epsilon^{\,2}).\label{e291}
\end{align}

\subsection{Toroidal Electromagnetic Angular Momentum Flux}
By analogy with Eq.~(\ref{torque}), the outward flux of toroidal electromagnetic angular momentum across the plasma boundary,
which is equal to the flux of  toroidal electromagnetic angular momentum across a surface of constant $\mu$ (in the direction of
decreasing $\mu$) in the vacuum region, is given by 
\begin{align}
T_\phi(1) &= -\oint\oint (\nabla\mu\times \nabla\eta\cdot\nabla\phi)^{-1}\,b_\phi\,b^{\,\mu}\,d\eta\,d\phi\nonumber\\[0.5ex]
&= -\frac{{\rm i}\,\pi\,n}{2}\oint \frac{z^2-1}{z-\cos\eta}\left(\frac{\partial V}{\partial z}\,V^{\,\ast}-\frac{\partial V^{\,\ast}}{\partial z}\,V\right)d\eta.
\end{align}
From Eq.~(\ref{e233}), we get
\begin{equation}
T_\phi(1)=-{\rm i}\,\pi^2\,n\sum_m(z^2-1)\left[\frac{dP_{m-1/2}^{\,n}}{dz}\,Q_{m-1/2}^{\,n}- P_{m-1/2}^{\,n}\,\frac{dQ_{m-1/2}^{\,n}}{dz}\right]
\left(A_m\,B_m^{\,\ast}- A_m^{\,\ast}\,B_m\right).
\end{equation}
However,\cite{morse3}
\begin{equation}
\frac{dP_{m-1/2}^{\,n}}{dz}\,Q_{m-1/2}^{\,n}- P_{m-1/2}^{\,n}\,\frac{dQ_{m-1/2}^{\,n}}{dz}= \frac{\Gamma(|m|+1/2+n)}{\Gamma(|m|-1/2-n)\,(z^2-1)}.
\end{equation}
Thus, making use of Eqs.~(\ref{e11cc})--(\ref{e15cc}), we obtain
\begin{equation}\label{e90cc}
T_\phi(1) =- {\rm i}\,\pi^2\,n\sum_m h_m\,(a_m\,b_m^{\,\ast}-a_m^{\,\ast}\,b_m).
\end{equation}

\subsection{Homogeneous Boundary Condition at Plasma/Vacuum Interface}
In the absence of non-axisymmetric currents flowing in  external magnetic field-coils, the vacuum expansion  (\ref{e233})
must contain none of the terms involving the $Q_{m-1/2}^{\,n}(z)$, because these terms are badly behaved a long way from the plasma,
and could not, therefore, be generated by perturbed currents flowing within the plasma (which are the only types of perturbed current present in the problem).  It follows that the $B_m$ coefficients must all be zero. Hence, according to
Eqs.~(\ref{e13cc}) and (\ref{e15cc}), 
\begin{equation}\label{e296}
b_m= 0
\end{equation}
for all $m$. Thus, Eq.~(\ref{e291}) yields the following homogenous boundary condition at the plasma/vacuum interface:
\begin{equation}\label{e297}
\psi_{m}(1)= \sum_{m'}H_{mm'}\,\frac{Z_{m'}(1) }{m'-n\,q(1)},
\end{equation}
where
\begin{equation}\label{e298}
\sum_{m''}{\cal P}_{m''}^{\,m\,\ast}\,H_{m'' m'}= {\cal R}_{m'}^{\,m\,\ast}.
\end{equation}

Now, from Eq.~(\ref{etorque}), (\ref{e27cc}), (\ref{e54cc}), and (\ref{e286})--(\ref{e288}), 
\begin{align}\label{e299}
T_\phi(1) &= {\rm i}\,\pi^2\,n\sum_{m}\frac{Z_m^{\,\ast}(1)\,\psi_m(1)-\psi_m^{\,\ast}(1)\,Z_m(1)}{m-n\,q(1)}\nonumber\\[0.5ex]
&={\rm i}\,\pi^2\,n\sum_{m,m'}\left(a_m^{\,\ast}\,{\cal A}^{mm'}\,a_{m'} + a_m^{\,\ast}\,{\cal C}^{mm'}\,b_{m'}-b_m^{\,\ast}\,{\cal C}^{m'm\,\ast}\,a_{m'}
+b_m^{\,\ast}\,{\cal B}^{mm'}\,b_{m'}\right)\nonumber\\[0.5ex]
&= -{\rm i}\,\pi^2\,n\sum_m h_m\,(a_m\,b_m^{\,\ast}-a_m^{\,\ast}\,b_m) + {\cal O}(\epsilon^{\,2}).
\end{align}
Given that Eq.~(\ref{e90cc}) is an exact result (i.e., it is independent of our inverse aspect-ratio expansion), it is clear that the 
${\cal O}(\epsilon^{\,2})$ and ${\cal O}(\epsilon^{\,4})$ residuals on the right-hand sides of Eqs.~(\ref{e286})--(\ref{e288})
would all be zero in an exact calculation (such as the one described in Sect.~\ref{vacxx}).

Equations~(\ref{e90cc}) and (\ref{e296}) imply that 
\begin{equation}\label{e300}
T_\phi(1)=0. 
\end{equation}
In other words, the flux of toroidal electromagnetic angular momentum across the plasma boundary is
exactly zero, as must be the case for an isolated plasma.  According to Eqs.~(\ref{e297}) and (\ref{e299}), this constraint can only be satisfied if  $H_{mm'}$ is an Hermitian matrix. 
Now, Eqs.~(\ref{e298}) yields 
\begin{equation}
\sum_{m'',m'''}{\cal P}_{m''}^{m\,\ast}\,(H_{m''m'''}-H_{m'''m''}^{\,\ast})\,{\cal P}_{m'''}^{\,m'}=\sum_{m''}\left({\cal R}_{m''}^{\,m\,\ast}\,{\cal P}_{m''}^{\,m'}- {\cal P}_{m''}^{\,m\,\ast}\,{\cal R}_{m''}^{\,m'}\right)={\cal O}(\epsilon^{\,2}),
\end{equation}
where use has been made of Eq.~(\ref{e286}). Thus, it is clear that $H_{mm'}$ is Hermitian to ${\cal O}(\epsilon^{\,2})$. 

\subsection{Inhomogeneous Boundary Condition at Plasma/Vacuum Interface}
Non-axisymmetric currents (with $n$ periods in the toroidal direction) flowing in magnetic field-coils external to the plasma generate a
vacuum magnetic field at the plasma boundary that is characterized by the $b_m$ coefficients  appearing in Eqs.~(\ref{e27cc}) and (\ref{e54cc}).
If the $b_m$ are non-zero then Eqs.~(\ref{e291}) and (\ref{e298}) yield the following inhomogeneous boundary condition at the plasma/vacuum interface:
\begin{equation}\label{inhom}
\psi_{m}(1)= \sum_{m'}H_{mm'}\,\frac{Z_{m'}(1) }{m'-n\,q(1)}+ \sum_{m'}G_{mm'}\,b_{m'},
\end{equation}
where
\begin{equation}
\sum_{m''} {\cal P}_{m''}^{\,m\,\ast}\,G_{m''m'} = h_{m}\,\delta^{\,m}_{m'}.
\end{equation}
However, making use of Eqs.~(\ref{e288}) and (\ref{e298}), the previous expression can be transformed to give
\begin{equation}\label{e338}
G_{mm'} = {\cal S}_m^{\,m'} + \sum_{m''} H_{mm''}\,{\cal Q}_{m''}^{\,m'}.
\end{equation}

\section{General Vacuum Solution}\label{vacxx}
\subsection{Introduction}
According to the analysis of Sect.~\ref{vacx}, the homogeneous vacuum response matrix, $H_{mm'}$, which is defined in Eq.~(\ref{inhom}), must be
Hermitian, otherwise toroidal angular momentum is not conserved. However, the analysis of Sect.~\ref{vacx} only guarantees that
$H_{mm'}$ is Hermitian to ${\cal O}(\epsilon^2)$. If we wish to obtain a homogeneous vacuum response matrix that is Hermitian to greater accuracy than
this then we must extend our analysis, as described in Sects.~\ref{svacsoln}--\ref{svacsoln1}.

\subsection{Vacuum Solution}\label{svacsoln}
Let us assumed that all perturbed quantities  vary with the toroidal angle, $\phi$, as $\exp(-{\rm i}\,n\,\phi)$, which allows us to ignore $\phi$.
Consequently, the vacuum region that surrounds the plasma reduces to the section, $C$ (say), of the $R$, $Z$ plane that lies between the
curve $\hat{r}=1$ and the curve $z=1$. 

According to the analysis of Sect.~\ref{pertb}, the perturbed scalar magnetic potential in the vacuum region can be written
\begin{equation}
V(z,\eta)= V^p(z,\eta)+V^x(z,\eta),
\end{equation}
where
\begin{align}
V^p(z,\eta) &=\sum_m A_m\,(z-\cos\eta)^{1/2}\,P_{m-1/2}^{\,n}(z)\,\exp(-{\rm i}\,m\,\eta),\\[0.5ex]
V^x(z,\eta) &=\sum_m B_m\,(z-\cos\eta)^{1/2}\,Q_{m-1/2}^{\,n}(z)\,\exp(-{\rm i}\,m\,\eta),
\end{align}
where the $A_m$ and $B_m$ are arbitrary complex coefficients. 
Here, $V^p$ is the potential generated by currents flowing within the plasma, whereas $V^x$ is the potential
generated by non-axisymmetric currents flowing in magnetic field-coils external to the plasma. For the sake
of convenience, the latter
currents are assumed to lie at infinity.

In the immediate vicinity of the plasma, we can write
\begin{equation}
V(\hat{r},\theta) = \sum_m V_m(\hat{r})\,\exp(\,{\rm i}\,m\,\theta).
\end{equation}
It follows that
\begin{equation}\label{e343}
V_m(1) = V_m^{\,p} + V_m^{\,x},
\end{equation}
where
\begin{align}\label{e344}
V_m^{\,p} &= \oint_{\hat{r}=1}V^p\,\exp(-{\rm i}\,m\,\theta)\,\frac{d\theta}{2\pi}= \sum_{m'}{\cal P}_m^{\,m'}\,a_{m'},\\[0.5ex]
V_m^{\,x} &= \oint_{\hat{r}=1}V^x\,\exp(-{\rm i}\,m\,\theta)\,\frac{d\theta}{2\pi}= \sum_{m'}{\cal Q}_m^{\,m'}\,b_{m'},\label{e345}
\end{align}
and
\begin{align}
{\cal P}_m^{\,m'} &=
\frac{\cos(|m'|\,\pi)\,\sqrt{\pi}\,\Gamma(|m'|+1/2-n)}{2^{\,|m'|-1/2}\,|m'|!\,\epsilon^{-|m'|}}\oint_{\hat{r}=1}
(z-\cos\eta)^{1/2}\,P_{m'-1/2}^{\,n}(z)\,\exp[-{\rm i}\,(m\,\theta+m'\,\eta)]\,\frac{d\theta}{2\pi},\label{e346}\\[0.5ex]
{\cal Q}_m^{\,m'} &=\frac{\cos(|m'|\,\pi)\,2^{\,|m'|+1/2}\,(|m'|-1)!}{\sqrt{\pi}\,\Gamma(|m'|+1/2+n)\,\epsilon^{\,|m'|}}\
\oint_{\hat{r}=1}
(z-\cos\eta)^{1/2}\,Q_{m'-1/2}^{\,n}(z)\,\exp[-{\rm i}\,(m\,\theta+m'\,\eta)]\,\frac{d\theta}{2\pi}
\end{align}
for general $m'$, and 
\begin{align}
{\cal P}_m^{\,0} &=
\frac{\sqrt{\pi}\,\Gamma(1/2-n)}{\sqrt{2}}\oint_{\hat{r}=1}
(z-\cos\eta)^{1/2}\,P_{-1/2}^{\,n}(z)\,\exp(-{\rm i}\,m\,\theta)\,\frac{d\theta}{2\pi},\label{e348}\\[0.5ex]
{\cal Q}_m^{\,0} &=\frac{\sqrt{2}}{\sqrt{\pi}\,\Gamma(1/2+n)}
\oint_{\hat{r}=1}
(z-\cos\eta)^{1/2}\,Q_{-1/2}^{\,n}(z)\,\exp(-{\rm i}\,m\,\theta)\,\frac{d\theta}{2\pi}
\end{align}
for the special case $m'=0$. Here, use has been made of Eqs.~(\ref{e11cc})--(\ref{e15cc}).

In the immediate vicinity of the plasma, we can also write
\begin{equation}
\psi(\hat{r},\theta) = \sum_m \psi_m(\hat{r})\,\exp(\,{\rm i}\,m\,\theta).
\end{equation}
It follows that
\begin{equation}\label{e351}
\psi_m(1) = \psi_m^{\,p} + \psi_m^{\,x},
\end{equation}
where, according to the analysis of Sect.~\ref{pertb}, 
\begin{align}\label{e352}
\psi_m^{\,p} &= \oint_{\hat{r}=1}R^{\,2}\,\hat{\nabla}V^p\cdot \hat{\nabla}\hat{r}\,\exp(-{\rm i}\,m\,\theta)\,\frac{d\theta}{2\pi}= \sum_{m'}{\cal R}_m^{\,m'}\,a_{m'},\\[0.5ex]
\psi_m^{\,x} &= \oint_{\hat{r}=1}R^{\,2}\,\hat{\nabla}V^x\cdot\hat{\nabla}\hat{r}\,\exp(-{\rm i}\,m\,\theta)\,\frac{d\theta}{2\pi}= \sum_{m'}{\cal S}_m^{\,m'}\,b_{m'},
\label{e353}
\end{align}
and
\begin{align}\label{e354}
{\cal R}_m^{\,m'} &=
\frac{\cos(|m'|\,\pi)\,\sqrt{\pi}\,\Gamma(|m'|+1/2-n)}{2^{\,|m'|-1/2}\,|m'|!\,\epsilon^{-|m'|}}
\nonumber\\[0.5ex]&\times \oint_{\hat{r}=1}
R^{\,2}\left\{\left[\frac{1}{2}\,(z-\cos\eta)^{-1/2}\,P_{m'-1/2}^{\,n}(z)+(z-\cos\eta)^{1/2}\,\frac{dP_{m'-1/2}^{\,n}}{dz}\right]\hat{\nabla}\hat{r}\cdot\hat{\nabla} z
\right.\nonumber\\[0.5ex]&
\left.+\left[\frac{1}{2}\,(z-\cos\eta)^{-1/2}\,\sin\eta-{\rm i}\,m'\,(z-\cos\eta)^{1/2}\right]P_{m'-1/2}^{\,n}(z)\,\hat{\nabla}\hat{r}\cdot\hat{\nabla} \eta
\right\}\nonumber\\[0.5ex] &
\times\exp[-{\rm i}\,(m\,\theta+m'\,\eta)]\,\frac{d\theta}{2\pi},\\[0.5ex]
{\cal S}_m^{\,m'} &=\frac{\cos(|m'|\,\pi)\,2^{\,|m'|+1/2}\,(|m'|-1)!}{\sqrt{\pi}\,\Gamma(|m'|+1/2+n)\,\epsilon^{\,|m'|}}
\nonumber\\[0.5ex]&\times \oint_{\hat{r}=1}
R^{\,2}\left\{\left[\frac{1}{2}\,(z-\cos\eta)^{-1/2}\,Q_{m'-1/2}^{\,n}(z)+(z-\cos\eta)^{1/2}\,\frac{dQ_{m'-1/2}^{\,n}}{dz}\right]\hat{\nabla}\hat{r}\cdot\hat{\nabla} z
\right.\nonumber\\[0.5ex]&
\left.+\left[\frac{1}{2}\,(z-\cos\eta)^{-1/2}\,\sin\eta-{\rm i}\,m'\,(z-\cos\eta)^{1/2}\right]Q_{m'-1/2}^{\,n}(z)\,\hat{\nabla}\hat{r}\cdot\hat{\nabla} \eta
\right\}\nonumber\\[0.5ex] &
\times\exp[-{\rm i}\,(m\,\theta+m'\,\eta)]\,\frac{d\theta}{2\pi},
\end{align}
for general $m'$, and 
\begin{align}\label{e356}
{\cal R}_m^{\,0} &=
\frac{\sqrt{\pi}\,\Gamma(1/2-n)}{\sqrt{2}}
\nonumber\\[0.5ex]&\times \oint_{\hat{r}=1}
R^{\,2}\left\{\left[\frac{1}{2}\,(z-\cos\eta)^{-1/2}\,P_{-1/2}^{\,n}(z)+(z-\cos\eta)^{1/2}\,\frac{dP_{-1/2}^{\,n}}{dz}\right]\hat{\nabla}\hat{r}\cdot\hat{\nabla} z
\right.\nonumber\\[0.5ex]&
\left.+\frac{1}{2}\,(z-\cos\eta)^{-1/2}\,\sin\eta\,P_{-1/2}^{\,n}(z)\,\hat{\nabla}\hat{r}\cdot\hat{\nabla} \eta
\right\}\exp(-{\rm i}\,m\,\theta)\,\frac{d\theta}{2\pi}
,\\[0.5ex]
{\cal S}_m^{\,0} &=\frac{\sqrt{2}}{\sqrt{\pi}\,\Gamma(1/2+n)}
\nonumber\\[0.5ex]&\times \oint_{\hat{r}=1}
R^{\,2}\left\{\left[\frac{1}{2}\,(z-\cos\eta)^{-1/2}\,Q_{-1/2}^{\,n}(z)+(z-\cos\eta)^{1/2}\,\frac{dQ_{-1/2}^{\,n}}{dz}\right]\hat{\nabla}\hat{r}\cdot\hat{\nabla} z
\right.\nonumber\\[0.5ex]&
\left.+\frac{1}{2}\,(z-\cos\eta)^{-1/2}\,\sin\eta\,Q_{-1/2}^{\,n}(z)\,\hat{\nabla}\hat{r}\cdot\hat{\nabla} \eta
\right\}\exp(-{\rm i}\,m\,\theta)\,\frac{d\theta}{2\pi}
\end{align}
for the special case $m'=0$.

\subsection{Homogeneous Vacuum Response Matrix}
Let us define the function ${\cal E}_m(z,\mu)$ such that
\begin{align}\label{e358}
{\cal E}_m&= \exp(-{\rm i}\,m\,\theta) &\mbox{at $\hat{r}=1$},\\[0.5ex]
\nabla^2{\cal E}_m &= 0 &\mbox{throughout $C$},\label{e359}\\[0.5ex]
{\cal E}_m &= 0& \mbox{at $z=1$}.\label{e360}
\end{align}
Recall that
\begin{align}
\nabla^2 V^p &= 0 &\mbox{throughout $C$},\label{e361}\\[0.5ex]
V^p &= 0& \mbox{at $z=1$}.\label{e362}
\end{align}
It follows from Eq.~(\ref{e352}) that
\begin{equation}
\psi_m^{\,p} = \oint_{\hat{r}=1}R^{\,2}\,{\cal E}_m\,\hat{\nabla} V^p\cdot \hat{\nabla}\hat{r}\,\frac{d\theta}{2\pi}=\oint_{\hat{r}=1}{\cal J}\,{\cal E}_m\,\nabla V^p\cdot \nabla r\,\frac{d\theta}{2\pi},
\end{equation}
given that $r=\epsilon\,\hat{r}$, $\nabla = \epsilon^{-1}\,\hat{\nabla}$, and ${\cal J}=r\,R^{\,2}$. [See Eq.~(\ref{jac}).]
The previous equation can also be written
\begin{equation}
\psi_m^{\,p} =-\frac{1}{2\pi} \oint_{S}{\cal E}_m\,\nabla V^p\cdot d{\bf S}
\end{equation}
where $S$ is the bounding surface of the vacuum domain, $C$, and use has been made of Eqs.~(\ref{e360}) and (\ref{e362}). 
Now,
\begin{align}
\oint_S\left({\cal E}_m\,\nabla V^p-V^p\,\nabla{\cal E}_m\right)\cdot d{\bf S}&=
\int_C\nabla\cdot\left({\cal E}_m\,\nabla V^p-V^p\,\nabla{\cal E}_m\right)dC\nonumber\\[0.5ex]
&
=\int_C\left({\cal E}_m\,\nabla^2 V^p - V^p\,\nabla^2{\cal E}_m\right)dC = 0,
\end{align}
where use has been made of Eqs.~(\ref{e359}) and (\ref{e361}). The previous three equations imply that 
\begin{equation}
\psi_m^{\,p} = -\frac{1}{2\pi} \oint_{S}V^p\,\nabla {\cal E}_m\cdot d{\bf S}=  \oint_{\hat{r}=1}R^{\,2}\,V^p\,\hat{\nabla} {\cal E}_m\cdot \hat{\nabla}\hat{r}\,\frac{d\theta}{2\pi},
\end{equation}
where  use has been made of Eqs.~(\ref{e360}) and (\ref{e362}).
Thus, we can
write
\begin{equation}
\psi_m^{\,p} = \sum_{m'}\,H_{mm'}\,V_{m'}^{\,p},
\end{equation}
where 
\begin{equation}\label{e368}
H_{mm'}=\oint_{\hat{r}=1}R^{\,2}\,\hat{\nabla} {\cal E}_m\cdot \hat{\nabla}\hat{r}\,\exp(\,{\rm i}\,m'\,\theta)\,\frac{d\theta}{2\pi},
\end{equation}
and use has been made of Eq.~(\ref{e344}). 

The homogeneous vacuum response matrix can be written
\begin{equation}\label{e369}
H_{mm'} =-\frac{1}{2\pi}\oint_S {\cal E}_{m'}^{\,\ast}\,\nabla{\cal E}_m\cdot d{\bf S},
\end{equation}
where use has been made of Eqs.~(\ref{e358}), (\ref{e360}), and (\ref{e368}). It follows that
\begin{align}\label{e370}
H_{mm'}-H_{m'm}^{\,\ast} &= -\frac{1}{2\pi}\oint_S\left({\cal E}_{m'}^{\,\ast}\,\nabla{\cal E}_m-{\cal E}_m\,\nabla{\cal E}_{m'}^{\,\ast}\right)\cdot
d{\bf S}\nonumber\\[0.5ex]
&\phantom{=}-\frac{1}{2\pi}\int_C\nabla\cdot\left({\cal E}_{m'}^{\,\ast}\,\nabla{\cal E}_m-{\cal E}_m\,\nabla{\cal E}_{m'}^{\,\ast}\right)dC\nonumber\\[0.5ex]
&\phantom{=}-\frac{1}{2\pi}\int_C\left({\cal E}_{m'}^{\,\ast}\,\nabla^2{\cal E}_m-{\cal E}_m\,\nabla^2{\cal E}_{m'}^{\,\ast}\right)dC=0,
\end{align}
where use has been made of Eq.~(\ref{e359}). Thus, we conclude that $H_{mm'}$, as defined in Eq.~(\ref{e368}), is   Hermitian  to all orders in $\epsilon$. 

Equations~(\ref{e359}) and (\ref{e360}) can be automatically satisfied by writing
\begin{align}\label{e372}
{\cal E}_m&= a_m^{\,0}\,\frac{\sqrt{\pi}\,\Gamma(1/2-n)}{\sqrt{2}}\,(z-\cos\eta)^{1/2}\,P_{-1/2}^{\,n}(z)\\[0.5ex]
&\phantom{=}+ \sum_{m'\neq 0} a_{m}^{\,m'}\,\frac{\cos(|m'|\,\pi)\,\sqrt{\pi}\,\Gamma(|m'|+1/2-n)}{2^{\,|m'|-1/2}\,|m'|!\,\epsilon^{-|m'|}}\,
(z-\cos\eta)^{1/2}\,P_{m'-1/2}^{\,n}(z)\,\exp(\,{\rm i}\,m'\,\eta),\nonumber
\end{align}
where the $a_m^{\,m'}$ are complex coefficients. 
It follows from Eq.~(\ref{e358}) that
\begin{equation}
\oint_{\hat{r}=1}{\cal E}_m\,\exp(\,{\rm i}\,m'\,\theta)\,\frac{d\theta}{2\pi} = \delta_{mm'}.
\end{equation}
The previous two equations yield
\begin{equation}
\sum_{m''}a_m^{\,m''}\,{\cal P}_{m'}^{\,m''\ast}= \delta_{mm'},
\end{equation}
where use has been made of Eqs.~(\ref{e346}) and (\ref{e348}). 
It follows that 
\begin{equation}
\sum_{m''}{\cal P}_{m''}^{\,m\,\ast}\,a_{m''}^{\,m'}= \delta^{\,mm'}.
\end{equation}
Moreover, Eqs.~(\ref{e368}) and (\ref{e372}) give
\begin{equation}
H_{mm'} = \sum_{m''}a_m^{\,m''}\,{\cal R}_{m'}^{\,m''\ast},
\end{equation}
where use has been made of Eqs.~(\ref{e354}) and (\ref{e356}). The previous two equations can be combined to give
\begin{equation}\label{e376}
\sum_{m''}{\cal P}_{m''}^{\,m\,\ast}\,H_{m''m'}={\cal R}_{m'}^{\,m\,\ast},
\end{equation}
which is analogous in form to Eq.~(\ref{e298}). 

Alternatively, Eqs.~(\ref{e369}) and (\ref{e370}) can be combined to give
\begin{equation}
H_{mm'}=-\frac{1}{4\pi}\oint_S\left({\cal E}_{m'}\,\nabla{\cal E}_m+ {\cal E}_m\,\nabla{\cal E}_{m'}\right)\cdot d{\bf S},
\end{equation}
which yields 
\begin{equation}
H_{mm'} = \frac{1}{2}\sum_{m''}\left(a_m^{\,m''}\,{\cal R}_{m'}^{\,m''\ast}+{\cal R}_m^{\,m''}\,a_{m'}^{\,m''\ast}\right), 
\end{equation}
with the aid of Eqs.~(\ref{e354}),  (\ref{e356}), and (\ref{e372}). Of course, the previous expression for $H_{mm'}$ is manifestly Hermitian. 

\subsection{Inhomogeneous Boundary Condition at Plasma/Vacuum Interface}\label{svacsoln1}
Equations~(\ref{evdef}), (\ref{e343}), (\ref{e345}), (\ref{e351}), (\ref{e353}),  and (\ref{e376}) yield the
following inhomogeneous boundary condition at the plasma/vacuum interface:
\begin{equation}
\psi_m(1) = \sum_{m'}H_{mm'}\,\frac{Z_{m'}(1)}{m'-n\,q(1)}+\sum_{m'}G_{mm'}\,b_{m'},
\end{equation}
where 
\begin{equation}
G_{mm'} = {\cal S}_m^{\,m'}-\sum_{m''}\,H_{mm''}\,{\cal Q}_{m''}^{\,m'}.
\end{equation}
Note that the previous two equations are analogous in form to Eqs.~(\ref{inhom}) and (\ref{e338}), respectively. 

\subsection{Summary}
In Sects.~\ref{svacsoln}--\ref{svacsoln1}, we demonstrated that if the algebraic quantities ${\cal P}_m^{\,m'}$, ${\cal Q}_m^{\,m'}$, ${\cal R}_m^{\,m'}$,
and ${\cal S}_m^{\,m'}$, defined in Sect.~\ref{pertb}, are replaced by the corresponding  one-dimensional integrals defined in Sect.~\ref{svacsoln} then the
remainder of the vacuum analysis of Sect.~\ref{vacx} is unchanged, but the homogenous vacuum response matrix becomes exactly Hermitian, as is required by
toroidal angular momentum conservation. Obviously, algebraic quantities can be evaluated far more quickly than integrals. Nevertheless,
in situations in which angular momentum conservation is important, the additional computational cost involved in the evaluation of the integrals
is justified. 

\section{Calculation of  Tearing-Mode Dispersion Relation}
\subsection{Introduction}
Let the $m_j$, for $j=1,J$, be the poloidal mode numbers included in the calculation. Here, it is assumed that $m_{j+1}=m_j+1$ for $j=1,J-1$. 
Let there be $K$ rational surfaces in the plasma, and let the $k$th surface lie at radius $\hat{r}_k$, for $k=1,K$. Here,
it is assumed that $\hat{r}_{k+1}>\hat{r}_k$ for $k=1,K-1$. 

\subsection{Well-Behaved Solutions Launched from Magnetic Axis}\label{axisl}
Let us launch $J$ linearly independent, well-behaved solutions of the outer-region ODEs, (\ref{e61x}) and (\ref{e62x}), from the
magnetic axis, as described in Sect.~\ref{axis}. Let us then numerically integrate these solutions to the plasma/vacuum interface. The poloidal harmonics of
the solutions are denoted $\psi^a_{m_{j'}\,m_j}(\hat{r})$ and $Z^a_{m_{j'}\,m_j}(\hat{r})$, for $j,j'=1,J$. Here,
$m_{j'}$ is the poloidal mode number of the harmonic, whereas $m_j$ is the dominant poloidal mode number of the solution close to the
magnetic axis. The asymptotic matching conditions imposed at the rational surfaces are
\begin{align}
A_{L\,k}^{-} &= A_{L\,k}^+,\\[0.5ex]
{\mit\Delta\Psi}_k &= 0,
\end{align}
for $k=1,K$. (See Sect.~\ref{sa7}). 
Let ${\mit\Pi}_{kj}^a$, for   $k=1,K$ and $j=1,J$,  be the value of ${\mit\Psi}_k$ at the $k$th rational surface associated with a solution launched
from the magnetic axis with dominant poloidal mode number $m_j$. 

A scheme similar to A.H.~Glasser's ``fixups'' \cite{ham} is employed to periodically re-orthogonalize the set of solutions. These  re-orthogonalizations are implemented at user-defined locations between the magnetic axis and the plasma boundary.  At each re-orthogonalization location, the matrix of solutions is forced to become upper triangular, via a process similar to Gaussian elimination, such that only one solution has a non-zero amount of the highest poloidal  harmonic, two solutions have non-zero amounts of the next highest harmonic, and so on. The solutions are then renormalized to their largest component. 
 The re-orthogonalizations are
 necessary to prevent the solutions for becoming colinear as a result of rounding errors, given the significantly
 different rates at which poloidal harmonics with different poloidal mode numbers grow with increasing $\hat{r}$  close to the magnetic axis. (In fact, a
 harmonic with mode number $m$ grows as $\hat{r}^{\,|m|}$. Hence, an $m=10$ harmonic  grows far faster than an $m=1$ harmonic. Unchecked, each solution
 would quickly become dominated by its component with the largest mode number.)

\subsection{Small Solutions Launched from Rational Surfaces}\label{smalll}
Let us launch a ``small'' solution of the outer-region ODEs  from each rational surface in the plasma, as described in Sects.~\ref{sgen} and \ref{sspec}, 
 and numerically integrate it to the plasma/vacuum interface. The poloidal harmonics of
the solutions are denoted $\psi^s_{m_{j}\,k}(\hat{r})$ and $Z^s_{m_{j}\,k}(\hat{r})$, for $j=1,J$ and $k=1,K$. Here,
$m_{j}$ is the poloidal mode number of the harmonic, whereas $k$ is the index of the rational surface from which the solution is launched.
The launch conditions are
\begin{align}
A_{L\,k}^+ &=  0,\\[0.5ex]
{\mit\Delta\Psi}_k &=1.
\end{align}
The asymptotic matching conditions imposed at the other rational surfaces are
\begin{align}
A_{L\,k'}^{-} &= A_{L\,k'}^+,\\[0.5ex]
{\mit\Delta\Psi}_{k'} &= 0,
\end{align}
for $k'=k+1,K$.
Let 
${\mit\Pi}_{k'k}^s$, for $k'=1,K$ and $k=1,K$,  be the value of ${\mit\Psi}_{k'}$ at the $k'$th rational surface associated with a small solution
launched from the $k$th rational surface. Note that ${\mit\Pi}_{k'k}^s=0$ for $k' \leq k$.  

\subsection{Homogeneous Tearing-Mode Dispersion Relation}
The most general expression for the solution of the outer-region ODEs at the plasma/vacuum interface is
\begin{align}\label{e385}
\psi_{m_j}(1) &= \sum_{j'=1,J}\psi_{m_j\,m_{j'}}^a(1)\,\alpha_{j'} + \sum_{k=1,K}\psi_{m_j\,k}^s(1)\,{\mit\Delta\Psi}_k,\\[0.5ex]
Z_{m_j}(1) &= \sum_{j'=1,J}Z_{m_j\,m_{j'}}^a(1)\,\alpha_{j'} + \sum_{k=1,K}Z_{m_j\,k}^s(1)\,{\mit\Delta\Psi}_k,\label{e386}
\end{align}
for $j=1,J$, where the $\alpha_j$ are complex coefficients. However, in the absence of non-axisymmetric currents flowing in  external
magnetic field-coils, the solution must satisfy the homogenous boundary condition (\ref{e297}). 
It follows that
\begin{equation}
\sum_{j'=1,J}X_{jj'}\,\alpha_{j'} = \sum_{k=1,K}Y_{jk}\,{\mit\Delta\Psi}_k
\end{equation}
for $j=1,J$, where
\begin{align}
X_{jj'} &=\sum_{j''=1,J}H_{m_j\,m_{j''}}\,\frac{Z_{m_{j''}\,m_{j'}}^{a}(1)}{m_{j''}-n\,q(1)}-\psi_{m_j\,m_{j'}}^{a}(1),\\[0.5ex]
Y_{jk} &=\psi_{m_j\,k}^{s}(1)- \sum_{j''=1,J}H_{m_j\,m_{j'}}\,\frac{Z^s_{m_{j'}\,k}(1)}{m_{j'}-n\,q(1)}
\end{align}
for $j,j'=1,J$ and $k=1,K$. Thus, we can write
\begin{equation}
\alpha_j = \sum_{k=1,K}{\mit\Omega}_{jk}\,{\mit\Delta\Psi}_k
\end{equation}
for $j=1,J$, 
where
\begin{equation}
\sum_{j'=1,J}X_{jj'}\,{\mit\Omega}_{j'k} = Y_{jk}
\end{equation}
for $j=1,J$ and $k=1,K$. Our general solution is now free of arbitrary coefficients.
 Finally, making use of the definitions of the ${\mit\Psi}_{kj}^a$ and the ${\mit\Pi}_{kk'}^s$ given in Sects.~\ref{axisl} and \ref{smalll}, we obtain the {\em homogeneous tearing-mode dispersion relation},\cite{connor,cht,am1,pletz}
\begin{equation}\label{e315}
{\mit\Psi}_k = \sum_{k'=1,K}F_{kk'}\,{\mit\Delta\Psi}_{k'}
\end{equation}
for $k,k'=1,K$, where
\begin{equation}
F_{kk'}=\sum_{j=1,J}{\mit\Pi}^a_{kj}\,{\mit\Omega}_{jk'} + {\mit\Pi}^s_{kk'}.
\end{equation}
Equation~(\ref{e315}) specifies the reconnected magnetic flux, ${\mit\Psi}_k$, driven at each rational surface in the plasma
as a consequence of the current sheets, ${\mit\Delta\Psi}_k$, flowing at the surfaces. It is clear that $F_{kk'}$ is a dimensionless
inductance matrix.\cite{rfbook}

We can construct the ``fully reconnected" tearing eigenfunction\,\cite{am1} associated with the $k$th rational surface,
which is defined to have the following properties, 
\begin{align}
{\mit\Psi}_{k' }&= F_{k'k},\\[0.5ex]
{\mit\Delta\Psi}_{k'} &= \delta_{k'k}
\end{align}
for $k'=1,K$, 
as follows: 
\begin{align}
\psi_{m_j\,k}^{f}(\hat{r}) &=\psi^s_{m_j\,k}(\hat{r})+ \sum_{j'=1,J}\psi_{m_j\,m_{j'}}^a(\hat{r})\,{\mit\Omega}_{j'k},\\[0.5ex]
Z_{m_j\,k}^{f}(\hat{r}) &=Z^s_{m_j\,k}(\hat{r})+ \sum_{j'=1,J}Z_{m_j\,m_{j'}}^a(\hat{r})\,{\mit\Omega}_{j'k}
\end{align}
for $j=1,J$. Note that the fully reconnected solution associated with the $k$th rational surface only has a current sheet at that surface. 
(The current sheets at the other surfaces are all zero.)

The tearing-mode dispersion relation can be written in the alternative form\,\cite{cht,am1}
\begin{equation}\label{e317}
{\mit\Delta\Psi}_k = \sum_{k'=1,K}E_{kk'}\,{\mit\Psi}_{k'}
\end{equation}
for $k=1,K$, where $E_{kk'}$ is the inverse of $F_{kk'}$. The previous equation specifies the current sheets driven at each rational surface
in the plasma as a consequence of the reconnected fluxes at the surfaces. 

We can construct the ``unreconnected" tearing eigenfunction\,\cite{am1} associated with the $k$th rational surface,
which is defined to have the following properties, 
\begin{align}
{\mit\Psi}_{k' }&= \delta_{k'k},\\[0.5ex]
{\mit\Delta\Psi}_{k'} &= E_{k'k}
\end{align}
for $k'=1,K$, 
as follows: 
\begin{align}
\psi_{m_j\,k}^{u}(\hat{r}) &=\sum_{k'=1,K}\psi_{m_j\,k'}^{f}(\hat{r})\,E_{k'k},\\[0.5ex]
Z_{m_j\,k}^{u}(\hat{r}) &=\sum_{k'=1,K}Z_{m_j\,k'}^{f}(\hat{r})\,E_{k'k},
\end{align}
for $j=1,J$.  Note that the unreconnected solution associated with the $k$th rational surface only has reconnected flux at that surface. 
(The reconnected fluxes at the other surfaces are all zero.)

Let
\begin{equation}\label{e403}
{\mit\Delta}_k = \frac{{\mit\Delta\Psi}_k}{{\mit\Psi}_k}
\end{equation}
 be the complex quantity that characterizes the
tearing response of the resonant layer at the $k$th rational surface to the ideal-MHD solution in the outer region.\cite{fkr}
In general, ${\mit\Delta}_k$ is a function of the growth-rate and phase-velocity of the reconnected magnetic flux at the surface.\cite{am1,layer,layer1}
The previous two equations can be combined to give the ultimate form of the homogeneous tearing-mode dispersion relation,
\begin{equation}
\sum_{k'=1,k}({\mit\Delta}_k\,\delta_{kk'}-E_{kk'})\,{\mit\Psi}_{k'} = 0
\end{equation}
for $k=1,K$. It is clear that $E_{kk}$ is the tearing stability index\,\cite{fkr} at the $k$th rational surface when magnetic reconnection  takes place
at this surface, but is suppressed at the other surfaces (as is likely to be the case in the presence of sheared plasma rotation\,\cite{am1}). 

\subsection{Toroidal Electromagnetic Torques}
According to Eqs.~(\ref{e194t}) and (\ref{e317}), the net toroidal electromagnetic torque exerted on an isolated plasma
in the immediate vicinity of the $k$th rational surface is
\begin{equation}\label{e405}
\delta T_k = 2\pi^2\,n\sum_{k'=1,K} {\rm Im}({\mit\Psi}_k^{\,\ast}\,E_{kk'}\,{\mit\Psi}_{k'}).
\end{equation}
The total electromagnetic torque exerted on the plasma is
\begin{equation}
T_\phi(1) = \sum_{k=1,K}\delta T_k =  2\pi^2\,n\sum_{k,k'=1,K} {\rm Im}({\mit\Psi}_k^{\,\ast}\,E_{kk'}\,{\mit\Psi}_{k'}).
\end{equation}
However, we have already established that this total torque is zero, irrespective of the values of the ${\mit\Psi}_k$. [See Eq.~(\ref{e300})].
Thus, it follows that
\begin{equation}
E_{kk'} = E_{k'k}^{\,\ast}.
\end{equation}
In other words, the matrix $E_{kk'}$ is Hermitian,\cite{am1} which implies that $F_{kk'}$ is also Hermitian (as must  be the case if $F_{kk'}$ can
be interpreted as 
a dimensionless inductance matrix). 

\subsection{Inhomogeneous Tearing-Mode Dispersion Relation}
In the presence of non-axisymmetric currents flowing in external magnetic field-coils, the solution to the outer-region ODEs
must satisfy the inhomogeneous boundary condition (\ref{inhom}) at the plasma/vacuum interface. Combining this
condition with Eqs.~(\ref{e385}) and (\ref{e386}), we obtain the {\em inhomogeneous tearing-mode dispersion relation},
\begin{equation}\label{e408}
{\mit\Delta\Psi}_k=\sum_{k'=1,K}E_{kk'}\,{\mit\Psi}_k+\sum_{j=1,J}\chi_{k\,j}\,b_{m_j}
\end{equation}
for $k=1,K$. 
Here, the resonant magnetic perturbation (RMP) generated by the non-axisymmetric currents has a magnetic scalar potential  that takes the form (see Sect.~\ref{vacxx})
\begin{align}
V^x(z,\eta)&= b_{m_{j_0}}\,\frac{\sqrt{\pi}\,\Gamma(1/2-n)}{\sqrt{2}}\,(z-\cos\eta)^{1/2}\,Q_{-1/2}^{\,n}(z)\\[0.5ex]
&\phantom{=}+ \sum_{j=1,J}^{j\neq j_0} b_{m_j}\,\frac{\cos(|m_j|\,\pi)\,\sqrt{\pi}\,\Gamma(|m_j|+1/2-n)}{2^{\,|m_j|-1/2}\,|m_j|!\,\epsilon^{-|m_j|}}\,
(z-\cos\eta)^{1/2}\,Q_{m_j-1/2}^{\,n}(z)\,\exp(\,{\rm i}\,m_j\,\eta),\nonumber
\end{align}
where $m_{j_0}=0$. Moreover,
\begin{align}
\chi_{kj} &= \sum_{k'=1,K}E_{kk'}\,\Upsilon_{k'j},\\[0.5ex]
\Upsilon_{kj}&= \sum_{j'=1,J}{\mit\Pi}_{kj'}^{\,a}\,{\mit\Xi}_{j'j},\\[0.5ex]
\sum_{j'=1,J} X_{jj'}\,{\mit\Xi}_{j'j}&= G_{m_j\,m_{j'}}
\end{align}
for $k=1,K$ and $j=1,J$. Here, we have made use of the fact that $E_{kk'}$ is the inverse of $F_{kk'}$.

Under normal circumstances, magnetic reconnection driven by the RMP at the various rational surfaces in the plasma 
is strongly suppressed by plasma rotation.\cite{am1,rfbook} This situation corresponds to $|{\mit\Delta}_k|\gg |E_{kk}|$ for $k=1$, $K$.
It follows from Eqs.~(\ref{e403}) and (\ref{e408}) that the reconnected magnetic flux driven at the $k$th rational surface is
\begin{equation}
{\mit\Psi}_k\simeq\frac{ \sum_{j=1,J}\chi_{k\,j}\,b_{m_j}}{{\mit\Delta}_k}.
\end{equation}
Thus, according to Eq.~(\ref{e405}), the toroidal electromagnetic locking torque exerted at the surface by the RMP is
\begin{equation}
\delta T_k = 2\pi^2\,n\,\frac{{\rm Im}({\mit\Delta}_k)}{|{\mit\Delta}_k|^2}\,\left|\sum_{j=1,J}\chi_{k\,j}\,b_{m_j}\right|^2.
\end{equation}
It is clear that, for a fixed RMP amplitude, the RMP that maximizes the torque exerted at the $k$th rational surface has a
poloidal spectrum such that 
\begin{equation}
b_{m_j}\propto \chi_{k\,j},
\end{equation}
and a magnetic scalar potential such that $V_x\propto V_k^x$, where 
\begin{align}
V^x_k(z,\eta)&= \chi_{k\,j_0}\,\frac{\sqrt{\pi}\,\Gamma(1/2-n)}{\sqrt{2}}\,(z-\cos\eta)^{1/2}\,Q_{-1/2}^{\,n}(z)\\[0.5ex]
&\phantom{=}+ \sum_{j=1,J}^{j\neq j_0} \chi_{k\,j}\,\frac{\cos(|m_j|\,\pi)\,\sqrt{\pi}\,\Gamma(|m_j|+1/2-n)}{2^{\,|m_j|-1/2}\,|m_j|!\,\epsilon^{-|m_j|}}\,
(z-\cos\eta)^{1/2}\,Q_{m_j-1/2}^{\,n}(z)\,\exp(\,{\rm i}\,m_j\,\eta).\nonumber
\end{align}

\section*{Acknowledgements}
This research was directly funded by the U.S.\ Department of Energy, Office of Science, Office of Fusion Energy Sciences, under  contract DE-SC0021156. 
The analytic calculations  described  this paper have been verified using the {\sc REDUCE} computer algebra system.

\section*{Data Availability Statement}
The digital data used in the figures in this paper can be obtained from the author upon reasonable request.

\appendix
\section{Nonorthogonal Curvilinear Coordinates}\label{s2}
Consider the nonorthogonal curvilinear coordinate system, $r$, $\theta$, $\phi$,  introduced in Sect.~\ref{coords}, and let
${\cal J}=  (\nabla r\times \nabla\theta\cdot\nabla\phi)^{-1}$.

Let
\begin{align}
{\bf A} &= A^{\,r}\,{\cal J}\,\nabla\theta\times\nabla\phi
+  A^{\,\theta}\,{\cal J}\,\nabla\phi\times\nabla r
+  A^{\,\phi}\,{\cal J}\,\nabla r\times\nabla\theta,\\[0.5ex]
{\bf A} &= A_r\,\nabla r+A_\theta\,\nabla\theta+A_\phi\,\nabla\phi,
\end{align}
where ${\bf A}$ is a general vector.
It is easily demonstrated that
\begin{align}
{\bf A}\cdot{\bf B} &= A_r\,B^{\,r}+A_\theta \,B^{\,\theta}+A_\phi\,B^{\,\phi}=A^{\,r}\,B_r+A^{\,\theta}\,B_{\theta}+A^{\,\phi}\,B_\phi,\\[0.5ex]
({\bf A}\times {\bf B})_r &= {\cal J}\,(A^{\,\theta}\,B^{\,\phi}-A^{\,\phi}\,B^{\,\theta}),\label{crossdown1}\\[0.5ex]
({\bf A}\times {\bf B})_\theta &= {\cal J}\,(A^{\,\phi}\,B^{\,r}-A^{\,r}\,B^{\,\phi}),\label{crossdown2}\\[0.5ex]
({\bf A}\times {\bf B})_\phi &= {\cal J}\,(A^{\,r}\,B^{\,\theta}-A^{\,\theta}\,B^{\,r}),\label{crossdown3}\\[0.5ex]
{\cal J}\,({\bf A}\times {\bf B})^{\,r} &= A_{\theta}\,B_{\phi}-A_{\phi}\,B_{\theta},\label{crossup1}\\[0.5ex]
{\cal J}\,({\bf A}\times {\bf B})^{\,\theta} &= A_{\phi}\,B_{r}-A_{r}\,B_{\phi},\\[0.5ex]
{\cal J}\,({\bf A}\times {\bf B})^{\,\phi} &=A_{r}\,B_{\theta}-A_{\theta}\,B_{r},\label{crossup3}
\end{align}
where ${\bf B}$ is another general vector.
Furthermore,
\begin{align}
{\cal J}\,\nabla \cdot{\bf C} &= \frac{\partial\,({\cal J}\,C^{\,r})}{\partial r} + 
\frac{\partial\,({\cal J}\,C^{\,\theta})}{\partial \theta}+ 
\frac{\partial\,({\cal J}\,C^{\,\phi})}{\partial \phi},\label{div}\\[0.5ex]
{\cal J}\,(\nabla\times {\bf C})^{\,r}&= \frac{\partial C_\phi}{\partial \theta}
-\frac{\partial C_\theta}{\partial \phi},\label{curl1}\\[0.5ex]
{\cal J}\,(\nabla\times {\bf C})^{\,\theta}&= \frac{\partial C_r}{\partial \phi}
-\frac{\partial C_\phi}{\partial r},\\[0.5ex]
{\cal J}\,(\nabla\times {\bf C})^{\,\phi}&= \frac{\partial C_\theta}{\partial r}
-\frac{\partial C_r}{\partial \theta},\label{curl3}
\end{align}
where ${\bf C}({\bf r})$ is a general vector field.

\begin{thebibliography}{99}\baselineskip 5ex

\bibitem{fkr} H.P.~Furth,  J.~Killeen, and M.N.~Rosenbluth,  Phys.\ Fluids {\bf 6}, 459 (1963).

\bibitem{con0} J.W.~Connor, and R.J.~Hastie, {\em The Effect of Shaped Plasma Cross Sections on the Ideal Kink Mode in a Tokamak}. (Rep. CLM-M106, Culham Laboratory, Abingdon UK, 1985)

\bibitem{connor} J.W.~Connor,  S.C.~Cowley, R.J.~Hastie,  T.C.~Hender,  A.~Hood, and T.J.~Martin,  Phys.\ Fluids {\bf 31}, 577 (1988).

\bibitem{cht} J.W.~Connor, R.J.~Hastie, and J.B.~Taylor, Phys.\ Fluids B {\bf 3}, 1539 (1991).

\bibitem{am1} R.~Fitzpatrick, R.J.~Hastie, T.J.~Martin, and C.M.~Roach, Nucl.\ Fusion {\bf 33}, 1533 (1993).

\bibitem{tokuda} S.~Tokuda, Nucl.\ Fusion {\bf 41}, 1037 (2001).

\bibitem{brennan} D.P.~Brennan, R.J.~La Haye, A.D.~Turnbull, M.S.~Chu, T.H.~Jensen, L.L.~Lao, T.C.~Luce, P.A.~Politzer, and E.J.~Strait, Phys.\ Plasmas {\bf 10}, 1643 (2003).

\bibitem{ham} C.J.~Ham, J.W.~Connor, S.C.~Cowley, C.G.~Gimblett, R.J.~Hastie, T.C.~Hender, and T.J.~Martin, Plasma Phys.\ Controlled Fusion {\bf 54}, 025009 (2012). 

\bibitem{am3} R.~Fitzpatrick, Phys.\ Plasmas {\bf 24}, 072506 (2017). 

\bibitem{nish} Y.~Nishimura, J.D.~Callen, and C.C.~Hegna, Phys.\ Plasmas {\bf 5}, 4292 (1998).

\bibitem{pletz} A.~Pletzer, and R.L.~Dewar, J.\ Plasma Physics {\bf 45}, 427 (1991).

\bibitem{am2} A.H.~Glasser, Z.R.~Wang, and J.-K.~Park, Phys.\ Plasmas {\bf 23}, 112506 (2016).

\bibitem{aglas} A.S.~Glasser, E.~Kolemen, and A.H.~Glasser, Phys.\ Plasmas {\bf 25}, 032507 (2018).

\bibitem{aglas1} A.S.~Glasser, and E.~Koleman, Phys.\ Plasmas {\bf 25}, 082502 (2018). 

\bibitem{greene} J.M.~Greene, J.L.~Johnson, and K.E.~Weimer,  Phys.\  Fluids  {\bf 14}, 671 (1971).

\bibitem{chance} M.S.~Chance, Phys.\ Plasmas {\bf 4}, 2161 (1997).

\bibitem{xu} T.Xu,  R. Fitzpatrick, Nucl.\ Fusion {\bf 59}, 064002 (2019).

\bibitem{bussac} M.N.~Bussac, R.~Pellat, D.~Edery, and J.L.~Soule, Phys.\ Rev.\ Lett.\ {\bf 35}, 1638 (1975).

\bibitem{gs1} J.P.~Freidberg, {\em Ideal Magnetohydrodynamics}. (Plenum, New York NY, 1987.)

\bibitem{flow} R.~Iacono, A.~Bondeson, F.~Troyon, and R.~Gruber, Phys.\ Fluids B {\bf 2}, 1794 (1990).

\bibitem{flow1} L.~Guazzotto,  R.~Betti, J.~Manickam, and  S.~Kaye, Phys.\ Plasmas {\bf 11}, 604 (2004).

\bibitem{rfa} R.~Fitzpatrick, Nucl.\ Fusion {\bf 33}, 1049 (1993).

\bibitem{ggj} A.H.~Glasser, J.M.~Greene, and J.L.~Johnson, Phys.\ Fluids {\bf 18}, 875 (1975).

\bibitem{mercier} C.~Mercier, Nucl.\ Fusion {\bf 1}, 47 (1960).

\bibitem{twist} R. Fitzpatrick, Phys.\ Plasmas {\bf 1}, 3308 (1994).

\bibitem{gim} R.~Fitzpatrick, C.G.~Gimblett, and R.J.~Hastie, Plasma Phys.\ Control.\ Fusion {\bf 34}, 161 (1992). 

\bibitem{fitz2024} R.~Fitzpatrick, {\em Inverse Aspect-Ratio Expanded Tokamak Equilibria}, submitted to Physics of Plasmas (2024).

\bibitem{shaf} V.D.~Shafranov, Atomnaya \'{E}nergiya {\bf 3}, 521  (1962).

\bibitem{morse} P.M.~Morse, and H.~Feshbach, {\em Methods of Theoretical Physics}, p.~1301. (McGraw-Hill, New York NY, 1953)

\bibitem{morse1} P.M.~Morse, and H.~Feshbach, {\em Methods of Theoretical Physics}, p.~1302. (McGraw-Hill, New York NY, 1953)

\bibitem{abrama} M.~Abramowitz, and I.A.~Stegun, {\em Handbook of Mathematical Functions}, sect.~8.11. (Dover, New York NY, 1964)

\bibitem{abram2} M.~Abramowitz, and I.A.~Stegun, {\em Handbook of Mathematical Functions}, sect.~8.2. (Dover, New York NY, 1964)

\bibitem{morse2} P.M.~Morse, and H.~Feshbach, {\em Methods of Theoretical Physics}, p.~1302, p.~1329. (McGraw-Hill, New York NY, 1953)

\bibitem{bate} A.~Erd\'{e}lyi, W.~Magnus, F.~Oberhettinger, and F.C.~Tricomi, {\em Higher Transcendental Functions}, vol.~1, sect.~1.7.3. 
(McGraw-Hill, New York NY, 1953)

\bibitem{abram1} M.~Abramowitz, and I.A.~Stegun, {\em Handbook of Mathematical Functions}, ch.~6. (Dover, New York NY, 1964)

\bibitem{morse3} P.M.~Morse, and H.~Feshbach, {\em Methods of Theoretical Physics}, p.~1330. (McGraw-Hill, New York NY, 1953)

\bibitem{rfbook} R.~Fitzpatrick, {\em Tearing Mode Dynamics in Tokamak Plasmas}. (IOP, Bristol UK,  2023)

\bibitem{layer} R.~Fitzpatrick, Phys.\ Plasmas {\bf 5}, 3325 (1998).

\bibitem{layer1} A. Cole, and R. Fitzpatrick, Phys.\ Plasmas {\bf 13}, 032503 (2006).

\end{thebibliography}

\end{document}


